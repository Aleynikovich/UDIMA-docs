\documentclass{article}
\usepackage{lipsum}
\usepackage[backend=biber]{biblatex}
\usepackage{authoraftertitle}
\usepackage[top=2cm,bottom=1.5cm,left=1.5cm, right=3cm,includeheadfoot]{geometry}
\usepackage{graphicx}
\usepackage{fancyhdr}
\usepackage[spanish]{babel}
\usepackage{mathtools}
\usepackage{csquotes}
\usepackage{amssymb}
\usepackage{fancybox, graphicx}
\usepackage{array}
\usepackage{hhline}
\usepackage{hyperref}
\usepackage{parskip}
\usepackage{tikz}
\usepackage{amsmath}
\usepackage{wrapfig}
\usepackage{float}
\usepackage{siunitx}
\usepackage{amsmath}
\usepackage{caption}
\usepackage{esvect}
\usepackage{siunitx}
\usepackage{commath}
\newcommand{\ihat}{\textbf{\^\i}}
\newcommand{\jhat}{\textbf{\^\j}}
%Header & Footer

\pagestyle{fancy}
%\fancyhead[LE]{\MyTitle}
\fancyhead[LO]{Técnicas de optimización de sistemas industriales}
%\fancyhead[RO]{\leftmark}
%\fancyhead[RE]{\leftmark}
\fancyfoot[L]{\raisebox{-1cm}{\includegraphics[height=1.5cm]{D:/KUKADisk/UDIMA/DocumentGraphics/LOGOUDIMA.jpg}}}
\fancyfoot[R]{Corregido:\\ Dr. Fco. David de la Peña Esteban}
%\fancyfoot[RO]{07/12/2018}


%Vars
\author{Alexander Sebastian Kalis}
\title{AEC 2 - Unidades 2, 4 y 6}


%DOC


\begin{document}

\begin{titlepage}

    \begin{center}

        \line(1,0){300}\\
        [0.2in]
        \huge{\bfseries {\MyTitle}}\\
        [1mm]
        \line(2,0){200}\\
        [0.75cm]
        \textsc{\LARGE Técnicas de optimización de sistemas industriales}\\
        [2cm]
        \includegraphics[height=10cm]{D:/KUKADisk/UDIMA/OptimizacionIndustrial/AEC2/portada.png}\\
        [3cm]

    \end{center}

    \begin{flushright}

        Autor: {\MyAuthor}\\
        Profesor: Dr. Fco. David de la Peña Esteban\\
        Curso: Ingeniería de Organización Industrial\\
        UDIMA         

    \end{flushright}
    
\end{titlepage}

\section*{Caso 1. Teoría de juegos}

\subsection*{1.A.}

Existen dos plataformas de streaming de películas y series que se reparten un mismo
segmento de mercado, y están analizando las estrategias a seguir en España para el próximo
año. En principio este segmento de mercado está saturado, y los clientes que gana una
compañía, los pierde la otra.
La matriz de pagos, donde pone los clientes (en miles) que gana uno y pierde el otro es:

\begin{center}
    \includegraphics[height=4cm]{D:/KUKADisk/UDIMA/OptimizacionIndustrial/AEC2/p1ae1.PNG}\\
\end{center}

Se pide:

-Aplicando el método de maximin – minimax ¿existe punto de equilibrio? ¿cuál es?\\
-Aplicar estrategias dominadas y determinar el punto de equilibrio, si es que existe.\\

Aplicando maximin - minimax podemos ver que existe el punto de equilibrio cuando el jugador 1 aplica la estrategia 1 y el jugador
2 aplica la estrategia 3, pues ambos son el máximo y el mínimo y además coinciden en la misma celda. Su valor es -10.

\begin{center}
    \includegraphics[height=4cm]{D:/KUKADisk/UDIMA/OptimizacionIndustrial/AEC2/p1ar1.PNG}\\
\end{center}

Aplicando estrategias dominadas, podemos observar que la estrategia 3 siempre resulta más favorable que la estrategia 4 para 
el jugador 2. Por otro lado, la estrategia 1 siempre es mejor que la 2 para el jugador 1.

Se ha programado un simple simulador en C\# que permite introducir los datos de la matriz de pagos y asegurarnos del resultado. 

El profesor puede consultar el código \href{https://github.com/Aleynikovik/DecisionMatrix/blob/master/Program.cs}{pulsando en este enlace}.

\begin{center}
    \includegraphics[height=2.9cm]{D:/KUKADisk/UDIMA/OptimizacionIndustrial/AEC2/p1ae2.PNG}
\end{center}


Si realizamos el proceso de eliminación de forma continua, podemos observar que obtenemos el mismo resultado que utilizando
el método maximin-minimax. El punto de equilibrio es -10 y se consigue cuando el jugador 1 aplica la estrategia 1 y el 
jugador 2 aplica la estrategia 3.


\begin{center}
    \includegraphics[width=0.7\textwidth]{D:/KUKADisk/UDIMA/OptimizacionIndustrial/AEC2/p1art.PNG}\\
\end{center}

\newpage

\section*{1.B.}

Dado la matriz de pago del siguiente juego, aplicando el método de estrategias 
dominadas, ¿existe punto de equilibrio?

\begin{center}
    \includegraphics[width=0.7\textwidth]{D:/KUKADisk/UDIMA/OptimizacionIndustrial/AEC2/p1be1.PNG}\\
\end{center}

Aplicando el método de estrategias dominadas, obtenemos el siguiente resultado:

\begin{center}
    \includegraphics[width=0.7\textwidth]{D:/KUKADisk/UDIMA/OptimizacionIndustrial/AEC2/p1br1.PNG}\\
\end{center}

Las celdas verdes representan la estrategia dominante mientras que las celdas amarillas representan las 
estrategias dominadas y por tanto son eliminadas de matriz de pagos.

El punto de equilibrio existe cuando la compañía 1 aplica la estrategia 1 y la compañía 2 aplica la estrategia 2,
resultando en una ganancia de 10 para la compañía 1 y 100 para la compañía 2.


\newpage

\subsection*{1.C.}

Poner el siguiente juego simultáneo en su forma extensiva, donde hay dos jugadores (J1 y 
J2), teniendo cada uno de ellos 3 posibles estrategias (E1, E2 y E3). ¿Cuál sería el punto de 
equilibrio?

\begin{center}
    \includegraphics[width=0.7\textwidth]{D:/KUKADisk/UDIMA/OptimizacionIndustrial/AEC2/p1ce1.PNG}\\
\end{center}


Representamos el modelo de árbol:

\begin{center}
    \includegraphics[width=1\textwidth]{D:/KUKADisk/UDIMA/OptimizacionIndustrial/AEC2/p1cr1.PNG}\\
\end{center}


Eliminando desde el punto de vista de J2 nos queda:


\begin{center}
    \includegraphics[width=1\textwidth]{D:/KUKADisk/UDIMA/OptimizacionIndustrial/AEC2/p1cr2.PNG}\\
\end{center}

Y por tanto ahora J1 decide la estrategia que mas le conviene de las restantes. En este caso podemos ver que será la estrategia E2.
De este modo concluimos que el punto de equilibrio está cuando ambos jugadores escogen E2 y corresponde a (7,6).

\newpage 


\section*{Caso 2. Árbol de decisión}

Una empresa dedicada a la comercialización de material deportivo especializado está
decidiendo la apertura de una nueva sede en España, debiendo elegir entre planta grande o
pequeña. 

La primera decisión que debe tomar es si realizar con carácter previo un estudio de mercado que
tiene un coste de 25.000 €. Se estima que la probabilidad de que el estudio indique que el
mercado es favorable para los intereses de la empresa es del 70\%. Si el estudio indica que el
mercado será favorable, existe un 90\% de probabilidades de que la demanda sea alta, y de un
10\% de que la demanda fuese baja.

Si el estudio indicara que el mercado es desfavorable la probabilidad de que la demanda sea alta
es de un 10\%, mientras que en un 90\% de los casos la demanda será baja. 

Si no se realiza el estudio de mercado, las probabilidades de que la demanda sea alta o baja se
estiman en un 50\% ante la falta de información adicional.

En todas las posibilidades si la empresa se decidiese por la planta pequeña los beneficios serían
de 35.000 € anuales con demanda alta y de 8.000 € anuales con demanda baja. 

En caso de decidirse por la planta grande los beneficios serían de 80.000 € anuales con demanda
alta y unas pérdidas de 50.000 € anuales con demanda baja, para todas las posibilidades.

Se pide: Realizar el árbol de decisión asociado a esta problemática, considerando los
próximos 3 años. ¿Qué decisión debe tomar la empresa para maximizar su beneficio?

\hspace{1cm}


\begin{center}
    \includegraphics[width=.9\textwidth]{D:/KUKADisk/UDIMA/OptimizacionIndustrial/AEC2/p2cr1.PNG}
\end{center}

Con todas estas ramas obtendríamos la siguiente tabla:

\hspace{2cm}

\begin{center}
    \includegraphics[width=.9\textwidth]{D:/KUKADisk/UDIMA/OptimizacionIndustrial/AEC2/p2cr2.PNG}
\end{center}

Pasamos ahora a simplificar las alternativas de derecha a izquierda:

\hspace{2cm}


\begin{center}
    \includegraphics[width=.5\textwidth]{D:/KUKADisk/UDIMA/OptimizacionIndustrial/AEC2/p2cr3.PNG}
\end{center}

\hspace{1cm}

En este caso como lo que buscamos es maximizar beneficios, escogemos entonces la opción de la planta grande
realizando el estudio del mercado que resulta en un beneficio trienal total ponderado de 82400.



\end{document}