\documentclass{article}
\usepackage{lipsum}
\usepackage{authoraftertitle}
\usepackage[top=2cm,bottom=1.5cm,left=1.5cm, right=3cm,includeheadfoot]{geometry}
\usepackage{graphicx}
\usepackage{fancyhdr}
\usepackage[utf8]{inputenc}
\usepackage[spanish]{babel}
\usepackage{mathtools}
\usepackage{csquotes}
\usepackage{enumitem}
\usepackage{amssymb}
\usepackage{fancybox, graphicx}
\usepackage{array}
\usepackage{hhline}
\usepackage{textcomp}
\usepackage{eurosym}
\usepackage{hyperref}
\usepackage{parskip}
\usepackage{tikz}
\usepackage{amsmath}
\usepackage{wrapfig}
\usepackage{float}
\usepackage{siunitx}
\usepackage{amsmath}
\usepackage{caption}
\usepackage{esvect}
\usepackage{siunitx}
\usepackage{commath}

%Header & Footer
\pagestyle{fancy}
%\fancyhead[LE]{\MyTitle}
\fancyhead[LO]{Técnicas de optimización de sistemas industriales}
\fancyhead[RO]{}
%\fancyhead[RE]{\nouppercase\leftmark}
\fancyfoot[L]{\raisebox{-1cm}{\includegraphics[height=1.5cm]{D:/KUKADisk/UDIMA/DocumentGraphics/LOGOUDIMA.jpg}}}
\fancyfoot[R]{Corregido:\\ Dr. Fco. David de la Peña Esteban}
%\fancyfoot[RO]{07/12/2018}


%Vars
%\author{Alexander Sebastian Kalis}
\title{AEC 3 - Unidades 8 y 9}


%DOC


\begin{document}

\begin{titlepage}

    \begin{center}

        \line(1,0){300}\\
        [0.2in]
        \huge{\bfseries {\MyTitle}}\\
        [1mm]
        \line(2,0){200}\\
        [0.75cm]
        \textsc{\LARGE Técnicas de optimización de sistemas industriales}\\
        [2cm]
        \includegraphics[height=10cm]{Items/portada.png}\\

    \end{center}

    \begin{flushright}

        José Luís Galán\\
        Jaume Lorente\\
        Raquel Domingo Lazaro\\
        Alexander Sebastian Kalis\\
        [.5cm]
        Profesor: Dr. Fco. David de la Peña Esteban\\
        Curso: Ingeniería de Organización Industrial\\
        UDIMA         

    \end{flushright}
    
\end{titlepage}

\thispagestyle{plain}
\tableofcontents

\newpage

\section{Modelo con aprovisionamiento y consumo simultáneo}

Una parte de nuestro productivo necesita 40.000 unidades anuales de un componente. Ese
componente lo suministra otra parte de nuestro sistema productivo con una tasa de 500
unidades por día. Se estima su coste en 5\euro \ por unidad. Su coste de almacenaje anual es de 1\euro \ por unidad.
El coste de preparación del pedido se estima en 200\euro.

Un año se considera que tiene 250 días laborables.

Se pide:

\begin{enumerate}

    \item ¿Cuál será el tamaño del lote económico? El lote económico debe ser múltiplo de 500
    unidades. Redondear por debajo.
    \item Realizar la representación gráfica del problema.
    \item Obtener también el punto de pedido para estas dos situaciones:
     
    \begin{enumerate}[label*=\arabic*.]
        \item Si el periodo de suministro es de 5 días
        \item Si el periodo de suministro es de 20 días
    \end{enumerate}     

\end{enumerate}

\subsection{Lote económico}

Datos:

Da = 40.000 unidades/año

Ce = 200\euro \ por pedido

Cal = 1\euro \ unidad/año

Cc = 5\euro \ unidad

p = 500 unidades/día

d = Da/N días año = 160 unidades/día

\[
    Q^*=\sqrt{\cfrac[]{2\cdot Da \cdot Ce}{Cal}\cdot \cfrac[]{p}{(p-d)}}=
    \sqrt{\cfrac[]{2\cdot 40000\cdot 200}{1}\cdot \cfrac[]{500}{(500-160)}}=
    4850.71 \approx 4500 \ uds
\]


\subsection{Representación gráfica}

Tiempo necesario de producción:
\[
    t=\frac{Q^*}{p}=\frac{4500}{500}=9 \ d
\]

Pedidos anuales: 

\[
    \frac{Da}{Q^*}=9 \ pedidos
\]

Tiempo entre dos pedidos (días de año / pedidos por año): 

\[
    T = \frac{160}{9} = 28 \ d
\]

\begin{figure}[!htb]
    \begin{center}
        \includegraphics[height=7cm]{Items/p1f1.PNG}
        \caption{Representación gráfica}
    \end{center}
\end{figure}

\subsection{Suministro a 5 días}

\[
    Pp=d \cdot PS1 = 160 \cdot 5 = 800 \ uds
\]

\subsection{Suministro a 20 días}

\[
    Pp=(p-d)\cdot (T-PS)=2720 \ uds
\]



\newpage

\section{Modelo de descuentos por cantidad}

Nuestra empresa necesita un componente para la línea de producción. Anualmente necesita
75.000 unidades, y se tiene una empresa proveedora que nos lo suministra a un precio de 82\euro \
por unidad si la cantidad comprada es inferior a 1000 unidades, de 80\euro \ por unidad en caso de
superar esta cifra, y de 75\euro \ por unidad si se igualan o superan las 1800 unidades. Cada vez
que se hace un pedido hay unos costes fijos de 200\euro.

Para este componente el factor K que relaciona su coste de almacenamiento con el precio es de
$0.15$.

Se pide:

\begin{enumerate}

    \item Realizar la representación gráfica del problema.
    \item ¿Qué cantidad interesa pedir a nuestro proveedor?
    
\end{enumerate}


\subsection{Representación gráfica}


$Da = 75.000$

$Ce = 200$

$Cc_1 = 82$ para $ Q < 1000$

$Cc_2 = 80$ para $ 1000 \leq Q < 1800$

$Cc_3 = 75$ para $ Q \geq 1800$

$K = 0,15$



\[
    Q^*_1=\sqrt{\frac{2\cdot Da \cdot Ce}{Cc_1 \cdot K}}=
    \sqrt{\frac{2\cdot 75000 \cdot 200}{82 \cdot 0.15}}=
    1561.73 \approx 1561 \ uds
\]

\[
    Q^*_2=\sqrt{\frac{2\cdot Da \cdot Ce}{Cc_2 \cdot K}}=
    \sqrt{\frac{2\cdot 75000 \cdot 200}{80 \cdot 0.15}}=
    1581.13 \approx 1581 \ uds
\]

\[
    Q^*_3=\sqrt{\frac{2\cdot Da \cdot Ce}{Cc_3 \cdot K}}=
    \sqrt{\frac{2\cdot 75000 \cdot 200}{75 \cdot 0.15}}=
    1632.99 \approx 1632 \ uds
\]

\begin{figure}[!htb]
    \begin{center}
        \includegraphics[height=6.5cm]{Items/p2f1.PNG}
        \caption{Representación gráfica}
    \end{center}
\end{figure}


\subsection{Cantidad a pedir al proveedor}

Calculamos las opciones disponibles:

\[
    Q=Q^*_2=1581 \ uds
\]

\[
    CT2=Da\cdot Cc_2 + \frac{Q}{2}\cdot Cc_2 \cdot K + \frac{Da}{Q} \cdot Ce=
    75000\cdot 80 + \frac{1581}{2} \cdot 80 \cdot 0.15 + \frac{75000}{1581}=
    6009533.44 \ eur
\]

Para $Q=q_2=1800$ unidades:

\[
    CT2=Da\cdot Ccq_2 + \frac{Q}{2}\cdot Ccq_2 \cdot K + \frac{Da}{Q} \cdot Ce=
    750000\cdot 75+\frac{1800}{2}\cdot 75 \cdot 0.15 + \frac{75000}{1800} = 56351666.67 \ eur
\]

Con lo cual interesa pedir 1800 unidades.


\newpage 

\section{Simulación}

Una empresa de distribución quiere analizar su sistema de entrega de paquetes por drones,
contando en la actualidad con 2 drones. Los paquetes a enviar llegan a la oficina de
expediciones cada cierto tiempo, según la siguiente distribución de probabilidad:

\begin{figure}[!htb]
    \begin{center}
        \includegraphics[width=.4\textwidth]{Items/p3e1.PNG}
        %\caption{Representación gráfica}
    \end{center}
\end{figure}

Al paquete a enviar se le asigna un dron para el envío, aquel dentro de los disponibles que lleve
más tiempo parado. El primer envío lo realiza el dron 1. El tiempo que emplea cada dron (en
minutos) en el total del envío y su vuelta al centro se rige por la siguiente distribución de
probabilidades:

\begin{figure}[!htb]
    \begin{center}
        \includegraphics[width=.8\textwidth]{Items/p3e2.PNG}
        %\caption{Representación gráfica}
    \end{center}
\end{figure}


Se pide:

Realizar una simulación de los primeros 12 envíos, calculando el tiempo total en realizarlos. La
simulación comienza en $t=0$ justo cuando llega el primer paquete al centro de expedición.

Las semillas a tomar para la generación de números aleatorios son las siguientes:

\begin{itemize}

    \item Tiempo que transcurre entre la llegada de paquetes a la oficina de expediciones: 4526
    \item Tiempo del dron 1: 3678
    \item Tiempo del dron 2: 7210
     
\end{itemize}

\newpage

\subsection{Simulación de 12 envíos}

\begin{figure}[!htb]
    \begin{center}
        \includegraphics[width=\textwidth]{Items/p3f1.PNG}
        \caption{RNG}
    \end{center}
\end{figure}


\begin{figure}[!htb]
    \begin{center}
        \includegraphics[width=.9\textwidth]{Items/p3f2.PNG}
        \caption{Rangos}
    \end{center}
\end{figure}

En este caso debemos tener en cuenta los drones que están disponibles mientras uno va y el otro vuelve:

\begin{figure}[H]
    \begin{center}
        \includegraphics[width=.9\textwidth]{Items/p3f3.PNG}
        \caption{Solución}
    \end{center}   
\end{figure}


\end{document}