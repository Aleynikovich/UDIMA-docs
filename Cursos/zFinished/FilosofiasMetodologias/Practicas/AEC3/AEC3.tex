\documentclass{article}
\usepackage{lipsum}
\usepackage{authoraftertitle}
\usepackage[top=2cm,bottom=1.5cm,left=1.5cm, right=3cm,includeheadfoot]{geometry}
\usepackage{graphicx}
\usepackage{fancyhdr}
\usepackage[utf8]{inputenc}
\usepackage[spanish]{babel}
\usepackage{mathtools}
\usepackage{csquotes}
\usepackage{enumitem}
\usepackage{amssymb}
\usepackage{fancybox, graphicx}
\usepackage{array}
\usepackage{hhline}
\usepackage{textcomp}
\usepackage{eurosym}
\usepackage{hyperref}
\usepackage{parskip}
\usepackage{tikz}
\usepackage{amsmath}
\usepackage{wrapfig}
\usepackage{float}
\usepackage{siunitx}
\usepackage{amsmath}
\usepackage{caption}
\usepackage{esvect}
\usepackage{siunitx}
\usepackage{commath}

%Header & Footer
\pagestyle{fancy}
%\fancyhead[LE]{\MyTitle}
\fancyhead[LO]{Filosofías y metodologías industriales}
\fancyhead[RO]{}
%\fancyhead[RE]{\nouppercase\leftmark}
\fancyfoot[L]{\raisebox{-1cm}{\includegraphics[height=1.5cm]{D:/KUKADisk/UDIMA/DocumentGraphics/LOGOUDIMA.jpg}}}
\fancyfoot[R]{Corregido:\\ Dr. Fco. David de la Peña Esteban}
%\fancyfoot[RO]{07/12/2018}


%Vars
%\author{Alexander Sebastian Kalis}
\title{AEC 3 - Unidad 9 - Casos prácticos Seis Sigma}


%DOC


\begin{document}

\begin{titlepage}

    \begin{center}

        \line(1,0){300}\\
        [0.2in]
        \huge{\bfseries {\MyTitle}}\\
        [1mm]
        \line(2,0){200}\\
        [0.75cm]
        \textsc{\LARGE Filosofías y metodologías industriales}\\
        [2cm]
        \includegraphics[height=9cm]{Items/portada.png}\\
        [2cm]

    \end{center}

    \begin{flushright}

        Juan Galdón\\
        Milagros Villena Pérez\\
        Alexander Sebastian Kalis\\
        [.5cm]
        Profesor: Dr. Fco. David de la Peña Esteban\\
        Curso: Ingeniería de Organización Industrial\\
        UDIMA         

    \end{flushright}
    
\end{titlepage}

\thispagestyle{plain}
\tableofcontents

\newpage

\section{Estratificación con ayuda de diagramas de dispersión}

En un equipo de mejora de procesos se obtienen los siguientes datos que recogen el nº de
defectos por lote en conjuntos de 10.000 piezas, en función del tiempo de tratamiento al que se
someten y la materia prima utilizada, para una muestra de 50 datos:

\begin{center}
    \includegraphics[height=11cm]{Items/p1e1.png}\\
\end{center}

Se pide:

a) Mediante un diagrama de dispersión, analizar la relación entre el número de defectos
y el tiempo de tratamiento.

b) Indicar qué tipo de relación se observa. Calcular el coeficiente de correlación y
explicar su valor.

c) Realizar una estratificación por las materias primas (A y B) utilizadas. Indicar los
coeficientes de correlación y qué tipo de relación se observa.



\subsection{Diagrama de dispersión}

En la Figura 1 podemos observar que no hay una aparente correlación entre el tiempo de tratamiento y los defectos pues muchos de los puntos
se encuentran fuera de la línea de moda.

\begin{figure}[!htb]
    \begin{center}
        \includegraphics[height=7cm]{Items/p1f1.PNG}
        \caption{Diagrama de dispersión}
    \end{center}
\end{figure}

\subsection{Relación y coeficiente de correlación}

Como se ha comentado en el apartado anterior, la correlación es positiva débil puesto que la línea de moda es positiva y los puntos
no forman una línea recta sobre ella.

Calculando el coeficiente de correlación obtenemos $r=0.18$.

Al ser un valor cercano a 0 podemos decir que hay muy poca correlación entre estos datos.

\subsection{Estratificación por materias primas}

Separamos los diagramas por materia prima:

\begin{figure}[!htb]
    \begin{center}
        \includegraphics[width=\textwidth]{Items/p1f2.PNG}
        \caption{Diagramas de dispersión por materia prima}
    \end{center}
\end{figure}

De la misma forma observamos en ambos diagramas una correlación positiva débil. El coeficiente de la materia prima A resulta ser $r_a=0.16$
y el coeficiente de la materia prima B es $r_b=0.20$. Podríamos decir que la materia prima B responde mejor a mayor tiempo de tratamiento
pero al ser correlaciones tan débiles, no sería una conclusión válida o útil.

\newpage

\section{Estratificación}

El Departamento de informática de una universidad da tres servicios principales en el
edificio:

-Atención al puesto de trabajo

-Wifi invitados

-Atención a aulas de laboratorio

Normalmente se tienen 3 tipos de incidencias principales asociadas a estos servicios. En
la siguiente tabla se detallan el número de incidencias que han tenido lugar para cada
servicio y día de la semana pasada:

\begin{center}
    \includegraphics[height=7cm]{Items/p2e1.png}\\
\end{center}

También se tienen datos del tiempo medio de resolución de las incidencias:

\begin{center}
    \includegraphics[height=7cm]{Items/p2e2.png}\\
\end{center}

Se pide:

Considerando que la gravedad de los tres tipos de incidencias son iguales, así como el
volumen de los servicios ofertados, realizar un análisis estratificado de la situación
actual y detectar qué factor es el más significativo.

¿A qué conclusiones se llega? ¿Echáis en falta algún dato adicional?


\subsection{Análisis estratificado}

Extraemos la información sobre las incidencias y las ordenamos por el tipo de incidencia. Representamos gráficamente el diagrama 
de Pareto:

\begin{figure}[!htb]
    \begin{center}
        \includegraphics[height=7.5cm]{Items/p2f1.PNG}
        \caption{Diagrama de Pareto, tipos de incidencias}
    \end{center}
\end{figure}

Lo que nos dice que las incidencias de tipo 2 son las más comunes representando más de la mitad de ellas y que las incidencias
de tipo 3 son las que menos peso tienen.

Procedemos a analizar las incidencias respecto al tipo de servicio:


\begin{figure}[!htb]
    \begin{center}
        \includegraphics[height=7.5cm]{Items/p2f2.PNG}
        \caption{Diagrama de Pareto, tipos de servicios}
    \end{center}
\end{figure}

En este caso observamos que tanto las aulas de laboratorio como la atención al puesto de trabajo son las más importantes.


Ahora vamos a centrarnos en las inidencias de tipo 2 puesto que representan la mayor parte de incidencias.

Por último nos fijamos en el tiempo de respuesta en cada tipo de incidencia:

\begin{figure}[!htb]
    \begin{center}
        \includegraphics[height=7.5cm]{Items/p2f3.PNG}
        \caption{Diagrama de Pareto, tiempo de respuesta}
    \end{center}
\end{figure}

\subsection{Conclusión}

Podemos ver que las incidencias de tipo 3, aunque anteriormente se ha observado que son las menos comunes, son aquellas que toman
más tiempo de resolución con lo cual no podemos quitarles importancia.

De igual forma no disponemos de datos que nos permitan saber qué provoca las incidencias como para poder darles una solución. Es un
dato que se echaría en falta.

\newpage

\section{Poka-Yoke}

\subsection{Ejemplo 1}

Un ejemplo de Poka-Yoke utilizado en un componente eléctrico de la maquinaria con la que trabajo es esta fuente de 
alimentación de servomotores (KUKA KPP).

Aunque los dos conectores negros superiores parecen iguales a simple vista, realizan una función muy diferente. El conector
de la izquierda sirve para que los motores descarguen su energía residual sobre una resistencia de lastre, mientras que el 
conector de la derecha es donde se conecta la alimentación. Conectarlos de forma inversa provocaría un sobrecalentamiento
de la resistencia y posiblemente haría que prenda fuego. Los conectores están fabricados de tal forma que no se pueden 
llegar a conectar de forma inversa.

\begin{figure}[!htb]
    \begin{center}
        \includegraphics[height=16cm]{Items/kpp.jpg}
        \caption{KUKA KPP, Poka-Yoke}
    \end{center}
\end{figure}

\subsection{Ejemplo 2}

El segundo ejemplo es algo más común y es la memoria RAM utilizada en ordenadores convencionales. Cada placa base admite un tipo de RAM
distinta.

Dependiendo de su tipo (DDR-2-3-4) tienen un slot con una muesca colocada en una poición distinta. También hace que sea imposible 
de colocar al revés:

\begin{figure}[!htb]
    \begin{center}
        \includegraphics[height=11cm]{Items/ram.png}
        \caption{Memoria RAM, Poka-Yoke}
    \end{center}
\end{figure}


\end{document}