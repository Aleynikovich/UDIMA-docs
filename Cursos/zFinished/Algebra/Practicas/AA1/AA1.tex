\documentclass{article}
\usepackage{lipsum}
\usepackage[backend=biber]{biblatex}
\addbibresource{aec2.bib}
\usepackage{authoraftertitle}
\usepackage[top=2cm,bottom=1.5cm,left=1.5cm, right=3cm,includeheadfoot]{geometry}
\usepackage{graphicx}
\usepackage{fancyhdr}
\usepackage[spanish]{babel}
\usepackage{mathtools}
\usepackage{csquotes}
\usepackage{amssymb}
\usepackage{fancybox, graphicx}
\usepackage{array}
\usepackage{hhline}
\usepackage{hyperref}
\usepackage{tikz}
\usepackage{amsmath}
\usepackage{wrapfig}
\usepackage{float}
\usepackage{amsmath}
\usepackage{caption}
\usepackage{esvect}
\usepackage{siunitx}
\usepackage{commath}
\usepackage{tcolorbox}
\makeatletter
\renewcommand*\env@matrix[1][*\c@MaxMatrixCols c]{%
   \hskip -\arraycolsep
   \let\@ifnextchar\new@ifnextchar
   \array{#1}}
\makeatother

%Header & Footer
\pagestyle{fancy}
\fancyhead[LE]{\MyTitle}
\fancyhead[LO]{Álgebra Lineal}
\fancyhead[RO]{\leftmark}
\fancyhead[RE]{\leftmark}
\fancyfoot[L]{\raisebox{-1cm}{\includegraphics[height=2cm]{C:/Users/XYZ/Dropbox/DocumentGraphics/LOGOUDIMA.jpg}}}
\fancyfoot[R]{Corregido:\\ Dr. Juan José Moreno García}
%\fancyfoot[RO]{07/12/2018}
%Vars
\author{Alexander Sebastian Kalis}
\title{Actividad de Aprendizaje 1}


%DOC

\begin{document}

\begin{titlepage}

    \begin{center}

        \line(1,0){300}\\
        [0.2in]
        \huge{\bfseries {\MyTitle}}\\
        [1mm]
        \line(2,0){200}\\
        [0.75cm]
        \textsc{\LARGE Álgebra Lineal}\\
        [2cm]
        \includegraphics[height=10cm]{C:/Users/XYZ/Dropbox/DocumentGraphics/Calculus.png}\\
        [3cm]

    \end{center}

    \begin{flushright}

        Autor: {\MyAuthor}\\
        Profesor: Dr. Juan José Moreno García\\
        Curso: 1o, Ingeniería de Organización Industrial\\
        UDIMA\\
        Domingo, 28 de Abril de 2019         

    \end{flushright}
    
\end{titlepage}

%\tableofcontents \thispagestyle{empty}

\newpage

\section{Problema 1}

\includegraphics[width=\textwidth]{C:/Users/XYZ/Dropbox/DocumentGraphics/AA1ampliacion/enunciadoprob1.png}

\subsection{Apartado a}

Aplicar Gauss-Jordan usando Octave para obtener la solución del sistema.\\

\textbf{Resolución:}\\

Primeramente asignamos los valores de A y b en Octave mediante las instrucciones:\\

\includegraphics[width=\textwidth]{C:/Users/XYZ/Dropbox/DocumentGraphics/AA1ampliacion/asignacionvalores.png}\\

Seguidamente creamos la matriz ampliada M de A y b:\\

\includegraphics[width=.4\textwidth]{C:/Users/XYZ/Dropbox/DocumentGraphics/AA1ampliacion/asignaiconm.png}\\

Ahora ya podemos aplicar la función rref() sobre M para obtener la solución de ésta.\\

\includegraphics[width=.4\textwidth]{C:/Users/XYZ/Dropbox/DocumentGraphics/AA1ampliacion/rrefm.png}\\

\newpage

\subsection{Apartado b}

Asignamos la matriz y el vector truncados manualmente a 2 posiciones decimales a AT y el vector a bt:\\

\includegraphics[width=\textwidth]{C:/Users/XYZ/Dropbox/DocumentGraphics/AA1ampliacion/asignaciontrunc.png}\\

Siguiendo el mismo proceso, obtenemos la solución siguiente:\\

\includegraphics[width=.5\textwidth]{C:/Users/XYZ/Dropbox/DocumentGraphics/AA1ampliacion/mtrmt.png}\\

Se puede comparar la diferencia de los resultados obtenidos restando las matrices:\\

\includegraphics[width=.5\textwidth]{C:/Users/XYZ/Dropbox/DocumentGraphics/AA1ampliacion/diferencia.png}\\

Supongamos que la primera matriz [A,b] contiene los valores reales y que [AT,bt], que contienen los datos truncados, son los
valores tomados con cierto error de medición.

Podemos extraer la 5a columna de las matrices para operar con ellas más fácilmente:\\

\includegraphics[width=.15\textwidth]{C:/Users/XYZ/Dropbox/DocumentGraphics/AA1ampliacion/d.png}\\

Recordando la fórmula del error relativo:

\[\epsilon_r = \frac{\epsilon_a}{X}\cdot 100\]

Aplicamos la fórmula dónde $\epsilon_a = d$ y $X=rm$:\\

\includegraphics[width=.15\textwidth]{C:/Users/XYZ/Dropbox/DocumentGraphics/AA1ampliacion/ea.png}\\

Nótese el uso del operador punto ./, que indica que queremos realizar la operación fila con fila, y no dividir el vector $d$ por el vector $rm$.\\

Entonces el vector `ea' nos indica, porcentualmente, el error que cometemos en cada fila al hacer la medición del valor utilizando sólo los 2 primeros 
decimales como significativos. 

Podemos observar que los errores generados por el truncamiento son muy elevados. Esto es normal ya que estamos trabajando con números decimales cercanos al 0 con lo que cualquier error decimal, sea en las centésimas o milésimas
provocará un desfase significativo en la solución.
\end{document}
