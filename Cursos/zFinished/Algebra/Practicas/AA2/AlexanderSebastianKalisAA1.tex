\documentclass{article}
\usepackage{lipsum}
\usepackage[backend=biber]{biblatex}
\addbibresource{aec2.bib}
\usepackage{authoraftertitle}
\usepackage[top=2cm,bottom=1.5cm,left=1.5cm, right=3cm,includeheadfoot]{geometry}
\usepackage{graphicx}
\usepackage{fancyhdr}
\usepackage[spanish]{babel}
\usepackage{mathtools}
\usepackage{csquotes}
\usepackage{amssymb}
\usepackage{fancybox, graphicx}
\usepackage{array}
\usepackage{hhline}
\usepackage{hyperref}
\usepackage{tikz}
\usepackage{amsmath}
\usepackage{wrapfig}
\usepackage{float}
\usepackage{amsmath}
\usepackage{caption}
\usepackage{esvect}
\usepackage{siunitx}
\usepackage{commath}
\usepackage{tcolorbox}
\makeatletter
\renewcommand*\env@matrix[1][*\c@MaxMatrixCols c]{%
   \hskip -\arraycolsep
   \let\@ifnextchar\new@ifnextchar
   \array{#1}}
\makeatother

%Header & Footer
\pagestyle{fancy}
%\fancyhead[LE]{\MyTitle}
\fancyhead[LO]{Álgebra Lineal}
\fancyhead[RO]{\leftmark}
%\fancyhead[RE]{\leftmark}
\fancyfoot[L]{\raisebox{-1cm}{\includegraphics[height=2cm]{C:/Users/XYZ/Dropbox/DocumentGraphics/LOGOUDIMA.jpg}}}
\fancyfoot[R]{Corregido:\\ Dr. Juan José Moreno García}
%\fancyfoot[RO]{07/12/2018}
%Vars
\author{Alexander Sebastian Kalis}
\title{Actividad de Aprendizaje 2}


%DOC

\begin{document}

\begin{titlepage}

    \begin{center}

        \line(1,0){300}\\
        [0.2in]
        \huge{\bfseries {\MyTitle}}\\
        [1mm]
        \line(2,0){200}\\
        [0.75cm]
        \textsc{\LARGE Álgebra Lineal}\\
        [2cm]
        \includegraphics[height=10cm]{C:/Users/XYZ/Dropbox/DocumentGraphics/Calculus.png}\\
        [3cm]

    \end{center}

    \begin{flushright}

        Autor: {\MyAuthor}\\
        Profesor: Dr. Juan José Moreno García\\
        Curso: 1o, Ingeniería de Organización Industrial\\
        UDIMA\\
        Domingo, 28 de Abril de 2019         

    \end{flushright}
    
\end{titlepage}

%\tableofcontents \thispagestyle{empty}

\newpage

\section{Problema 1}

Proporcionar las corrientes I1, I2 y I3 que circulan según el esquema de figura aplicando la ley de Ohm
y las reglas de Kirchhoff. Para ello guiarse por la página 83 del manual.

\begin{center}
    \includegraphics{D:/KUKADisk/UDIMA/Algebra/Practicas/AA2/problema1enun.png}\\
\end{center}

Apoyándonos del problema resuelto del manual, que es exactamente el mismo problema con diferentes valores, representamos las ecuaciones lineales y las introducimos en variables en Octave:\\

\includegraphics[height=5cm]{D:/KUKADisk/UDIMA/Algebra/Practicas/AA2/problema1a.png}\\

\newpage

Posteriormente podemos asignar estas ecuaciones a una matriz aumentada:\\

\includegraphics[height=3cm]{D:/KUKADisk/UDIMA/Algebra/Practicas/AA2/problema1b.png}\\

Finalmente hacemos una reducción y obtenemos los resultados para $I_1$, $I_2$, $I_3$:\\

\includegraphics[height=3cm]{D:/KUKADisk/UDIMA/Algebra/Practicas/AA2/problema1c.png}\\

\section{Problema 2}

Usar el método de valores singulares descrito en el apédice para calcular la solución del sistema:\\

\begin{center}
    \includegraphics[height=2cm]{D:/KUKADisk/UDIMA/Algebra/Practicas/AA2/problema2enun.png}\\    
\end{center}

Siguiendo el procedimiento del apéndice:\\

Introducimos la matriz A y b en Octave:

\includegraphics[height=4cm]{D:/KUKADisk/UDIMA/Algebra/Practicas/AA2/problema2a.png}\\ 

\newpage

Realizamos la descomposición en valores singulares de la matriz A:\\


\includegraphics[height=5cm]{D:/KUKADisk/UDIMA/Algebra/Practicas/AA2/problema2b.png}\\ 


Observamos el valor singular $0.014$ y consideramos $\beta=0.1$ entonces el rango de $A$
 sería $q=2$ y podemos truncar la matriz y guardarla en la variable Dt:\\

\includegraphics[height=3cm]{D:/KUKADisk/UDIMA/Algebra/Practicas/AA2/problema2c.png}\\

\includegraphics[height=3cm]{D:/KUKADisk/UDIMA/Algebra/Practicas/AA2/problema2d.png}\\

Calculamos la pseudoinversa invirtenido los valores de la diagonal de Dt:\\

\includegraphics[height=3cm]{D:/KUKADisk/UDIMA/Algebra/Practicas/AA2/problema2e.png}\\

\newpage
Finalmente calculamos la matriz pseudo-inversa truncada:\\

\includegraphics[height=2.5cm]{D:/KUKADisk/UDIMA/Algebra/Practicas/AA2/problema2f.png}\\

Y resolvemos el sistema $Ax=b$ utilizando la descomposición en valores singulares:\\

\includegraphics[height=2.5cm]{D:/KUKADisk/UDIMA/Algebra/Practicas/AA2/problema2g.png}\\

\newpage

\section{Problema 3}

\includegraphics[width=\textwidth]{D:/KUKADisk/UDIMA/Algebra/Practicas/AA2/problema3enun.png}\\

Considerando cada habitante y sus porcentajes de consumición, creamos una matriz aumentada con éstos datos
conocidos:\\

\includegraphics[height=3.5cm]{D:/KUKADisk/UDIMA/Algebra/Practicas/AA2/problema3a.png}\\

Seguidamente aplicamos $\lambda=1$:\\

\includegraphics[height=3.5cm]{D:/KUKADisk/UDIMA/Algebra/Practicas/AA2/problema3b.png}\\

\newpage

Creamos un vector 0 $b$ y lo añadimos a la matriz de aldeab. Reducimos a forma escalonada:

\includegraphics[height=7cm]{D:/KUKADisk/UDIMA/Algebra/Practicas/AA2/problema3c.png}\\

Con lo que podemos observar que el producto más caro será el aguardiente que denotaremos con $t$ y este producto se utiliza como 
referencia para calcular los precios de los demás productos:

\[
  \begin{cases}
      x (comida) = 0.67932t\\
      y (ropa) = 0.62561t\\
      z (vivienda) = 0.68785t\\
      u (energia) = 0.73754t\\
      w (aguardiente) = t  
  \end{cases}  
\]

Con lo cual concluimos que el tercer producto más caro es la vivienda.
\end{document}
