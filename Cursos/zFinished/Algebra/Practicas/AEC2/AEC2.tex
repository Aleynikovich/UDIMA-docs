\documentclass{article}
\usepackage{lipsum}
\usepackage[backend=biber]{biblatex}
\addbibresource{aec2.bib}
\usepackage{authoraftertitle}
\usepackage[top=2cm,bottom=1.5cm,left=1.5cm, right=3cm,includeheadfoot]{geometry}
\usepackage{graphicx}
\usepackage{fancyhdr}
\usepackage[spanish]{babel}
\usepackage{mathtools}
\usepackage{csquotes}
\usepackage{amssymb}
\usepackage{fancybox, graphicx}
\usepackage{array}
\usepackage{enumitem}
\usepackage{hhline}
\usepackage{hyperref}
\usepackage{tikz}
\usepackage{amsmath}
\usepackage{wrapfig}
\usepackage{float}
\usepackage{amsmath}
\usepackage{caption}
\usepackage{esvect}
\usepackage{siunitx}
\usepackage{commath}
\usepackage{tcolorbox}
\usepackage{multirow}
\makeatletter
\renewcommand*\env@matrix[1][*\c@MaxMatrixCols c]{%
  \hskip -\arraycolsep
  \let\@ifnextchar\new@ifnextchar
  \array{#1}}
\makeatother

%Header & Footer
\pagestyle{fancy}
\fancyhead[LE]{\MyTitle}
\fancyhead[LO]{Álgebra Lineal}
\fancyhead[RO]{\leftmark}
\fancyhead[RE]{\leftmark}
\fancyfoot[L]{\raisebox{-1cm}{\includegraphics[height=2cm]{C:/Users/XYZ/Dropbox/DocumentGraphics/LOGOUDIMA.jpg}}}
\fancyfoot[R]{Corregido:\\ Dr. Juan José Moreno García}
%\fancyfoot[RO]{07/12/2018}
%Vars
\author{Alexander Sebastian Kalis}
\title{Actividad de Evaluación Continua 2}


%DOC

\begin{document}

\begin{titlepage}

    \begin{center}

        \line(1,0){300}\\
        [0.2in]
        \huge{\bfseries {\MyTitle}}\\
        [1mm]
        \line(2,0){200}\\
        [0.75cm]
        \textsc{\LARGE Álgebra Lineal}\\
        [2cm]
        \includegraphics[height=10cm]{C:/Users/XYZ/Dropbox/DocumentGraphics/Calculus.png}\\
        [3cm]

    \end{center}

    \begin{flushright}

        Autor: {\MyAuthor}\\
        Profesor: Dr. Juan José Moreno García\\
        Curso: 1o, Ingeniería de Organización Industrial\\
        UDIMA         

    \end{flushright}
    
\end{titlepage}

\newpage

\section*{Actividades}

\subsection*{Problema 1}

¿Son linealmente independientes los vectores del siguiente conjunto?\\
$C={2x^3-4x^2+2x,2x^3-3x^2-2x+7,x^3-3x^2-2x+2,-2x^2+x-5}$\\\\

Podemos representar el conjunto C de forma matricial:\\

\[
C=
\begin{bmatrix*}[r]
    2  & 2  & 1  & 0\\
    -4 & -3 & -3 & -2\\
    2  & -2 & -2 &  1\\
    0  & 7  & 2  & -5
\end{bmatrix*}
\]

Para que los vectores de C sean linealmente independientes, entre otros, se debe cumplir que $Null(C)$ tenga
una única solución, el vector $\textbf{0}$.

Sabiendo que $Null(C)=Null(rref(C))$ primero obtenemos $rref(C)$ ayudándonos de Octave:

\[
    rref(C)=
    \begin{bmatrix*}[r]
        1 & 0 & 0 & .5\\
        0 & 0 & 0 & -1\\
        0 & 0 & 1 & 1 \\
        0 & 0 & 0 & 0 
    \end{bmatrix*}
\]

Viendo esto inmediatamente podemos deducir que los vectores del conjunto $C$ \textbf{no son linealmente independientes} ya que $c_2$
se puede producir mediante la combinación lineal de las otras columnas escalándolas con coeficientes diferentes a $0$.\\\\


\subsection*{Problema 2}

Encuentra una base del espacio columna y una para el espacio fila de A. Si esta matriz corresponde a una aplicación $Tx=Ax$ hallar
una base para la imagen y otra para el núcleo. ¿Es la transformación inyectiva, sobreyectiva o biyectiva?


\[
    A=
    \begin{bmatrix*}[r]
        1 & 1 & 3 & 1 & 6\\
        -3 & 2 & 1 & -2 & 1\\
        4 & -2 & 0 & 2 & -2\\
        2 & 1 & 4 & 1 & 5
    \end{bmatrix*}
\]

\[
    rref(A)=
    \begin{bmatrix*}[r]
        \textbf{1} & \textbf{0} & 1 & \textbf{0} & -1\\
        \textbf{0} & \textbf{1} & 2 & \textbf{0} & 3\\
        \textbf{0} & \textbf{0} & 0 & \textbf{1} & 4\\
        \textbf{0} & \textbf{0} & 0  & \textbf{0} & 0
    \end{bmatrix*}
\]\\

Tras haber encontrado la forma escalonada reducida, simplemente seleccionamos las columnas pivote para obtener la base del espacio columna y
las filas no nulas para la base del espacio fila:

\[
    \mathcal{B}_c(A) =
    \left\{
        \begin{bmatrix*}[r]
            1\\
            -3\\
            4\\
            2
        \end{bmatrix*}
        ,
        \begin{bmatrix*}[r]
            1\\
            2\\
            -2\\
            1
        \end{bmatrix*}
        ,
        \begin{bmatrix*}[r]
            1\\
            -2\\
            2\\
            1
        \end{bmatrix*}
    \right\}  
\]

\[
    \mathcal{B}_f(A) =
    \left\{
        \begin{bmatrix*}[r]
            1 & 0 & 1 & 0 & -1\\
        \end{bmatrix*}
        ,
        \begin{bmatrix*}[r]
            0 & 1 & 2 & 0 & 3\\
        \end{bmatrix*}
        ,
        \begin{bmatrix*}[r]
            0 & 0 & 0 & 1 & 4\\
        \end{bmatrix*}
    \right\}  
\]\\

Para determinar el núcleo o espacio nulo de $A$, debemos resolver el sistema $Ax=0$:

\[
    \begin{bmatrix*}[r]
        1 & 0 & 1 & 0 & -1\\
        0 & 1 & 2 & 0 & 3\\
        0 & 0 & 0 & 1 & 4\\
        0 & 0 & 0 & 0 & 0
    \end{bmatrix*}
    \begin{bmatrix*}[r]
        x_1 \\
        x_2\\
        x_3\\
        x_4\\
        x_5
    \end{bmatrix*}
    =
    \begin{bmatrix}
        0\\
        0\\
        0\\
        0
    \end{bmatrix}
\]

Representado en forma de matriz aumentada:

\begin{equation}
    \begin{bmatrix}[ccccc|c]
        1 & 0 & 1 & 0 & -1 & 0\\
        0 & 1 & 2 & 0 & 3 & 0\\
        0 & 0 & 0 & 1 & 4 & 0\\
        0 & 0 & 0 &0 & 0 & 0
    \end{bmatrix}
\end{equation}

Obtenemos el conjunto de soluciones:

\[
    \begin{bmatrix*}[r]
        x_1 \\
        x_2\\
        x_3\\
        x_4\\
        x_5
    \end{bmatrix*}
    =
    x_3
    \begin{bmatrix*}[r]
        -1\\
        -2\\
        1\\
        0\\
        0
    \end{bmatrix*}
    +
    x_5    
    \begin{bmatrix*}[r]
        1\\
        -3\\
        0\\
        -4\\
        1
    \end{bmatrix*}
\]

Entonces:

\[
    \mathcal{B}(Null(A))=
    \left\{
        \begin{bmatrix*}[r]
            -1\\
            -2\\
            1\\
            0\\
            0
        \end{bmatrix*}
        ,
        \begin{bmatrix*}[r]
            1\\
            -3\\
            0\\
            -4\\
            1
        \end{bmatrix*}
    \right\}  
\]

Y recuperando del apartado anterior:


\[
    \mathcal{B}_c(A) =
    \mathcal{B}_c(T) =
    \left\{
        \begin{bmatrix*}[r]
            1\\
            -3\\
            4\\
            2
        \end{bmatrix*}
        ,
        \begin{bmatrix*}[r]
            1\\
            2\\
            -2\\
            1
        \end{bmatrix*}
        ,
        \begin{bmatrix*}[r]
            1\\
            -2\\
            2\\
            1
        \end{bmatrix*}
    \right\}  
\]

Para determinar qué tipo de aplicación es, recuperamos el conjunto solución: 

\[
    \begin{bmatrix*}[r]
        x_1 \\
        x_2\\
        x_3\\
        x_4\\
        x_5
    \end{bmatrix*}
    =
    x_3
    \begin{bmatrix*}[r]
        -1\\
        -2\\
        1\\
        0\\
        0
    \end{bmatrix*}
    +
    x_5    
    \begin{bmatrix*}[r]
        1\\
        -3\\
        0\\
        -4\\
        1
    \end{bmatrix*}
\]

Sabemos que no será inyectiva ni biyectiva ya que el espacio nulo son todas las combinaciones lineales para cualquier $\left\{x_3, x_5 \in \mathbb{R} \right\}$.

Por otro lado la aplicación tampoco será sobreyectiva ya que la imagen está en $\mathbb{R}^3$ mientras que el codominio está en $\mathbb{R}^4$.

\newpage



\subsection*{Problema 3}

Sea en $\mathbb{R}$ las bases siguientes: 

\[
    B=
    \left\{
        \begin{pmatrix}
            1\\
            0\\
            0
        \end{pmatrix},
        \begin{pmatrix}
            0\\
            1\\
            0
        \end{pmatrix},
        \begin{pmatrix}
            0\\
            0\\
            1
        \end{pmatrix}   
    \right\}
    \]
    \[
    C=
    \left\{
        \begin{pmatrix*}
            0\\
            2\\
            1
        \end{pmatrix*},
        \begin{pmatrix*}
            1\\
            1\\
            -2
        \end{pmatrix*},
        \begin{pmatrix*}
            1\\
            4\\
            -1
        \end{pmatrix*}   
    \right\}  
\]

Sea además en dicho espacio el vector $v=(1,2,-3)^T$. Calcular lo siguiente:\\
-La matriz de cambio de base que pasa de B a C.\\

Podemos encontrar la matriz de cambio de base tomando la inversa de $C$ mediante Gauss-Jordan:\\

\[
    rref
    \left(
        \begin{bmatrix}[ccc|ccc]
            0 & 1 & 1 & 1 & 0 & 0\\
            2 & 1 & 4 & 0 & 1 & 0\\
            1 & -2 & -1 & 0 & 0 & 1 \\
       \end{bmatrix} 
    \right)
    =
    \begin{bmatrix}[ccc|ccc]
        1 & 0 & 0 & 7 & -1 & 3\\
        0 & 1 & 0 & 6 & -1 & 2\\
        0 & 0 & 1 & -5 & 1 & -2 \\
   \end{bmatrix}
\]

Entonces la matriz utilizada para hacer el cambio de base es:

\[
    M=
    \begin{bmatrix*}[r]
        7 & -1 & 3\\
        6 & -1 & 2\\
        -5 & 1 & -2 \\
    \end{bmatrix*}
\]\\


-El vector de coordenadas de \textbf{v} en la base $B$. El vector de coordenadas de \textbf{v} en C usando la matriz de
cambios de base y \textbf{comprobar los resultados}.\\

El vector $\textbf{v}$ en base $B$ será el mismo, ya que $B$ es la base canónica en $\mathbb{R}^3$.
Para encontrar las coordenadas de $\textbf{v}$ tras aplicarle el cambio de base en $C$, multiplicamos $M\textbf{v}$:

\[
    M\textbf{v}=
    \begin{bmatrix*}[r]
        7 & -1 & 3\\
        6 & -1 & 2\\
        -5 & 1 & -2 \\
    \end{bmatrix*}
    \begin{bmatrix*}[r]
        1\\
        2\\
        -3
    \end{bmatrix*}
    =
    \begin{bmatrix*}[r]
        -4\\
        -2\\
        3
    \end{bmatrix*}
\]\\

Para comoprobar los resultados, multiplicamos cada vector por su base y el resultado final debería ser el mismo:

\[
    \textbf{v}=
     \begin{bmatrix}
        1 & 0 & 0\\
        0 & 1 & 0\\
        0 & 0 & 1
     \end{bmatrix}
     \begin{bmatrix*}[r]
        1\\
        2\\
        -3
    \end{bmatrix*}
    =
    \begin{bmatrix*}[r]
        0 & 1 & 1\\
        2 & 1 & 4\\
        1 & -2 & -1
    \end{bmatrix*}
    \begin{bmatrix*}[r]
        -4\\
        -2\\
        3
    \end{bmatrix*}
    =
    \begin{bmatrix*}[r]
        1\\
        2\\
        -3
    \end{bmatrix*}
    =
    \textbf{v}
\]

\newpage















\subsection*{Problema 4}

Sea $f:\mathbb{R}^4 \rightarrow \mathbb{R}^4$ una transformación definida por

\[
    f
    \begin{pmatrix}
        a\\
        b\\
        c\\
        d
    \end{pmatrix}
    =
    \begin{bmatrix}
        a+b+3c-4d\\
        6a+4b+8c-10d\\
        -8a-6b-14c+18d\\
        -7a-5b-11c+14d
    \end{bmatrix}
\]

Calcular su matriz asociada respecto a la base estandar o canónica, una base de su núcleo y una base de su imagen.\\

La matriz asociada respecto a $I_4$ es la propia transformación que aplica $f$ sobre los vectores:

\[
    B=
    \begin{bmatrix*}[r]
        1  & 1  &  3  & -4\\
        6  & 4  &  8  & -10\\
        -8 & -6 & -14 & 18\\
        -7 & -5 & -11 & 14
    \end{bmatrix*}  
\]

Para obtener la base de su núcleo resolvemos $Bx=0$:

\[
    Bx=
    \begin{bmatrix*}[r]
        1  & 1  &  3  & -4\\
        6  & 4  &  8  & -10\\
        -8 & -6 & -14 & 18\\
        -7 & -5 & -11 & 14
    \end{bmatrix*} 
    \begin{bmatrix}
        x_1\\
        x_2\\
        x_3\\
        x_4
    \end{bmatrix}
    =
    \begin{bmatrix}
        0\\
        0\\
        0\\
        0
    \end{bmatrix} 
\]\\

\[
    rref
    \left(
        \begin{bmatrix}[cccc|c]
            1  & 1  &  3  & -4 & 0\\
            6  & 4  &  8  & -10 & 0\\
            -8 & -6 & -14 & 18 & 0\\
            -7 & -5 & -11 & 14 & 0
        \end{bmatrix}  
    \right)
    =
    \begin{bmatrix}[cccc|c]
        1  & 0  & -2  & 3  & 0      \\
        0  & 1  & 5   & -7 & 0      \\
        0  & 0  & 0   & 0  & 0      \\
        0  & 0  & 0   & 0  & 0
    \end{bmatrix}  
\]\\

Obtenemos entonces como solución:


\[
    \begin{bmatrix}
        x_1\\
        x_2\\
        x_3\\
        x_4
    \end{bmatrix}
    =
    x_3
    \begin{bmatrix}
        2\\
        -5\\
        1\\
        0        
    \end{bmatrix}
    +
    x_4
    \begin{bmatrix}
        -3\\
        7\\
        0\\
        1        
    \end{bmatrix}
\]

Siendo $x_3$ y $x_4$ variables libres. Y por lo tanto:


\[
    \mathcal{B}(Null(B))=
    \left\{
        \begin{bmatrix}
            2\\
            -5\\
            1\\
            0        
        \end{bmatrix}
        ,
        \begin{bmatrix}
            -3\\
            7\\
            0\\
            1        
        \end{bmatrix}
    \right\}  
\]

Y la base de la imagen serán las columnas pivote de $B$, dicho de otra forma, $C(B)$:

\[
    \mathcal{C}(B)=
    \left\{
        \begin{bmatrix}
            1\\
            6\\
            -8\\
            -7       
        \end{bmatrix}
        ,
        \begin{bmatrix}
            1\\
            4\\
            -6\\
            -5       
        \end{bmatrix}
    \right\}  
\]












\newpage




\subsection*{Problema 5}

Demostrar si el siguiente conjunto de vectores en $\mathcal{P}_2(\mathbb{R})$ es linealmente independiente: 

\[
    p_1(x)=1+3x+2x^2,p_2(x)=2+4x,p_3(x)=3+6x-x^2  
\]

Si es así formar la base de $\mathcal{P}_2(\mathbb{R}) siguiente:$

\[
    C=\left\{p_1,p_2,p_3\right\}    
\]

Siendo la base canónica de $\mathcal{P}_2(\mathbb(R))$:

Sea en dicho espacio el vector $p(x)=2+6x+8x^2$. Calcular el vector de coordenadas de p en la base $C$ usando una matriz de cambio base y comprobar el resultado.\\\\


Para comprobar si el conjunto es linealmente independiente resolvemos:
\[
    rref
    \left(
        \begin{bmatrix}[ccc|c]
            1 & 2 & 3 &0\\
            3 & 4 & 6 & 0\\
            2 & 0 & -1 & 0
        \end{bmatrix}
    \right)
    =
    \begin{bmatrix}[ccc|c]
        1 & 0 & 0 & 0\\
        0 & 1 & 0 & 0\\
        0 & 0 & 1 & 0
    \end{bmatrix}
\]

Ya que la única solución a $Ax=0$ es $\textbf{x}=0$, sabemos que el conjunto $\mathcal{P}_2$ es linealmente independiente. \\\\




Ya que los 3 vectores son linealmente independientes o columnas pivote, forman la base de $\mathcal{P}_2(\mathbb{R})$:

\[
    C=
    \left\{
        \begin{bmatrix*}[r]
            1\\
            3\\
            2
        \end{bmatrix*} 
        ,
        \begin{bmatrix*}[r]
            2 \\
             4 \\
             0
        \end{bmatrix*} 
        ,\begin{bmatrix*}[r]
            3\\
             6 \\
            -1
        \end{bmatrix*} 
    \right\}
\]

Para hacer la transformación primero necesitamos encontrar la matriz de cambio $M$, que lo haremos como en los problemas anteriores
utilizando Gauss-Jordan:

\[
    rref
    \left(
        \begin{bmatrix}[ccc|ccc]
            1 & 2 & 3 & 1 & 0 & 0\\
            3 & 4 & 6 & 0 & 1 & 0 \\
            2 & 0 & -1 & 0 & 0 & 1
        \end{bmatrix}
    \right)
    =
    \begin{bmatrix}[ccc|ccc]
         1 & 0 & 0 & -2 & 1 & 0\\
         0 & 1 & 0 & 7.5 & -3.5 & 1.5\\
         0 & 0 & 1 & -4 & 2 & -1
    \end{bmatrix}
\]

\[
    M=
    \begin{bmatrix}
        -2 & 1 & 0\\
        7.5 & -3.5 & 1.5\\
        -4 & 2 & -1
    \end{bmatrix}  
\]
Aplicamos la transformación a $p(x)$:

\[
    Mp(x)=
      \begin{bmatrix*}[r]
        -2 & 1 & 0\\
        7.5 & -3.5 & 1.5\\
        -4 & 2 & -1
    \end{bmatrix*}
    \begin{bmatrix}
        2\\
        6\\
        8
    \end{bmatrix}  
    =
    \begin{bmatrix*}[r]
        2\\
        6\\
        -4
    \end{bmatrix*}
\]

Finalmente comprobamos el resultado:

\[
    \begin{bmatrix*}[r]
        1 & 2 & 3\\ 
        3 & 4 & 6\\
        2 & 0 & -1 
    \end{bmatrix*}
    \begin{bmatrix*}[r]
        2\\
        6\\
        -4
    \end{bmatrix*}
    =
    \begin{bmatrix}
        2\\
        6\\
        8
    \end{bmatrix}  
    =p(x)
\]



\newpage


\subsection*{Problema 6}

Estudia si la matriz

\[
    A=
    \begin{bmatrix}
        4 & -3 & 1\\
        5 & -4 & 1\\
        4 & -3 & 1
    \end{bmatrix}
\]  

es o no diagonalizable. En caso afirmativo:

\begin{enumerate}[label=(\alph*)]
    \item  Escribe explícitamente una matriz diagonal $D$ y dos matrices $P$ y $P^{-1}$ tales que $D=P^{-1}AP$. Ordenar las entradas de $D$ de menor a mayor según se lee de izquierda a derecha.
    \item  Calcular no explícitamente y sin usar fuerza bruta $A^{5050}$.\\
\end{enumerate}

Para determinar si $A$ es diagonalizable, encontramos sus autovalores y auto vectores:

\[
   \left(\lambda I_3-A\right)\textbf{v}=\textbf{0}
   \iff 
   det \left(\lambda I_3-A\right)=0
\]

\[
    det
    \left(
        \begin{bmatrix}
            4 & -3 & 1\\
            5 & -4 & 1\\
            4 & -3 & 1
        \end{bmatrix}      
        -
        \begin{bmatrix}
            \lambda & 0 & 0\\
            0 & \lambda & 0\\
            0 & 0 & \lambda
        \end{bmatrix}
    \right)
    =
    0
    \rightarrow
    det
    \left(
    \begin{bmatrix}
        4 - \lambda   & -3 & 1\\
        5 &  - 4 -\lambda & 1\\
        4 & -3 &   1 -\lambda
    \end{bmatrix}
    \right)
    =
    0
    \rightarrow
    -\lambda^3+\lambda^2+2\lambda=0
\]

Resolviendo $-\lambda^3+\lambda^2+2\lambda=0$ obtenemos $\lambda_1=0$,$\lambda_2=2$,$\lambda_3=-1$ lo que significa
que la matriz \textbf{es diagnoalizable} pues obtenemos 3 autovalores distintos.

Podemos representar la matriz diagonal $D=P^{-1}AP$:

\[
    D=
    \begin{bmatrix}
        \textbf{-1} & 0 & 0\\
        0 & \textbf{0} & 0\\
        0 & 0 & \textbf{2}
    \end{bmatrix}  
\]

Para encontrar las columnas de $P$ necesitamos primero encontrar los auto vectores para cada valor en:

\[
    \begin{bmatrix}
        4 - \lambda   & -3 & 1\\
        5 &  - 4 -\lambda & 1\\
        4 & -3 &   1 -\lambda
    \end{bmatrix}
    \begin{bmatrix}
        v_1\\
        v_2\\
        v_3
    \end{bmatrix}
    =
    \begin{bmatrix}
        0\\
        0\\
        0
    \end{bmatrix}
\]

Para $\lambda_1=-1$:


\[
    rref
    \left(
        \begin{bmatrix}[ccc|c]
            5 & -3 & 1 & 0\\
            5 & -3 & 1 & 0\\
            4 & -3 & 2 & 0
        \end{bmatrix}
    \right)
    =
    \begin{bmatrix}[ccc|c]
        1 & 0 & -1 & 0\\
        0 & 1 & -2 & 0\\
        0 & 0 & 0 & 0
    \end{bmatrix}
    \rightarrow
    \begin{bmatrix}
        v_1\\
        v_2\\
        v_3
    \end{bmatrix}
    =
    v_3
    \begin{bmatrix}
       1\\
       2\\
       1 
    \end{bmatrix}
\]

Para $\lambda_2=0$:

\[
    rref
    \left(
        \begin{bmatrix}[ccc|c]
            4 & -3 & 1 & 0\\
            5 & -4 & 1 & 0\\
            4 & -3 & 1 & 0
        \end{bmatrix}
    \right)
    =
    \begin{bmatrix}[ccc|c]
        1 & 0 & 1 & 0\\
        0 & 1 & 1 & 0\\
        0 & 0 & 0 & 0
    \end{bmatrix}
    \rightarrow
    \begin{bmatrix}
        v_1\\
        v_2\\
        v_3
    \end{bmatrix}
    =
    v_3
    \begin{bmatrix}
       -1\\
       -1\\
       1 
    \end{bmatrix}
\]


Para $\lambda_3=2$:

\[
    rref
    \left(
        \begin{bmatrix}[ccc|c]
            2 & -3 & 1 & 0\\
            5 & -6 & 1 & 0\\
            4 & -3 & -1 & 0
        \end{bmatrix}
    \right)
    =
    \begin{bmatrix}[ccc|c]
        1 & 0 & -1 & 0\\
        0 & 1 & -1 & 0\\
        0 & 0 & 0 & 0
    \end{bmatrix}
    \rightarrow
    \begin{bmatrix}
        v_1\\
        v_2\\
        v_3
    \end{bmatrix}
    =
    v_3
    \begin{bmatrix}
       1\\
       1\\
       1 
    \end{bmatrix}
\]

\newpage

Dejando la variable libre $v_3=1$, podemos construir la matriz $P$:

\[
    P
    =
    \begin{bmatrix}
       1 & -1 & 1\\
       2 & -1 & 1\\
       1 & 1 & 1 
    \end{bmatrix}  
\]

y su matriz inversa:

\[
    P^{-1}
    =
    \begin{bmatrix}
        -1 & 1 & 0\\
        -.5 & 0 & .5\\
        1.5 & -1 & .5
    \end{bmatrix}  
\]

Sabiendo que $A=PDP^{-1}$, podemos calcular $(PDP^{-1})^{5050}$, que resultará mucho más sencillo a nivel computacional pues $P$ y $P^{-1}$ se irán
cancelando a medida que evolucione la série y nos quedamos con:

\[
    A=PD^{5050}P^{-1}
    =
    \begin{bmatrix}
        1 & -1 & 1\\
        2 & -1 & 1\\
        1 & 1 & 1 
     \end{bmatrix} 
     \begin{bmatrix}
        (-1)^{5050} & 0 & 0\\
        0 & 0 & 0\\
        0 & 0 & 2^{5050}
     \end{bmatrix}
     \begin{bmatrix}
        -1 & 1 & 0\\
        -.5 & 0 & .5\\
        1.5 & -1 & .5
    \end{bmatrix} 
\]\\\\









\subsection*{Problema 7}

Dada la matriz:

\[
    A=
    \begin{bmatrix}
        6 & -1 & 3\\
        -10 & 3 & -6\\
        -10 & 2 & -5
    \end{bmatrix}
\]

Calcula los autovalores de A, sus autovectores correspondintes y argumentar por qué la matriz es o no es diagonalziable. En caso positivo hallar $D,P,P^{-1}$ y la diagonalización. Ordenar los autovalores
en $D$ de menor a mayor en sentido de lectura.

\[
    det
    \left(
        A-\lambda I_3
    \right)
    =0
    \rightarrow
    det 
    \left(
        \begin{bmatrix}
            6 - \lambda & -1 & 3\\
            -10 & 3 - \lambda & -6\\
            -10 & 2 & -5- \lambda
        \end{bmatrix}
    \right)
    =
    0
    \rightarrow
    -(\lambda-2)(\lambda-1)^2=0
\]

Lo que nos otorga los valores $\lambda_1=1$ con multiplicidad 2 y $\lambda_2=2$ con multiplicidad 1 con los que procederemos a calcular los auto vectores:



Para $\lambda_1=1$:

\[
    rref
    \left(
        \begin{bmatrix}[ccc|c]
            5 & -1  & 3 & 0\\
            -10 & 2 & -6 & 0\\
            -10 & 2 & -6 & 0
        \end{bmatrix}
    \right)
    =
    \begin{bmatrix}[ccc|c]
        5 & -1 & 3 & 0\\
        0 & 0 & 0 & 0\\
        0 & 0 & 0 & 0
    \end{bmatrix}
    \rightarrow
    \begin{bmatrix}
        v_1\\
        v_2\\
        v_3
    \end{bmatrix}
    =
    v_2
    \begin{bmatrix}
       1\\
       5\\
       0 
    \end{bmatrix}
    +
    v_3
    \begin{bmatrix}
       -3\\
       0\\
       5 
    \end{bmatrix}
\]

Para $\lambda_2=2$:

\[
    rref
    \left(
        \begin{bmatrix}[ccc|c]
            4 & -1  & 3 & 0\\
            -10 & 1 & -6 & 0\\
            -10 & 2 & -7 & 0
        \end{bmatrix}
    \right)
    =
    \begin{bmatrix}[ccc|c]
        2 & 0 & 1 & 0\\
        0 & 2 & -2 & 0\\
        0 & 0 & 0 & 0
    \end{bmatrix}
    \rightarrow
    \begin{bmatrix}
        v_1\\
        v_2\\
        v_3
    \end{bmatrix}
    =
    v_3
    \begin{bmatrix}
       -1\\
       2\\
       2 
    \end{bmatrix}
\]

Ya que las dimensiones de ambos espacios coinciden con la multiplicidad de sus autovalores, sabemos que la matriz
es diagonizable:

\[  
    P=
    \begin{bmatrix}
        1 & -3 & -1\\
        5 & 0 & 2\\
        0 & 5 & 2
    \end{bmatrix}
    \rightarrow
    P^{-1}=
    \begin{bmatrix}
        10 & 1 & 6\\
        10 & -2 & 7\\
        -10 & 2 & -6
    \end{bmatrix}
    \rightarrow
    D=P^{-1}AP=
    \begin{bmatrix}
    1 & 0 & 0\\
    0 & 1 & 0\\
    0 & 2 & 2   
    \end{bmatrix}
\]







\subsection*{Problema 8}

Resuélvase el siguiente sistema incompatible por mínimos cuadrados:

\[
    \begin{matrix}
        x+3y & = & 1\\
        3x+4y & = & 0\\
        x+2y & = & -8
    \end{matrix}
\]

Ya que no hay ningún punto que cumple el sistema de ecuaciones anterior, podemos buscar el punto más cercano a todas
las rectas trazadas por el sistema mediante el método de mínimos cuadrados.

\[
    A^TAx=A^Tb
    \rightarrow
    \begin{bmatrix}
        1 & 3 & 1\\
        3 & 4 & 2
    \end{bmatrix}
    \begin{bmatrix}
        1 & 3\\
        3 & 4\\
        1 & 2
    \end{bmatrix}
    \begin{bmatrix}
        x\\
        y
    \end{bmatrix}
    =
    \begin{bmatrix}
        1 & 3 & 1\\
        3 & 4 & 2
    \end{bmatrix}
    \begin{bmatrix}
        1\\
        0\\
        -8
    \end{bmatrix}
    \rightarrow
    \begin{bmatrix}
        11 & 17\\
        17 & 29
    \end{bmatrix}
    \begin{bmatrix}
        x\\
        y
    \end{bmatrix}
    =
    \begin{bmatrix}
        -7\\
        -13
    \end{bmatrix}
\]  

Que tiene forma $Ax=b$ y se puede resolver mediante Gauss-Jordan:
\[
    rref
    \left(
        \begin{bmatrix}[cc|c]
            11 & 17 & -7\\
            17 & 29 & -13
        \end{bmatrix} 
    \right)
   =
   \begin{bmatrix}
       1 & 0 & .6\\
       0 & 1 & -.8
   \end{bmatrix}
\]

Por lo tanto la solución más cercana, mediante el método de mínimos cuadrados es $x=0.6$ e $y=-0.8$:\\

\includegraphics[height=10cm]{C:/Users/XYZ/Dropbox/DocumentGraphics/AEC2Algebra/problema8.png}\\

\newpage

\subsection*{Problema 9}

Calcula la descomposición QR de la siguiente matriz:

\[
    A=
    \begin{bmatrix}
        0 & 1 & 1\\
        1 & -1 & 1\\
        1 & 1 & 0
    \end{bmatrix}
\]

Nota importante: conservar los números en forma de raíces cuadradas y sus combinaciones y no explicitar
sus valores en forma decimal. Si se da en forma decimal se calificará con un cero.\\

Procedemos a descomponer $A$ en $Q$, una matriz ortonormal y $R$, una matriz triangular superior:

\[
    A=
     \begin{bmatrix}
        0 & 1 & 1\\
        1 & -1 & 1\\
        1 & 1 & 0
    \end{bmatrix}=
    QR  
\]

Para encontrar las columnas de Q, podemos utilizar la ortogonalización Gram-Schmidt:

\[
    u_1=
    \begin{bmatrix}
        0\\
        1\\
        1
    \end{bmatrix}  
    \rightarrow
    e_1=
    \frac{1}{\| \mathbf{v_1} \|} v_1=
    \frac{1}{\sqrt{2}}
    \begin{bmatrix}
        0\\
        1\\
        1
    \end{bmatrix} 
    =
    \begin{bmatrix}
        0\\
        \frac{1}{\sqrt{2}}\\
        \frac{1}{\sqrt{2}}
    \end{bmatrix}
\]

\[
    u_2=
    \begin{bmatrix}
        1\\
        -1\\
        1
    \end{bmatrix}
    -0e_1
    \rightarrow
    e_2
    \frac{1}{\| \mathbf{v_2} \|} v_2=
    \frac{1}{\sqrt{3}}
    \begin{bmatrix}
        1\\
        -1\\
        1
    \end{bmatrix}  
    =
    \begin{bmatrix}
        \frac{1}{\sqrt{3}}\\
        -\frac{1}{\sqrt{3}}\\
        \frac{1}{\sqrt{3}}
    \end{bmatrix}
\]

\[
    u_3=
    \begin{bmatrix}
        1\\
        1\\
        0
    \end{bmatrix} 
    -\frac{1}{\sqrt{2}}e_1
    -0e_2
    \rightarrow
    e_3=
    \frac{1}{\| \mathbf{v_3} \|} v_3=
    \frac{\sqrt{6}}{2}
    \begin{bmatrix}
        1\\
        \frac{1}{2}\\
        -\frac{1}{2}
    \end{bmatrix}  
    =
    \begin{bmatrix}
        \frac{2}{\sqrt{6}}\\
        \frac{1}{\sqrt{6}}\\
        -\frac{1}{\sqrt{6}}
    \end{bmatrix}
\]


Recopilamos todos los vectores en la matriz $Q$:

\[
    Q=
    \begin{bmatrix}
        0 & \frac{1}{\sqrt{3}}& \frac{2}{\sqrt{6}}\\
        \frac{1}{\sqrt{2}} & -\frac{1}{\sqrt{3}}& \frac{1}{\sqrt{6}}\\
        \frac{1}{\sqrt{2}} & \frac{1}{\sqrt{3}} & -\frac{1}{\sqrt{6}}
    \end{bmatrix}  
\]

Y finalmente:

\[
    R=Q^TA=
    \begin{bmatrix}
        0 & \frac{1}{\sqrt{2}} & \frac{1}{\sqrt{2}}\\
        \frac{1}{\sqrt{3}} & -\frac{1}{\sqrt{3}}& \frac{1}{\sqrt{3}} &\\
        \frac{2}{\sqrt{6}} &  \frac{1}{\sqrt{6}} & -\frac{1}{\sqrt{6}}
    \end{bmatrix}  
    \begin{bmatrix}
        0 & 1 & 1\\
        1 & -1 & 1\\
        1 & 1 & 0
    \end{bmatrix}
    =
    \begin{bmatrix}
        \sqrt{2} & -\frac{1}{\sqrt{2}} & \frac{1}{\sqrt{2}}\\
        0 & \sqrt{3} & 0\\
        0 & 0 & \frac{\sqrt{6}}{2}
    \end{bmatrix}
\]















\newpage


\subsection*{Problema 10}

Dada la matriz simétrica

\[
    \begin{bmatrix}
        -1 & 1 & 1\\
        1 & 0 & 1\\
        1 & 1 & -1
    \end{bmatrix},
\]

calcular los autovalores y autovectores. ¿Es diagonalizable? Obtener un conjunto de autovectores ortogonales a partir de los autovectores
asociados. Escribe explícitamente una matriz diagonal $D$ (con autovalores de menor a mayor leídos de izquierda a derecha) y dos matrices $Q$ y $Q^T$ tales
que $D=Q^T AQ$. Es decir, realiza la diagonalización ortogonal de esa matriz simétrica.\\

Encontramos los autovalores:

\[
    det
    \left(
    \begin{bmatrix}
        -1 - \lambda & 1 & 1\\
        1 & 0- \lambda & 1\\
        1 & 1 & -1 - \lambda
    \end{bmatrix}    
    \right) 
    =
    0
    \rightarrow
    \lambda^3+2\lambda^2-2\lambda-4=0 
\]

Y obtenemos: $\lambda_1=-2$, $\lambda_2=\sqrt{2}$ y $\lambda_3=-\sqrt{2}$. La matriz es diagnoalizable pues los 3 valores son 
únicos.

Procedemos a computar los autovectores y dividimos por su módulo para obtener los autovectores ortogonales:

Para $\lambda_1=-2$:

\[
    rref
    \left(
        \begin{bmatrix}[ccc|c]
            1 & 1 & 1 & 0\\
            1 & 2 & 1 & 0\\
            1 & 1 & 1 & 0
        \end{bmatrix}
    \right)
    =
    \begin{bmatrix}[ccc|c]
        1 & 0 & 1 & 0\\
        0 & 1 & 0 & 0\\
        0 & 0 & 0 & 0
    \end{bmatrix}
    \rightarrow
    \begin{bmatrix}
        v_1\\
        v_2\\
        v_3
    \end{bmatrix}
    =
    v_3
    \begin{bmatrix}
       -1\\
       0\\
       1 
    \end{bmatrix}
\]

Normalizado:

\[
    \frac{1}{\sqrt{2}}
    \begin{bmatrix}
        -1\\
        0\\
        1 
     \end{bmatrix}
     =
     \begin{bmatrix}
        \frac{-1}{\sqrt{2}}\\
        0\\
        \frac{1}{\sqrt{2}} 
     \end{bmatrix}
\]




Para $\lambda_2=\sqrt{2}$:

\[
    rref
    \left(
        \begin{bmatrix}[ccc|c]
            -1-\sqrt{2} & 1 & 1 & 0\\
            1 & -\sqrt{2} & 1 & 0\\
            1 & 1 & -1-\sqrt{2} & 0
        \end{bmatrix}
    \right)
    =
    \begin{bmatrix}[ccc|c]
        1 & 0 & -1 & 0\\
        0 & 1 & -\sqrt{2} & 0\\
        0 & 0 & 0 & 0
    \end{bmatrix}
    \rightarrow
    \begin{bmatrix}
        v_1\\
        v_2\\
        v_3
    \end{bmatrix}
    =
    v_3
    \begin{bmatrix}
       1\\
       \sqrt{2}\\
       1 
    \end{bmatrix}
\]

Normalizado:

\[
    \frac{1}{2}
    \begin{bmatrix}
        1\\
        \sqrt{2}\\
        1 
     \end{bmatrix}
     =
     \begin{bmatrix}
        \frac{1}{2}\\
        \frac{\sqrt{2}}{2}\\
        \frac{1}{2} 
     \end{bmatrix}
\]

Para $\lambda_3=-\sqrt{2}$:

\[
    rref
    \left(
        \begin{bmatrix}[ccc|c]
            -1+\sqrt{2} & 1 & 1 & 0\\
            1 & \sqrt{2} & 1 & 0\\
            1 & 1 & -1+\sqrt{2} & 0
        \end{bmatrix}
    \right)
    =
    \begin{bmatrix}[ccc|c]
        1 & 0 & -1 & 0\\
        0 & 1 & \sqrt{2} & 0\\
        0 & 0 & 0 & 0
    \end{bmatrix}
    \rightarrow
    \begin{bmatrix}
        v_1\\
        v_2\\
        v_3
    \end{bmatrix}
    =
    v_3
    \begin{bmatrix}
       1\\
       -\sqrt{2}\\
       1 
    \end{bmatrix}
\]\\

\newpage 


Normalizado:

\[
    \frac{1}{2}
    \begin{bmatrix}
        1\\
        -\sqrt{2}\\
        1 
     \end{bmatrix}
     =
     \begin{bmatrix}
        \frac{1}{2}\\
        -\frac{\sqrt{2}}{2}\\
        \frac{1}{2} 
     \end{bmatrix}
\]

Finalmente podemos formar la matriz $Q$ con los vecotres ortonormales:

\[
    Q=
    \begin{bmatrix}
        \frac{-1}{\sqrt{2}}     &  \frac{1}{2}          & \frac{1}{2}\\
        0                       & \frac{\sqrt{2}}{2}    &  -\frac{\sqrt{2}}{2}\\
        \frac{1}{\sqrt{2}}      & \frac{1}{2}           & \frac{1}{2} 
    \end{bmatrix}     
\]

Y su traspuesta:

\[
    Q^T=
    \begin{bmatrix}
        \frac{-1}{\sqrt{2}}     & 0                       & \frac{1}{\sqrt{2}}\\
        \frac{1}{2}          & \frac{\sqrt{2}}{2}    &   \frac{1}{2}         \\
        \frac{1}{2} &  -\frac{\sqrt{2}}{2} & \frac{1}{2} 
    \end{bmatrix}     
\]

Podemos entonces computar $D$:

\[
    D=Q^TAQ
    =
    \begin{bmatrix}
        \frac{-1}{\sqrt{2}}     & 0                       & \frac{1}{\sqrt{2}}\\
        \frac{1}{2}          & \frac{\sqrt{2}}{2}    &   \frac{1}{2}         \\
        \frac{1}{2} &  -\frac{\sqrt{2}}{2} & \frac{1}{2} 
    \end{bmatrix}   
    \begin{bmatrix}
        -1 & 1 & 1\\
        1 & 0 & 1\\
        1 & 1 & -1
    \end{bmatrix}
    \begin{bmatrix}
        \frac{-1}{\sqrt{2}}     &  \frac{1}{2}          & \frac{1}{2}\\
        0                       & \frac{\sqrt{2}}{2}    &  -\frac{\sqrt{2}}{2}\\
        \frac{1}{\sqrt{2}}      & \frac{1}{2}           & \frac{1}{2} 
    \end{bmatrix}  
    =
    \begin{bmatrix}
        -2 & 0 & 0\\
        0 & -\sqrt{2} & 0\\
        0 & 0 & \sqrt{2}
    \end{bmatrix}
\]
\end{document}
