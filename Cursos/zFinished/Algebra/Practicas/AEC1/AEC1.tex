\documentclass{article}
\usepackage{lipsum}
\usepackage[backend=biber]{biblatex}
\addbibresource{aec2.bib}
\usepackage{authoraftertitle}
\usepackage[top=2cm,bottom=1.5cm,left=1.5cm, right=3cm,includeheadfoot]{geometry}
\usepackage{graphicx}
\usepackage{fancyhdr}
\usepackage[spanish]{babel}
\usepackage{mathtools}
\usepackage{csquotes}
\usepackage{amssymb}
\usepackage{fancybox, graphicx}
\usepackage{array}
\usepackage{hhline}
\usepackage{hyperref}
\usepackage{tikz}
\usepackage{amsmath}
\usepackage{wrapfig}
\usepackage{float}
\usepackage{amsmath}
\usepackage{caption}
\usepackage{esvect}
\usepackage{siunitx}
\usepackage{commath}
\usepackage{tcolorbox}
\makeatletter
\renewcommand*\env@matrix[1][*\c@MaxMatrixCols c]{%
   \hskip -\arraycolsep
   \let\@ifnextchar\new@ifnextchar
   \array{#1}}
\makeatother

%Header & Footer
\pagestyle{fancy}
\fancyhead[LE]{\MyTitle}
\fancyhead[LO]{Álgebra Lineal}
\fancyhead[RO]{\leftmark}
\fancyhead[RE]{\leftmark}
\fancyfoot[L]{\raisebox{-1cm}{\includegraphics[height=2cm]{C:/Users/XYZ/Dropbox/DocumentGraphics/LOGOUDIMA.jpg}}}
\fancyfoot[R]{Corregido:\\ Dr. Juan José Moreno García}
%\fancyfoot[RO]{07/12/2018}
%Vars
\author{Alexander Sebastian Kalis}
\title{Actividad de Evaluación Continua 1}


%DOC

\begin{document}

\begin{titlepage}

    \begin{center}

        \line(1,0){300}\\
        [0.2in]
        \huge{\bfseries {\MyTitle}}\\
        [1mm]
        \line(2,0){200}\\
        [0.75cm]
        \textsc{\LARGE Álgebra Lineal}\\
        [2cm]
        \includegraphics[height=10cm]{C:/Users/XYZ/Dropbox/DocumentGraphics/Calculus.png}\\
        [3cm]

    \end{center}

    \begin{flushright}

        Autor: {\MyAuthor}\\
        Profesor: Dr. Juan José Moreno García\\
        Curso: 1o, Ingeniería de Organización Industrial\\
        UDIMA         

    \end{flushright}
    
\end{titlepage}

\tableofcontents \thispagestyle{empty}

\newpage

\section{Actividades}

\subsection{Problema 1}

Calcular la distancia desde el punto $Q = (1,-1,-1)$ a la recta intersección de los planos:

\begin{center}
    \begin{eqnarray*}
        x-2y+z=3\\
        2x-3y-z=6  
    \end{eqnarray*}  
\end{center}


RESOLUCIÓN:

\textit{Disponemos de 2 planos cuyas ecuaciones son:}
\begin{center}
    \begin{eqnarray*}
        \pi_1: x-2y+z=3\\
        \pi_2: 2x-3y-z=6  
    \end{eqnarray*}
\end{center}


\textit{Y sus rectas normales: }
\begin{eqnarray*}
    \vec{n_1}=<1,-2,1>\\
    \vec{n_2}=<2,-3,-1>
\end{eqnarray*}

\textit{Con lo cual son planos no paralelos. Para definir la línea que forman al cruzarse necesitaremos un punto y su vector director el cual
obtenemos calculando el \textbf{producto vectorial} de ambos:}
\[\vec{v}=\vec{n_1}\times\vec{n_2}\]

\[
\vec{v}=
\begin{vmatrix} 
    \hat{\imath} & \hat{\jmath} & \hat{k} \\
    2 & -3 & -1 \\
    1 & -2 & 1
\end{vmatrix}=
5\hat{\imath} + 3 \hat{\jmath} + \hat{k}\\
\]



\textit{Para encontrar el punto resolvemos el sistema tomando $y=0$:}
\[
    \begin{cases}
        y=0\\
        x+z=3\\
        2x-z=6
    \end{cases}\rightarrow
    P(x,y,z)=(3,0,0)
\]    
\textit{Con lo que podemos definir la ecuación de la recta y calculamos la distancia del punto $Q(1,-1,-1)$:}

\[ 
r:\frac{x-3}{5}=\frac{y}{3}=z   
\]
\[
\vec{u}=\vec{PQ}=<-2,-1,-1>    
\]
\[
    Distancia(r,Q)=
    \frac{\lVert\vec{u}\times\vec{v}\rVert}{\lVert\vec{v}\rVert}=
    \frac{\lVert<2,-3,1>\rVert}{\sqrt{35}}=
    \sqrt{\frac{2}{5}}
\]

\newpage

\subsection{Problema 2}

Resolver el siguiente sistema por gauss y sustitución hacia atrás:

\[
    \begin{cases}
        2x+5y+2z=4\\
        x-y+3z=0\\
        -x+7y-7z=4
    \end{cases}
    \xrightarrow[]{\text{Matriz aumentada}}
    \begin{bmatrix}[*2cr@{\quad}|@{\quad}>{\bf\color{black}}r]
        2 & 5 & 2  &  4 \\
        1 & -1 & 3 & 0 \\
        -1 & 7 & -7 & 4
    \end{bmatrix}
    \xrightarrow[]{R_1 \leftrightarrow R_2}
    \begin{bmatrix}[*2cr@{\quad}|@{\quad}>{\bf\color{black}}r]
        1 & -1 & 3 & 0 \\
        2 & 5 & 2  &  4 \\
        -1 & 7 & -7 & 4
    \end{bmatrix}
    \xrightarrow[]{}
    \begin{bmatrix}[*2cr@{\quad}|@{\quad}>{\bf\color{black}}r]
        1 & -1 & 3 & 0 \\
        0 & 7 & -4  &  4 \\
        0 & 6 & -4 & 4
    \end{bmatrix}
    \]
    \[
        \xrightarrow[]{C_2 \leftrightarrow C_3}
        \begin{bmatrix}[*2cr@{\quad}|@{\quad}>{\bf\color{black}}r]
            1 & 3 & -1 & 0 \\
            0 & -4 & 7  &  4 \\
            0 & -4 & 6 & 4
        \end{bmatrix}
        \xrightarrow[]{}
        \begin{bmatrix}[*2cr@{\quad}|@{\quad}>{\bf\color{black}}r]
            1 & 3 & -1 & 0 \\
            0 & -4 & 7  &  4 \\
            0 & 0 & -1 & 0
        \end{bmatrix}\\
    \]\\

    \textit{Queda el sistema reducido:}
   
    \[
        \begin{cases}
            x+3z-y=0\\
            -4z+7y=4\\
            y=0
        \end{cases}\rightarrow
        \begin{cases}
            x=3\\
            y=0\\
            z=-1
        \end{cases}    
    \]\\\\

\subsection{Problema 3}

Resolver el siguiente sistema por Gauss-Jordan:

\[
    \begin{matrix}
        2x & +6y & -2z & & = 2\\
        3x & +9y & -3z & +3w & = 1\\
        -x & -3y & +z & -3w & = 1\\
        -2x & -6y & +2z & -6w & = 2
    \end{matrix}
\]

Obtener forma escalonada reducida por fila de la matriz ampliada del sistema. Comprobar si el sistema es compatible. Comproba si es determinado
o indeterminado. En el caso de que sea lo segundo expresar la solución en función de un parámetro $t$ que se corresponda a la variable libre.\\

RESOLUCIÓN:\\

\[
    \begin{bmatrix}[*3cr@{\quad}|@{\quad}>{\bf\color{black}}r]
        2 & 6 & -2  & 0 & 2 \\
        3 & 9 & -3 & 3 & 1\\
        -1 & -3 & 1 & -3 & 1\\
        -2 & -6 & 2 & -6 & 2
    \end{bmatrix}
    \xrightarrow[]{\substack{R_4+R_1 \\ R1/2}}
    \begin{bmatrix}[*3cr@{\quad}|@{\quad}>{\bf\color{black}}r]
        1 & 3 & -1  & 0 & 1 \\
        3 & 9 & -3 & 3 & 1\\
        -1 & -3 & 1 & -3 & 1
    \end{bmatrix}
    \xrightarrow[]{\substack{R_3+R_1 \\ R_2+3R_1}}
    \begin{bmatrix}[*3cr@{\quad}|@{\quad}>{\bf\color{black}}r]
        1 & 3 & -1  & 0 & 1 \\
        0 & 0 & 0 & 3 & -2\\
        0 & 0 & 0 & -3 & 2
    \end{bmatrix}=
\]
\[
    \begin{bmatrix}[*3cr@{\quad}|@{\quad}>{\bf\color{black}}r]
        1 & 3 & -1  & 0 & 1 \\
        0 & 0 & 0 & 1 & -\frac{2}{3}\\
    \end{bmatrix}
\]\\

\textit{Podemos comprobar que el sistema es compatible indeterminado ya que tenemos infinitas soluciones
para las variables libres $z=s$ e $y=t$.}

\textit{Representamos de forma paramétrica:}

\[
    \begin{cases}
        x = s-3t+1\\
        y = t\\
        z =s \\
        w= -\frac{2}{3}
    \end{cases}
    \xrightarrow[]{}
    \begin{bmatrix} 
        x \\
        y \\
        z \\
        w 
    \end{bmatrix}
    =
    \begin{bmatrix}
        1 \\
        0 \\
        0 \\
        -\frac{2}{3}
    \end{bmatrix}
    +s
    \begin{bmatrix}
        1 \\
        0 \\
        1 \\
        0 
    \end{bmatrix}
    +t
    \begin{bmatrix}
        -3 \\
        1 \\
        0 \\
        0 
    \end{bmatrix}
\]


\subsection{Problema 4}

Sea el sistema de ecuaciones cuya matriz aumpliada es la siguiente:

\[
\begin{bmatrix}[*2cr@{\quad}|@{\quad}>{\bf\color{black}}r]
    1 & a & -1 & 2 \\
    2 & -1 & a & 5 \\
    1 & 10 & -6 & 1
\end{bmatrix}
\]

Discutir en función del parámetro $a$. \\

RESOLUCIÓN:\\

\textit{Se determina qué valores de $a$ hacen que el sistema sea incompatible con el determinante de la matriz:}

% \[
% \begin{bmatrix}[*2cr@{\quad}|@{\quad}>{\bf\color{black}}r]
%     1 & a & -1 & 2 \\
%     2 & -1 & a & 5 \\
%     1 & 10 & -6 & 1
% \end{bmatrix}
% \xrightarrow[]{\substack{R_2-2R_1 \\ R_3-R_1}}
% \begin{bmatrix}[*2cr@{\quad}|@{\quad}>{\bf\color{black}}r]
%     1 & a & -1 & 2 \\
%     0 & -1-2a & a+2 & 1 \\
%     0 & 10-a & -5 & -1
% \end{bmatrix}
% \xrightarrow[]{\substack{C_2 \leftrightarrow C_3}}
% \]

\[
det \left(
\begin{bmatrix}
    1 & a & -1  \\
    2 & -1 & a  \\
    1 & 10 & -6 
\end{bmatrix}
\right)=
a^2+2a-15=
(a-3)(a+5)
\]

\textit{Sabemos entonces que el sistema es compatible determinado para valores de $a$ diferentes a 3 y -5.}

\textit{Se evalúa el sistema para $a=3$:}

\[
\begin{bmatrix}[*2cr@{\quad}|@{\quad}>{\bf\color{black}}r]
    1 & 3 & -1 & 2 \\
    2 & -1 & 3 & 5 \\
    1 & 10 & -6 & 1
\end{bmatrix}\rightarrow
\begin{bmatrix}[*2cr@{\quad}|@{\quad}>{\bf\color{black}}r]
    1 & 3 & -1 & 2 \\
    0 & -7 & 5 & 1 \\
    0 & 0 & 0 & 0
\end{bmatrix}
\]

\textit{Podemos observar que el sistema queda compatible e indeterminado ya que existen más incógnitas que ecuaciones 
linealmente independientes.\\}


\textit{Se realiza el mismo procedimiento para $a=-5$:}

\[
\begin{bmatrix}[*2cr@{\quad}|@{\quad}>{\bf\color{black}}r]
    1 & -5 & -1 & 2 \\
    2 & -1 & -5 & 5 \\
    1 & 10 & -6 & 1
\end{bmatrix}\rightarrow
\begin{bmatrix}[*2cr@{\quad}|@{\quad}>{\bf\color{black}}r]
    1 & -5 & -1 & 2 \\
    0 & 9 & -3 & 1 \\
    0 & 0 & 0 & -\frac{8}{3}
\end{bmatrix}
\]

\textit{Lo que resulta en un sistema incompatible, ya que obtenemos el resultado $0=-\frac{8}{3}$.}

\newpage 

\subsection{Problema 5}

Realiza la descomposición LU de la matriz A

\textit{Se reduce de forma escalonada para obtener los coeficientes de $LU$}
\[
    A=
    \begin{bmatrix}
        1 & -2 & 1 \\
        2 & -5 & 4 \\
        1 & -4 & 6 
    \end{bmatrix}
    \xrightarrow[]{\substack{-2R_1+R_2 \\ -1R_1+R_3}}
    \begin{bmatrix}
        1 & -2 & 1 \\
        0 & -1 & 2 \\
        0 & -2 & 5 
    \end{bmatrix}
    \xrightarrow[]{\substack{-2R_2+R_3}}
    \begin{bmatrix}
        1 & -2 & 1 \\
        0 & -1 & 2 \\
        0 & 0 & 1 
    \end{bmatrix}
\]

\textit{Lo que es equivalente a multiplicarlo por:}

\[
    L= 
    \begin{bmatrix}
        1 & 0 & 0 \\
        2 & 1 & 0 \\
        1 & 2 & 1 
    \end{bmatrix}
    U=
    \begin{bmatrix}
        1 & -2 & 1 \\
        0 & -1 & 2 \\
        0 & 0 & 1 
    \end{bmatrix}     
\]\\

Si la matriz A corresponde a la matriz de un sistema, calcular la solución de
dicho sistema si el vector del lado derecho es:

\[
  b=
  \begin{bmatrix}
      5\\
      -3\\
      10
  \end{bmatrix}  
\]\\

\textit{Resolvemos el sistema $Ax=b \rightarrow LUx=b$}

\[
    Ly=b\rightarrow
    \begin{bmatrix}
        1 & 0 & 0 \\
        2 & 1 & 0 \\
        1 & 2 & 1 
    \end{bmatrix}
    \cdot
    \begin{bmatrix}
        y_1\\
        y_2\\
        y_3
    \end{bmatrix}
    =
    \begin{bmatrix}
        5\\
        -3\\
        10
    \end{bmatrix}\rightarrow
    \begin{bmatrix}
        y_1\\
        y_2\\
        y_3
    \end{bmatrix}=
    \begin{bmatrix}
        5\\
        -13\\
        31
    \end{bmatrix}
\]

\textit{Y finalmente:}
\[
    Ux=b\rightarrow
    \begin{bmatrix*}[r]
        1 & -2 & 1 \\
        0 & -1 & 2 \\
        0 & 0 & 1 
    \end{bmatrix*}
    \cdot
    \begin{bmatrix}
        x_1\\
        x_2\\
        x_3
    \end{bmatrix}=
    \begin{bmatrix*}[r]
        5\\
        -13\\
        31
    \end{bmatrix*}
    \rightarrow
    \begin{bmatrix*}[r]
        x_1\\
        x_2\\
        x_3
    \end{bmatrix*}=
    \begin{bmatrix}[r]
        124\\
        75\\
        31
    \end{bmatrix}
\]


\subsection{Problema 6}

Realiza la descomposición LU de la matriz A y comprueba el resultado:

\[
A=
\begin{bmatrix*}[r]
    2 & 4 & -1 & 5\\
    -4 & -5 & 3 & -8\\
    2 & -5 & -4 & 1\\
    -6 & 0 & 7 & -3
\end{bmatrix*}\\    
\]\\

\textit{Se reduce de forma escalonada para obtener los coeficientes de $LU$}

\[
    \begin{bmatrix*}[r]
        2 & 4 & -1 & 5\\
        -4 & -5 & 3 & -8\\
        2 & -5 & -4 & 1\\
        -6 & 0 & 7 & -3
    \end{bmatrix*}
    \xrightarrow[]{\substack{2R_1+F_2 \\ \substack{-1R_1+R_3 \\ 3R_1+R_4}}}
    \begin{bmatrix*}[r]
        2 & 4 & -1 & 5\\
        0 & 3 & 1 & 2\\
        0 & -9 & -3 & -4\\
        0 & 12 & 4 & -12
    \end{bmatrix*}
    \xrightarrow[]{\substack{3R_2 + R_3 \\ -4R_2+R_4}}
    \begin{bmatrix*}[r]
        2 & 4 & -1 & 5\\
        0 & 3 & 1 & 2\\
        0 & 0 & 0 & 2\\
        0 & 0 & 0 & -20
    \end{bmatrix*}
\]

\newpage
\textit{Entonces los coeficientes de $LU$ y su comprobación son:}

\[
  A=L \cdot U \rightarrow
  \begin{bmatrix*}[r]
    2 & 4 & -1 & 5\\
    -4 & -5 & 3 & -8\\
    2 & -5 & -4 & 1\\
    -6 & 0 & 7 & -3
\end{bmatrix*}=
\begin{bmatrix*}[r]
    1 & 0 & 0 & 0\\
    -2 & 1 & 0 & 0\\
    1 & -3 & 1 & 0\\
    -3 & 4 & 0 & 1
\end{bmatrix*}
\cdot
\begin{bmatrix*}[r]
    2 & 4 & -1 & 5\\
    0 & 3 & 1 & 2\\
    0 & 0 & 0 & 2\\
    0 & 0 & 0 & 20
\end{bmatrix*}
\]\\

\subsection{Problema 7}

Obtener por el método de Gauss-Jordan la inversa de la matriz A, si es
posible.

\[
    \begin{bmatrix*}[r]
        2 & 1 & -4  \\
        -4 & -1 & 6   \\
        -2 & 2 & -1 
    \end{bmatrix*}
\]

Si $A$ es la matriz de un sistema de ecuaciones con el vector del lado derecho

\[
    b=
    \begin{bmatrix*}[r]
        3 \\
        -1   \\
        -2 
    \end{bmatrix*}
\]

¿Cuál es la solución del sistema usando dicha inversa?\\

\[
    \left[
        \begin{array}{ccc|ccc}
            2 & 1 & -4 & 1 & 0 & 0 \\
            -4 & -1 & 6 & 0 & 1 & 0  \\
            -2 & 2 & -1 & 0 & 0 & 1
        \end{array}
    \right]
    \xrightarrow[]{\substack{\frac{1}{2}R_1 \\ \substack{ 2R_1 + R_2\\ R_1 + R_3}}}
    \left[
        \begin{array}{ccc|ccc}
            1 & \frac{1}{2} & -2        & \frac{1}{2} & 0 & 0 \\
            0 & 1 & -2                   & 2 & 1 & 0  \\
            0 & 3 & -5                  & 1 & 0 & 1
        \end{array}
    \right]
    \xrightarrow[]{\substack{-3R_2 + R_3 \\ }}
    \left[
        \begin{array}{ccc|ccc}
            1 & \frac{1}{2} & -2        & \frac{1}{2} & 0 & 0 \\
            0 & 1 & -2                   & 2 & 1 & 0  \\
            0 & 0 & 1                  & -5 & -3 & 1
        \end{array}
    \right]
    \xrightarrow[]{\substack{2R_3 + R_2 \\ 2R_3+R_1}}
\]
\[
    \left[
        \begin{array}{ccc|ccc}
            1 & \frac{1}{2} & 0        & -\frac{19}{2} & -6 & 2 \\
            0 & 1 & 0                   & -8 & -5 & 2  \\
            0 & 0 & 1                  & -5 & -3 & 1
        \end{array}
    \right]
    \xrightarrow[]{\substack{-\frac{1}{2}R_2+R_1}}
    \left[
        \begin{array}{ccc|ccc}
            1 & 0 & 0        & -\frac{11}{2} & -\frac{-7}{2} & 1 \\
            0 & 1 & 0                   & -8 & -5 & 2  \\
            0 & 0 & 1                  & -5 & -3 & 1
        \end{array}
    \right]
\]\\

\textit{Entonces la matriz inversa es:}

\[
    A^{-1}=\frac{1}{2}
    \begin{bmatrix*}[r]
        -11 & -7 & 2\\
        -16 & -10 & 4\\
        -10 & -6 & 2
    \end{bmatrix*}  
\]

\textit{Y $Ax=b$ es:}

\[
    x=A^{-1} \cdot b =
    \frac{1}{2}
    \begin{bmatrix*}[r]
        -11 & -7 & 2\\
        -16 & -10 & 4\\
        -10 & -6 & 2
    \end{bmatrix*}  
    \cdot
    \begin{bmatrix*}[r]
        3 \\
        -1   \\
        -2 
    \end{bmatrix*}
    =
    \begin{bmatrix*}[r]
        -15 \\
        -23   \\
        -14 
    \end{bmatrix*}
\]

\newpage

\subsection{Problema 8}


Encontrar en $\mathbb{R}^2$ la matriz canónica de la aplicación que en el orden indicado
efectúa las siguientes operaciones: giro de 60º en el sentido contrario a las
agujas del reloj, dilatación positiva de factor 2 horizontal, dilatación positiva
vertical de factor $\sqrt{3}$ y reflexión especular respecto al eje Y .
Nota: no usar notación decimal\\

\textit{Para efectuar la rotación de 60º utilizaremos la matriz}

\[
    R=
    \begin{bmatrix*}[r]
        \cos(60) & -\sin(60) \\
         \sin(60) & \cos(60)     
    \end{bmatrix*} 
    =
    \begin{bmatrix*}[r]
        \frac{1}{2} & -\frac{\sqrt{3}}{2} \\
         \frac{\sqrt{3}}{2} & \frac{1}{2}     
    \end{bmatrix*} 
\]

\textit{Para la dilatación positiva horizontal de factor 2 y vertical de factor $\sqrt{3}$:}

\[
    D=
    \begin{bmatrix*}[r]
        2 & 0 \\
         0 & \sqrt{3}     
    \end{bmatrix*}    
\]

\textit{Para efectuar la reflexión especular respecto al eje Y:}

\[
    E=
    \begin{bmatrix*}[r]
        -1 & 0 \\
         0 & 1     
    \end{bmatrix*}    
\]

\textit{Entonces podemos representar estas 3 transformaciones conjuntamente mediante su producto en el siguiente orden:}

\[
  A=EDR=
  \begin{bmatrix*}[r]
    -1 & 0 \\
     0 & 1     
\end{bmatrix*}   
  \begin{bmatrix*}[r]
    2 & 0 \\
     0 & \sqrt{3}     
\end{bmatrix*}    
  \begin{bmatrix*}[r]
    \frac{1}{2} & -\frac{\sqrt{3}}{2} \\
     \frac{\sqrt{3}}{2} & \frac{1}{2}     
\end{bmatrix*}
=
\begin{bmatrix*}[r]
    -1 & \sqrt{3} \\
     \frac{3}{2} & \frac{\sqrt{3}}{2}     
\end{bmatrix*}
\]
\subsection{Problema 9}

Calcula el siguiente determinante:

\[
    \begin{vmatrix}
        1 & 1 & 1 & 1\\
        1 & a & 0 & 0\\
        1 & 0 & a & 0\\
        1 & 0 & 0 & a
    \end{vmatrix}
    \xrightarrow[]{\substack{-R_1+R_2 \\ \substack{-R_1+R_3 \\ -R_1+R_4}}}
    \begin{vmatrix}
        1 & 1 & 1 & 1\\
        0 & a-1 & -1 & -1\\
        0 & -1 & a-1 & -1\\
        0 & -1 & -1 & a-1
    \end{vmatrix}
    \xrightarrow[]{R_2 \leftrightarrow R_3}
    -
    \begin{vmatrix}
        1 & 1 & 1 & 1\\
        0 & -1 & a-1 & -1\\
        0 & a-1 & -1 & -1\\
        0 & -1 & -1 & a-1
    \end{vmatrix}
    \xrightarrow[]{(a-1)R_2+R_3}
\]
\[
    \begin{vmatrix}
        1 & 1 & 1 & 1\\
        0 & -1 & a-1 & -1\\
        0 & 0 & (a-2)a & -a\\
        0 & -1 & -1 & a-1
    \end{vmatrix}
    \xrightarrow[]{-R_2+R_4}
    \begin{vmatrix}
        1 & 1 & 1 & 1\\
        0 & -1 & a-1 & -1\\
        0 & 0 & (a-2)a & -a\\
        0 & 0 & -a & a
    \end{vmatrix}
    \xrightarrow[]{\frac{1}{-2+a}R_3+R_4}
    \begin{vmatrix}
        \mathbf{1} & 1 & 1 & 1\\
        0 & \mathbf{-1} & a-1 & -1\\
        0 & 0 & \mathbf{(a-2)a} & -a\\
        0 & 0 & 0 & \mathbf{\frac{(-3+a)a}{-2+a}}
    \end{vmatrix}
\]\\

\textit{Una vez reducida, podemos calcular el determinante de la matriz con el producto de sus elementos
diagonales}

\[
    det
    \left(
        \begin{vmatrix}
        1 & 1 & 1 & 1\\
        1 & a & 0 & 0\\
        1 & 0 & a & 0\\
        1 & 0 & 0 & a
    \end{vmatrix}
    \right)
    =
    -\frac{-2((-2+a)a)((-3+a)a)}{-2+a}
    =
    a^3-3a^2
\]

\newpage

\subsection{Problema 10}

Sea la matriz 

\[
    \begin{bmatrix}
        2 & -6\\
        -3 & 0
    \end{bmatrix}
\]

Encuéntrese la matrix 2 x 2 a la que llamaremos $B$ cuyas columnas sean distintas de cero y que cumpla
que $AB=0$. Nota: en realidad se trata de una familia de soluciones.

\[
    Tomamos \ A=
    \begin{bmatrix}
        2 & -6\\
        -3 & 0
    \end{bmatrix} 
    \ y \ B=
    \begin{bmatrix}
        a & b\\
        c & d
    \end{bmatrix}
\]

\[
  AB=
    \begin{bmatrix}
        2 & -6\\
        -3 & 0
    \end{bmatrix} 
    \begin{bmatrix}
        a & b\\
        c & d
    \end{bmatrix}
    =
    \begin{bmatrix}
        2a-6c & 2b-6d\\
        -3a+9c & -3b+9d
    \end{bmatrix}
    =
    \begin{bmatrix}
        0 & 0\\
        0 & 0
    \end{bmatrix}
\]\\

\textit{Resolvemos el sistema $AB=0$ de forma matricial por eleminación Gaussiana:}

\[
    \begin{bmatrix}[*3cr@{\quad}|@{\quad}>{\bf\color{black}}r]
        2 & 0 & -6  & 0 &  0 \\
        0 & 2 & 0 & -6 & 0 \\
        -3 & 0 & 9 & 0 & 0 \\
        0 & -3 & 0 & 9 & 0
    \end{bmatrix}
    \xrightarrow[]{\substack{3R_1+2R_3 \\ 3R_2+2R_4}}
    \begin{bmatrix}[*3cr@{\quad}|@{\quad}>{\bf\color{black}}r]
        2 & 0 & -6  & 0 &  0 \\
        0 & 2 & 0 & -6 & 0 \\
        0 & 0 & 0 & 0 & 0 \\
        0 & 0 & 0 & 0 & 0
    \end{bmatrix}
\]

\textit{Lo que nos devuelve un sistema compatible indeterminado con variables libres $c$ y $d$}

\textit{Tomamos c=t y d=s para parametrizar el sistema, entonces:}

\[
    \begin{cases}
        a=3t\\
        b=3s\\
        c=t\\
        d=s
    \end{cases}
    \text{con lo cual la familia de soluciones del sistema tiene la forma: }
    B=  
    \begin{bmatrix}
        3t & 3s\\
        t & s
    \end{bmatrix}
\]



\end{document}
