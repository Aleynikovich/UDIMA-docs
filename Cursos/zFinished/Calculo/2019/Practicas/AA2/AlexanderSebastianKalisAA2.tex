\documentclass{article}
\usepackage[backend=biber]{biblatex}
\usepackage{authoraftertitle}
\usepackage[top=2cm,bottom=1.5cm,left=1.5cm, right=3cm,includeheadfoot]{geometry}
\usepackage{graphicx}
\usepackage{fancyhdr}
\usepackage[spanish]{babel}
\usepackage{mathtools}
\usepackage{csquotes}
\usepackage{amssymb}
\usepackage{fancybox, graphicx}
\usepackage{array}
\usepackage{hhline}
\usepackage{hyperref}
\usepackage{amsmath}
\usepackage{float}
\usepackage{amsmath}
\usepackage{esvect}
\usepackage{siunitx}
\usepackage{commath}
\newcommand{\ihat}{\textbf{\^\i}}
\newcommand{\jhat}{\textbf{\^\j}}



%Header & Footer

\pagestyle{fancy}
\fancyhead[LE]{\MyTitle}
\fancyhead[LO]{Análisis Mateático}
\fancyhead[RO]{\leftmark}
\fancyhead[RE]{\leftmark}
\fancyfoot[L]{\raisebox{-1cm}{\includegraphics[height=2cm]{E:/KUKADisk/UDIMA/DocumentGraphics/LOGOUDIMA.jpg}}}
\fancyfoot[R]{Corregido:\\ Dr. J.J. Moreno García}


%Vars
\author{Alexander Sebastian Kalis}
\title{Actividad de Aprenizaje 2\\ Octave}


%DOC


\begin{document}

\begin{titlepage}

    \begin{center}

        \line(1,0){300}\\
        [0.2in]
        \huge{\bfseries {\MyTitle}}\\
        [1mm]
        \line(2,0){200}\\
        [0.75cm]
        \textsc{\LARGE Análisis Matemático}\\
        [3cm]
        \includegraphics[height=10cm]{E:/KUKADisk/UDIMA/DocumentGraphics/octave.png}\\
        [2cm]

    \end{center}

    \begin{flushright}

        Autor: {\MyAuthor}\\
        Profesor: Dr. Juan José Moreno García\\
        Curso: 1o, Ingeniería de Organización Industrial\\
        UDIMA         

    \end{flushright}
    
\end{titlepage}

\tableofcontents \thispagestyle{empty}

\newpage

\section{Introducción}

El objetivo de la actividad es la contstrucción de un programa mediante el lenguaje MatLab/Octave que nos permita computar la resolución 
numérica de integrales por varios métodos, es decir, calcular el área que hay por debajo de una curva. 

El método utilizado para la resolución de la actividad será el método de los trapecios. Este método trata de dar los valores que tiene la función
$f$ para una serie de valores equidistantes de $x$. De este modo tenemos trapecios de distintas áreas que podemos sumar para dar el área total. También
se utilizará el método de Simpson, que aproxima las curvas a un poliniomio de grado tres.

\newpage
\section{Desarrollo de la actividad}

    \subsection{Apartado A}

    \textit{Calcular numéricamente la siguiente integral definida:\\}
    \[I=\int_0^{2\pi} \left| x\sen x^2 \right| dx\]

    \textit{Usando el sistema de los trapecios y Simpson, para 11, 101, 1001 y 10001 puntos e interpretar ambos casos.}

    \subsubsection{Sistema de trapecios}

    Para el sistema de trapecios, se ha desarrollado una función que toma 4 argumentos.

    \begin{itemize}
        \item Límite inferior de integración.
        \item Límite superior de integración.
        \item La función a integrar.
        \item El número de trapecios en los que descomponemos el área de la integral.\\
    \end{itemize}

    \begin{center}
        \includegraphics[height=10cm]{E:/KUKADisk/UDIMA/DocumentGraphics/AA2Mates/AA2Matestrapz.png}\\
    \end{center}

    \textbf{Comparativa de resultados}\\

    Esto nos permite comparar los resultados que obtenemos dependiendo de la cantidad de trapecios en los
    que dividimos el área.
    
    A continuación se verá unos ejemplos de cómo influyen las subdivisiones en la aproximación del resultado.

    \newpage

    \textbf{Resultado por cálculo analítico}\\

    Mediante el cálculo analítico podemos concluir que

    \[I=\int_0^{2\pi} \left| x\sen x^2 \right| dx = 12.604\]

    \textbf{Resultado por cálculo numérico}\\

        Utilizando la función programada en Octave, se evalúa el resultado de la integral dividiendo el área en 10, 100, 1000 y 10000 trapecios:\\

        \begin{center}
            \includegraphics[height=6cm]{E:/KUKADisk/UDIMA/DocumentGraphics/AA2Mates/AAMatestrapza.png}
        \end{center}
        
        Podemos ver claramente cómo el valor se aproxima cada vez más al valor analítico a medida que aumentamos los trapecios. Sim embargo
        aplicando la fórmula del error absoluto $\epsilon_a=\bar{X}-X_i$ podemos ver que \textbf{hay picos de aumento en error absoluto cuando nos acercamos a las decenas:}

        \begin{figure}[!htb]
            \begin{minipage}{0.48\textwidth}
              \centering
              \includegraphics[height=5.5cm]{E:/KUKADisk/UDIMA/DocumentGraphics/AA2Mates/AA2MatestrapzAprox.png}
              \caption{Subdivisiones de $f$}
            \end{minipage}\hfill
            \begin{minipage}{0.48\textwidth}
              \centering
              \includegraphics[height=5.5cm]{E:/KUKADisk/UDIMA/DocumentGraphics/AA2Mates/AA2Mateserror.png}
              \caption{Error absoluto respecto a las subdivisiones}
            \end{minipage}
        \end{figure}

        \newpage  
    
        \subsubsection{Sistema de Simpson}

        Para el sistema de Simpson, también se ha buscado una forma alternativa de plantear el programa con la finalidad de practicar. En este caso la
        función solo admite 1 argumento, el número de subdivisiones, las cuales deben ser pares ya que van de 2 en 2.

        \begin{center}
            \includegraphics[height=10cm]{E:/KUKADisk/UDIMA/DocumentGraphics/AA2Mates/AA2Matessimp13.png}\\
        \end{center}
        
        \textbf{Resultado por cálculo numérico}\\

        Se calculan los resultados para 10, 100, 1000 y 10000 subdivisiones:

        \begin{center}
            \includegraphics[height=6cm]{E:/KUKADisk/UDIMA/DocumentGraphics/AA2Mates/AA2Matessimpa.png}\\
        \end{center}

        \newpage 

        Volvemos a consultar los gráficos de subdivisión y aproximación:

        \begin{figure}[!htb]
            \begin{minipage}{0.48\textwidth}
              \centering
              \includegraphics[height=5.5cm]{E:/KUKADisk/UDIMA/DocumentGraphics/AA2Mates/AA2MatessimpAprox.png}
              \caption{Subdivisiones de $f$}
            \end{minipage}\hfill
            \begin{minipage}{0.48\textwidth}
              \centering
              \includegraphics[height=5.5cm]{E:/KUKADisk/UDIMA/DocumentGraphics/AA2Mates/AA2Matessimperror.png}
              \caption{Error absoluto respecto a las subdivisiones}
            \end{minipage}
        \end{figure}

        Donde se ve claramente la diferencia con el método de los trapecios, ya que la función del error absoluto baja constantemente a medida que
        se aumentan las subdivisiones, a diferencia del método de los trapecios, en el cual había altibajos propios de una función sinusoidal que eran
        provocados por los trapecios saliéndose de la gráfica de la función en las bajadas o bien no llegando hasta el final en las subidas.\\\\

        \textbf{Conclusión}\\

        Para concluir con ambos métodos conjutamente, utilizamos la función plot() para ver los resultados de ambos métodos juntos en una gráfica:

        \begin{center}
            \includegraphics[width=\textwidth]{E:/KUKADisk/UDIMA/DocumentGraphics/AA2Mates/AA2Matescomparagraph.png}\\
        \end{center}

        \newpage 
        \subsection{Apartado B}

        \textit{Calcular analíticamente la integral definida}
        
        \[  
            I = \int_0^\pi \sen^3x \ dx =
            \int _0^{\pi }\left(1-\cos ^2\left(x\right)\right)\sin \left(x\right)dx
            \int _1^{-1}-1+u^2du=
            -\int _{-1}^1-1+u^2du=
        \]
        \[
            -\left(-\int _{-1}^11du+\int _{-1}^1u^2du\right)=
            -\left(-2+\frac{2}{3}\right)=
            \frac{4}{3}
        \]

        \textit{Entonces, por comparación, calcular la cantidad mínima de puntos que habría que usar con Simpson para tener un resultado con al menos 5 cifras 
        significativas correctas (4 decimales).}\\

        Para comparar los resultados, se ha realizado otro script que se ha llamado comparacion.m. Este script consta de un bucle \textit{while} que va ir iterando
        entre posibles valores de subdivisiones (de 2 en 2) mientras que el resultado no cumpla la condición. Una vez cumplida la condición propuesta, el bucle se romperá e imprimirá
        por consola el valor que estamos buscando.

        \begin{center}
            \includegraphics[width=1\textwidth]{E:/KUKADisk/UDIMA/DocumentGraphics/AA2Mates/AA2Matescomparacion.png}\\
        \end{center}

        
        Entonces ejecutando este nuevo programa obtenemos el resultado de \textbf{22 subdivisiones o 23 puntos}
        
        \begin{center}
            \includegraphics[height=4cm]{E:/KUKADisk/UDIMA/DocumentGraphics/AA2Mates/AA2Matescomparaciono.png}\\
        \end{center}


    \end{document}