\documentclass{article}
\title{Repaso Examen Analisis Matematico}

\begin{document}
    
    \section{Que hay que saber hacer}

    \subsection{Unidad 1}
    \begin{itemize}
        \item Problemas básicos de funciones
        \item Cálculo de límites
        \item Continuidad de una función en un punto
        \item Cálculo de asíntotas
    \end{itemize}

    \subsection{Unidad 2}
    \begin{itemize}
        \item Derivadas
        \item Cálculo de extremos, concavidad, crecimiento, puntos de inflexión
        \item Representación gráfica de funciones
        \item Optimización
        \item Polinomio de Taylor       
    \end{itemize}

    \subsection{Unidad 3}
    \begin{itemize}
        \item Integrales
        \item Cálculo de áreas y volúmenes
    \end{itemize}

    \subsection{Unidad 4}
    \begin{itemize}
        \item Expresar funciones de forma paramétrica
        \item Coordenadas polares y esféricas
    \end{itemize}

    \subsection{Unidad 5}
    \begin{itemize}
        \item Diferencia entre serie y sucesión
        \item Convergencia 
        \item Desarrollo de series de Taylor
    \end{itemize}

    \subsection{Unidad 6}
    \begin{itemize}
        \item Límites para funciones multivariable
        \item Continuidad
        \item Derivadas parciales. Regla de la cadena.
        \item Extremos absolutos y relativos
        \item Puntos críticos. Hessiana.
        \item Extremos condicionados.
    \end{itemize}

    \subsection{Unidad 7}
    \begin{itemize}
        \item Resolución de Integrales dobles o triples
        \item Cálculo de superfícies y volúmenes.
        \item Integrales de línea.
        \item Cambio de variables (XYZ-POLAR-SPHERIC).
        \item Gradiente.
        \item Derivada direccional y rotacional.
        \item Divergencia.
        \item Teoremas de Green, Stokes y Gauss. (Identificar integral dificil y hacerla facil con estos teoremas).
    \end{itemize}

    \subsection{Unidad 8}
    \begin{itemize}
        \item Ecuaciones diferenciales
        \begin{itemize}
            \item Separables
            \item Homogeneas
            \item Por cambio de variable
            \item Exactas
            \item Lineales de primera orden
            \item Bernoulli y Ricatti
        \end{itemize}
    \end{itemize}

    \subsection{Unidad 9}
    \begin{itemize}
        \item Descomposicion en fracciones simples
        \item Laplace
        \item Oscilador armónico.
        \item RLC
    \end{itemize}
   



\end{document}