\documentclass[a4paper,12pt]{article}
\usepackage[latin1]{inputenc}
\usepackage[T1]{fontenc}
\usepackage[spanish]{babel}
\usepackage{amsmath}
\usepackage{fancyhdr}
\usepackage{latexsym,pifont,mathrsfs}
\usepackage{amssymb,amsfonts,textcomp}
\usepackage{multicol}
\usepackage{color}
\usepackage{fancyhdr}
\usepackage[top=1cm,bottom=2cm,left=2cm,right=3cm,nohead,nofoot]{geometry}
\usepackage{array}
\usepackage{hhline}
\usepackage{hyperref}
\usepackage{tikz}
\usepackage{fancybox}
\usepackage{caption}
\usepackage[shortlabels]{enumitem}
\usepackage[document]{ragged2e}
\usepackage{authoraftertitle}

\pagestyle{fancy}
\title{AA1 \LaTeX}

\author{Alexander Sebastian Kalis}
\date{\today}

\begin{document}
Actividad realizada por: \MyAuthor \newline
\vspace{1em}
    \begin{center}  
        \setlength{\fboxsep}{12pt}
            \shadowbox
            {
                \parbox{1\textwidth}
                {   
                    
                    \textbf{\underline{EJEMPLO 17}}
                    \break 
                    
                    Usar la regla de L'Hopital para hallar los siguientes limites:
                    \break

                    \begin{enumerate}[a)]
                        \item $\lim_{x \to 0}\:\cfrac{\sin \:x}{x}$
                        \item $\lim_{x \to 0}\:\cfrac{1-\cos \:x}{x^2}$
                        \item $\lim_{x \to 0}\:\cfrac{x-\tan \:x}{x^3}$
                        \item $\lim_{x \to \infty}\:x\:ln\:\cfrac{x-1}{x+1}$
                        \item $\lim_{x \to 0^+}\:xe^{1/x}$
                    \end{enumerate}

                    \vspace{2em}
                    \textbf{Solucion}
                    \vspace{1em}

                    En cada caso, despues de verificar la indeterminacion, se aplica la
                    regla de L'Hopital las \newline veces necesarias.
                    
                    \begin{enumerate}[a)]
                        \item %a)
                            $
                                \lim_{x \to 0}\:\cfrac{\sin x}{x}= 
                                \left(\:\cfrac{0}{0}\:\right)\overset{H}{=} 
                                \lim_{x \to 0}\: \cfrac{\cos x}{1}= 
                                \cfrac{1}{1}=
                                1
                            $ 
                        \item %b)
                            $
                                \lim_{x \to 0}\:\cfrac{1-\cos x}{x^2} =
                                \left(\:\cfrac{0}{0}\:\right) \overset{H}{=}
                                \lim_{x \to 0}\: \cfrac{\sin x}{2x} =
                                \left(\:\cfrac{0}{0}\:\right) \overset{H}{=}
                                \lim_{x \to 0}\: \cfrac{\cos x}{2}=
                                \cfrac{1}{2}
                            $
                        \item %c)
                            $
                                \lim_{x \to 0}\:\cfrac{x-\tan x}{x^3} =
                                \left(\:\cfrac{0}{0}\:\right) \overset{H}{=}
                                \lim_{x \to 0} \cfrac{1-(1+\tan^2 x)}{3x^2}=
                                lim_{x \to 0} \cfrac{-\tan^2 x}{3x^2}=
                                \left(\:\cfrac{0}{0}\:\right) \overset{H}{=}
                                \lim_{x \to 0} \cfrac{-2tan\:x(1+\tan^2 x)}{6x}=
                                \lim_{x \to 0} \cfrac{-\tan x}{3x}=
                                \left(\:\cfrac{0}{0}\:\right) \overset{H}{=}
                                \lim_{x \to 0} \cfrac{-(1+\tan^2 x)}{3}=
                                \cfrac{-1}{3}
                            $
                        \item %d)
                            $
                                \lim_{x \to \infty}\:x\:ln\:\cfrac{x-1}{x+1}=
                                (\infty \cdot 0)=
                                \lim_{x \to \infty}\: \cfrac{ln\: \cfrac{x-1}{x+1}}{\cfrac{1}{x}}=
                                \left(\:\cfrac{0}{0}\:\right) \overset{H}{=}
                                \lim_{x \to \infty}\: \cfrac{\cfrac{1}{x-1}-\cfrac{1}{x+1}}{\cfrac{-1}{x^2}}=\newline
                                =\lim_{x \to \infty}\: \cfrac{-2x^2}{x^2-1}=
                                -2
                            $
                        \item %e)
                            $
                                \lim_{x \to 0^+}\:xe^{1/x}=
                                (0 \cdot \infty)=
                                \lim_{x \to 0}\:\cfrac{e^{1/x}}{\cfrac{1}{x}}=
                                \cfrac{\infty}{\infty}\overset{H}{=}
                                \lim_{x \to 0^+}\: \cfrac{\cfrac{-1}{x^2}e^{1/x}}{\cfrac{-1}{x^2}}=
                                \lim_{x \to 0^+}\: e^{1/x}=
                                \infty
                            $
                    \end{enumerate}
                }
            }

        %FOOTER
        \lfoot{\textbf{www.udima.es}}
        \cfoot{}
        \rfoot{85}

    \end{center}

\end{document}
