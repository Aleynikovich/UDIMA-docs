\documentclass{article}
\usepackage{lipsum}
\usepackage[backend=biber]{biblatex}
\addbibresource{aec2.bib}
\usepackage{authoraftertitle}
\usepackage[top=2cm,bottom=1.5cm,left=1.5cm, right=3cm,includeheadfoot]{geometry}
\usepackage{graphicx}
\usepackage{fancyhdr}
\usepackage[spanish]{babel}
\usepackage{mathtools}
\usepackage{csquotes}
\usepackage{amssymb}
\usepackage{fancybox, graphicx}
\usepackage{array}
\usepackage{hhline}
\usepackage{hyperref}
\usepackage{tikz}
\usepackage{amsmath}
\usepackage{wrapfig}
\usepackage{float}
\usepackage{amsmath}
\usepackage{caption}
\usepackage{esvect}
\usepackage{siunitx}
\usepackage{commath}
\newcommand{\ihat}{\textbf{\^\i}}
\newcommand{\jhat}{\textbf{\^\j}}
%Header & Footer

\pagestyle{fancy}
%\fancyhead[LE]{\MyTitle}
\fancyhead[LO]{Análisis Matemático}
%\fancyhead[RO]{\leftmark}
%\fancyhead[RE]{\leftmark}
\fancyfoot[L]{\raisebox{-1cm}{\includegraphics[height=1.5cm]{}}}
\fancyfoot[R]{Corregido:\\ Dr. Juan José Moreno García}
%\fancyfoot[RO]{07/12/2018}


%Vars
\author{Alexander Sebastian Kalis}
\title{Actividad de Evaluación Continua 1}


%DOC


\begin{document}

\begin{titlepage}

    \begin{center}

        \line(1,0){300}\\
        [0.2in]
        \huge{\bfseries {\MyTitle}}\\
        [1mm]
        \line(2,0){200}\\
        [0.75cm]
        \textsc{\LARGE Análisis Matemático}\\
        [2cm]
        \includegraphics[height=10cm]{}\\
        [3cm]

    \end{center}

    \begin{flushright}

        Autor: {\MyAuthor}\\
        Profesor: Dr. Juan José Moreno García\\
        Curso: Ingeniería de Organización Industrial\\
        UDIMA         

    \end{flushright}
    
\end{titlepage}

\section*{Problema 1}

Resolver los siguientes problemas:\\

a) Halla el siguiente límite:\\

\[
    \lim_{x\to 2} \cfrac[]{x^2+x-6}{2-x}
\]

Factorizamos la cuadrática para simplificar el problema:\\

\[
    \lim_{x\to 2} \cfrac[]{x^2+x-6}{2-x} \implies
    \lim_{x\to 2} \frac{(x-2)(x+3)(-1)}{(2-x)(-1)} \implies 
    \lim_{x\to 2} (-x-3) = -2 -3 = 5
\]

b) Calcular este límite:\\

\[
    \lim_{x\to 7} \frac{x-7}{\sqrt{x-4}-\sqrt{3}}
\]

Para resolver este problema haremos uso de la razionalización del denominador:\\

\[
    \frac{x-7}{\sqrt{x-4}-\sqrt{3}} = 
    \frac{(x-7)(\sqrt{x-4}+\sqrt{3})}{(\sqrt{x-4}-\sqrt{3})(\sqrt{x-4}+\sqrt{3})} =
    \sqrt{x-4}+\sqrt{3}
\]

\[
    \lim_{x\to 7} \sqrt{x-4}+\sqrt{3}=\sqrt{7-4}+\sqrt{3}=2\sqrt{3}
\]

c) Determinar el siguiente límite sin usar Hôpital:\\

\[
    \lim_{x\to \infty} \frac{2x^3}{2+x^5}
\]

Podemos deducir que el límite será 0 pues valores cada vez más altos de x, el denominador resultará más grande
que el numerador y por lo tanto tenderá a 0.\\

d) Hallar este límite:\\

\[
    \lim_{x\to \infty} \ln \left(e^{\frac{x^2-1}{2x^2-x+1}}\right)
\]

En este caso podemos sacar factor común para simplificar la expresión:


\[
    \ln \left(e^{\frac{x^2-1}{2x^2-x+1}}\right)=
    \ln \left(e^{\frac{x^2\left(1-\frac{1}{x^2}\right)}{x^2\left(2-\frac{1}{x}+\frac{1}{x^2}\right)}}\right)
\]

\[
    \ln \left(e\left(\lim_{x\to \infty} \left(\frac{1-\frac{1}{x^2}}{2-\frac{1}{x}+\frac{1}{x^2}}\right)\right)\right) \implies
    \lim _{x\to \infty \:}\left(\frac{1-\frac{1}{x^2}}{2-\frac{1}{x}+\frac{1}{x^2}}\right)
\]

Entonces:

\[
    \frac{\lim _{x\to \infty \:}\left(1-\frac{1}{x^2}\right)}{\lim _{x\to \infty \:}\left(2-\frac{1}{x}+\frac{1}{x^2}\right)} \implies
\]

\[
    \lim_{x\to\infty\:}\left(1-\frac{1}{x^2}\right)=1
\]

\[
    \lim_{x\to\infty\:}\left(2-\frac{1}{x}+\frac{1}{x^2}\right)=2
\]

Entonces el límite es $\frac{1}{2}$\\

e) Resolver el siguiente límite:

\[
    \lim _{x\to \infty \:}\frac{\sin ^2x}{x^2}
\]

En este caso podemos aplicar Hôpital. Derivamos la expresión del numerador y denominador y aplicamos límites de nuevo:

\[
    \lim _{x\to \:0}\left(\frac{\cos \left(2x\right)\cdot \:2}{2}\right) \implies
    \lim _{x\to \:0}\left(\cos \left(2x\right)\right) =
    \cos \left(2\cdot \:0\right) =
    1
\]


\section*{Problema 2}
Un fabricante de latas de refresco desea optimizar la construcción de las mismas. Asumamos que dichas
latas tienen forma cilíndrica y que la chapa de aluminio con la que se hacen tiene el mismo grosor en
todas las piezas y que este es despreciable a la hora de calcular el volumen. El costo de material
dependerá de la cantidad de chapa de aluminio que se use para confeccionar la lata. Teniendo en
cuenta que las latas tienen que contener 330 $cm^3$ ,cuánto tienen que valer el radio y altura de la lata
para que el coste de material sea mínimo?

En este caso tenemos un problema de optimización. Debemos tener en cuenta el área de las dos tapas circulares y
luego el área radial del cilindro.

\[
    A_b=\pi r^2
\]

\[
    A_L=2\pi r h
\]

Entonces el área total:

\[
    A_t=\pi r^2+2\pi r h
\]

Sabiendo que el volumen del cilindro es $V=Ab \cdot h$ y que éste debe ser de 0.33L:

\[
    \pi r^2=0.33
\]

\[
  h=\frac{0.33}{\pi r^2}  
\]

Entonces nos queda que:

\[
    A_t=2\pi r^2 + 2 \pi r \frac{0.33}{\pi r^2}=
    2\pi r^2 + \frac{0.66}{r}
\]

Derivando el área total obtenemos:

\[
    A'=4\pi r - \frac{0.66}{r^2}
\]

Con esto obtenemos los mínimos. Entonces igualamos a 0:

\[
    4\pi r - \frac{0.66}{r^2}=0 \implies
    r=\sqrt[3]{0.053}
\]

Para encontrar la altura $h$ sustituyo:

\[
    h=\frac{0.33}{\pi r^2} \implies
    h=0.28052 m
\]




\section*{Problema 3}

a) Hallar dominio y asíntotas de la siguiente función:

\[
    f\left(x\right)=\frac{x^2}{\left(x^2+x-2\right)}
\]

El dominio es el conjunto de inputs que hacen que la función sea real y definida. Entonces Los puntos $x$ no definidos serán aquellas
que cumplan que $x^2+x-2=0$. Y en este caso es $x=1$ y $x=-2$. Por tanto el dominio es:\\

\[
    \:\left(-\infty \:,\:-2\right)\cup \left(-2,\:1\right)\cup \left(1,\:\infty \:\right)
)\]

Las asíntotas verticales serán $x=1$ y $x=-2$. Como ambas funciones son del mismo grado, la asíntota horizontal será $y=1$.\\

b) Hallar dominio y asíntotas de:

\[
    f\left(x\right)=\frac{2x^2+x}{\left(x-1\right)}
\]

De la misma forma el valor de $x=1$ hace que la función no esté definida y por tanto es asíntota vertical. Esto hace que el dominio sea:

\[
    \left(-\infty \:,\:1\right)\cup \left(1,\:\infty \:\right)
\]

Como el numerador es un grado superior al denominador, nos encontraremos con una asíntota oblícua. Para encontrarla hacemos la división 
y nos queda que la asíntota oblícua es de la forma:

\[
    y=2x+3
\]


c) Hallar dominio y asíntotas de la función logística:

\[
    f\left(x\right)=ln\:\frac{1}{x}
\]

La funcion de un logaritmo sólo está definida para valores positivos por tanto debe cumplirse que $\frac{1}{x}>0$.

Esto hace que la asíntota vertical esté en $x=0$ y el dominio sea $\left(0,\:\infty \:\right)$. No tiene asíntotas horizontales.\\

d) Estudiar la continuidad de la función

\[
    f\left(x\right)=
    \begin{cases}
        \cos x \ si \ x \le 0 \\
        x+1 \ si \ x>0
    \end{cases}
\]

Para que la funcion sea contínua se debe cumplir que $\lim_{x\to 0} \cos x = \lim_{x \to 0} x+1$. Si los igualamos obtenemos que:

\[
    1=1
\]

Por lo tanto la función es contínua en el punto $(0,1)$.

e) Estudiar el dominio y continuidad de la función f(x).

Evaluamos el dominio para cada función:

\[
    x\le \:1
\]

Que no tiene puntos indefinidos por tanto es su dominio.

\[
    \frac{-1}{x-2}
\]

Para su dominio $x>1$ tiene como puntos no definidos $x=2$ y por tanto su dominio es $\frac{-1}{x-2}$.


De la misma forma igualamos los ímites de las funciones y evaluamos si son iguales.

\[
    \lim_{x\to 1} e^{1-x} = \lim_{x\to 1} \frac{-1}{x-2} \implies
    1 = 1
\]

Entonces la función es contínua.


\section*{Problema 4}

Calcular la ecuación de la recta tangente en el punto $(x, y) = (1, 1)$ a la función $y = y(x)$ definida
implícitamente por la relación $x^2+xy+y^2=3$.\\

Sabemos que la ecuación de la recta tangente es  $y-y_0=m(x-x_0)$ donde $y_0=y(x_0 )$.\\

Ya tenemos el punto de la función, nos faltaría la pendiente $m$. La podremos encontrar derivando implícitamente la expresión
$x^2+xy+y^2=3$.\\

\[
    \frac{d}{dx}\left(x^2+xy+y^2\right)=\frac{d}{dx}\left(3\right)
\]

Nos quedaría que:

\[
    \frac{dy}{dx} = \frac{-2x-y}{x+2y}
\]

Si sustituimos x e y por nuestro punto obtenemos:

\[
    \frac{dy}{dx} = \frac{-2(1)-(1)}{(1)+2(1)}
\]

Y por tanto la pendiente es $m=-1$.


\section*{Problema 5}

Descomponer en fracciones simples las siguientes expresiones:\\

\[
    - \frac{2x^3+3x^2-49x+56}{x^4-5x^3+x^2+21x-1} 
\]

Factorizamos denominador utilizando ruffini:

\[
    x^4-5x^3+x^2+21x-18 = \left(x-1\right)\left(x+2\right)\left(x-3\right)^2
\]

Igualamos la expresión a la nueva expresión que utilizará fracciones simples:

\[
    -\left(\frac{2x^3+3x^2-49x+56}{\left(x-1\right)\left(x+2\right)\left(x-3\right)^2}\right)=-\left(\frac{a}{x-1}+\frac{b}{x+2}+\frac{c}{x-3}+\frac{d}{\left(x-3\right)^2}\right)
\]

Multiplicamos la ecuación por el denominador:

\[
    \frac{\left(2x^3+3x^2-49x+56\right)\left(x-1\right)\left(x+2\right)\left(x-3\right)^2}{\left(x-1\right)\left(x+2\right)\left(x-3\right)^2}=
\]

\[
    \frac{a\left(x-1\right)\left(x+2\right)\left(x-3\right)^2}{x-1}+\frac{b\left(x-1\right)\left(x+2\right)\left(x-3\right)^2}{x+2}+\frac{c\left(x-1\right)\left(x+2\right)\left(x-3\right)^2}{x-3}+\frac{d\left(x-1\right)\left(x+2\right)\left(x-3\right)^2}{\left(x-3\right)^2}
\]

Simplificamos:

\[
    2x^3+3x^2-49x+56=a\left(x+2\right)\left(x-3\right)^2+b\left(x-1\right)\left(x-3\right)^2+c\left(x-1\right)\left(x+2\right)\left(x-3\right)+d\left(x-1\right)\left(x+2\right)
\]

Resolvemos la ecuacion para cada raíz. Para $x=1$:

\[
    2\cdot \:1^3+3\cdot \:1^2-49\cdot \:1+56=a\left(1+2\right)\left(1-3\right)^2+b\left(1-1\right)\left(1-3\right)^2+c\left(1-1\right)\left(1+2\right)\left(1-3\right)+d\left(1-1\right)\left(1+2\right)
\]

Obtenemos $12=12a \implies a=1$.

Para $x=-2$ obtenemos $b=-2$.

Para $x=3$ obtenemos $d=-1$

Finalmente resolvemos con los valoes de $a,b,d$:

\[
    2x^3+3x^2-49x+56=x^3\left(c-1\right)+x^2\left(-2c+9\right)+x\left(-5c-34\right)+\left(6c+38\right)
\]

Agrupamos los elementos por potencia y resolvemos:

\[
    2x^3+3x^2-49x+56=x^3\left(c-1\right)+x^2\left(-2c+9\right)+x\left(-5c-34\right)+\left(6c+38\right)
\]

\[
    6c+38=56 \implies c=3
\]

Y entonces nos quedaría la decomposición de esta forma:

\[
    -\frac{1}{x-1}+\frac{2}{x+2}-\frac{3}{x-3}+\frac{1}{\left(x-3\right)^2}
\]

\newpage

\[
    \frac{3x^4-2x^3-31x^2+60x-16}{x^4-12x^2+16x}
\]

En este caso puesto que ambas expresiones son del mismo grado podemos simplifiar el proceso haciendo directamente
la división.

Nos quedaria entonces la expresiones simplificada:

\[
    3+\frac{-2x^3+5x^2+12x-16}{x^4-12x^2+16x}
\]

A partir de aquí dejamos la constante 3 de lado y procedemos a hacer la decomposición:

\[
    \frac{-2x^3+5x^2+12x-16}{x^4-12x^2+16x}=\frac{-2x^3+5x^2+12x-16}{x\left(x-2\right)^2\left(x+4\right)}=\frac{a}{x}+\frac{b}{x-2}+\frac{c}{\left(x-2\right)^2}+\frac{d}{x+4}
\]

Multipicamos por el denominador y simplificamos:

\[
    \frac{x\left(-2x^3+5x^2+12x-16\right)\left(x-2\right)^2\left(x+4\right)}{x\left(x-2\right)^2\left(x+4\right)}=\frac{ax\left(x-2\right)^2\left(x+4\right)}{x}+\frac{bx\left(x-2\right)^2\left(x+4\right)}{x-2}+\frac{cx\left(x-2\right)^2\left(x+4\right)}{\left(x-2\right)^2}+\frac{dx\left(x-2\right)^2\left(x+4\right)}{x+4}
\]

\[
    2x^3+5x^2+12x-16=a\left(x-2\right)^2\left(x+4\right)+bx\left(x-2\right)\left(x+4\right)+cx\left(x+4\right)+dx\left(x-2\right)^2
\]

Procedemos a solucionar la ecuación para cada raíz y obtenemos:

Para $x=0, \ a=-1$.

Para $x=2, \ c=1$.

Para $x=-4, \ d=-1$.


Entonces solucionamos:

\[
    -2x^3+5x^2+12x-16=\left(-1\right)\left(x-2\right)^2\left(x+4\right)+bx\left(x-2\right)\left(x+4\right)+1\cdot \:x\left(x+4\right)+\left(-1\right)x\left(x-2\right)^2
\]

\[
    -2x^3+5x^2+12x-16=bx^3-2x^3+2bx^2+5x^2-8bx+12x-16
\]

\[
    -2x^3+5x^2+12x-16=x^3\left(b-2\right)+x^2\left(2b+5\right)+x\left(-8b+12\right)-16
\]

\[
    -8b+12=12 \implies b=0
\]

Entonces nos queda:

\[
    -\frac{1}{x}+\frac{1}{\left(x-2\right)^2}-\frac{1}{x+4}
\]

\newpage


\section*{Problema 6}

Resolver la siguiente integral:

\[
    \int \frac{2x^2-17x+29}{x^3-10x^2+31x-30}dx
\]

Aplicamos la regla de la suma de integrales:

\[
    \int \frac{2x^2}{x^3-10x^2+31x-30}dx-\int \frac{17x}{x^3-10x^2+31x-30}dx+\int \frac{29}{x^3-10x^2+31x-30}dx
\]

Resolvemos cada integral por separado:

\[
    \int \frac{2x^2}{x^3-10x^2+31x-30}dx
    =2\cdot \int \frac{x^2}{x^3-10x^2+31x-30}dx =
\]

Simplificamos en fracciones simples como hemos hecho en la actividad anterior y nos queda:


\[
    =2\cdot \int \frac{4}{3\left(x-2\right)}-\frac{9}{2\left(x-3\right)}+\frac{25}{6\left(x-5\right)}dx
\]

Volvemos a aplicar la regla de la suma y resolvemos cada integral por separado:


\[
    \int \frac{4}{3\left(x-2\right)}dx=\frac{4}{3}\cdot \int \frac{1}{x-2}dx
\]

Hacemos un cambio de variable $u=x-2$:

\[
    \frac{4}{3}\cdot \int \frac{1}{u}du =\frac{4}{3}\ln \left|u\right|
\]

Sustituimos u por su valor original.

\[
    =\frac{4}{3}\ln \left|x-2\right|
\]

La siguiente integral:

\[
    \int \frac{9}{2\left(x-3\right)}dx =\frac{9}{2}\cdot \int \frac{1}{x-3}dx =\frac{9}{2}\cdot \int \frac{1}{u}du =\frac{9}{2}\ln \left|u\right|   =\frac{9}{2}\ln \left|x-3\right|
\]

Y la última:

\[
    \int \frac{25}{6\left(x-5\right)}dx =\frac{25}{6}\cdot \int \frac{1}{x-5}dx =\frac{25}{6}\cdot \int \frac{1}{u}du =\frac{25}{6}\ln \left|u\right| =\frac{25}{6}\ln \left|x-5\right|
\]

Sustituimos los resultados en la integral original:

\[
    2\left(\frac{4}{3}\ln \left|x-2\right|-\frac{9}{2}\ln \left|x-3\right|+\frac{25}{6}\ln \left|x-5\right|\right)
\]

Que simplificado quedaría tal que así:

\[
    \frac{8}{3}\ln \left|x-2\right|-9\ln \left|x-3\right|+\frac{25}{3}\ln \left|x-5\right|
\]


Ahora resolvemos:

\[
    \int \frac{17x}{x^3-10x^2+31x-30}dx =17\cdot \int \frac{x}{x^3-10x^2+31x-30}dx
\]

Volvemos a descomponer en fracciones simples y queda:

\[
    =17\cdot \int \frac{2}{3\left(x-2\right)}-\frac{3}{2\left(x-3\right)}+\frac{5}{6\left(x-5\right)}dx
\]

Aplicamos regla de la suma y resolvemos cada integral por separado (los pasos intermedios son iguales a la anterior integral):

\[
    =17\left(\int \frac{2}{3\left(x-2\right)}dx-\int \frac{3}{2\left(x-3\right)}dx+\int \frac{5}{6\left(x-5\right)}dx\right)
\]

\[
    \int \frac{2}{3\left(x-2\right)}dx=\frac{2}{3}\ln \left|x-2\right|
\]

\[
    \int \frac{3}{2\left(x-3\right)}dx=\frac{3}{2}\ln \left|x-3\right|
\]

\[
    \int \frac{5}{6\left(x-5\right)}dx=\frac{5}{6}\ln \left|x-5\right|
\]

Y nos quedará, simplificada:

\[
    17\left(\frac{2}{3}\ln \left|x-2\right|-\frac{3}{2}\ln \left|x-3\right|+\frac{5}{6}\ln \left|x-5\right|\right)
\]

Atacamos la última integral:

\[
    \int \frac{29}{x^3-10x^2+31x-30}dx =29\cdot \int \frac{1}{x^3-10x^2+31x-30}dx
\]

Descomponemos y resolvemos cada una por separado:

\[
    =29\cdot \int \frac{1}{3\left(x-2\right)}-\frac{1}{2\left(x-3\right)}+\frac{1}{6\left(x-5\right)}dx==29\left(\int \frac{1}{3\left(x-2\right)}dx-\int \frac{1}{2\left(x-3\right)}dx+\int \frac{1}{6\left(x-5\right)}dx\right)
\]

\[
    \int \frac{1}{3\left(x-2\right)}dx=\frac{1}{3}\ln \left|x-2\right|
\]

\[
    \int \frac{1}{2\left(x-3\right)}dx=\frac{1}{2}\ln \left|x-3\right|
\]

\[
    \int \frac{1}{6\left(x-5\right)}dx=\frac{1}{6}\ln \left|x-5\right|
\]

Obtenemos:

\[
    29\left(\frac{1}{3}\ln \left|x-2\right|-\frac{1}{2}\ln \left|x-3\right|+\frac{1}{6}\ln \left|x-5\right|\right)
\]


Ya tenemos resueltas las 3 integrales iniciales y podemos decir que, tras simplificar la expresión, obtenemos:\\

\[
    \int \frac{2x^2-17x+29}{x^3-10x^2+31x-30}dx=\ln \left|x-2\right|-\ln \left|x-5\right|+2\ln \left|x-3\right|+C
\]


\section*{Problema 7}

Usar el método de integración por partes para hallar la primitiva de:

\[
    \int \:arcsin\:x\:dx
\]


Aplicamos la definición de integración por partes.

\[
    \int \:uv'=uv-\int \:u'v
\]

Donde $u=\arcsin \left(x\right)$ y $v'=1$

\[
    \frac{d}{dx}\left(\arcsin \left(x\right)\right)= \frac{1}{\sqrt{1-x^2}}
\]

\[
    \int \:1dx=x
\]

Nos queda:

\[
    =\arcsin \left(x\right)x-\int \frac{1}{\sqrt{1-x^2}}xdx
\]

Resolvemos 

\[
    \int \frac{x}{\sqrt{1-x^2}}dx
\]

Aplicamos cambio de variable $u=1-x^2$:

\[
    \int \:-\frac{1}{2\sqrt{u}}du =-\frac{1}{2}\cdot \int \frac{1}{\sqrt{u}}du =-\frac{1}{2}\cdot \int \frac{1}{u^{\frac{1}{2}}}du =-\frac{1}{2}\cdot \int \:u^{-\frac{1}{2}}du
    =-\frac{1}{2}\cdot \frac{u^{-\frac{1}{2}+1}}{-\frac{1}{2}+1}
\]

Sustituimos $u$ por su valor original:

\[
    =-\frac{1}{2}\cdot \frac{\left(1-x^2\right)^{-\frac{1}{2}+1}}{-\frac{1}{2}+1}=-\sqrt{1-x^2}
\]

Entonces la solución sería:

\[
    x\arcsin \left(x\right)+\sqrt{1-x^2}
\]

Y por tanto, añadiendo la constante de integración obtenemos la primitiva:

\[
    x\arcsin \left(x\right)+\sqrt{1-x^2}+C
\]


\section*{Problema 8}

Hallar la integral:

\[
    \int \:cos^4x\:sin^3x\:dx
\]

En este caso podemos resolver la integral utilizando las propiedades trigonométricas:


\[
    \int \:cos^4x\:sin^3x\:dx =\int \cos ^4\left(x\right)\sin ^2\left(x\right)\sin \left(x\right)dx
\]

Sabemos que $\sin ^2\left(x\right)=1-\cos ^2\left(x\right)$ entonces:

\[
    =\int \left(1-\cos ^2\left(x\right)\right)\sin \left(x\right)\cos ^4\left(x\right)dx
\]

Hacemos un cambio de variable: $u=\cos(x)$

\[
    \frac{d}{dx}\left(\cos \left(x\right)\right) = -\sin \left(x\right)
\]

\[
    du=-\sin \left(x\right)dx
\]

\[
    dx=\left(-\frac{1}{\sin \left(x\right)}\right)du
\]

\[
    =\int \left(1-u^2\right)\sin \left(x\right)u^4\left(-\frac{1}{\sin \left(x\right)}\right)du
\]

Simplificamos la expresión:

\[
    \left(1-u^2\right)\sin \left(x\right)u^4\left(-\frac{1}{\sin \left(x\right)}\right) =-\left(1-u^2\right)\sin \left(x\right)u^4\frac{1}{\sin \left(x\right)}
    =-\frac{1\cdot \left(1-u^2\right)\sin \left(x\right)u^4}{\sin \left(x\right)}
    =-u^4\left(-u^2+1\right)
\]

Aplicamos propiedad distributiva:

\[
    \int \:-u^4+u^6du
\]

Separamos en dos integrales y resolvemos cada una:

\[
    -\int \:u^4du+\int \:u^6du
\]

\[
    \int \:u^4du=\frac{u^5}{5}
\]

\[
    \int \:u^6du=\frac{u^7}{7}
\]

No debemos olvidar de sustituir u por su valor original:

\[
    \int \:cos^4x\:sin^3x\:dx=-\frac{\cos ^5\left(x\right)}{5}+\frac{\cos ^7\left(x\right)}{7}+C
\]



\section*{Problema 9}

Calcular, si es posible, la siguiente integral:

\[
    \int _1^{\infty \:}\frac{\ln \left(x\right)}{x^3}dx
\]

Procedemos a calcular la integral indefinida:

\[
    \int \frac{\ln \left(x\right)}{x^3}dx
\]

Aplicamos integración por partes: $\int \:uv'=uv-\int \:u'v$.

\[
    u=\ln \left(x\right)
\]

\[
    v'=\frac{1}{x^3} 
\]

\[
    u'=\frac{d}{dx}\left(\ln \left(x\right)\right)=\frac{1}{x}
\]

\[
    v=\int \frac{1}{x^3}dx=-\frac{1}{2x^2}
\]

Obtenemos finalmente:

\[
    \ln \left(x\right)\left(-\frac{1}{2x^2}\right)-\int \frac{1}{x}\left(-\frac{1}{2x^2}\right)dx
\]

Que si simplificamos es:

\[
    -\frac{\ln \left(x\right)}{2x^2}-\int \:-\frac{1}{2x^3}dx
\]

Resolvemos la integral:

\[
    \int \:-\frac{1}{2x^3}dx=\frac{1}{4x^2}
\]

Obtenemos que la integral indefinida sería:

\[
    \frac{\ln \left(x\right)}{2x^2}-\frac{1}{4x^2}+C
\]

Ahora debemos calcularla teniendo en cuenta sus límites de integración:

\[
    \lim _{x\to \:1+}\left(-\frac{\ln \left(x\right)}{2x^2}-\frac{1}{4x^2}\right)=-\frac{1}{4}
\]

Aplicando Hôpital obtenemos:

\[
    \lim _{x\to \infty \:}\left(-\frac{\ln \left(x\right)}{2x^2}-\frac{1}{4x^2}\right)=0
\]

Por tanto:

\[
    \int _1^{\infty \:}\frac{\ln \left(x\right)}{x^3}dx=0-\left(-\frac{1}{4}\right)=\frac{1}{4}
\]


\section*{Problema 10}

a) Calcula el desarrollo de Taylor en el origen de la función hasta el término de $x^2$:

\[
    f\left(x\right)=\frac{\sqrt{x+1}}{x+1}
\]

Por definición, la serie de Taylor tiene la forma:

\[
    f\left(x\right)=f\left(a\right)+\frac{f'\left(a\right)}{1!}\left(x-a\right)+\frac{f''\left(a\right)}{2!}\left(x-a\right)^2+\frac{f'''\left(a\right)}{3!}\left(x-a\right)^3+\ldots 
\]

Tomamos $x=0$:

\[
    f\left(0\right)=  1
\]

Procedemos a encontrar $f'$ y $f''$ ya que el enunciado nos pide llegar hasta el término $x^2$. 

\[
    =1+\frac{\frac{d}{dx}\left(\frac{\sqrt{x+1}}{x+1}\right)\left(0\right)}{1!}x+\frac{\frac{d^2}{dx^2}\left(\frac{\sqrt{x+1}}{x+1}\right)\left(0\right)}{2!}x^2
\]

Evaluamos siempre en el punto $x=0$:

\[
    \frac{d}{dx}\left(\frac{\sqrt{x+1}}{x+1}\right)=-\frac{1}{2\left(x+1\right)^{\frac{3}{2}}}=-\frac{1}{2}
\]

\[
    \frac{d^2}{dx^2}\left(\frac{\sqrt{x+1}}{x+1}\right)=\frac{3}{4\left(x+1\right)^{\frac{5}{2}}} =\frac{3}{4}
\]

Entonces podemos montar la série:
\[
    1+\frac{-\frac{1}{2}}{1!}x+\frac{\frac{3}{4}}{2!}x^2+ \ldots
\]

Que simplificada queda:

\[
    1-\frac{1}{2}x+\frac{3}{8}x^2+ \ldots
\]


b) Estudiar el carácter de la serie que tiene como término general el siguiente:

\[
    a_n=\frac{\left(n+1\right)^n}{n!3^n}
\]

Estudiaremos la convergencia de la série:

\[
    \sum _{n=0}^{\infty }\:\frac{\left(n+1\right)^n}{n!3^n}
\]


Para ello podemos hacer uso del criterio de d'Alembert:

Entonces si existe una $N$ que para todo $n\ge N, \ a_n \ \neq 0$ y que $\lim _{n\to \infty }|\frac{a_{n+1}}{a_n}|=L$ se puede dar uno de estos casos:\\


Si $L<1$, la serie converge.\\

Si $L>1$, la serie diverge.\\

Si $L=1$, hay que calcular el límite mediante otra forma.\\

Entonces expresamos la serie de la forma:

\[
    \left|\frac{a_{n+1}}{a_n}\right|=\left|\frac{\frac{\left(\left(n+1\right)+1\right)^{\left(n+1\right)}}{\left(n+1\right)!3^{\left(n+1\right)}}}{\frac{\left(n+1\right)^n}{n!3^n}}\right|
\]

\[
    =\left|\frac{\frac{\left(\left(n+1\right)+1\right)^{n+1}}{\left(n+1\right)!\cdot \:3^{n+1}}}{\frac{\left(n+1\right)^n}{n!\cdot \:3^n}}\right|
\]

\[
    =\left|\frac{\frac{\left(n+2\right)^{n+1}}{\left(n+1\right)!\cdot \:3^{n+1}}}{\frac{\left(n+1\right)^n}{n!\cdot \:3^n}}\right|
\]

\[
    =\left|\frac{\left(n+2\right)^{n+1}n!\cdot \:3^n}{\left(n+1\right)!\cdot \:3^{n+1}\left(n+1\right)^n}\right|
\]

\[
    =\frac{\left(n+2\right)^{n+1}}{3\left(n+1\right)\left(n+1\right)^n}
\]

\[
    =\frac{\left|\left(n+2\right)^{n+1}\right|}{3\left|\left(n+1\right)^{1+n}\right|}
\]

Ahora podemos evaluar el límite:

\[
    \lim _{n\to \infty \:}\left(\frac{\left|\left(n+2\right)^{n+1}\right|}{3\left|\left(n+1\right)^{1+n}\right|}\right)
\]

\[
    =\lim _{n\to \infty \:}\left(\frac{\left(n+2\right)^{n+1}}{3\left(n+1\right)^{1+n}}\right)
\]

\[
    =\frac{1}{3}\cdot \lim _{n\to \infty \:}\left(\frac{\left(n+2\right)^{n+1}}{\left(n+1\right)^{1+n}}\right)
\]

\[
    =\frac{1}{3}\cdot \lim _{n\to \infty \:}\left(\left(\frac{n+2}{n+1}\right)^{n+1}\right)
\]

\[
    \frac{1}{3}\cdot \lim _{n\to \infty \:}\left(e^{\left(n+1\right)\ln \left(\frac{n+2}{n+1}\right)}\right)
\]

Aplicando la regla de la cadena nos queda:

\[
    =\lim _{n\to \infty \:}\left(\frac{\ln \left(\frac{2+n}{1+n}\right)}{\frac{1}{1+n}}\right)
\]

Aplicamos Hôpital:

\[
    =\lim _{n\to \infty \:}\left(\frac{\left(\ln \left(\frac{2+n}{1+n}\right)\right)^{'\:}}{\left(\frac{1}{\left(1+n\right)}\right)^{'\:}}\right)
\]

\[
    \left(\ln \left(\frac{2+n}{1+n}\right)\right)'=-\frac{1}{\left(n+2\right)\left(n+1\right)}
\]

\[
    \left(\frac{1}{\left(1+n\right)}\right)'=-\frac{1}{\left(1+n\right)^2} \implies
\]

\[
    \lim _{n\to \infty \:}\left(\frac{-\frac{1}{\left(n+2\right)\left(n+1\right)}}{-\frac{1}{\left(1+n\right)^2}}\right)
\]

\[
    =\lim _{n\to \infty \:}\left(\frac{1+n}{n+2}\right)
\]

Volvemos a aplicar Hôpital:

\[
    \lim _{n\to \infty \:}\left(\frac{\left(1+n\right)^{'\:}}{\left(n+2\right)^{'\:}}\right)
\]

\[
    \left(1+n\right)'=1
\]

\[
    \left(n+2\right)'=1 \implies
\]

\[
    =\lim _{n\to \infty \:}\left(\frac{1}{1}\right)=1
\]

Aplicando la regla de exponentes:

\[
    \frac{1}{3}\cdot \lim _{n\to \infty \:}\left(e^{\left(n+1\right)\ln \left(\frac{n+2}{n+1}\right)}\right)
\]

Y la regla de la cadena:


\[
    L = \frac{1}{3}e > 1
\]

Por lo tanto podemos decir que la serie \textbf{Converge}




\end{document}