\documentclass{article}
\usepackage{lipsum}
\usepackage[backend=biber]{biblatex}
\addbibresource{aec2.bib}
\usepackage{authoraftertitle}
\usepackage[top=2cm,bottom=1.5cm,left=1.5cm, right=3cm,includeheadfoot]{geometry}
\usepackage{graphicx}
\usepackage{fancyhdr}
\usepackage[spanish]{babel}
\usepackage{mathtools}
\usepackage{nicefrac}
\usepackage{csquotes}
\usepackage{amssymb}
\usepackage{fancybox, graphicx}
\usepackage{array}
\usepackage{hhline}
\usepackage{hyperref}
\usepackage{tikz}
\usepackage{amsmath}
\usepackage{wrapfig}
\usepackage{float}
\usepackage{amsmath}
\usepackage{esint}
\usepackage{caption}
\usepackage{esvect}
\usepackage{siunitx}
\usepackage{commath}
\newcommand{\ihat}{\textbf{\^\i}}
\newcommand{\jhat}{\textbf{\^\j}}
%Header & Footer

\pagestyle{fancy}
%\fancyhead[LE]{\MyTitle}
\fancyhead[LO]{1506: Fundamentos Matemáticos}
%\fancyhead[RO]{\leftmark}
%\fancyhead[RE]{\leftmark}
%\fancyfoot[L]{\raisebox{-1cm}{\includegraphics[height=1.5cm]{E:/KUKADisk/UDIMA/DocumentGraphics/LOGOUDIMA.jpg}}}
\fancyfoot[R]{Corregido:\\ Dr. Juan José Moreno García}
%\fancyfoot[RO]{07/12/2018}


%Vars
\author{Alexander Sebastian Kalis}
\title{Actividad de Evaluación Continua 2}


%DOC


\begin{document}

\begin{titlepage}

    \begin{center}

        \line(1,0){300}\\
        [0.2in]
        \huge{\bfseries {\MyTitle}}\\
        [1mm]
        \line(2,0){200}\\
        [0.75cm]
        \textsc{\LARGE 1506: Fundamentos Matemáticos}\\
        [2cm]
        %\includegraphics[height=10cm]{E:/KUKADisk/UDIMA/Calculo/2020/Practicas/portada.png}\\
        [3cm]

    \end{center}

    \begin{flushright}

        Autor: {\MyAuthor}\\
        Profesor: Dr. Juan José Moreno García\\
        Curso: Ingeniería de Organización Industrial\\
        UDIMA         

    \end{flushright}
    
\end{titlepage}

\section*{Problema 1}

Hallar los puntos críticos, y determinar su naturaleza, de la función:

\[
    f(x,y)=ln(x^2+y^2+1)
\]

Para hallar los puntos críticos	debemos primero encontrar las derivadas parciales de la función, tanto las de 
primer orden como las de segundo orden.

\[
    f_{x}(x,y)=
    \frac{\partial \:}{\partial \:x}\left(\ln \left(x^2+y^2+1\right)\right)=
    \frac{\partial \:}{\partial \:u}\left(\ln \left(u\right)\right)\frac{\partial \:}{\partial \:x}\left(x^2+y^2+1\right)= 
    \frac{1}{x^2+y^2+1}\frac{\partial \:}{\partial \:x}\left(x^2+y^2+1\right)=
    \frac{2x}{x^2+y^2+1}
\]

Y por tanto

\[
    f_y(x,y)=
    \frac{\partial \:}{\partial \:y}\left(\ln \left(x^2+y^2+1\right)\right)=
    \frac{2y}{x^2+y^2+1}
\]

Encontramos las derivadas parciales de segundo orden:

\[
    f_{xx}(x,y)=
    \frac{\partial }{\partial \:x\:}\left(\frac{2x}{x^2+y^2+1}\right)=
    2\frac{\partial \:}{\partial \:x}\left(\frac{x}{x^2+y^2+1}\right)=
    2\cdot \frac{\frac{\partial \:}{\partial \:x}\left(x\right)\left(x^2+y^2+1\right)-\frac{\partial \:}{\partial \:x}\left(x^2+y^2+1\right)x}{\left(x^2+y^2+1\right)^2}=
\]
\[
    2\cdot \frac{1\cdot \left(x^2+y^2+1\right)-2xx}{\left(x^2+y^2+1\right)^2}=
    \frac{2\left(-x^2+y^2+1\right)}{\left(x^2+y^2+1\right)^2}
\]

Y por tanto

\[
    f_{yy}(x,y)=
    \frac{\partial \:}{\partial \:y}\left(\frac{2y}{x^2+y^2+1}\right)=
    \frac{2\left(-y^2+x^2+1\right)}{\left(x^2+y^2+1\right)^2}
\]

Por último la derivada cruzada

\[
    f_{xy}(x,y)=\frac{\partial }{\partial \:y\:}\left(\frac{2x}{x^2+y^2+1}\right)=
    2x\frac{\partial \:}{\partial \:y}\left(\frac{1}{x^2+y^2+1}\right)=
    2x\frac{\partial \:}{\partial \:y}\left(\left(x^2+y^2+1\right)^{-1}\right)=
    -\frac{1}{\left(x^2+y^2+1\right)^2}\frac{\partial \:}{\partial \:y}\left(x^2+y^2+1\right)=
\]

\[
    =2x\left(-\frac{1}{\left(x^2+y^2+1\right)^2}\cdot \:2y\right)=
    -\frac{4xy}{\left(x^2+y^2+1\right)^2}
\]


Para lograr los puntos críticos, igualamos a 0 las derivadas parciales primeras y lo resolvemos como un sistema:

\[
    \begin{dcases}
        \dfrac{2y}{x^2+y^2+1} = 0\\
        \dfrac{2x}{x^2+y^2+1} = 0
    \end{dcases}
    \implies
    x=0, y=0
\]

Entonces el punto $(0,0)$ es el único punto crítico. Para saber de qué tipo de punto se trata, computamos el discriminante:

\[
    D(f(x,y))=f_{xx}f_{yy}-f^2_{xy} \implies
    D(f(0,0))=0
\]

Esto significa que no podemos clasificar el punto.

\newpage


\section*{Problema 2}


Halla los máximos y mínimos de la función $f(x,y)=x^2-y^2$ condicionado al círculo de radio 1
centrado en el origen.

Para resolver este problema disponemos del método de multiplicadores de Lagrange.

\[
    \begin{array}{lcl}
        f_x(x,y)=\lambda g_x(x,y)\\
        f_y(x,y)=\lambda g_y(x,y)\\
        g(x,y)=0
    \end{array}
\]

Donde

\[
    f_x(x,y)=\frac{\partial }{\partial \:x}\left(x^2-y^2\right)=
    2x
\]

\[
    f_y(x,y)=\frac{\partial }{\partial \:y}\left(x^2-y^2\right)=
    -2y
\]

\[
    D=g(x,y)=x^2+y^2-1=0
\]

\[
    g_x(x,y)=
    \frac{\partial }{\partial \:x}\left(x^2+y^2-1\right)=
    2x
\]

\[
    g_y(x,y)=
    \frac{\partial }{\partial \:y}\left(x^2+y^2-1\right)=
    2y
\]

Teniendo todos los componentes podemos montar el sistema de ecuaciones:

\[
    \begin{dcases}
        2x=\lambda 2x\\
        -2y=\lambda 2y\\
        x^2+y^2-1=0
    \end{dcases}
    \implies
    \begin{dcases}
        x=-1,y=0\\
        x=0,y=-1\\
        x=0,y=1\\
        x=1,y=0
    \end{dcases}
\]

Se evalúan los puntos en la función:

\[
    f(-1,0)=1 \ 
    f(0,-1)=-1 \
    f(0,1)=-1 \
    f(1,0)=1 \
\]

Lo que significa que el máximo de la función se encontrará en los puntos (-1,0) (1,0) y sus mínimos
en (0,-1) (0,1).

%\newpage

\section*{Problema 3}

Sea $R$ el rectángulo $[-2,1]\times[0, 1]$ y sea la función $f(x, y) = y(x^3 - 12x)$. Calcular la siguiente integral:

\[
    I=\int \int_R f(x,y)dx \ dy
\]

El enunciado nos proporciona el área cerrada del rectángulo siendo $[-2,1]$ el intervalo de integración en $x$
y $[0,1]$ el intervalo de integración en $y$:

\[
    \int_0^1 \int_{-2}^1 y(x^3-13x)  \ dx \ dy=
    \int_0^1 I_x \ dy
\]
Integramos $I_x$:

\[
    I_x=\int_{-2}^1 y(x^3-13x) \ dx=
    y \int_{-2}^1 (x^3-13x) \ dx=
    y \left(\int _{-2}^1x^3 \ dx-\int _{-2}^1 13x \ dx\right)=
\]
\[
    =y\left(\left[\frac{x^4}{4}\right]^1_{-2}-13\left[\frac{x^2}{2}\right]^1_{-2}\right)=
    y\left(-\frac{15}{4}-\left(-\frac{39}{2}\right)\right)=
    \frac{63}{4}y
\]
Integramos $I$:

\[
    \int_0^1 I_x \ dy=
    \int_0^1 \frac{63}{4}y \ dy=
    \frac{63}{4}\left[\frac{y^2}{2}\right]^1_0=
    \frac{63}{8}
\]

\section*{Problema 4}

Calcular la integral

\[
    \iint\limits_D (3-x-y) \ dA
\]

En donde D es la región en forma triangular encerrada por el eje x y las líneas $y = x$ y $x = 1$.\\

Podemos entonces interpretar que el valor de la integral será la mitad del área del cuadrado $[0,1]\times[0,1]$.

\[
    \frac{1}{2}\left[\int_0^1 \int_0^1 (3-x-y) \ dx \ dy\right]=
    \frac{1}{2}\left[\int_0^1 3-y-\frac{1}{2} \ dy\right]=
    \frac{1}{2}\left[ 3-\frac{1}{2}-\frac{1}{2}\right]=
    1
\]

O lo que sería lo mismo, integrar utilizando como límite de integración la recta $y=x$:

\[
    \int_0^1 \int_0^x (3-x-y) \ dy \ dx=
    1
\]

\section*{Problema 5}

Usando el cambio a coordenadas esféricas, calcular la integral

\[
    I=\iiint\limits_\Omega \frac{1}{\left[1+\left(x^2+y^2+z^2\right)^\frac{3}{2}\right]^\frac{3}{2}} dx \ dy \ dz,     
\]

En donde $\Omega$ es todo $\mathbb{R}^3$.


Empezamos haciendo el cambio a coordenadas esféricas, sabiendo que $x^2+y^2+z^2=\rho^2$,

\[
    \Omega=\left\{(\theta,\phi,\rho):0\leq\theta<2\pi,0\leq\phi\leq\pi,0\leq\rho<\infty\right\}
\]

\[
    \int_0^\infty\int_0^\pi\int_0^{2\pi} \frac{\rho^2 \sin \phi}{\left[1+[{\rho^2}]^\frac{3}{2}\right]^\frac{3}{2}} \ d\theta \ d\phi \ d\rho=
    \int_0^\infty\int_0^\pi\int_0^{2\pi} \frac{\rho^2 \sin \phi}{\left[1+{\rho^3}\right]^\frac{3}{2}} \ d\theta \ d\phi \ d\rho=
    \int_0^\infty\int_0^\pi \frac{2\pi \rho^2 \sin \phi }{\left(1+\rho^3\right)^{\frac{3}{2}}}   \ d\phi \ d\rho =
\]
\[
    =\int_0^\infty\frac{2\pi \rho^2}{\left(1+\rho^3\right)^{\frac{3}{2}}}\left[-\cos \phi\right]^{\pi }_0 \ d\rho=
    \int_0^\infty \frac{4\pi \rho^2}{\left(1+\rho^3\right)^{\frac{3}{2}}}\ d\rho=
    4\pi \frac{1}{3}\cdot \int\frac{1}{\left(1+u\right)^{\frac{3}{2}}} \ du=
    4\pi \frac{1}{3}\cdot \int \frac{1}{v^{\frac{3}{2}}} \ dv=
\]
\[
    =4\pi \frac{1}{3}\cdot \int \:v^{-\frac{3}{2}} \ dv=
    4\pi \frac{1}{3}\cdot \frac{\left(1+\rho^3\right)^{-\frac{3}{2}+1}}{-\frac{3}{2}+1}=
    \left[-\frac{8\pi }{3\left(1+\rho^3\right)^{\frac{1}{2}}}\right]_0^\infty=
    \frac{8\pi }{3}
\]


\section*{Problema 6}

Calcular la integral

\[
    I=\iiint\limits_\Omega(x^2+y^2) \ dx \ dy \ dz.
\]
En donde $\Omega$ la región limitada por un cono recto de revolución centrado
en el origen de altura $h = 2$, cuya base se asienta en el plano $XY$ y cuyo
radio de la base es $r = 1$. La ecuación del cono será entonces

\[
    a^2(h-z)^2=h^2(x^2+y^2),
\]
siendo $a=1$ al ser el radio de la base del cono y nuestro caso $h=2$.\\

En este caso se resolverá el problema aplicando una conversión a coordenadas cilíndricas:

\[
    \iiint\limits_\Omega \rho^3 \ dV
\]

Ahora evaluamos los límites de integración.
\begin{itemize}
    \item $\theta$ Va de $0$ a $2\pi$, pues el cono completa la vuelta al eje.
    \item $\rho$ Va de $0$ hasta el radio máximo, $1$.
    \item $z$ Empieza desde el plano XY y su límite superior (cono inferior) es el cono $z=2-2\rho$ .
\end{itemize}

\[
    \int_0^{2\pi} \int_0^1 \int_0^{2-2\rho} \rho^3 \ dz \ d\rho \ d\theta=
    \int_0^{2\pi} \int_0^1  \rho^3\left(-2\rho+2\right)  \ d\rho \ d\theta=
    \int_0^{2\pi} 2\int _0^1-\rho^4+\rho^3 \ d\rho \ d\theta=
    \int _0^{2\pi}\frac{1}{10} \ d\theta=
    \frac{\pi}{5}
\]

\section*{Problema 7}

Evaluar la integral de linea

\[
    \int_C \left(x^3y+\frac{y^3x}{3}\right) \ dx + x^2 dy
\]

en donde la trayectoria $C$ es el contorno de la región definida por el círculo $x^2 + y^2 - 2y = 0$ e $y > 1$.
Nota: la figura no es más que un círculo centrado en el punto $(0,1)$, así que cuando $y > 1$ lo que estamos
definiendo es el semicírculo superior. Por tanto, la parametrización para el diámetro será $(t, 1)$ y para
el semicírculo $(\cos t, 1 + \sin t)$ (al que llamaremos $C_2$).\\


Ya que se trata de una integral de curva simple cerrada podemos utilizar el Teorema de Green para simplificar la integral.

\[
    \oint P \ dx + Q \ dy = \iint\limits_R \left[\frac{\partial Q}{\partial x}-\frac{\partial P}{\partial y}  \right] \ dA
\]

Nos quedaría entonces:

\[
    \iint\limits_R -xy^2-x^3+2x \ dA =
    \int_{-1}^1 \int_1^{1+\sqrt{1-x^2}} -xy^2-x^3+2x \ dy \ dx=
\]

\[
    \int_{-1}^1 \left(-\int _1^{1+\sqrt{1-x^2}}x^3dy+\int _1^{1+\sqrt{1-x^2}}2xdy-\int _1^{1+\sqrt{1-x^2}}xy^2dy\right) \ dx=
\]
\[
    \int_{-1}^1  \left(\frac{1}{3}x\left(1-\left(\sqrt{1-x^2}+1\right)^3\right)-x\sqrt{1-x^2}(x^2-2)\right) \ dx=
\]
\[
    \int_{-1}^1 \left(-\frac{1}{3}\left(x^2-1\right)\left(2\sqrt{1-x^2}-3\right)\right) \ dx
\]

Puesto que $f(x)=-\frac{1}{3}\left(x^2-1\right)\left(2\sqrt{1-x^2}-3\right)$ es una función impar,

\[
    \int_{-1}^1 \left(-\frac{1}{3}\left(x^2-1\right)\left(2\sqrt{1-x^2}-3\right)\right) \ dx = 0
\]



\section*{Problema 8}

Usar el teorema de Green para calcular la integral

\[
    \int_C y^2 \ dx+ 3xy \ dy
\]

De nuevo aplicamos Green y ya que es una región similar al problema 7, los límites de integración también lo serán
En este caso la circunferencia está centrada en el origen XY y su radio también es 1:

\[
    \oint P \ dx + Q \ dy = \iint\limits_R \left[\frac{\partial Q}{\partial x}-\frac{\partial P}{\partial y}  \right] \ dA
\]

Nos queda:

\[
    \iint\limits_R y \ dA=
    \int_{-1}^1 \int_0^{\sqrt{1-x^2}} y \ dy \ dx=
    \int_{-1}^1 \frac{1-x^2}{2} \ dx=
    \frac{1}{2}\left(\int _{-1}^11dx -\int _{-1}^1x^2dx \right)=
    \frac{2}{3}
\]


\section*{Problema 9}

Resolver el siguiente problema de valores iniciales:

\[
    y''-7y'+6y=0,\:y\left(0\right)=0,\:y'\left(0\right)=5
\]

Es una ecuación homogénea de coeficientes constantes y polinomio característico:

\[
    \lambda^2-7\lambda+6=0
\]

Lo que nos da las raíces $\lambda_1=1$,$\lambda_2=6$ que son soluciones reales y distintas y por lo tanto podemos escribir:

\[
    y=c_1e^{t}+c_2e^{6t}, y'=c_1e^t+c_2e^{6t}
\]

\[
    \begin{cases}
        0=c_1+c_2\\
        5=c_1+6c_2
    \end{cases}
    \implies
    \begin{array}{l}
        c_1=-1\\
        c_2=1
    \end{array}
\]

Sustituimos los valores y obtenemos:

\[
    y=e^{6t}-e^t
\]



\section*{Problema 10}

Resolver el siguiente problema de valores iniciales:

\[
    y''+4y'+3y=\frac{2}{e^t},y(0)=1,y'(0)=2
\]

En este caso es una ecuación no homogénea con coeficientes constantes.

Resolvemos la ecuación correspondiente homogénea:

\[
    y''+4y'+3y=0
\]

Con polinomio característico

\[
    \lambda^2+4\lambda+3=0
\]

Con raíces $\lambda_1=-3, \lambda_2=-1$. Entonces la solución a la homogénea es:

\[
    y_h=c_1e^{-3t}+c_2e^{-t}
\]

Utilizando el método de coeficientes indeterminados podemos encontrar la solución particular.

\[
    \frac{d^2y}{dt^2}+4\frac{dy}{dt}+3y=2e^{-t}
    \implies
    y_p=t(k_1e^{-t})
\]

Calculamos las derivadas:

\[
    \frac{d y}{dt}=k_1e^{-t}-k_1e^{-t}t
\]

\[
    \frac{d^2 y}{dt^2}=k_1\left(-2e^{-t}+e^{-t}t\right)
\]

Ahora podemos sustituir los valores en nuestra ecuación no homogénea:

\[
    k_1\left(-2e^{-t}+e^{-t}t\right)+4\left( k_1e^{-t}-k_1e^{-t}t \right)+3(k_1e^{-t}t)=2e^{-t}
\]

Resolviendo la ecuación obtenemos $k_1=1$ y por tanto $y_p=e^{-t}t$.

La solución general será:

\[
    y=y_h+y_p \implies
    y=c_1e^{-3t}+c_2e^{-t}+e^{-t}t,
    y'=e^{-t}-e^{-t}t-3c_1e^{-3t}-c_2e^{-t}
\]

Conociendo los datos de las condiciones iniciales obtenemos las constantes:



\[
    \begin{cases}
        c_1+c_2=1\\
        -3c_1-c_2+1=2
    \end{cases}
    \implies
    \begin{array}{l}
        c_1=-1\\
        c_2=2
    \end{array}
\]

Y finalmente podemos escribir:

\[
    y=-e^{-3t}+2e^{-t}+e^{-t}t
\]


\newpage

\section*{Problema 11}

Resolver, usando transformada de Laplace, la siguiente ecuación diferencial:

\[
    y''+9y=0, y(0)=1, y'(0)=9
\]

Aplicamos la transformada:


\[
     \mathcal{L}\left[f^{''}\left(t\right)\right]=s^2\mathcal{L}\left[f\left(t\right)\right]-sf\left(0\right)-f^{'}\left(0\right)
\]

\[
    \mathcal{L}\left[y''+9y\right]=\mathcal{L}\left[0\right]
\]
\[
    s^2\mathcal{L}\left[y\right]-sy\left(0\right)-y'\left(0\right)+9\mathcal{L}\left[y\right]=0 
\]

Insertamos los valores iniciales:

\[
    s^2\mathcal{L}\left[y\right]-s-9+9\mathcal{L}\left[y\right]=0
\]

Resolviendo para $\mathcal{L}\left[y\right]$ obtenemos:

\[
    \mathcal{L}\left[y\right]=\frac{s+9}{s^2+9}
\]

Aplicamos la inversa de la transformada de Laplace y reescribimos las fracciones de tal forma que podamos utilizar
la tabla de transformadas de Laplace:

\[
    y=\mathcal{L}^{-1}\left[\frac{s+9}{s^2+9}\right]=\mathcal{L}^{-1}\left[\frac{s}{s^2+3^2}+3\frac{3}{s^2+3^2}\right]
\]


\[
    y=\cos \left(3t\right)+3\sin \left(3t\right)
\]


\section*{Problema 12}

Resolver la siguiente ecuación diferencial por transformada de Laplace:

\[
    y''-3y'-10y=12e^t, y(0)=2,y'(0)=7
\]

Aplicamos Laplace:

\[
    \mathcal{L}\left[y''\:-3y'\:-10y\right]=\mathcal{L}\left[12e^t\right]
\]

\[
    s^2\mathcal{L}\left[y\right]-sy\left(0\right)-y'\left(0\right)-3\left(s\mathcal{L}\left[y\right]-y\left(0\right)\right)-10\mathcal{L}\left[y\right]=\frac{12}{s-1}
\]

Insertamos las condiciones iniciales y simplificamos:

\[
    s^2\mathcal{L}\left[y\right]-3s\mathcal{L}\left[y\right]-2s-10\mathcal{L}\left[y\right]-1=\frac{12}{s-1}
\]

Resolvemos la ecuación para $\mathcal{L}\left[y\right]$:

\[
    \mathcal{L}\left[y\right]=\frac{2s^2-s+11}{\left(s-1\right)\left(s^2-3s-10\right)}
\]


Aplicamos la inversa de Laplace. En este caso debemos simplificar la fracción en fracciones parciales simples para 
que nos cuadre con la tabla de transformadas:

\[
    y=
    \mathcal{L}^{-1}\left[\frac{2s^2-s+11}{\left(s-1\right)\left(s^2-3s-10\right)}\right]=
    \mathcal{L}^{-1}\left[-\frac{1}{s-1}+\frac{1}{s+2}+\frac{2}{s-5}\right]=
    -e^t+e^{-2t}+2e^{5t}
\]










\end{document}