\documentclass{article}
\usepackage{lipsum}
\usepackage{authoraftertitle}
\usepackage[top=2cm,bottom=1.5cm,left=1.5cm, right=3cm,includeheadfoot]{geometry}
\usepackage{graphicx}
\usepackage[parfill]{parskip}
\usepackage{fancyhdr}
\usepackage[spanish]{babel}
\usepackage{mathtools}
\usepackage{csquotes}
\usepackage{amssymb}
\usepackage[shortlabels]{enumitem}
\usepackage{fancybox, graphicx}
\usepackage{array}
\usepackage{hhline}
\usepackage{subfigure}
\usepackage{gensymb}
\usepackage{hyperref}
\usepackage{tikz}
\usepackage{amsmath}
\usepackage{leftidx}
\usepackage{wrapfig}
\usepackage{float}
\usepackage{upgreek}
\usepackage{amsmath} 
\usepackage{caption}
\usepackage{esvect}
\usepackage{siunitx}
\usepackage{commath}
\usepackage{bigints}


%Vars
\author{Alexander Sebastian Kalis}
\title{Actividad de Evaluación Continua 1}
\pagestyle{fancy}
\fancyhead[LO]{1508 Fundamentos de Economía de Empresa}
\fancyhead[RO]{\author}
%\fancyhead[RE]{\leftmark}
\fancyfoot[L]{\raisebox{-1cm}{\includegraphics[height=1.5cm]{E:/KUKADisk/UDIMA/DocumentGraphics/LOGOUDIMA.jpg}}}
\fancyfoot[R]{Corregido:\\ Dra. Patricia Madrigal Barrón}
%\fancyfoot[RO]{07/12/2018}


\begin{document}

\begin{titlepage}

    \begin{center}

        \line(1,0){300}\\
        [0.2in]
        \huge{\bfseries {\MyTitle}}\\
        [1mm]
        \line(2,0){200}\\
        [0.75cm]
        \textsc{\LARGE Fundamentos de Economía de Empresa}\\
        [2cm]
        \includegraphics[height=10cm]{E:/KUKADisk/UDIMA/EconomiaEmpresa/Practicas/portada.jpg}\\
        [3cm]

    \end{center}

    \begin{flushright}

        {\MyAuthor}\\
        Profesora: Dra. Patricia Madrigal Barrón\\
        Curso: Ingeniería de Organización Industrial\\
        UDIMA\\
        \today        

    \end{flushright}
    
\end{titlepage}

%\tableofcontents \thispagestyle{empty}
%\newpage

\section*{Ejercicio 1}

\textit{¿Cuáles son las posibles razones y circunstancias que asemejan a la empresa con un ser vivo?}
\newline 

Cuando comparamos un ser vivo con una empresa, nos podemos dar cuenta que a nivel básico, ambos buscan
sobrevivir, crecer y/o multiplicarse.

Para ello necesitan de lo siguiente: 

\begin{itemize}
    \item En el caso de los seres vivos, deben nutrirse con agua y comida. Podríamos ver el símil 
    en este caso ya que necesitan una fuente de ingresos para poder seguir operando.
    \item Ambos buscan la forma de reproducirse o expandirse.
    \item Tanto los seres vivos como las empresas se adaptan constantemente a su entorno.
    \item Los seres vivos están pendientes de no ser devorados por depredadores, que en caso de las empresas
    (asumiendo que trabajan en un mercado libre) podría considerarse como la competencia.
\end{itemize}

Por otro lado, a nivel sistemático, podemos observar que ambos son organismos complejos que funcionan a base 
de cooperación entre sus distintos componentes (órganos en caso de los seres vivos y departamentos en caso 
de las empresas). En este caso podemos hacer las siguientes observaciones:

\begin{itemize}
    \item Las células de los órganos se renuevan constantemente al igual que una empresa no tiene una plantilla
    constante, siempre hay movimiento de personal, talento nuevo, personas que se jubilan, etc.
    \item El cerebro y el sistema nervioso podría verse como el canal de comunicación que enlaza todos estos 
    órganos y por donde fluye la información, que hoy en día serían los sistemas informáticos.
    \item El sistema endocrino, que es un bucle cerrado que regula los órganos, podría verse como los objetivos
    y metas que se proponen las empresas. Si no se llega a los objetivos, se realiza un análisis y se aplican
    estretegias distintas.
\end{itemize}

\newpage


\section*{Ejercicio 2}


\textit{Clasifique la empresa española Zara según criterios económicos y organizativos.}
\newline

Buscando información sobre ZARA ESPAÑA S.A. podemos clasificarla de la siguiente forma:

Según criterios económicos:

\begin{itemize}
    \item Sector económico: Empresa Industrial ya que se dedica a la producción y distribución textil.
    \item Tamaño: Se trata de una empresa muy grande pues tiene 12.716 empleados según la información proporcionada por www.infocif.es y una facturación de 19.56 billones de EUR en 2019.
    \item Estructura social de producción: Se trata de una sociedad capitalista pues es una sociedad anónima.
    \item Sistema técnico: Empresa multiproducto y de producción en serie.
    \item Localización: Empresa multiplanta.
    \item Ámbito de competencia: Empresas multinacionales.
\end{itemize}

Según criterios organizativos:

\begin{itemize}
    \item Configuración básica: Empresa jerárquica.
    \item Estilo de decisión: Empresa centralizada.
    \item Ejercicio de autoridad: Empresa autoritaria.
    \item Organización jurídica: Empresa plurisocietaria.
    \item Estilo de dirección: Empresas creativas y flexibles.
\end{itemize}

\newpage

\section*{Ejercicio 3}

\textit{Busque información de Ingvar Kamprad y detalla las características personales y las motivaciones que le
llevaron a la actividad empresarial que ha ejercido.}
\newline

Kamprad empezó desde muy pequeño con su espíritu emprendedor vendiendo fósforos que compraba al por mayor. 
Fundó su empresa IKEA a los 17 años con el dinero que le otorgó su padre por obtener buenos resultados académicos. Sus 
primeros pasos los dio vendiendo réplicas de mesas. Más adelante añadió otros tipos de mobiliario a su portfolio.

Debido a que los precios ofrecidos por IKEA eran bajos, los compradores veían el mobiliario de IKEA como un producto de 
baja calidad. Esto motivó a Kamprad a ofrecer un showroom con sus muebles para que los clientes puedan probarlos y 
ver que realmente no solo eran baratos, si no que también de buena calidad. Hoy en día este modelo de negocios lo utilizan
muchas empresas como Apple.

Como figura de empresario, según la información encontrada, era una persona que buscaba siempre dar ejemplo y empatizar con
sus empleados y clientes. Se le conocía como un hombre rico, pero muy austero y frugal. Podríamos decir que de esta forma
motivaba a sus empleados haciéndolos sentir parte de la familia y parte de un proyecto común, es decir, un liderazgo transformador.

\newpage

\section*{Ejercicio 4}

\textit{Realice un esquema que le sirva de estudio sobre la evolución de la teoría del empresario.}



\end{document}