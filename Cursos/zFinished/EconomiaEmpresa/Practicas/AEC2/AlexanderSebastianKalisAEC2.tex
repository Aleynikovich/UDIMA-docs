\documentclass{article}
\usepackage{lipsum}
\usepackage{authoraftertitle}
\usepackage[top=2cm,bottom=1.5cm,left=1.5cm, right=3cm,includeheadfoot]{geometry}
\usepackage{graphicx}
\usepackage[parfill]{parskip}
\usepackage{fancyhdr}
\usepackage[spanish]{babel}
\usepackage{mathtools}
\usepackage{csquotes}
\usepackage{amssymb}
\usepackage[shortlabels]{enumitem}
\usepackage{fancybox, graphicx}
\usepackage{array}
\usepackage{hhline}
\usepackage{subfigure}
\usepackage{gensymb}
\usepackage{hyperref}
\usepackage{tikz}
\usepackage{amsmath}
\usepackage{leftidx}
\usepackage{wrapfig}
\usepackage{float}
\usepackage{upgreek}
\usepackage{amsmath} 
\usepackage{caption}
\usepackage{esvect}
\usepackage{siunitx}
\usepackage{commath}
\usepackage{bigints}


%Vars
\author{Alexander Sebastian Kalis}
\title{Actividad de Evaluación Continua 2}
\pagestyle{fancy}
\fancyhead[LO]{1508 Fundamentos de Economía de Empresa}
\fancyhead[RO]{\author}
%\fancyhead[RE]{\leftmark}
\fancyfoot[L]{\raisebox{-1cm}{\includegraphics[height=1.5cm]{E:/KUKADisk/UDIMA/DocumentGraphics/LOGOUDIMA.jpg}}}
\fancyfoot[R]{Corregido:\\ Dra. Patricia Madrigal Barrón}
%\fancyfoot[RO]{07/12/2018}


\begin{document}

\begin{titlepage}

    \begin{center}

        \line(1,0){300}\\
        [0.2in]
        \huge{\bfseries {\MyTitle}}\\
        [1mm]
        \line(2,0){200}\\
        [0.75cm]
        \textsc{\LARGE Fundamentos de Economía de Empresa}\\
        [2cm]
        \includegraphics[height=10cm]{E:/KUKADisk/UDIMA/EconomiaEmpresa/Practicas/portada.jpg}\\
        [3cm]

    \end{center}

    \begin{flushright}

        {\MyAuthor}\\
        Profesora: Dra. Patricia Madrigal Barrón\\
        Curso: Ingeniería de Organización Industrial\\
        UDIMA\\
        \today        

    \end{flushright}
    
\end{titlepage}

%\tableofcontents \thispagestyle{empty}
%\newpage

\section*{Ejercicio 1}

\textit{Explique el significado del punto muerto o umbral de rentabilidad. Calcule el punto muerto con los datos
de la empresa Zeta que se recogen a continuación.}

Conocemos como punto muerto o umbral de rentabilidad, el punto en el cual los ingresos totales equivalen a los costes totales de la actividad de una empresa.

\[
    I=CT
\]

Considerando los diferentes tipos de costes tenemos que el umbral de rentabilidad es:

\[
    Q=\frac{CF}{P-CVu}
\]

Entonces según los datos del enunciado tenemos que:

\[
    Q=\frac{15000000}{1000-100}=16666.67 \ uds.
\]

Lo que significa que la empresa debe vender 16667 unidades para mantenerse por encima del umbral de rentabilidad.


\textit{Represente gráficamente el punto muerto.}

Apoyándonos sobre la imagen de Bueno, podemos observar que según la gráfica los elementos serían los siguientes:\\

\begin{center}
    \includegraphics[height=5cm]{E:/KUKADisk/UDIMA/EconomiaEmpresa/Practicas/AEC2/grafica1.PNG}
\end{center}

\begin{itemize}
    \item Los ingresos totales son la cantidad de unidades vendidas multiplicadas por su precio y en nuestro caso suman $250000000$ unidades monetarias.
    \item Los costes fijos anuales son aquellos que no dependen de la cantidad producida por lo tal se mantienen constantes a lo largo del ciclo de producción. En nuestro ejemplo son $15000000$.
    \item Los costes variables son aquellos que varían y aumentan con cada unidad producida. Los totales son $25000000$ pero los unitarios serían $100$ ya que los dividimos por la cantidad de unidades de producto.
    \item Los costes totales son la suma de los fijos y los variables, es decir, $50000000$.
    \item El punto Q de la gráfica muestra el umbral de rentabilidad. Hemos calculado que $Q=16666.67$.
\end{itemize}

\newpage


\section*{Ejercicio 2}

\textit{Seleccione una empresa que opere en el mercado español y estudie las acciones que la empresa ha
realizado durante el año 2020 para adaptarse al entorno en el que participa. Esas actuaciones que
demostrarían que hoy en día las empresas están influenciadas por el entorno y preocupadas por los
mismos hechos que afectan a sus consumidores}

Por supuesto vamos a hablar de una de las empresas clave del mercado español que es Mercadona.

La principal actuación de Mercadona frente a la pandemia que ha azotado el mundo en el 2020 ha sido la gran inversión en su capacidad logística. Hablando
de números, 500 millones de euros entre 2020 y 2021. Considerando que tenían planeado invertir alrededor de 1.000 millones de 2020 a 2025, esto supone
una inversión del 50 \% de esa cantidad en tan solo un cuarto del tiempo.

El motivo está claro y es que, al principio de la crisis por COVID-19, la gente arrasó con los supermercados incluso varias semanas después del estado de alarma. Esto hizo que
Mercadona (y muchos otros supermercados) no diesen abasto para reponer todas sus mercancías. Según Rosa Aguado, la directora general de logística de Mercadona, 
los viajes diarios de camiones han incrementado en un 30\%. Esto ha hecho que la inversión en logística, que ya estaba planeada, diese un estirón y se invirtiese más
aún de lo previsto.

Por otro lado, gran parte de la nueva inversión logística se hará al aspecto del reparto a domicilio. Esto es debido a que cuentan con que muchas personas, debido al confinamiento, se hayan 
acostumbrado a los servicios de reparto a domicilio y por este motivo cambien su modalidad de compra a esta última, en lugar de hacerlo presencialmente.






\end{document}