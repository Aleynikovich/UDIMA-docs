\documentclass{article}
\usepackage{lipsum}
\usepackage{authoraftertitle}
\usepackage[top=2cm,bottom=1.5cm,left=1.5cm, right=3cm,includeheadfoot]{geometry}
\usepackage{graphicx}
\usepackage[parfill]{parskip}
\usepackage{fancyhdr}
\usepackage[spanish]{babel}
\usepackage{mathtools}
\usepackage{csquotes}
\usepackage{amssymb}
\usepackage[shortlabels]{enumitem}
\usepackage{fancybox, graphicx}
\usepackage{array}
\usepackage{hhline}
\usepackage{subfigure}
\usepackage{gensymb}
\usepackage{hyperref}
\usepackage{tikz}
\usepackage{amsmath}
\usepackage{leftidx}
\usepackage{wrapfig}
\usepackage{float}
\usepackage{upgreek}
\usepackage{amsmath} 
\usepackage{caption}
\usepackage{esvect}
\usepackage{siunitx}
\usepackage{commath}
\usepackage{bigints}


%Vars
\author{Alexander Sebastian Kalis}
\title{Actividad de Evaluación Continua 3}
\pagestyle{fancy}
\fancyhead[LO]{1508 Fundamentos de Economía de Empresa}
\fancyhead[RO]{\author}
%\fancyhead[RE]{\leftmark}
\fancyfoot[L]{\raisebox{-1cm}{\includegraphics[height=1.5cm]{E:/KUKADisk/UDIMA/DocumentGraphics/LOGOUDIMA.jpg}}}
\fancyfoot[R]{Corregido:\\ Dra. Patricia Madrigal Barrón}
%\fancyfoot[RO]{07/12/2018}


\begin{document}

\begin{titlepage}

    \begin{center}

        \line(1,0){300}\\
        [0.2in]
        \huge{\bfseries {\MyTitle}}\\
        [1mm]
        \line(2,0){200}\\
        [0.75cm]
        \textsc{\LARGE Fundamentos de Economía de Empresa}\\
        [2cm]
        \includegraphics[height=10cm]{E:/KUKADisk/UDIMA/EconomiaEmpresa/Practicas/portada.jpg}\\
        [3cm]

    \end{center}

    \begin{flushright}

        {\MyAuthor}\\
        Profesora: Dra. Patricia Madrigal Barrón\\
        Curso: Ingeniería de Organización Industrial\\
        UDIMA\\
        \today        

    \end{flushright}
    
\end{titlepage}

%\tableofcontents \thispagestyle{empty}
%\newpage

\section*{Ejercicio 1}

\textit{Selecciona una empresa que forme parte del IBEX 35 y analiza los siguientes puntos en relación con su
sistema cultural. Recuerda siempre explicar tu respuesta.}

En este caso analizaremos la empresa Grifols ya que es bastante conocida en mi sector y puede resultar más fácil de encontrar información.

\textbf{Valores}

Según Grifols sus valores se basan en el enfoque empresarial ético, sostenible y transparente para generar sus positivos resultados. Los valores que destacan
son:

\begin{itemize}
    \item Orgullo. Como muchas empresas líder en su sector, son orgullosos de sus productos. Esto implica una fuerte creencia en que pocos productos de la 
    competencia pueden ser igual de eficaces o mejores que los suyos. 
    \item Esfuerzo. Que se utiliza como base para superar obstáculos a nivel personal y del grupo.
    \item Trabajo en equipo. Nos dan a entender que no es una empresa individualista. Esto significa que los diversos departamentos cooperan entre sí por el bien
    general del grupo. Muchas veces nos encontramos con empresas en las que los resultados personales y/o de cada departamento están por encima de de los del conjunto 
    y esto a veces crea roces entre ellos.
    \item Innovación y mejora. Nos dan a entender que no es una empresa conformista. Busca siempre avanzar a crear mejores productos y acepta que no siempre están en lo correcto.    
\end{itemize}

\textbf{Normas}

Como toda empresa madura, Grifols incorpora un código de conducta y normas que los identifican y los hacen únicos. Entre ellos podemos destacar:

\begin{itemize}
    \item El respeto a terceros. Grifols es una empresa que trabaja mucho mediante la subcontratación. Muchas veces, las personas subcontratadas sienten que no forman
    parte de la empresa y esto puede llegar a ser desmotivante. Haciendo hincapié en el respeto al personal subcontratado y a otros terceros como clientes y proveedores,
    consiguen un ambiente saludable también para aquellos que no forman parte directa del grupo. De esta forma aseguran que los trabajos realizados por terceros sean de máxima
    calidad pues nadie quiere trabajar para una empresa si no se siente respetado por ella.
    \item Medio ambiente, salud y seguridad. Como todas las empresas, Grifols tiene su propia estrategia para lidiar con asuntos de medio ambiente como puede ser la gestión
    residual. Por otro lado tienen una serie de normas de seguridad para reducir los accidentes laborales en sus fábricas.
    \item Protección de datos y privacidad. El conocimiento y la información son, hoy en dia, considerados unos de los activos más importantes de las empresas. Esto hace que tengan activo un plan
    de protección de datos para asegurar que aquellos que disponen de información confidencial sobre la empresa no la utilicen para asuntos ajenos.
    \item Fraude y corrupción. En estas normas estipulan la prohibición de utilizar la pertenencia al grupo para el beneficio propio. Esto incluye solicitar o recibir de terceros
    dinero u otros objetos de valor.
    \item Transparencia financiera. En los que se comprometen a difundir de forma transparente y veraz los registros y cuentas de acuerdo con los requisitos legales y principios contables.
\end{itemize}

\textbf{Símbolos}

Los símbolos de la empresa son su anagrama y logotipo. Estos se conforman como el conjunto de identidad corporativa de la empresa, y serán los que 
obligatoriamente estarán, en las formas que se establecen seguidamente, en todos y cada uno de los documentos y elementos publicitarios.

En caso de Grifols es bastante simple. Su anagrama es el mismo nombre de la empresa pues es simplemente el apellido de la família fundadora y su logotipo es GRIFOLS en color azul:

\begin{center}
    \includegraphics[height=4cm]{E:/KUKADisk/UDIMA/EconomiaEmpresa/Practicas/AEC3/grifols.png}
\end{center}

\textbf{Mito y orígenes}

Laboratorios Grifols fue fundada por el médico hematólogo osé Antonio Grifols Roig, junto con sus dos hijos José Antonio Grifols Lucas y Víctor Grifols Lucas.
En sus orígenes fue un laboratorio especializado en transfusiones y analíticas de sangre. Tras patentar el proceso de liofilización de plasma, se convirtieron en el
primer banco de sangre privado de España.

Hoy cuenta con filiales en 24 países y es la tercera empresa del mundo en producción de medicamentos derivados del plasma sanguíneo.\\

\section*{Ejercicio 2}

\textit{Una empresa española se pone en contacto con nosotros como expertos en economía de la empresa para
que participemos en un proceso en el que se encuentran inmersos de decisión de lanzamiento de
productos. La empresa tiene la posibilidad de producir dos tipos diferentes de productos, A y B, con tres
escenarios de ventas alternativos posibles; 1, 2 y 3; los cuales se presentan en la tabla siguiente. }

\begin{center}
    \includegraphics[height=2cm]{E:/KUKADisk/UDIMA/EconomiaEmpresa/Practicas/AEC3/ee2.png}
\end{center}

\textit{Nos piden aplicar el criterio de Hurwicz, tomando un coeficiente de optimismo del 60 \%. Indica la decisión
de éxito qué debe tomar la empresa en función de los datos facilitados.}

Tomamos como coeficientes $\alpha_0 = 0.6$ y $\alpha_p=0.4$.

Calculamos el peso promedio: $\alpha_0*(max) + \alpha_p*(min)$.

A:
\[
    0.6 \cdot 80 + 0.4 \cdot 20 = 56
\]

B:

\[
    0.6 \cdot 60 + 0.4 \cdot 10 = 40
\]

Con lo cual la empresa debería centrarse en producir el producto A.


\section*{Ejercicio 3}

\textit{Hay una expresión empresarial que dice: “La información es poder”. Explica en detalle, en base a lo
estudiado en la asignatura, cuál es el significado de dicha expresión.}

Desde el punto de vista empresarial, podemos relacionarlo con el llamado "Business Intelligence", que podemos definir como un conjunto de técnicas y estrategias que
se utilizan para administrar y generar conocimiento.

La palabra "poder" en este caso se puede entender como una metáfora que puede significar una serie de conceptos:

\begin{itemize}
    \item Eficiencia.
    \item Margen de respuesta.
    \item Información precisa.
    \item Análisis.
    \item Control.
\end{itemize}

Como se ha comentado anteriormente en el caso de Grifols, hoy en dia las empresas valoran el conocimiento y la información como un activo muy valioso ya que les permite
obtener beneficios a partir de los conceptos comentados arriba.

Podemos distinguir entre dos tipos de información o conocimiento.

El conocimiento interno podemos verlo como su "know-how". Es importante que las empresas desarrollen una base de conocimiento, lo almacenen y organicen correctamente.
De esta forma es posible afrontar situaciones parecidas a partir de datos antiguos o bien entrenar a la plantilla con unos protocolos definidos. Así se contribuye
a una plantilla competente y disminución del tiempo en el que un empleado es capaz de empezar a trabajar por su cuenta.

Por otro lado tenemos el conocimiento externo. Esto se consigue mediante técnicas "Big Data". Es decir, recopilación masiva de información anónima de millones de clientes,
proveedores, etc. Esto permite a la empresa saber qué productos lanzar, cómo reaccionar frente a ciertas situaciones y en general estar siempre un paso por delante de la competencia.

\newpage


\section*{Ejercicio 4}

\textit{Realiza un esquema que te sirva de estudio de los procesos de información}

\begin{center}
    \includegraphics[height=20cm]{E:/KUKADisk/UDIMA/EconomiaEmpresa/Practicas/AEC3/esquema.png}
\end{center}



\end{document}