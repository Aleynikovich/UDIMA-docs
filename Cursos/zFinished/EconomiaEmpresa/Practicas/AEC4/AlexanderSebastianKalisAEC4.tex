\documentclass{article}
\usepackage{lipsum}
\usepackage{authoraftertitle}
\usepackage[top=2cm,bottom=1.5cm,left=1.5cm, right=3cm,includeheadfoot]{geometry}
\usepackage{graphicx}
\usepackage[parfill]{parskip}
\usepackage{fancyhdr}
\usepackage[spanish]{babel}
\usepackage{mathtools}
\usepackage{csquotes}
\usepackage{amssymb}
\usepackage[shortlabels]{enumitem}
\usepackage{fancybox, graphicx}
\usepackage{array}
\usepackage{hhline}
\usepackage{subfigure}
\usepackage{gensymb}
\usepackage{hyperref}
\usepackage{tikz}
\usepackage{amsmath}
\usepackage{leftidx}
\usepackage{wrapfig}
\usepackage{float}
\usepackage{upgreek}
\usepackage{amsmath} 
\usepackage{caption}
\usepackage{esvect}
\usepackage{siunitx}
\usepackage{commath}
\usepackage{bigints}
\usepackage[official]{eurosym}

%Vars
\author{Alexander Sebastian Kalis}
\title{Actividad de Evaluación Continua 4}
\pagestyle{fancy}
\fancyhead[LO]{1508 Fundamentos de Economía de Empresa}
\fancyhead[RO]{\author}
%\fancyhead[RE]{\leftmark}
\fancyfoot[L]{\raisebox{-1cm}{\includegraphics[height=1.5cm]{E:/KUKADisk/UDIMA/DocumentGraphics/LOGOUDIMA.jpg}}}
\fancyfoot[R]{Corregido:\\ Dra. Patricia Madrigal Barrón}
%\fancyfoot[RO]{07/12/2018}


\begin{document}

\begin{titlepage}

    \begin{center}

        \line(1,0){300}\\
        [0.2in]
        \huge{\bfseries {\MyTitle}}\\
        [1mm]
        \line(2,0){200}\\
        [0.75cm]
        \textsc{\LARGE Fundamentos de Economía de Empresa}\\
        [2cm]
        \includegraphics[height=10cm]{E:/KUKADisk/UDIMA/EconomiaEmpresa/Practicas/portada.jpg}\\
        [3cm]

    \end{center}

    \begin{flushright}

        {\MyAuthor}\\
        Profesora: Dra. Patricia Madrigal Barrón\\
        Curso: Ingeniería de Organización Industrial\\
        UDIMA\\
        \today        

    \end{flushright}
    
\end{titlepage}

%\tableofcontents \thispagestyle{empty}
%\newpage

\section*{Ejercicio 1}

Se presenta a continuación alguna de las partidas principales del balance de situación de la empresa Alfa
para el ejercicio 2020 y en base a las mismas se plantea una serie de cuestiones a realizar por el
estudiante.

\begin{center}
    \includegraphics[height=4cm]{E:/KUKADisk/UDIMA/EconomiaEmpresa/Practicas/AEC4/tablee1.PNG}\\
\end{center}

a) Calcular los siguientes ratios de análisis de solvencia empresarial: ratio de solvencia, ratio de
liquidez, ratio de liquidez inmediata, ratio de tesorería.

Ratio de solvencia:

\[
    RS=\frac{Activo}{Pasivo}\cdot 100=\frac{1150000}{700000}\cdot 100=164.3 \%
\]

Ratio de liquidez:

\[
    RL=\frac{Activo \ circulante}{Pasivo \ circulante}\cdot 100 =\frac{450000}{250000}\cdot 100=180 \%
\]

Ratio de liquidez inmediata:

\[
   RLI= \frac{Activo \ circulante - Existencias}{Pasivo \ circulante}\cdot 100=\frac{450000-250000}{250000}\cdot 100=80 \%
\]

Ratio de tesorería:

\[
    RT=\frac{Tesoreria}{Pasivo \ circulante} \cdot 100 =\frac{100000}{250000} \cdot 100 =40 \%
\]


b) Explicación del significado de cada uno de los ratios calculados. 

\begin{itemize}
    \item El ratio de solvencia mide la capacidad de una empresa para afrontar todas sus deudas.
    \item El ratio de liquidez mide la capacidad de una empresa para devolver los préstamos en el corto plazo.
    \item El ratio de liquidez inmediata mide la posibilidad de hacer frente a un pago de forma inmediata ante un imprevisto.
    \item El ratio de tesorería mide la capacidad de una compañía o institución para pagar las deudas que vencen en el corto plazo o deudas ya vencidas.
\end{itemize}

c) Análisis de cada uno de los ratios calculados. 

En general los resultados obtenidos han sido muy positivos. Esto significa que si la empresa necesitara pedir un préstamo o bien riesgo a un 
arrendador, tendrían posibilidades de recibir una buena cantidad.

\newpage

\section*{Ejercicio 2}

De los siguientes costes, indique cuáles son costes variables y cuáles costes fijos. Justifique su respuesta.

-El coste de materias primas es un 30 \% del valor de las ventas.

En este caso es un coste variable, pues el coste de la materia prima es directamente proporcional a la cantidad de ventas.

-El alquiler mensual de su nave es de 1.200\euro.

El alquiler resulta un coste fijo ya que independientemente si haya producción o actividad en el local alquilado, se debe
pagar una cantidad fija al mes.

-El coste de personal asciende anualmente a 60.000 euros.

En general los costes de los empleados suelen considerarse como costes fijos. Si una persona está contratada, deberá recibir el sueldo
tanto si produce como si no. Es por ello que en casos de baja producción, se ven casos de ERTEs o EREs como hemos vivido en la situación
del COVID-19. Esto se hace como medida para eliminar costes fijos que no aportan nada.

-El mínimo por potencia de energía eléctrica contratada es de 16 \euro /mes, además es imputable a
cada producto 1 kw, que paga a 0,12\euro/kw.

En este caso hay una parte fija (la potencia contratada de 16 \euro /mes) que se pagará tanto si se utiliza como si no. Luego hay otra parte
que es variable y es la que depende de la cantidad de producto producido.

-La amortización lineal de la maquinaria asciende a 5.000\euro \ anuales sobre 12 años de vida útil.

Al ser una amortización lineal, el coste cada año será el mismo y por tanto será fijo.
\end{document}