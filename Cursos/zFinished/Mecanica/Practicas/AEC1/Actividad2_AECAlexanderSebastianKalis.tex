\documentclass{article}
\usepackage{lipsum}
\usepackage{authoraftertitle}
\usepackage[top=2cm,bottom=1.5cm,left=1.5cm, right=3cm,includeheadfoot]{geometry}
\usepackage{graphicx}
\usepackage[parfill]{parskip}
\usepackage{fancyhdr}
\usepackage[spanish]{babel}
\usepackage{mathtools}
\usepackage{csquotes}
\usepackage{amssymb}
\usepackage[shortlabels]{enumitem}
\usepackage{fancybox, graphicx}
\usepackage{array}
\usepackage{hhline}
\usepackage{subfigure}
\usepackage{gensymb}
\usepackage{hyperref}
\usepackage{tikz}
\usepackage{amsmath}
\usepackage{wrapfig}
\usepackage{float}
\usepackage{amsmath} 
\usepackage{caption}
\usepackage{esvect}
\usepackage{siunitx}
\usepackage{commath}
%Header & Footer

\pagestyle{fancy}
\fancyhead[LE]{\MyTitle}
\fancyhead[LO]{Mecánica}
%\fancyhead[RO]{\leftmark}
%\fancyhead[RE]{\leftmark}
\fancyfoot[L]{\raisebox{-1cm}{\includegraphics[height=2cm]{D:/KUKADisk/UDIMA/DocumentGraphics/LOGOUDIMA.jpg}}}
\fancyfoot[R]{Corregido:\\ Dra. Isabel Cristina Gil García}
%\fancyfoot[RO]{07/12/2018}


%Vars
\author{Alexander Sebastian Kalis}
\title{Ejercicios propuestos Unidades 1-3}


%DOC


\begin{document}

\begin{titlepage}

    \begin{center}

        \line(1,0){300}\\
        [0.2in]
        \huge{\bfseries {\MyTitle}}\\
        [1mm]
        \line(2,0){200}\\
        [0.75cm]
        \textsc{\LARGE Mecánica Clásica}\\
        [2cm]
        \includegraphics[height=10cm]{D:/KUKADisk/UDIMA/Mecanica/portada.jpg}\\
        [3cm]

    \end{center}

    \begin{flushright}

        {\MyAuthor}\\
        %Profesora: Dra. Isabel Cristina Gil García\\
        %Curso: Ingeniería de Organización Industrial\\
        %UDIMA\\
        \today        

    \end{flushright}
    
\end{titlepage}

%\tableofcontents \thispagestyle{empty}
%\newpage

%\section*{Introducción}


\newpage

\section*{Actividades}
\subsection*{Problema 1}

\subsubsection*{Enunciado del problema}

La posición de una partícula que se mueve en línea recta está definida por la relación\\

$x=t^3-6t^2-15t+40$\\

$x$ está expresada en pies y $t$ en segundos.\\
Determinar:

\begin{enumerate}[a)]
    \item El tiempo para el cual la velocidad será nula.
    \item La posición y la distancia recorrida por la partícula en ese tiempo.
    \item La aceleración de la partícula en ese instante.
    \item La distancia recorrida por la partícula desde $t=4s$ hasta $t=6s$
    \item Grafique el movimiento de posición, velocidad y aceleración.
\end{enumerate}

Expresar los resultados en SI.

\subsubsection*{Datos}
Función que define la posición en el eje $x$ de la partícula: 
\begin{equation}
    x(t)=t^3-6t^2-15t+40\\
\end{equation}
Función que define la velocidad de la partícula:
\begin{equation}
    v(t)=\frac{dx}{dt}=3t^2-12t-15\\
\end{equation} 
Función que define la aceleración de la parícula:
\begin{equation}
    a(t)=\frac{dv}{dt}=6t-12\\
\end{equation} 
Tipo de movimiento: Movimiento Rectilíneo Uniformemente Acelerado\\
%Aceleración de la gravedad: $-9.8m/s^2$

\subsubsection*{Solución y explicación del problema}

Disponiendo de la función de la velocidad, simplemente igualamos $v(t)=0$ y extraemos las raíces de las
ecuación para obtener el tiempo en el que la velocidad es nula:

\begin{equation}
    3t^2-12t-15=0 \rightarrow t=5s
\end{equation}

Para la posición, recurrimos a la función (1) que es la que nos proporciona 
esa información, donde el tiempo será $t=5$.

\begin{equation}
    x(5)=5^3-6\cdot5^2-15\cdot5+40 \rightarrow x(5)=-60ft=-18.29m
\end{equation}
\begin{equation}
    x(0)=40ft=12.19m
\end{equation}

La distancia total recorrida entre 0s y 5s será la integral del valor absoluto de la velocidad en ese tramo de tiempo:

\begin{equation}
    \int_{0}^{5}\lvert 3t^2-12t-15 \lvert \ dt = 100ft = 30.48m
\end{equation}

Para la aceleración en el instante $t=5s$ recurrimos a la función (3) y obtenemos:

\begin{equation}
    a(5)=6\cdot5-12=18ft/s^2=5.49m/s^2
\end{equation}

Nuevamente para calcular la distancia total recorrida entre 4s y 6s integramos el valor absoluto de la 
velocidad en ese tramo:

\begin{equation}
    \int_{4}^{6}\lvert 3t^2-12t-15 \lvert \ dt = 18 ft = 5.49m
\end{equation}

Por último representamos gráficamente las funciones utilizando Symbolab:

\begin{figure}[!htb]
    \begin{minipage}{0.50\textwidth}
    \centering
    \includegraphics[height=5cm]{D:/KUKADisk/UDIMA/Mecanica/Practicas/AEC1/posiciontiempo.PNG}
    \caption{Posición-tiempo}
    \end{minipage}\hfill
    \begin{minipage}{0.50\textwidth}
    \centering
    \includegraphics[height=5cm]{D:/KUKADisk/UDIMA/Mecanica/Practicas/AEC1/velocidadtiempo.PNG}
    \caption{Velocidad-tiempo}
    \end{minipage}
\end{figure}

\begin{figure}
    \centering
    \includegraphics[height=5cm]{D:/KUKADisk/UDIMA/Mecanica/Practicas/AEC1/aceleraciontiempo.PNG}
    \caption{Aceleración-tiempo}
\end{figure}

\newpage

\subsection*{Problema 2}

\subsubsection*{Enunciado del problema}

Un helicóptero de rescate deja caer un paquete de suministros a alpinistas que se encuentran
aislados en la cima de una colina peligrosa, situada 200 m abajo del helicóptero. Si
éste vuela horizontalmente con una rapidez de 250 km/h:\\
a) ¿A qué distancia horizontal antes de los alpinistas debe dejarse caer el paquete de suministros?\\
En vez de esto, suponga que el helicóptero lanza los suministros a una distancia horizontal
de 400 m antes de donde se encuentran los alpinistas.\\
b) ¿Qué velocidad vertical debería darse a los suministros (hacia arriba o hacia abajo)
para que éstos caigan precisamente en la cima de la montaña?\\
c) ¿Con qué rapidez aterrizan los suministros en este último caso?

\subsubsection*{Datos}

\textbf{Primer apartado}\\
Distancia en $y$ del helicóptero al punto de impacto: $200m$\\
Velocidad de la partícula en el instante de lanzamiento: $v_{x0}=250km/h=69.5m/s, \ v_{y0}=0m/s$\\
Gravedad (aceleración componente $y$): $g=a_y=-9.8m/s^2$\\
Se omiten posibles fuerzas de desaceleración en $x$: $a_x=0m/s^2$\\
\textbf{Segundo apartado}\\
Se añade la condición de distancia en $x$ del helicóptero al punto de impacto: $400m$

\subsubsection*{Solución y explicación del problema}

\textbf{Primer apartado}\\
Empezamos calculando el tiempo que tarda la partícula en llegar a $y=-200m$, el tiempo de impacto. Tomamos como referencia 
la posición del helicóptero.

\begin{equation}
    y=v_{y0}t_i+\frac{1}{2}gt_i^2 \rightarrow
    -200=-\frac{1}{2}9.8t^2 \rightarrow
    t_i \approx 6.4s
\end{equation}

Sabiendo el tiempo que tarda la partícula en impactar, podemos calcular la distancia que recorre durante ese tiempo en su
componente $x$, que será la distancia con la cual debe lanzar la partícula con antelación:

\begin{equation}
    x=v_xt_i \rightarrow x=69.5\cdot6.4 \rightarrow x \approx 445m
\end{equation}

\textbf{Segundo apartado}

Conociendo la posición en $x$ y la velocidad en $x$ de la partícula calculamos el tiempo de impacto, que será la
rapidez con la que se suministran:
\begin{equation}
    x=v_xt_i \rightarrow 400=69.5t_i \rightarrow t_i \approx 5.8s
\end{equation}


El objetivo es que la partícula llegue a $y=-200m$ en $5.8s$. Entonces podemos calcular qué velocidad inicial en $y$ debemos darle
para que llegue en el tiempo programado.

\begin{equation}
    y=v_{y0}t_i+\frac{1}{2}gt_i^2 \rightarrow
    -200=5.8v_{y0}-\frac{9.8}{2}\cdot5.8^2 \rightarrow
    v_{y0} \approx -6m/s
\end{equation}

Como es negativa, esto significa que el helicóptero deberá soltar los suministros mientras desciende a $6m/s$.
\newpage


\subsection*{Problema 3}

\subsubsection*{Enunciado del problema}
Un automóvil que pesa 1000 Kg desciende por una cuesta de 10º de inclinación. El conductor
divisa un obstáculo y aplica los frenos produciendo una fuerza total de frenado (aplicada por
la carretera sobre los neumáticos) constante y de 5000 N. Hasta que el coche se detiene
recorre una distancia de 200 m.\\
Determinar la velocidad inicial del coche

\subsubsection*{Datos}
Masa del automóvil: $m=1000kg$\\
Inclinación del plano: $\alpha=10\degree$\\
Fuerza total de frenado: $\vec{F}=-5000N$\\
Distancia recorrida: $s=200m$
\subsubsection*{Solución y explicación del problema}
Aunque podría resolverse mediante el teorema de fuerzas vivas, al tratarse de un ejercicio de plano inclinado se resolverá por descomposición de fuerzas:

\begin{figure}[H]
    \begin{centering}
        \includegraphics[height=6cm]{D:/KUKADisk/UDIMA/Mecanica/Practicas/AEC1/descomp3.png}
        \caption{Descomposición de fuerzas}
    \end{centering}   
\end{figure}
\begin{equation}
    \vec{P}=mg=1000\cdot9.8=9800N
\end{equation}
\begin{equation}
    \vec{P_x}=\vec{P}\cdot \sin(\alpha)=9800\cdot \sin(10\degree) \approx 1700N
\end{equation}
\begin{equation}
    \vec{P_y}=\vec{N}=\vec{P}\cdot \cos(\alpha)=9800\cdot \cos(10\degree) \approx 9650N
\end{equation}
\begin{equation}
    \vec{F_r}=\vec{P_x}+\vec{F}=1700-5000=-3300N
\end{equation}

Sabiendo esto podemos calcular la aceleración del vehículo:
\begin{equation}
    a=\frac{\vec{F_r}}{m}=-\frac{3300}{1000}=-3.3m/s^2
\end{equation}

Podemos encontrar la velocidad inicial aplicando la fórmula:

\begin{equation}
    v^2-v^2_0=2as \rightarrow 
    0-v^2_0=2(-3.3)(200) \rightarrow
    v_0 \approx 36.3 m/s \approx 130km/h
\end{equation}

\newpage


\subsection*{Problema 4}

\subsubsection*{Enunciado del problema}

Hallar los valores de las tensiones en las cuerdas en cada uno de los casos siguientes si los
sistemas se encuentran en equilibrio estático.

\begin{figure}[H]
    \begin{centering}
        \includegraphics[height=5cm]{D:/KUKADisk/UDIMA/Mecanica/Practicas/AEC1/e4.png}
    \end{centering}   
\end{figure}

\subsubsection*{Datos}

Se carece de datos con lo cual simplemente se procederá a aplicar la segunda Ley de Newton:

\subsubsection*{Solución y explicación del problema}

\textbf{Primer apartado}\\
Al haber solo un cable y una masa afectada por la gravedad con equilibrio estático:
\begin{equation}
    \vec{T}+\vec{P}=0 \rightarrow
    \vec{T}=m\cdot g
\end{equation}

\textbf{Segundo apartado}\\
En este caso hay 2 cuerdas y se descompone la tensión:
\begin{equation}
    m\cdot g = \vec{T}_1+\vec{T}_2
\end{equation}
\begin{equation}
    T_1 \cos(\alpha)-T_2 \cos(\beta)=m \cdot g
\end{equation}
\begin{equation}
    T_1=\frac{mg \cos(\beta)}{\sin(\alpha) \cos(\beta) + \cos(\alpha)\sin(\beta)}
\end{equation}
\begin{equation}
    T_2=\frac{mg \cos(\alpha)}{\sin(\alpha) \cos(\beta) + \cos(\alpha)\sin(\beta)}
\end{equation}

\textbf{Tercer apartado}\\
En este caso, la fuerza generada perpendicularmente al plano inclinado se cancela con la normal por lo tanto solo se tiene en cuenta la fuerza paralela al 
plano inclinado:

\begin{equation}
    T=F_x=mg \sin(\theta)
\end{equation}
\newpage

\subsection*{Formulario}
\textbf{Algebra vectorial}\\
Formas de expresar un vector en función de sus componentes de coordenadas:
\begin{equation*}
    v=x+y+z \ \ \ v(x,y,z) \ \ \ v=xi+yj+zk
\end{equation*}
Vector unitario:
\begin{equation*}
    u=\frac{xi+yj+zk}{\sqrt{x^2+y^2+z^2}}
\end{equation*}
Producto vectorial:
\begin{equation*}
    v \ x \ v' =
    \begin{vmatrix}
        i & j & k\\
        x & y & z\\
        x'& y'&z'
    \end{vmatrix}
\end{equation*}

\textbf{Cinemática}\\
Eacuación vectorial horaria:
\begin{equation*}
    r=x(t)i+(t)j+z(t)k
\end{equation*}
Velocidad:
\begin{equation*}
    v=\frac{dr}{dt}
\end{equation*}
Aceleración:
\begin{equation*}
    a=\frac{dv}{dt}
\end{equation*}
MRU:
\begin{equation*}
    x=x_0+v_xt
\end{equation*}
MRUA:
\begin{equation*}
    V_x=V_{ox}+at
\end{equation*}
\begin{equation*}
    x=x_0+(\frac{v_x+v_0x}{2})t
\end{equation*}
\begin{equation*}
    x=x_0+v_{0x}t+\frac{1}{2}at^2
\end{equation*}
\begin{equation*}
    v^2-v_2^2=2as
\end{equation*}
\textbf{Dinámica}\\
Descomposición de fuerzas:
\begin{equation*}
    P=P_x\cdot sin(\alpha) + P_y \cdot cos(\alpha)
\end{equation*}
Segunda ley de Newton
\begin{equation*}
    F=ma
\end{equation*}
Lei de Hooke
\begin{equation*}
    F=kx    
\end{equation*}
Momento de una fuerza
\begin{equation*}
    M=Fd
\end{equation*}
Trabajo:
\begin{equation*}
    W=Fd
\end{equation*}
Energía cinética:
\begin{equation*}
    E_c=\frac{mv^2}{2}
\end{equation*}
Energia potencial:
\begin{equation*}
    E_p=mgh
\end{equation*}
Teorema de fuerzas vivas:
\begin{equation*}
    W = \Delta E_c = E_{c2} - E_{c1}
\end{equation*}
\end{document}