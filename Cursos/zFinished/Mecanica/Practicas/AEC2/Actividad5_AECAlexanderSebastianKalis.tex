\documentclass{article}
\usepackage{lipsum}
\usepackage{authoraftertitle}
\usepackage[top=2cm,bottom=1.5cm,left=1.5cm, right=3cm,includeheadfoot]{geometry}
\usepackage{graphicx}
\usepackage[parfill]{parskip}
\usepackage{fancyhdr}
\usepackage[spanish]{babel}
\usepackage{mathtools}
\usepackage{csquotes}
\usepackage{amssymb}
\usepackage[shortlabels]{enumitem}
\usepackage{fancybox, graphicx}
\usepackage{array}
\usepackage{hhline}
\usepackage{subfigure}
\usepackage{gensymb}
\usepackage{hyperref}
\usepackage{tikz}
\usepackage{amsmath}
\usepackage{leftidx}
\usepackage{wrapfig}
\usepackage{float}
\usepackage{amsmath} 
\usepackage{caption}
\usepackage{esvect}
\usepackage{siunitx}
\usepackage{commath}
\usepackage{bigints}
%Header & Footer

\pagestyle{fancy}
\fancyhead[LE]{\MyTitle}
\fancyhead[LO]{Mecánica}
%\fancyhead[RO]{\leftmark}
%\fancyhead[RE]{\leftmark}
\fancyfoot[L]{\raisebox{-1cm}{\includegraphics[height=2cm]{D:/KUKADisk/UDIMA/DocumentGraphics/LOGOUDIMA.jpg}}}
\fancyfoot[R]{Corregido:\\ Dra. Isabel Cristina Gil García}
%\fancyfoot[RO]{07/12/2018}


%Vars
\author{Alexander Sebastian Kalis}
\title{Ejercicios propuestos Unidades 4-6}


%DOC


\begin{document}

\begin{titlepage}

    \begin{center}

        \line(1,0){300}\\
        [0.2in]
        \huge{\bfseries {\MyTitle}}\\
        [1mm]
        \line(2,0){200}\\
        [0.75cm]
        \textsc{\LARGE Mecánica Clásica}\\
        [2cm]
        \includegraphics[height=10cm]{D:/KUKADisk/UDIMA/Mecanica/portada.jpg}\\
        [3cm]

    \end{center}

    \begin{flushright}

        {\MyAuthor}\\
        %Profesora: Dra. Isabel Cristina Gil García\\
        %Curso: Ingeniería de Organización Industrial\\
        %UDIMA\\
        \today        

    \end{flushright}
    
\end{titlepage}

%\tableofcontents \thispagestyle{empty}
%\newpage

%\section*{Introducción}


\newpage

\section*{Actividades}
\subsection*{Problema 1}

\subsubsection*{Enunciado del problema}

Una avioneta A vuela con una velocidad constante de $1000 ft/s$, su trayectoria describe un
arco de circunferencia con un radio de 10.000 ft. Otra avioneta B viaja en línea recta con una
velocidad de $500 ft/s$, que aumenta a razón de $50 ft/s^2$.\\
Determinar la velocidad y aceleración relativas de la avioneta A respecto al B. Expresarlo en unidades del SI. 

\begin{figure}[!htb]
    \centering
        \includegraphics[height=7cm]{p1e.PNG}
\end{figure}

\subsubsection*{Datos}
Se toma como sistema de referencia el frame mundo, representación esquemática:

\begin{figure}[!htb]
    \centering
    \includegraphics[height=6cm]{p1e2.PNG}
\end{figure}

\newpage
Entonces los datos serán los siguientes:\\

$\leftidx{^W}{v}_A = 1000ft/s \approx -305 m/s$\\
$\leftidx{^W}{v}_B = 500ft/s \approx 150m/s$\\
$\leftidx{^W}{a}_A = \cfrac{v^2}{r} \approx -\cfrac{305^2}{3050} \approx -30.5m/s^2$\\
$\leftidx{^W}{a}_B = 50ft/s^2 \approx  15 m/s^2$
\subsubsection*{Solución y explicación del problema}
Aplicamos la fórmula del movimiento relativo para determinar la velocidad relativa entre los aviones:

\begin{equation}
    \vec{v}_{A/B} = \vec{v}_A - \vec{v}_B \approx -305 - 150 \approx - 455 \vec{i} \ m/s 
\end{equation}

Hacemos lo mismo con la aceleración, teniendo en cuenta que la aceleración del avión A es centrípeta y en este caso es
paralela en sentido negativo al eje  $\leftidx{^A}{y}$ como se puede observar en el esquema.

\begin{equation}
    \vec{a}_{A/B} = \vec{a}_A - \vec{a}_B = -30.5 \vec{j} - 15 \vec{i} \ m/s^2
\end{equation}



\subsection*{Problema 2}

\subsubsection*{Enunciado del problema}

El diámetro exterior de una polea es de 0,8 m y la sección transversal de su borde es como vemos en la figura.
Determina la masa y el peso del borde conociendo que la polea es de acero y que la densidad del acero es $\rho=7,85 \cdot 10^3 kg/m^3$ 

\begin{figure}[!htb]
    \centering
        \includegraphics[height=7cm]{p2e.PNG}
\end{figure}

\newpage

\subsubsection*{Datos}

Primeramente realizamos la representación en 3D de la polea en Autocad según los datos del enunciado:\\
Radio exterior de la polea: $0.4m$.\\
Material y su densidad: $\rho_i = 7.85 \cdot 10^3kg/m^3$.\\

\begin{figure}[!htb]
    \centering
        \includegraphics[height=7cm]{p2f1.PNG}
\end{figure}

\subsubsection*{Solución y explicación del problema}

Se pide encontrar la masa y peso de la polea. Al tratarse de un sólido en revolución, se aplicará el segundo teorema de Pappus-Guldin para encontrar el volumen:

\begin{equation}
    V=\int 2\pi y \ dA = 2\pi y_G A
\end{equation}

Para ello será necesario conocer $y_G$, que es la distancia recorrida por el centro de gravedad cuando se engendra el volumen y el área generatriz, que es
el área de la sección transversal de la polea.

Determinamos el centro de gravedad de la sección transversal de la polea:

\begin{figure}[!htb]
    \centering
        \includegraphics[height=5cm]{p2f2.PNG}
\end{figure}

Como se puede observar en la figura, esta dispone de un eje de simetría. Sabemos entonces que $X_G$ debe situarse sobre ese eje con lo cual $X_G=50mm$.

Falta calcular $Y_G$:
\[
    A_1=A_3=20 \cdot 50=100mm^2    
\]
\[
    A_2=60 \cdot 20 = 120mm^2
\]
\[
    y_{G_1}=y_{G_3}=25mm
\]
\[
    y_{G_2}=30+10=40mm
\]

\begin{equation}
    Y_G=\cfrac{\sum y_{Gi}m_i}{\sum m_i} \rightarrow
    Y_G=\cfrac{100 \cdot 25 + 120 \cdot 40 + 100 \cdot 25   }{100 + 120 + 100} = 30.625mm
\end{equation}

Entonces nos queda el siguiente centro de gravedad:

\begin{figure}[!htb]
    \centering
        \includegraphics[height=5cm]{p2f3.PNG}
\end{figure}

Como ya disponemos de las áreas de los rectángulos individuales que forman la figura, la sumamos para obtener el área total $A$:

\begin{equation}
    A=320mm^2
\end{equation}

Ahora podemos restar al radio exterior de la polea la componente $Y$ del centro de gravedad de la sección transversal de la polea:

\begin{equation}
    Y_G=400-30.625=369.375mm
\end{equation}

Aplicamos Pappus-Guldin:

\begin{equation}
    V=2 \pi \cdot 369.375 \cdot 320 = 7.4267 \cdot 10^{-4} m^3
\end{equation}

Entonces su masa será 

\begin{equation}
    m=V\rho = 7.4267 \cdot 10^{-4} \cdot  7.85 \cdot 10^3 \approx 5.83Kg
\end{equation}
Y su peso en la tierra
\begin{equation}
    W=mg=5.83 \cdot 9.8 \approx 57.1 N
\end{equation}
\newpage


\subsection*{Problema 3}

\subsubsection*{Enunciado del problema}
Una placa delgada de acero de 4 mm de espesor se corta y se dobla para formar la siguiente pieza. Determinar el momento de inercia de la pieza respecto al origen de coordenadas

$\rho_{acero}=7850kg/m^3$

\begin{figure}[!htb]
    \centering
        \includegraphics[height=7cm]{p3e1.PNG}
\end{figure}

\subsubsection*{Datos}

De la imagen del enunciado disponemos de las medidas del objeto y su espesor.

$\rho_{acero}=7850kg/m^3$

\subsubsection*{Solución y explicación del problema}

Al tratarse de una figura compleja, empezaremos por obtener las masas descomponiendo en figuras simples y calculando sus momentos de inercia
consultando la tabla del libro:\\

\textbf{Semidisco}\\
\[
    m_{semidisco}=\rho \cdot V = \rho \cdot A \cdot d=7850 \cdot \cfrac{0.08^2\pi}{2} \cdot 0.004 \approx 0.32kg
\]

En este caso ya tenemos el origen de coordenadas ubicado en el centroide del <<disco completo>> y no hace falta aplicar Steiner. Simplemente usamos
las fórmulas relativas a un disco completo pero que tiene la mitad de su masa.

\[
    I_{x_{semidisco}}=\cfrac{m \cdot r^2}{2} = \cfrac{0.32 \cdot 0.08^2}{2}=    1.024  \cdot 10 ^{-3} kg \cdot m^2 \ \vec{i}
\]

\[
    I_{y_{semidisco}}=I_{z_{semidisco}}=\cfrac{m\cdot r^2}{4}=\cfrac{0.32\cdot 0.08^2}{4}=0.512 \cdot 10^{-3} kg \cdot m^2 \ \vec{jk}
\]


\textbf{Disco}\\
\[    
    m_{disco}=\rho \cdot V = 7850 \cdot 0.05^2  \pi \cdot 0.004 \approx 0.25kg
\]

\[
    I_{x_{disco}}=\cfrac{m \cdot r^2}{4} = \cfrac{0.25 \cdot 0.05^2}{4} =  -0.156   \cdot 10 ^{-3} kg \cdot m^2  \ \vec{i}
\]

En el caso de $I_y$ e $I_z$ tendremos que utilizar Steiner. Calculamos los momentos de inercia en el CG del disco y posteriormente trasladamos el frame 
de referencia.

\[
    I_{y_{disco}}=  I_{x_{disco}} + 0.25 \cdot 0.1^2= -2.656  \cdot 10 ^{-3} kg \cdot m^2 \ \vec{j} 
\]

\[
    I_{z_{disco}}= \cfrac{0.25 \cdot 0.05^2}{2}  + 0.25 \cdot 0.1^2= -2.812  \cdot 10 ^{-3} kg \cdot m^2 \ \vec{k}
\]

\textbf{Lámina}\\
\[
    m_{lamina}=\rho \cdot V = 7850 \cdot 0.2 \cdot 0.16 \cdot 0.004 \approx 1kg
\]  

Podemos calcular directamente $I_{x_{lamina}}$ ya que está sobre su centro de gravedad:

\[
    I_{x_{lamina}}=\cfrac{m \ b^2}{12}=\cfrac{0.16^2}{12}= 2.13 \cdot 10^{-3} kg \cdot m^2 \ \vec{i}
\]

Similarmente al semidisco, para los ejes $y$ y $z$ podemos distribuir la masa imaginando que la lámina tiene el doble de largo $2a$. De esta forma tenemos el origen de coordenadas 
en el centro de gravedad:

\[
    I_{y_{lamina}}=\cfrac{m \ 2a^2}{12}=\cfrac{0.4^2}{12}= 13.333 \cdot 10^{-3} kg \cdot m^2 \ \vec{j}
\] 
\[
    I_{z_{lamina}}=m\cfrac{2a^2 + b^2}{12}=\cfrac{0.4^2+0.16^2}{12}=  15.4666 \cdot 10^{-3} kg \cdot m^2 \ \vec{k}
\]

Finalmente realizamos el sumatorio de todos los momentos de inercia:

\begin{equation}
    I=[2.998 \vec{i} + 11.189 \vec{j} + 13.166 \vec{k}] \cdot 10^{-3}  kg \cdot m^2
\end{equation}

\newpage


\subsection*{Problema 4}

\subsubsection*{Enunciado del problema}
Determinar el momento de inercia y el radio de giro respecto al eje OX de la siguiente figura plana. Las cotas están dadas en centímetros. 
\begin{figure}[!htb]
    \centering
        \includegraphics[height=8cm]{p4e1.PNG}
\end{figure}
\subsubsection*{Datos}
Disponemos de las dimensiones de la figura gracias a la imagen.

A1: $A = 800mm^2, \ X_G=40mm, \ Y_G=5mm$

A2: $A= 300mm^2,\ X_G=5mm, \ Y_G=25mm$
\subsubsection*{Solución y explicación del problema}

Calculamos los momentos de inercia de la figura aplicando Steiner directamente:

\begin{equation}
    I=I_{f1}+I_{f2}=\left(\cfrac{80 \cdot 10^3}{12}+800 \cdot 5^2\right) + \left( \cfrac{10 \cdot 30^3}{12}+300 \cdot 25^2\right) \approx 236666.7 \ kg \cdot mm^2
\end{equation}

Ahora podemos calcular el radio de giro:

\begin{equation}
    k=\sqrt{\cfrac{I}{m}}=\sqrt{\cfrac{236666.7}{800+300}} \approx 14.68 \ mm \approx 0.01468 \ m
\end{equation}

\newpage

\subsection*{Problema extra}

Demostración numérica y con un programa CAD (AutoCAD, SolidWorks etc.) que calcule propiedades físicas, obtendrán 0,5 puntos extra
para el examen final presencial ordinario. Deben adjuntar pantallas en el PDF de la AEC (1 fichero) y fichero CAD (1 fichero).

Determine el Momento de Inercia centroidal de la siguiente figura plana. Las cotas están en cm.

\begin{figure}[!htb]
    \centering
        \includegraphics[height=7cm]{p5e1.PNG}
\end{figure}


Realizaremos el procedimiento paso a paso con ayuda de Autocad y posteriormente veremos si el resultado total coincide con el que calculamos:

\textbf{Triángulo}

Creamos el triángulo con las medidas del enunciado y lo convertimos en región. Aplicamos el comando $massprop$ para obtener los datos físicos:

\begin{figure}[!htb]
    \centering
        \includegraphics[width=1\textwidth]{p5triangulo.PNG}
\end{figure}

Obtenemos: $A=216.75 , \ CG_x=17, \ CG_y=5.67$
\newpage
\textbf{Rectángulo}

Seguimos el mismo procedimiento, dejamos el eje de coordenadas donde está para obtener el CG:

\begin{figure}[!htb]
    \centering
        \includegraphics[width=1\textwidth]{p5rectangulo.PNG}
\end{figure}

Obtenemos: $A=433.5, \ GC_x=38.25, \ GC_y=8.5$

\textbf{Círculo}

\begin{figure}[!htb]
    \centering
        \includegraphics[width=1\textwidth]{p5circulo.PNG}
\end{figure}

Obtenemos: $A=127.88, \ GC_x=38.25, \ GC_y=8.5$

Si calculamos el CG de la figura:

\begin{equation}
    X_G=\cfrac{\sum X_Gi \ m_i}{\sum m_i}=\cfrac{216.75 \cdot 17 + 433.5 \cdot 38.25 - 127.88 \cdot 38.25}{216.75 +  433.5 - 127.88}=29.43
\end{equation}

\begin{equation}
    Y_G=\cfrac{\sum Y_Gi \ m_i}{\sum m_i}=\cfrac{216.75 \cdot 5.67 + 433.5 \cdot 8.5 - 127.88 \cdot 8.5}{216.75 +  433.5 - 127.88}=7.32
\end{equation}

\newpage

Para comprobarlo en Autocad, convertimos todo el perímetro exterior de la figura en una sola región, el círculo en otra región, y subtraemos el círculo de la figura con el comando
$subtract$:

\begin{figure}[!htb]
    \centering
        \includegraphics[width=1\textwidth]{p5fx.PNG}
\end{figure}

Por último nos queda ver el el momento de inercia respecto al centroide de la figura. Para eso movemos el frame W en el punto anteriormente calculado y volvemos a aplicar $massprop$:


\begin{figure}[!htb]
    \centering
        \includegraphics[width=1\textwidth]{p5inertia.PNG}
\end{figure}

De esta forma obtenemos el siguiente resultado:

\[   
    I_x=13636.9097
\]

\[
    I_y=87283.1972
\]

\newpage

\subsection*{Formulario}

\textbf{Movimiento relativo}

$\vec{v}_{absoluta}=\vec{v}_{relativa}+\vec{v}_{arrastre}$

$\vec{v}_{A/B}=\vec{v}_{A}-\vec{v}_{B}$

$\vec{v}=\vec{\omega} \times  \vec{r}$

$\vec{v}_{absoluta}=\vec{v}_{relativa}+\vec{\omega} \times  \vec{r}$

$\vec{a}_{abs}=\vec{a}_{rel}+\vec{a}_{arrastre}+\vec{a}_{Coriolis}$

Fuerza de arrastre: $-m\vec{a}_{a}$

Fuerza tangencial: $-m(\vec{\alpha} \times \vec{r})$

Fuerza centrífuga: $-m(\vec{\omega}\times (\vec{\omega} \times \vec{r}))$

Fuerza de Coriolis: $-2m (\vec{\omega}\times\vec{v_{rel}})$

$\sum \vec{F}+\sum\vec{F}_{inercia}=m\vec{a}_{rel}$\\

\textbf{Centro de gravedad}

$XYZ_G=\cfrac{\bigint XYZ_i \ dm_i}{M}$

1er Teorema Pappus-Guldin: $A=\int 2 \pi y \ dl = 2 \pi \int y \ dl = 2 \pi y_G \ L$

2o Teorema Pappus-Guldin: $V=\int 2 \pi y \ dA = 2\pi \int y \ dA = 2\pi y_G A$

\newpage

\begin{figure}[!htb]
    \centering
        \includegraphics[height=.9\textheight]{centroide.jpg}
\end{figure}

\newpage

\textbf{Momentos de inercia}

Radio de giro: $k=\sqrt{\cfrac{I}{m}}$

Teorema de Steiner: $I_{OZ}=I_{GZ}+md^2$

Inercia en eje perpendicular: $I_z=I_x+I_y$

Teorema de Steiner, productos de inercia: $P_{X'Y'}=P_{XY}+\overline{x} \cdot \overline{y} \cdot A$


\begin{figure}[!htb]
    \centering
        \includegraphics[height=.8\textheight]{inercias1.jpg}
\end{figure}

\begin{figure}[!htb]
    \centering
        \includegraphics[height=.6\textheight]{inercias2.png}
\end{figure}
\end{document}