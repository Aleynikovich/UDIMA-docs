\documentclass{article}
\usepackage{lipsum}
\usepackage{authoraftertitle}
\usepackage[top=2cm,bottom=1.5cm,left=1.5cm, right=3cm,includeheadfoot]{geometry}
\usepackage{graphicx}
\usepackage[parfill]{parskip}
\usepackage{fancyhdr}
\usepackage[spanish]{babel}
\usepackage[T1]{fontenc}
\usepackage[utf8]{inputenc}
\usepackage{textcomp}
\usepackage{eurosym}
\usepackage{mathtools}
\usepackage{csquotes}
\usepackage{amssymb}
\usepackage[shortlabels]{enumitem}
\usepackage{fancybox, graphicx}
\usepackage{array}
\usepackage{hhline}
\usepackage{subfigure}
\usepackage{gensymb}
\usepackage{hyperref}
\usepackage{tikz}
\usepackage{amsmath}
\usepackage{wrapfig}
\usepackage{float}
\usepackage{amsmath} 
\usepackage{caption}
\usepackage{esvect}
\usepackage{siunitx}
\usepackage{commath}
%Header & Footer

\pagestyle{fancy}
\fancyhead[LE]{\MyTitle}
\fancyhead[LO]{Miroeconomía}
%\fancyhead[RO]{\leftmark}
%\fancyhead[RE]{\leftmark}
\fancyfoot[L]{\raisebox{-1cm}{\includegraphics[height=2cm]{D:/KUKADisk/UDIMA/DocumentGraphics/LOGOUDIMA.jpg}}}
\fancyfoot[R]{Corregido:\\Dr. Juan José Pintado Conesa}
%\fancyfoot[RO]{07/12/2018}


%Vars
\author{Alexander Sebastian Kalis}
\title{Ejercicios vinculados a las unidades 9 y 10}


%DOC


\begin{document}

\begin{titlepage}

    \begin{center}

        \line(1,0){300}\\
        [0.2in]
        \huge{\bfseries {\MyTitle}}\\
        [1mm]
        \line(2,0){200}\\
        [0.75cm]
        \textsc{\LARGE Miroeconomía}\\
        [2cm]
        \includegraphics[height=10cm]{D:/KUKADisk/UDIMA/Micoeconomia/AEC1/Graphics/portada.jpg}\\
        [3cm]

    \end{center}

    \begin{flushright}

        {\MyAuthor}\\
        %Profesora: Dra. Isabel Cristina Gil García\\
        %Curso: Ingeniería de Organización Industrial\\
        %UDIMA\\
        \today        

    \end{flushright}
    
\end{titlepage}

%\tableofcontents \thispagestyle{empty}
%\newpage

%\section*{Introducción}


\newpage

\section*{Problema 1}

Dada la siguiente curva de demanda del bien $X$, $QD = 150 - 5 P$ y la siguiente curva de oferta para dicho bien $QS = 120 + 10 P$.
Calcule el precio y la cantidad de equilibrio y el valor del excedente total en el mismo. Represéntelo
gráficamente. 

\textbf{Precio de equilibrio}

Para que exista equilibrio, se debe cumplir que $QD=QS$, entonces:
\begin{equation}
    150-5p=120+10P \rightarrow P^*=2
\end{equation}

\begin{equation}
    QD=150-5(2) \rightarrow Q^* = 140
\end{equation}

Averiguamos los puntos de corte de la curva de demanda:

\begin{equation}
    0=150-5P \rightarrow P=30
\end{equation}

Entonces, para $P=30$

\begin{equation}
    Q=150-5(30) \rightarrow Q=150
\end{equation}

Y los puntos de corte para la curva de la oferta:

\begin{equation}
    0=120+10P \rightarrow P=-12
\end{equation}

con lo cual para $P=-12$

\begin{equation}
    Q=120+10P \rightarrow Q=120
\end{equation}


\textbf{Valor del excedente total}

\begin{equation}
    EC=\frac{(30-2)\cdot 140}{2} = 1960
\end{equation}

\begin{equation}
    EP=\left|\frac{-12-2 \cdot 140}{2}\right|=980
\end{equation}

\begin{equation}
    ET=EC+IP=2940
\end{equation}

\newpage 
\textbf{Representación gráfica}

\begin{figure}[!htb]
    \centering
    \includegraphics[width=.90\textwidth]{p1f1.PNG}
    \caption{Representación gráfica de oferta-demanda}
\end{figure}

\section*{Problema 2}
A partir de la situación del ejercicio anterior, el Gobierno decide establecer un precio mínimo para el bien $X$
de 3 euros por unidad. Obtenga la pérdida de eficiencia provocada por esta medida. Represéntelo
gráficamente.

\begin{equation}
    QD=150-5(3)=135
\end{equation}

Para encontrar la pérdida de eficiencia se suman las areas que forman las curvas con el eje x.

\[
    A_1=0.5\cdot (15-2) \cdot (140-135)=32.5
\]
\[
    A_2=0.5\cdot (3-1.5) \cdot (140-135)=3.75
\]
\begin{equation}
    P_e=A_1+A_2=36.25    
\end{equation}

\begin{figure}[!htb]
    \centering
    \includegraphics[height=6cm]{p2f1.PNG}
    \caption{Representación gráfica de oferta-demanda con precio minimo 3}
\end{figure}


\section*{Problema 3}

Suponga que las curvas de oferta y demanda de carne de buey de Kobe de un país son,\\
$QD = 500.000 - 2.000 P$\\
$QS = 200.000 + 1.000 P$\\
A nivel internacional, el precio de la carne buey de Kobe es de 80 euros por kilogramo, el país representa una
porción pequeña en el mercado internacional de la carne.

Calcule el precio de equilibrio sin comercio, y calcule las importaciones o exportaciones una vez se abre el
país al comercio.

Para que esté equilibrado se debe cumplir que $QD=QS$ entonces:

\begin{equation}
    500000-2000P=200000+1000P \rightarrow P^*=100
\end{equation}

Al ser el precio nacional superior al internacional, el país se convertirá en importador. 
Se calcula entonces la cantidad de bien ofrecida a nivel nacional si se vende a precio internacional:

\[
    QS=200000+1000P=280000
\]

Y posteriormente la cantidad demandada nacionalmente pero a precio internacional:

\[
    QD=500000-2000P=340000
\]

La cantidad importada será entonces la diferencia entre cantidad demandada y de la ofrecida:

\begin{equation}
    QD-QS=60000
\end{equation}


\section*{Problema 4}

Suponga que las curvas de oferta y demanda de microchips son:\\
$QD = 800 – 5 P$\\
$QS = 590 + 2 P$\\
a) Obtenga el precio y la cantidad de equilibrio.\\
b) Si el Gobierno decide establecer un impuesto de 15 euros sobre los compradores, ¿cuál es la nueva
cantidad de equilibrio?, ¿a qué precio se vende y se compra el bien?, ¿cuál es la recaudación del
Gobierno por este impuesto?\\

\textbf{Apartado A}

De nuevo tenemo que $QS=QD$:
\begin{equation}
    800-5P=590+2P \rightarrow P^*=30, \  Q^*=650
\end{equation}

\textbf{Apartado B}

Introduciendo un impuesto de 15e fijos, se está desplazando la curva de oferta:

\begin{equation}
    PC=PV+15
\end{equation}

Volviendo a calcular los valores de equilibrio, obtenemos:

\begin{equation}
    800-5PV=590+2(PV+15) \rightarrow PV=25.7, \ PC=40.7, \ Q=596.5
\end{equation}

Entonces para esta cantidad el gobierno recaudará $QT*I=596.5 \cdot 15 = 8947.5e$

\end{document}