\documentclass{article}
\usepackage{lipsum}
\usepackage{authoraftertitle}
\usepackage[top=2cm,bottom=1.5cm,left=1.5cm, right=3cm,includeheadfoot]{geometry}
\usepackage{graphicx}
\usepackage[parfill]{parskip}
\usepackage{fancyhdr}
\usepackage[spanish]{babel}
\usepackage[T1]{fontenc}
\usepackage[utf8]{inputenc}
\usepackage{textcomp}
\usepackage{eurosym}
\usepackage{mathtools}
\usepackage{csquotes}
\usepackage{amssymb}
\usepackage[shortlabels]{enumitem}
\usepackage{fancybox, graphicx}
\usepackage{array}
\usepackage{hhline}
\usepackage{subfigure}
\usepackage{gensymb}
\usepackage{hyperref}
\usepackage{tikz}
\usepackage{amsmath}
\usepackage{wrapfig}
\usepackage{float}
\usepackage{amsmath} 
\usepackage{caption}
\usepackage{esvect}
\usepackage{siunitx}
\usepackage{commath}
%Header & Footer

\pagestyle{fancy}
\fancyhead[LE]{\MyTitle}
\fancyhead[LO]{Miroeconomía}
%\fancyhead[RO]{\leftmark}
%\fancyhead[RE]{\leftmark}
\fancyfoot[L]{\raisebox{-1cm}{\includegraphics[height=2cm]{D:/KUKADisk/UDIMA/DocumentGraphics/LOGOUDIMA.jpg}}}
\fancyfoot[R]{Corregido:\\Dr. Juan José Pintado Conesa}
%\fancyfoot[RO]{07/12/2018}


%Vars
\author{Alexander Sebastian Kalis}
\title{Ejercicios vinculados a las unidades 6, 7 y 8}


%DOC


\begin{document}

\begin{titlepage}

    \begin{center}

        \line(1,0){300}\\
        [0.2in]
        \huge{\bfseries {\MyTitle}}\\
        [1mm]
        \line(2,0){200}\\
        [0.75cm]
        \textsc{\LARGE Miroeconomía}\\
        [2cm]
        \includegraphics[height=10cm]{D:/KUKADisk/UDIMA/Micoeconomia/AEC1/Graphics/portada.jpg}\\
        [3cm]

    \end{center}

    \begin{flushright}

        {\MyAuthor}\\
        %Profesora: Dra. Isabel Cristina Gil García\\
        %Curso: Ingeniería de Organización Industrial\\
        %UDIMA\\
        \today        

    \end{flushright}
    
\end{titlepage}

%\tableofcontents \thispagestyle{empty}
%\newpage

%\section*{Introducción}


\newpage

\section*{Problema 1}

Dada la siguiente función de costes totales a corto plazo de una empresa:

\[
    CT=2x^2+4x+50  
\]
Se  le  pide  que señale  cuáles  son  los  costes  fijos, 
y  que  obtenga  la expresión de los costes totales medios, costes variables medios, costes fijos medios y coste marginal.


Costes fijos: $C_f=50$

Costes totales medios: $\frac{CT}{x}=2x+4+\frac{50}{x}$

Costes variables medios: $\frac{CV}{x}=2x+4$

Costes fijos medios: $\frac{C_f}{x}=\frac{50}{x}$

Coste marginal: $\frac{dCT}{dx}=\frac{4x}{4}$


\section*{Problema 2}

Cuál será el beneficio obtenido por el productor anterior, si opera en un mercado de competencia perfecta con la siguiente oferta y demanda:

\[
    x=640+3p
\]  

\[
    x=800-5p
\]

\textbf{Punto de equilibrio}

\begin{equation}
    640+3p=800-5p \rightarrow p=20
\end{equation}

\textbf{Precio igualado al Coste Marginal}

\begin{equation}
    20=4x+4 \rightarrow x=4
\end{equation}

\textbf{Beneficio}

\begin{equation}
    B=I-C \rightarrow
    B=20 \cdot 4 - \left(2x^2+4x+5-\right) \rightarrow
    B=-18
\end{equation}


\section*{Problema 3}

Obtenga la curva de oferta de la empresa que actúa en un mercado de competencia perfecta, y que tiene la siguiente función de costes a corto plazo: 
\[
    CTCP=25+50q^2    
\]

\begin{equation}
    CMg=\frac{CT}{q}=100q
\end{equation}

\begin{equation}
    P=100q
\end{equation}

Y su curva de oferta:
\begin{equation}
    q=\frac{P}{100}
\end{equation}

\section*{Problema 4}

En  un  mercado  de  competencia  perfecta  existen  20  empresas con idéntica función de costes: 

\[
    CT=x^2+4x+10
\]
Obtenga la curva de oferta de mercado de esta industria.\\

\begin{equation}
    CMg=\frac{dCT}{dx}=2x+4
\end{equation}

Entonces la curva de oferta es:
\begin{equation}
    P=2x+4 \rightarrow 2x=P-4 \rightarrow x=\frac{20P}{2}-40
\end{equation}
Y la curva de oferta de mercado:

\begin{equation}
    x=10P-40
\end{equation}


\section*{Problema 5}

Señale como pueden establecerse barreras de entrada en el mercado.\\


\textbf{Economías de escala}: El declive en el costo de las operaciones debido a un volumen más alto de producción

\textbf{Diferenciación de producto}: La fortaleza de la marca del producto como resultado de una comunicación efectiva de los beneficios hacia el público objetivo.

\textbf{Requerimientos de capital}: Se requieren recursos financieros para operar un negocio.

\textbf{Costes de cambio}: Los costos que el comprador hace una solo vez, debe hacer el cambio a un producto diferente.

\textbf{Acceso a todos los canales de distribución}: Un solo negocio los controla todos o son abiertos?

\textbf{Desventajas del costo independiente de la escala}: Cuando una compañía tiene ventajas que no pueden ser reproducidas por la competencia, como la propiedad de tecnología.

\textbf{Políticas gubernamentales}: Controles que el gobierno ha puesto en el mercado, como el requerimiento de licencias.


\newpage


\section*{Problema 6}
Obtenga la condición de máximo beneficio en un monopolio, y en un mercado de competencia perfecta.

Como podemos ver en el manual de la asignatura, ambas empresas buscan maximizar sus beneficios. Esto se puede conseguir
igualando los ingresos marginales a los costes marginales tal y como podemos consultar en la gráfica:

\begin{figure}[!htb]
    \centering
    \includegraphics[height=8cm]{competitivo.PNG}
    \caption{Maximización de beneficios de una empresa competitiva}
\end{figure}

\begin{figure}[!htb]
    \centering
    \includegraphics[height=8cm]{monopolio.PNG}
    \caption{Maximización de beneficios de una empresa monopolística}
\end{figure}

\newpage

\section*{Problema 7}

Razone  la  importancia  que  tiene  en  un  mercado  de  competencia monopolística la estrategia de “diferenciación del producto”.\\


La diferenciación del producto es una dotación de recurso que permite a la compañía obtener ventajas competitivas sobre otras empresas.
Al tener un producto diferente, este hace reduce la competencia pues no hay un bien sustitutivo con el que se pueda comparar.

La diferenciación permite desarrollar una posición en la que los potenciales clientes vean la marca como la única que produce ese bien,
o con esas características.

Una vez conseguida la diferenciación, el monopolio puede empezar a subir los precios de sus productos. Un claro ejemplo podría ser el de Apple inc.
Aunque fabrican dispositivos comunes como ordenadores y teléfonos móviles, estos tienen características únicas. Especialmente el sistema operativo iOS que
a diferencia de todos los otros fabricantes, no utiliza Android. Esto permite a la empresa cobrar un \textit{premium} ya que no hay otro competidor
que ofrezca lo mismo.



\end{document}