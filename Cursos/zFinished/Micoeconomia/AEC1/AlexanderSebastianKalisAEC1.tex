\documentclass{article}
\usepackage{lipsum}
\usepackage[backend=biber]{biblatex}
\addbibresource{aec2.bib}
\usepackage{authoraftertitle}
\usepackage[top=2cm,bottom=1.5cm,left=1.5cm, right=3cm,includeheadfoot]{geometry}
\usepackage{fancyhdr}
\usepackage[spanish]{babel}
\usepackage{mathtools}
\usepackage{csquotes}
\usepackage{amssymb}
\usepackage{fancybox, graphicx}
\usepackage{array}
\usepackage{hhline}
\usepackage{hyperref}
\usepackage{textcomp}
\usepackage{tikz}
\usepackage{amsmath}
\usepackage{eurosym}
\usepackage{wrapfig}
\usepackage{float}
\usepackage{amsmath}
\usepackage{caption}
\usepackage{esvect}
\usepackage{siunitx}
\usepackage{commath}
\newcommand{\ihat}{\textbf{\^\i}}
\newcommand{\jhat}{\textbf{\^\j}}
%Header & Footer

\pagestyle{fancy}
%\fancyhead[LE]{\MyTitle}
\fancyhead[LO]{Microeconomía}
%\fancyhead[RO]{\leftmark}
%\fancyhead[RE]{\leftmark}
%\fancyfoot[L]{\raisebox{-1cm}{\includegraphics[height=2cm]{C:/Users/XYZ/Dropbox/DocumentGraphics/LOGOUDIMA.jpg}}}
\fancyfoot[R]{Corregido:\\ Dr. Juan José Pintado Conesa}
%\fancyfoot[RO]{07/12/2018}


%Vars
\author{Alexander Sebastian Kalis}
\title{Tarea 1. Ejercicios vinculados a las Unidades 1 y 2}


%DOC


\begin{document}

\begin{titlepage}

    \begin{center}

        \line(1,0){300}\\
        [0.2in]
        \huge{\bfseries {\MyTitle}}\\
        [1mm]
        \line(2,0){200}\\
        [0.75cm]
        \textsc{\LARGE Microeconomía}\\
        [2cm]
        \includegraphics[height=10cm]{D:/KUKADisk/UDIMA/Micoeconomia/AEC1/Graphics/portada.jpg}\\
        [2.5cm]
        

    \end{center}

    \begin{flushright}

        Autor: {\MyAuthor}\\
        Profesor: Dr. Juan José Pintado Conesa\\
        Curso: Ingeniería de Organización Industrial\\
        UDIMA\\
        \today     

    \end{flushright}
    
\end{titlepage}

\tableofcontents \thispagestyle{empty}
\newpage

\newpage

\section{Actividades}
    
\subsection{Actividad 1}

En un mercado existen solamente cuatro consumidores A, B, C y D cuyas demandas individuales respecto al mismo bien, aparecen reflejadas en la siguiente Tabla para diferentes precios de ese bien. \\\\
\includegraphics[height=3cm]{D:/KUKADisk/UDIMA/Micoeconomia/AEC1/Graphics/Ejer1.PNG}\\

Se le solicita que obtenga la demanda de mercado de dicho bien, y su representación gráfica. \\\\

Podemos representar la demanda total mediante la suma horizontal de cada bien de cada consumidor:\\

\begin{table}[H]
    \begin{tabular}{llllll}
     Precio     & Consumidor A  & Consumidor B  & Consumidor C      &  Consumidor D     & Total mercado \\
     1 \euro{}  & 6 unidades    & 11 unidades   & 16 unidades       & 21 unidades       & 54 unidades            \\
     2 \euro{}  & 4 unidades    & 8 unidades    & 13 unidades       & 20 unidades       & 45 unidades            \\
     3 \euro{}  & 2 unidades    & 6 unidades    & 10 unidades       & 15 unidades       & 33 unidades           \\
     4 \euro{}  & 1 unidad      & 4 unidades    & 7 unidades        & 10 unidades       & 22 unidades            \\
     5 \euro{}  & 1 unidad      & 3 unidades    & 5 unidades        &  7 unidades       & 16 unidades
    \end{tabular}
\end{table}

Estos datos, representado gráficamente con la curva de demanda individual de mercado: \\\\

\includegraphics[height=7cm]{D:/KUKADisk/UDIMA/Micoeconomia/AEC1/Graphics/graficapreciodemanda.PNG}\\
\newpage

\subsection{Actividad 2}

La función de utilidad de un consumidor es $UT = 2xy$, dónde x e y son las cantidades respectivas de dos bienes. 
 
Los precios son para cada uno de ellos: \\

$P_x = 1 \EUR $


$P_y = 2 \EUR $\\

Se le pide que calcule la cantidad consumida de ambos bienes para una renta del consumidor de 100 euros. 
 
Nota: recuerde que la Utilidad Marginal del bien x es igual a la derivada de la función de utilidad con respecto al bien x. Igual ocurre con el bien y.\\\\

Sabemos que la utilidad marginal es el incremento de la utilidad que se consigue al aumentar la cantidad disponible, de forma que:\\

\begin{equation}
    Umg_x = \frac{\Delta U}{\Delta X} = 2y = 2 \cdot 2 = 4
\end{equation}
\begin{equation}
    Umg_y = \frac{\Delta U}{\Delta Y} = 2x = 2 \cdot1 = 2
\end{equation}

Sabiendo esto y la renta del consumidor de 100 $\EUR$ calculamos las cantidades de bien X e Y consumidos:\\

\begin{equation}
    100 = 2xy \rightarrow 100 = 4x \rightarrow x = 25
\end{equation}

\begin{equation}
    100 = 2xy \rightarrow 100 = 8y \rightarrow y = 12.5
\end{equation}

\subsection{Actividad 3}

Un consumidor adquiere 100 kilos de naranjas al año cuándo su precio es de 2 euros el kilo. Por otro lado, cuándo el precio sube a 3 euros demanda 80 kilos por año. \\
Calcule y califique a la elasticidad precio demanda de este consumidor. \\\\

Conocemos que la elasticidad del precio de la demanda se expresa como:\\

\begin{equation}
    \eta_p = \cfrac{\left(\cfrac{\Delta Q_d}{Q_d}\right)}{\left(\cfrac{\Delta P}{P}\right)}
\end{equation}

Donde donde el numerador es la variación de la cantidad del producto demandado y el denominador la variación del precio. Entonces:

\begin{equation}
    \eta_p = \cfrac{\cfrac{80-100}{100}}{\cfrac{3-2}{2}}=-0.2
\end{equation}

Como es lógico, obtenemos un valor negativo ya que al incrementar el precio de las naranjas, el consumidor reduce su demanda. Con lo cual podemos decir que la demanda respecto
al precio es \textbf{inelástica} ya que podemos observar que $0 < \left| \eta_p \right| < -1 $



\subsection{Actividad 4}

En la economía EVASA, durante el año 20X0, la Renta Nacional fue de 2 Billones de euros. En ese mismo año, las ventas de una determinada marca de automóviles ascendieron a 100.000 unidades. 
 
Durante el año 20X1, la Renta Nacional subió a 2,1 Billones (en euros constantes), y las ventas de ese modelo aumentaron a 106.500 unidades. 
 
¿Cómo calificaría a este bien?\\\\

Para calificar un bien, utilizamos la elasticidad renta de la demanda, que relaciona la variación porcentual de la cantidad demandada de un bien y la variaciación
de la renta que originó ese cambio. Utilizamos la siguiente fórmula:\\

\begin{equation}
    \eta_R = \cfrac{\cfrac{\Delta Q}{Q}}{\cfrac{\Delta R}{R}}
\end{equation}

Sustituyendo los datos del enunciado, obtenemos:\\

\begin{equation}
    \eta_R=\cfrac{\cfrac{106500-100000}{106500}}{\cfrac{2.1-2}{2.1}}=1.5
\end{equation}

Ya que $\eta_R > 1$, podemos calificar el bien como un \textbf{bien de lujo}.

\subsection{Actividad 5}

Como se ha visto en la actividad 3, la elasticidad precio demanda viene dada por la ecuación 5. Sustituyendo los datos del enunciado obtenemos:

\begin{equation}
    -1.2=\cfrac{\cfrac{\Delta Q_d}{Q_d}}{-0.5} \rightarrow \cfrac{\Delta Q_d}{Q_d}=0.6
\end{equation}

Lo que significa que el cambio porcentual demandado es del 60 \% y consecuentemente el incremento de gasto en ese producto es de:

\begin{equation}
    10000 \cdot \frac{60}{100} \cdot 50 = 300000
\end{equation}

Dado que tras la bajada de precio aumentará el gasto en 300.000 euros, entonces el gasto total será de $10000 \cdot 50 + 300000 = 800000 \EUR$
\end{document}