\documentclass{article}
\usepackage{lipsum}
\usepackage{authoraftertitle}
\usepackage[top=2cm,bottom=1.5cm,left=1.5cm, right=3cm,includeheadfoot]{geometry}
\usepackage{graphicx}
\usepackage[parfill]{parskip}
\usepackage{fancyhdr}
\usepackage[spanish]{babel}
\usepackage[T1]{fontenc}
\usepackage[utf8]{inputenc}
\usepackage{textcomp}
\usepackage{eurosym}
\usepackage{mathtools}
\usepackage{csquotes}
\usepackage{amssymb}
\usepackage[shortlabels]{enumitem}
\usepackage{fancybox, graphicx}
\usepackage{array}
\usepackage{hhline}
\usepackage{subfigure}
\usepackage{gensymb}
\usepackage{hyperref}
\usepackage{tikz}
\usepackage{amsmath}
\usepackage{wrapfig}
\usepackage{float}
\usepackage{amsmath} 
\usepackage{caption}
\usepackage{esvect}
\usepackage{siunitx}
\usepackage{commath}
%Header & Footer

\pagestyle{fancy}
\fancyhead[LE]{\MyTitle}
\fancyhead[LO]{Miroeconomía}
%\fancyhead[RO]{\leftmark}
%\fancyhead[RE]{\leftmark}
\fancyfoot[L]{\raisebox{-1cm}{\includegraphics[height=2cm]{D:/KUKADisk/UDIMA/DocumentGraphics/LOGOUDIMA.jpg}}}
\fancyfoot[R]{Corregido:\\Dr. Juan José Pintado Conesa}
%\fancyfoot[RO]{07/12/2018}


%Vars
\author{Alexander Sebastian Kalis}
\title{Ejercicios vinculados a las unidades 3, 4 y 5}


%DOC


\begin{document}

\begin{titlepage}

    \begin{center}

        \line(1,0){300}\\
        [0.2in]
        \huge{\bfseries {\MyTitle}}\\
        [1mm]
        \line(2,0){200}\\
        [0.75cm]
        \textsc{\LARGE Miroeconomía}\\
        [2cm]
        \includegraphics[height=10cm]{D:/KUKADisk/UDIMA/Micoeconomia/AEC1/Graphics/portada.jpg}\\
        [3cm]

    \end{center}

    \begin{flushright}

        {\MyAuthor}\\
        %Profesora: Dra. Isabel Cristina Gil García\\
        %Curso: Ingeniería de Organización Industrial\\
        %UDIMA\\
        \today        

    \end{flushright}
    
\end{titlepage}

%\tableofcontents \thispagestyle{empty}
%\newpage

%\section*{Introducción}


\newpage

\section*{Problema 1}
Analice como operan el efecto sustitución, el efecto renta y el efecto total en el caso de un:\\
a) Bien Normal.\\
b) Bien Inferior.\\ 
c) Bien Giffen\\

Podemos analizarlo utilizando la tabla que encontramos en el manual de la asignatura:\\
\begin{figure}[htb!]
    \centering
    \includegraphics[height=4cm]{D:/KUKADisk/UDIMA/Micoeconomia/AEC2/tabla1.png}
    \caption{Efectos sobre la cantidad demandada de un bien, provocados por cambios de su precio}
    \label{}
\end{figure}

En la tabla se puede observar como el efecto sustitución tiene el mismo efecto para los tres tipos de bienes,
cuando el precio sube, baja la cantidad demandada del mismo, a su vez, cuando el precio baja, aumenta la
cantidad demandada.\\
Cuando el precio de un bien normal sube, la demanda disminuye con todos sus efectos.\\
En el caso de un bien inferior, una subida del precio hace que la cantidad demandada de los mismos
disminuya en los casos del efecto de sustitución y del total.\\
En el caso de un bien Giffen, el sentido de los efectos renta y sustitución es igual al de los bienes inferiores,
sin embargo, podemos observar que ante una subida del precio del bien, la cantidad demandada aumenta.
Esto es debido a la magnitud del efecto renta es superior a la del efecto sustitución.



\section*{Problema 2}

La empresa ESTMARSA dedicada a la fabricación de Microchips utiliza únicamente dos factores productivos: “a” y “b”. 
 
Su función de producción viene determinada por la ecuación: 

\begin{equation*}
    y = 2ab 
\end{equation*}


Donde y es la cantidad de producto obtenido (número de microchips fabricados), y “a” y “b” son las cantidades utilizadas de cada uno de los factores productivos. 
 
El precio del factor “a” es de 1 \euro{}, y el de “b” de 2 \euro{} (ambos por unidad de factor). 
 
Para una producción de 100 microchips, se le pide que determine: 
 
a) El empleo de cada uno de los factores.\\
b) El mínimo coste en el que incurre. \\

\newpage

Podemos utilizar la ecuación donde minimizamos los costes de producción:

\begin{equation}
     \cfrac{PMg_b}{w}=\cfrac{PMg_a}{r}
\end{equation}

La productividad marginal: $f(a,b) - 2ab$ entonces $PMg_a=2b$ y ${PMg_b=2a}$. Sustituyendo en la anterior fórmula obtenemos:\\


\begin{equation}
    \cfrac{2a}{2b}=\cfrac{2}{1}
\end{equation}

Despejando y sustituyendo el valor de 100 microchips obtenemos:

\begin{equation}
    100=2ab \rightarrow
    b=\sqrt{25}=\rightarrow
    b=5
\end{equation}

y

\begin{equation}
    100=2a(5) \rightarrow 
    100=10a \rightarrow
    a=10
\end{equation}

Una vez sabido esto podemos calcular el coste mínimo que incurre utilizando la fórmula:

\begin{equation}
    CT=rPa+wPb \rightarrow
    1(10)+2(5)=20
\end{equation}

Luego 20eur es el coste mínimo.


\section*{Problema 3}

La empresa GONPABSA que tiene unos costes fijos de 100 u.m., presenta la siguiente
estructura de costes: 

\begin{figure}[htb!]
    \centering
    \includegraphics[height=4cm]{D:/KUKADisk/UDIMA/Micoeconomia/AEC2/tabla2a.png}
\end{figure}

Se le pide que represente en una tabla, para cada volumen de producción, el coste total, el coste
marginal, el coste total medio, el coste variable medio, y el coste fijo medio.

\newpage

Se realiza la actividad con una tabla en Excel:
\begin{figure}[htb!]
    \centering
    \includegraphics[height=4cm]{D:/KUKADisk/UDIMA/Micoeconomia/AEC2/tabla2b.png}
\end{figure}

\section*{Problema 4}

Si nos encontramos en los siguientes situaciones: 
\begin{figure}[htb!]
    \centering
    \includegraphics[height=2cm]{D:/KUKADisk/UDIMA/Micoeconomia/AEC2/tabla4a.png}
\end{figure}

Señale cuál es el producto medio del factor variable en cada una de las mismas. 

Podemos calcular el producto medio del factor variable utilizando la fórmula
\begin{equation}
    PMe_L=\cfrac{PT_L}{L}
\end{equation}

El resultado se muestra en la tabla:
\begin{figure}[htb!]
    \centering
    \includegraphics[height=2cm]{D:/KUKADisk/UDIMA/Micoeconomia/AEC2/tabla4b.png}
\end{figure}

\section*{Problema 5}

Dadas las siguientes posibilidades en un proceso productivo: 

\begin{figure}[htb!]
    \centering
    \includegraphics[height=2cm]{D:/KUKADisk/UDIMA/Micoeconomia/AEC2/tabla5a.png}
\end{figure}

Señale y argumente que clase de rendimientos a escala experimenta al pasar de la situación A a la
B. 

Se puede observar la variación proporcional del factor de producción $m=2$.

Esto significa que al pasar de situación A a situación B tenemos un rendimiento creciente de escala, aplicamos la fórmula:

\begin{equation}
    f(mK,mL)>mf(K,L) \rightarrow
    f(2K,2L)>2f(K,L)
\end{equation}

Entonces se multiplica el producto obtenido en un factor de $3.33$.


\section*{Problema 6}

Sea la función de costes: $CT = 100 + 10 q$\\
Se le pide que obtenga:\\
a) El coste total medio.\\
b) El coste variable medio.\\
c) El coste fijo medio.\\
d) El coste marginal. \\

Se representa con una tabla de valores:

\begin{figure}[htb!]
    \centering
    \includegraphics[height=4cm]{D:/KUKADisk/UDIMA/Micoeconomia/AEC2/tabla6a.png}
\end{figure}


\end{document}