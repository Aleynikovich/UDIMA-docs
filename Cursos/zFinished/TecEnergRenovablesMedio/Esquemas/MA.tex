%%%%%%%%%%%%%%%%%%%%%%%%%%%%%%%%%%%%%%%%%%%%%%%%%%%%%%%%%%%%%%%%%%%%%%%%%%%%%%%%
% PREÁMBULO - Configuración del documento
%%%%%%%%%%%%%%%%%%%%%%%%%%%%%%%%%%%%%%%%%%%%%%%%%%%%%%%%%%%%%%%%%%%%%%%%%%%%%%%%
\documentclass[a4paper, 11pt]{article}

% --- Paquetes necesarios ---
\usepackage[spanish]{babel}
\usepackage[utf8]{inputenc}
\usepackage{graphicx}
\usepackage{amsmath}
\usepackage[margin=2.5cm]{geometry}
\usepackage{hyperref}
\usepackage{titlesec}
\usepackage{amssymb} % Para símbolos matemáticos adicionales

% --- Configuración de títulos ---
\titleformat{\section}{\Large\bfseries}{\thesection}{1em}{}
\titleformat{\subsection}{\large\bfseries}{\thesubsection}{1em}{}
\titleformat{\subsubsection}{\normalsize\bfseries}{\thesubsubsection}{1em}{}

% --- Título del Documento ---
\title{Resumen Completo de Tecnología Energética y Medioambiente}
\author{Material de Estudio}
\date{\today}

%%%%%%%%%%%%%%%%%%%%%%%%%%%%%%%%%%%%%%%%%%%%%%%%%%%%%%%%%%%%%%%%%%%%%%%%%%%%%%%%
% INICIO DEL DOCUMENTO
%%%%%%%%%%%%%%%%%%%%%%%%%%%%%%%%%%%%%%%%%%%%%%%%%%%%%%%%%%%%%%%%%%%%%%%%%%%%%%%%
\begin{document}

\maketitle
\tableofcontents
\newpage

%%%%%%%%%%%%%%%%%%%%%%%%%%%%%%%%%%%%%%%%%%%%%%%%%%%%%%%%%%%%%%%%%%%%%%%%%%%%%%%%
% UNIDAD 1: ECOLOGÍA
%%%%%%%%%%%%%%%%%%%%%%%%%%%%%%%%%%%%%%%%%%%%%%%%%%%%%%%%%%%%%%%%%%%%%%%%%%%%%%%%

% --- Página del Esquema Visual ---
\newgeometry{margin=0pt}
\thispagestyle{empty}
\begin{figure}[p]
    \centering
    \includegraphics[width=\paperwidth, height=\paperheight, keepaspectratio]{MA1.png}
\end{figure}
\restoregeometry
\newpage

% --- Resumen de Texto Detallado ---
\section*{Unidad 1: ¿Qué es la ecología? La ciencia que entiende el planeta}

\subsection*{Introducción a la Ecología y el Medioambiente}
\begin{itemize}
    \item \textbf{Definición de Medioambiente}: Según la Real Academia Española (RAE), es el "conjunto de circunstancias o condiciones exteriores a un ser vivo que influyen en su desarrollo y en sus actividades". Esto engloba factores físicos, ecológicos, económicos, culturales e incluso estéticos.
    \item \textbf{Origen de la Preocupación Ambiental}: Aunque el estudio de la naturaleza se remonta a Aristóteles, la percepción del impacto del progreso humano en el medioambiente surgió a mediados del siglo XX, consolidando la ecología como ciencia.
    \item \textbf{Pioneros y Figuras Clave}:
    \begin{itemize}
        \item \textbf{Alexander von Humboldt (s. XIX)}: Considerado uno de los padres de la ciencia ecológica. Buscaba demostrar que todo en la naturaleza estaba interconectado y fue de los primeros en describir científicamente cómo el ser humano altera su entorno de forma irreversible.
        \item \textbf{Ernst Haeckel (1866)}: Acuñó el término \textbf{ecología} como "la ciencia de las relaciones de los organismos con su entorno".
        \item \textbf{Rachel Carson (1962)}: Con su libro \textit{Primavera silenciosa}, alertó sobre los efectos nocivos de los pesticidas en el medioambiente.
        \item \textbf{Eugene Odum}: Fundamental en el desarrollo de la ecología de sistemas y de poblaciones.
    \end{itemize}
\end{itemize}

\subsection*{La Problemática Ambiental Actual}
\begin{itemize}
    \item \textbf{Factores Desencadenantes}: Principalmente el \textbf{aumento de la población} ("explosión demográfica" post-siglo XX) y la \textbf{globalización}.
    \item \textbf{Consecuencias del Crecimiento Poblacional}:
    \begin{itemize}
        \item \textbf{Sobreexplotación de Recursos Naturales}: Agotamiento de agua dulce, degradación del suelo, sobrepesca y deforestación a nivel mundial.
        \item \textbf{Degradación del Medioambiente}: Contaminación del aire y del agua (eutrofización), pérdida de biodiversidad, erosión del paisaje y desertificación.
        \item \textbf{Cambio Climático}: Causado por el aumento de la concentración de \textbf{Gases de Efecto Invernadero (GEI)} por la actividad humana.
        \begin{itemize}
            \item \textbf{Principales GEI y su Potencial de Calentamiento (GWP)}:
                \begin{itemize}
                    \item \textbf{Dióxido de Carbono ($CO_2$)}: El más abundante, de la quema de combustibles fósiles.
                    \item \textbf{Metano ($CH_4$)}: GWP de 28. De agricultura, ganadería y vertederos.
                    \item \textbf{Óxido Nitroso ($N_2O$)}: GWP de 265. De fertilizantes y quema de combustibles.
                    \item \textbf{Otros gases} (HFC, PFC, $SF_6$): GWP muy elevados (ej. $SF_6$ tiene un GWP de 23000).
                \end{itemize}
            \item \textbf{Efectos del Cambio Climático}: Aumento de la temperatura global, deshielo de casquetes polares, alteración de patrones de precipitación y afectación a la biodiversidad.
        \end{itemize}
    \end{itemize}
    \item \textbf{Implicaciones Socioeconómicas}: Inseguridad alimentaria, desigualdad en el acceso a recursos, migraciones y conflictos.
\end{itemize}

\subsection*{Ecología vs. Ecologismo y el Desarrollo Sostenible}
\begin{itemize}
    \item \textbf{Diferencia Clave}: La \textbf{ecología} es una ciencia interdisciplinaria que estudia las relaciones en los ecosistemas. El \textbf{ecologismo} es un movimiento social y político que defiende la protección del medioambiente.
    \item \textbf{Desarrollo Sostenible} (Informe Brundtland, 1987): "El desarrollo que satisface las necesidades de la generación presente sin comprometer la capacidad de las generaciones futuras para satisfacer sus propias necesidades". Su pionero fue Ernst Friedrich Schumacher.
\end{itemize}

\subsection*{Hitos y Acuerdos Internacionales}
\begin{itemize}
    \item \textbf{Conferencia de Estocolmo (1972)}: Primera cumbre mundial sobre medioambiente. Creó el PNUMA (Programa de las Naciones Unidas para el Medio Ambiente).
    \item \textbf{Cumbre de la Tierra de Río (1992)}: Creó las 3 grandes convenciones (Cambio Climático - CMNUCC, Biodiversidad y Desertificación) y la Agenda 21. Estableció las Conferencias de las Partes (COP).
    \item \textbf{Protocolo de Kioto (1997)}: Primer acuerdo legalmente vinculante para reducir las emisiones de 6 GEI en un 5\% para 2012 respecto a 1990.
    \item \textbf{Acuerdo de París (2015)}: Sustituye a Kioto. Objetivo: limitar el calentamiento global por debajo de los 2$^{\circ}$C.
    \item \textbf{Agenda 2030 (2015)}: Plan de acción de la ONU que incluye los \textbf{17 Objetivos de Desarrollo Sostenible (ODS)}.
\end{itemize}

\subsection*{Análisis del Ciclo de Vida (ACV)}
\begin{itemize}
    \item \textbf{Definición}: Herramienta sistemática para evaluar el impacto ambiental de un producto, proceso o servicio a lo largo de todo su ciclo de vida ("de la cuna a la tumba"). Sirve para evitar el \textit{greenwashing} (lavado ecológico).
    \item \textbf{Normativa}: ISO 14040 y 14044.
    \item \textbf{Fases de un ACV}: 1. Definición de objetivos y alcance (incluyendo la \textbf{unidad funcional}); 2. Análisis del Inventario del Ciclo de Vida (ICV); 3. Evaluación del Impacto del Ciclo de Vida (EICV); 4. Interpretación de resultados.
\end{itemize}

\subsection*{Ecoetiquetas}
\begin{itemize}
    \item \textbf{Definición}: Sellos que identifican productos o servicios respetuosos con el medioambiente.
    \item \textbf{Clasificación (según norma ISO 14020)}:
    \begin{itemize}
        \item \textbf{Tipo 1 (ISO 14024)}: Certificadas por una tercera entidad independiente. Analizan todo el ciclo de vida. Alta credibilidad (ej. Etiqueta Ecológica Europea, Blue Angel).
        \item \textbf{Tipo 2 (ISO 14021)}: Autodeclaraciones informativas del fabricante sobre aspectos concretos (ej. "reciclable").
        \item \textbf{Tipo 3 (ISO 14025)}: Declaraciones ambientales de producto (DAP o EPD), que ofrecen información cuantitativa detallada basada en el ACV.
    \end{itemize}
\end{itemize}
\newpage

%%%%%%%%%%%%%%%%%%%%%%%%%%%%%%%%%%%%%%%%%%%%%%%%%%%%%%%%%%%%%%%%%%%%%%%%%%%%%%%%
% UNIDAD 2: CONTAMINACIÓN ATMOSFÉRICA
%%%%%%%%%%%%%%%%%%%%%%%%%%%%%%%%%%%%%%%%%%%%%%%%%%%%%%%%%%%%%%%%%%%%%%%%%%%%%%%%

% --- Página del Esquema Visual ---
\newgeometry{margin=0pt}
\thispagestyle{empty}
\begin{figure}[p]
    \centering
    \includegraphics[width=\paperwidth, height=\paperheight, keepaspectratio]{MA2.png}
\end{figure}
\restoregeometry
\newpage

% --- Resumen de Texto Detallado ---
\section*{Unidad 2: Contaminación atmosférica}

\subsection*{La Atmósfera}
\begin{itemize}
    \item \textbf{Estructura}: La atmósfera es una envoltura gaseosa heterogénea. El 75\% de su masa se concentra en los primeros 11 km.
    \item \textbf{Capas por Temperatura}:
    \begin{itemize}
        \item \textbf{Troposfera}: Capa más cercana (hasta 18 km). Contiene la mayor parte del vapor de agua y en ella ocurren los fenómenos meteorológicos. La temperatura disminuye con la altitud a razón de 0,64 °C cada 100 metros.
        \item \textbf{Estratosfera}: Hasta 50 km. Contiene la \textbf{capa de ozono}, que absorbe radiación UV de longitud de onda \textbf{menor de 360 nm}, provocando que la temperatura aumente con la altitud.
        \item \textbf{Mesosfera}: Hasta 85 km. Es la capa más fría, donde se desintegran los meteoritos. La temperatura disminuye con la altitud.
        \item \textbf{Termosfera}: Entre 85 y 600 km. La temperatura aumenta drásticamente por la ionización de gases (absorción de rayos X y gamma). Esta capa, también llamada \textbf{ionosfera}, refleja las ondas de radio. Aquí orbita la Estación Espacial Internacional.
        \item \textbf{Exosfera}: Capa más externa, donde los gases se dispersan hacia el espacio.
    \end{itemize}
    \item \textbf{Composición Química}:
    \begin{itemize}
        \item \textbf{Homosfera} (0-100 km): Composición constante: 78,084\% Nitrógeno ($N_2$), 20,946\% Oxígeno ($O_2$), 0,934\% Argón (Ar), 0,04\% $CO_2$.
        \item \textbf{Heterosfera} (>100 km): Composición estratificada y variable por la disociación de moléculas.
    \end{itemize}
\end{itemize}

\subsection*{Contaminación Atmosférica}
\begin{itemize}
    \item \textbf{Definición Legal}: "La presencia en el aire de materias, sustancias o formas de energía que impliquen molestia grave, riesgo o daño para personas, medioambiente y demás bienes".
    \item \textbf{Conceptos}: \textbf{Emisión} es la cantidad de contaminante vertido desde un foco; \textbf{Inmisión} es la concentración que llega al receptor.
    \item \textbf{Clasificación}: \textbf{Contaminantes primarios} (emitidos directamente, ej. $SO_2$) y \textbf{secundarios} (formados en la atmósfera, ej. ozono troposférico).
\end{itemize}

\subsection*{Principales Contaminantes Químicos}
\begin{itemize}
    \item \textbf{Dióxido de azufre ($SO_2$)}: De la quema de combustibles fósiles con azufre. Causa la lluvia ácida.
    \item \textbf{Óxidos de nitrógeno ($NO_x$)}: De la combustión a alta temperatura (tráfico, industria). Precursores de lluvia ácida y smog.
    \item \textbf{Monóxido de carbono (CO)}: De la combustión incompleta. Es tóxico porque se une a la hemoglobina con una afinidad 210 veces mayor que el oxígeno, impidiendo su transporte.
    \item \textbf{Partículas en suspensión (PM)}: Sólidos o líquidos. Las \textbf{PM10} (<10µm) y \textbf{PM2,5} (<2,5µm) son las más peligrosas para la salud respiratoria y cardiovascular.
    \item \textbf{Ozono troposférico ($O_3$)}: "Ozono malo". Contaminante secundario formado por $NO_x$ y COV con luz solar. Causa problemas respiratorios y daños a la vegetación.
\end{itemize}

\subsection*{Efectos Globales de la Contaminación Atmosférica}
\begin{itemize}
    \item \textbf{Lluvia Ácida}: Formada por la reacción de $SO_x$ y $NO_x$ con el agua atmosférica, creando ácido sulfúrico y nítrico. Acidifica suelos y aguas.
    \item \textbf{Destrucción de la Capa de Ozono}: Los CFCs liberan cloro en la estratosfera, que destruye el ozono ($O_3$) protector. El \textbf{Protocolo de Montreal} prohibió estos compuestos.
    \item \textbf{Smog Urbano}:
    \begin{itemize}
        \item \textbf{Fotoquímico}: Mezcla de $NO_x$, COV y luz solar, que produce ozono. Color pardo-rojizo.
        \item \textbf{Sulfuroso (Industrial)}: Mezcla de $SO_2$ y partículas de la quema de carbón.
    \end{itemize}
\end{itemize}
\newpage

%%%%%%%%%%%%%%%%%%%%%%%%%%%%%%%%%%%%%%%%%%%%%%%%%%%%%%%%%%%%%%%%%%%%%%%%%%%%%%%%
% UNIDAD 3: CONTAMINACIÓN DEL AGUA
%%%%%%%%%%%%%%%%%%%%%%%%%%%%%%%%%%%%%%%%%%%%%%%%%%%%%%%%%%%%%%%%%%%%%%%%%%%%%%%%

% --- Página del Esquema Visual ---
\newgeometry{margin=0pt}
\thispagestyle{empty}
\begin{figure}[p]
    \centering
    \includegraphics[width=\paperwidth, height=\paperheight, keepaspectratio]{MA3.png}
\end{figure}
\restoregeometry
\newpage

% --- Resumen de Texto Detallado ---
\section*{Unidad 3: Contaminación del agua}

\subsection*{El Agua y sus Propiedades}
\begin{itemize}
    \item \textbf{Ciclo Hidrológico}: Proceso continuo de evaporación, condensación, precipitación y escorrentía del agua, impulsado por la energía solar.
    \item \textbf{Propiedades Físico-Químicas Clave}:
    \begin{itemize}
        \item \textbf{Estructura Molecular}: La molécula de agua ($H_2O$) es polar, lo que le permite formar puentes de hidrógeno, siendo un "disolvente universal".
        \item \textbf{Anomalía de la Densidad}: El hielo (sólido) es menos denso que el agua líquida, por lo que flota. La densidad máxima del agua se da a 3,983 °C.
        \item \textbf{pH}: El agua pura es neutra (pH=7). La acidificación de los océanos, por la absorción de $CO_2$ atmosférico, dificulta la formación de caparazones de carbonato de calcio en organismos marinos.
        \item \textbf{Dureza}: Concentración de iones disueltos de calcio ($Ca^{2+}$) y magnesio ($Mg^{2+}$). Se clasifica en temporal (eliminable al hervir) y permanente.
    \end{itemize}
    \item \textbf{Parámetros de Calidad del Agua}:
    \begin{itemize}
        \item \textbf{Oxígeno Disuelto (OD)}: Esencial para la vida acuática (mínimo $\approx$ 5 mg/L para peces). La materia orgánica en descomposición consume OD.
        \item \textbf{Demanda Biológica de Oxígeno (DBO)}: Cantidad de oxígeno consumido por microorganismos para descomponer la materia orgánica biodegradable. Se mide en 5 días ($DBO_5$).
        \item \textbf{Demanda Química de Oxígeno (DQO)}: Cantidad de oxígeno necesaria para oxidar toda la materia orgánica (biodegradable y no) mediante un agente químico. Siempre $DQO \ge DBO$. La relación $DQO/DBO_5$ indica la tratabilidad biológica del agua residual.
    \end{itemize}
\end{itemize}

\subsection*{Tipos de Contaminantes y Fenómenos Asociados}
\begin{itemize}
    \item \textbf{Clasificación de Contaminantes}:
    \begin{itemize}
        \item \textbf{Físicos}: Sedimentos, residuos sólidos, contaminación térmica (aumento de temperatura).
        \item \textbf{Químicos}: Metales pesados (plomo, mercurio, cadmio), purines (exceso de nitratos y fosfatos), jabones y detergentes, petróleo, pesticidas y plásticos.
        \item \textbf{Biológicos}: Organismos patógenos como bacterias (\textit{E. coli}), virus y parásitos.
    \end{itemize}
    \item \textbf{Fenómenos Principales}:
    \begin{itemize}
        \item \textbf{Eutrofización}: Enriquecimiento excesivo de un cuerpo de agua con nutrientes (nitrógeno y fósforo), que provoca una proliferación masiva de algas. Al morir y descomponerse, estas algas agotan el oxígeno del agua, causando la muerte de la fauna acuática.
        \item \textbf{Acidificación}: Disminución del pH del agua, principalmente por lluvia ácida o drenaje ácido de minas.
        \item \textbf{Microplásticos}: Partículas de plástico de menos de 5 mm que contaminan los ecosistemas acuáticos e ingresan en la cadena alimentaria. Provienen de la degradación de plásticos mayores y de productos como cosméticos o fibras textiles.
    \end{itemize}
\end{itemize}

\subsection*{Tratamiento de Aguas}
\begin{itemize}
    \item \textbf{Potabilización (ETAP - Estación de Tratamiento de Agua Potable)}: Proceso para hacer el agua segura para el consumo humano.
    \begin{itemize}
        \item \textbf{Fases}: Captación $\rightarrow$ Pretratamiento (desbaste, desarenado) $\rightarrow$ Coagulación-Floculación (agrupación de partículas finas) $\rightarrow$ Decantación (sedimentación de flóculos) $\rightarrow$ Filtración (a través de lechos de arena) $\rightarrow$ Desinfección (generalmente con cloro).
    \end{itemize}
    \item \textbf{Depuración (EDAR - Estación Depuradora de Aguas Residuales)}: Proceso para limpiar las aguas residuales antes de devolverlas al medioambiente.
    \begin{itemize}
        \item \textbf{Línea de Agua}:
            \begin{itemize}
                \item \textbf{Pretratamiento}: Eliminación de sólidos gruesos, arenas y grasas.
                \item \textbf{Tratamiento Primario}: Decantación de sólidos en suspensión.
                \item \textbf{Tratamiento Secundario (Biológico)}: Microorganismos (fangos activos) eliminan la materia orgánica disuelta (DBO). Incluye procesos de nitrificación-desnitrificación para eliminar nitrógeno.
                \item \textbf{Tratamiento Terciario}: Procesos avanzados para una depuración más exhaustiva (eliminación de fósforo, desinfección).
            \end{itemize}
        \item \textbf{Línea de Fangos}: El lodo generado se trata mediante \textbf{espesamiento} (reducir volumen), \textbf{estabilización} (digestión anaerobia para producir biogás) y \textbf{deshidratación} (secado).
    \end{itemize}
\end{itemize}
\newpage

%%%%%%%%%%%%%%%%%%%%%%%%%%%%%%%%%%%%%%%%%%%%%%%%%%%%%%%%%%%%%%%%%%%%%%%%%%%%%%%%
% UNIDAD 4: DEGRADACIÓN DEL SUELO
%%%%%%%%%%%%%%%%%%%%%%%%%%%%%%%%%%%%%%%%%%%%%%%%%%%%%%%%%%%%%%%%%%%%%%%%%%%%%%%%

% --- Página del Esquema Visual ---
\newgeometry{margin=0pt}
\thispagestyle{empty}
\begin{figure}[p]
    \centering
    \includegraphics[width=\paperwidth, height=\paperheight, keepaspectratio]{MA4.png}
\end{figure}
\restoregeometry
\newpage

% --- Resumen de Texto Detallado ---
\section*{Unidad 4: Degradación y contaminación del suelo}

\subsection*{El Suelo y sus Características}
\begin{itemize}
    \item \textbf{Edafología}: Es la ciencia que se dedica al estudio del suelo.
    \item \textbf{Definición}: El suelo es la fina capa superior de la corteza terrestre, un recurso frágil y no renovable a corto plazo, formado por la interacción de procesos geológicos, climáticos y biológicos.
    \item \textbf{Funciones del Suelo}: Actúa como productor de biomasa, es un componente clave del ciclo hidrológico, funciona como filtro de contaminantes, es hábitat para una gran biodiversidad y proporciona soporte para las actividades humanas.
    \item \textbf{Composición}: Se compone de minerales (clasificados por tamaño: \textbf{grava} > 2 mm, \textbf{arena} 0,05-2 mm, \textbf{limo} 0,002-0,05 mm, y \textbf{arcilla} < 0,002 mm), materia orgánica (humus), agua, aire y organismos vivos.
    \item \textbf{Estructura por Horizontes}: El perfil del suelo se divide en capas:
    \begin{itemize}
        \item \textbf{Horizonte O}: Capa superficial orgánica.
        \item \textbf{Horizonte A}: Capa arable, rica en materia orgánica y minerales.
        \item \textbf{Horizonte E}: Capa de eluviación (lavado de minerales).
        \item \textbf{Horizonte B}: Subsuelo, zona de acumulación de los materiales lavados.
        \item \textbf{Horizonte C}: Material parental o roca madre parcialmente alterada.
        \item \textbf{Horizonte R}: Lecho de roca no alterado.
    \end{itemize}
\end{itemize}

\subsection*{Degradación del Suelo}
\begin{itemize}
    \item \textbf{Proceso}: La degradación es la modificación negativa de las propiedades del suelo, afectando su capacidad para producir bienes y servicios. Puede ser física, química o biológica.
    \item \textbf{Definición Legal (Ley 7/2022)}: Un \textbf{suelo contaminado} es aquel alterado negativamente por componentes químicos peligrosos de origen humano que comportan un riesgo inaceptable para la salud humana o el medioambiente.
    \item \textbf{Principales Procesos y Efectos}:
    \begin{itemize}
        \item \textbf{Erosión del Suelo}: Es la remoción del material superficial por la acción del agua (erosión hídrica) o del viento (erosión eólica). La actividad humana acelera este proceso.
        \item \textbf{Acidificación}: Disminución del pH del suelo, que reduce la disponibilidad de nutrientes para las plantas.
        \item \textbf{Pérdida de Materia Orgánica}: Debilita la estructura del suelo, disminuye la retención de agua y nutrientes, y aumenta la erosión.
        \item \textbf{Compactación}: Reduce el espacio poroso, dificultando la infiltración de agua y el crecimiento de las raíces.
        \item \textbf{Sellado del Suelo}: Cobertura permanente de la superficie con materiales artificiales (asfalto, hormigón) debido a la expansión urbana.
        \item \textbf{Desertificación}: Es la fase final de la degradación de la tierra en zonas áridas, semiáridas y subhúmedas secas, donde el suelo pierde su capacidad productiva.
    \end{itemize}
\end{itemize}

\subsection*{Contaminantes del Suelo}
\begin{itemize}
    \item Son sustancias que degradan la calidad del suelo y provienen de diversas fuentes. Los principales son:
    \begin{itemize}
        \item \textbf{Metales Pesados}: Plomo (Pb), mercurio (Hg), cadmio (Cd), cromo (Cr), arsénico (As). Son tóxicos y persistentes.
        \item \textbf{Compuestos Orgánicos Persistentes (COP)}: Pesticidas (como DDT), herbicidas y productos industriales (como PCBs) que son resistentes a la degradación.
        \item \textbf{Petróleo y sus derivados}: Por derrames y fugas.
        \item \textbf{Productos Químicos Agrícolas}: Uso excesivo de fertilizantes y pesticidas.
        \item \textbf{Desechos Sólidos}: Basura doméstica e industrial mal gestionada.
        \item \textbf{Radiactivos}: De accidentes nucleares o residuos de la industria.
    \end{itemize}
\end{itemize}
\newpage

%%%%%%%%%%%%%%%%%%%%%%%%%%%%%%%%%%%%%%%%%%%%%%%%%%%%%%%%%%%%%%%%%%%%%%%%%%%%%%%%
% UNIDAD 5: GESTIÓN DE RESIDUOS
%%%%%%%%%%%%%%%%%%%%%%%%%%%%%%%%%%%%%%%%%%%%%%%%%%%%%%%%%%%%%%%%%%%%%%%%%%%%%%%%

% --- Página del Esquema Visual ---
\newgeometry{margin=0pt}
\thispagestyle{empty}
\begin{figure}[p]
    \centering
    \includegraphics[width=\paperwidth, height=\paperheight, keepaspectratio]{MA5.png}
\end{figure}
\restoregeometry
\newpage

% --- Resumen de Texto Detallado ---
\section*{Unidad 5: Gestión de residuos y gestión ambiental}

\subsection*{Residuos: Definición y Clasificación}
\begin{itemize}
    \item \textbf{Definición Legal (Ley 7/2022)}: Un \textbf{residuo} es "cualquier sustancia u objeto que su poseedor deseche o tenga la intención o la obligación de desechar".
    \item \textbf{Economía Circular}: Modelo que contrasta con la economía lineal de "usar y tirar", promoviendo la reutilización, reparación y reciclaje para extender la vida útil de los productos y minimizar los residuos.
    \item \textbf{Clasificación de Residuos}:
    \begin{itemize}
        \item \textbf{Por Origen}: Domésticos, municipales, industriales, comerciales, agrarios, de construcción y demolición, etc.
        \item \textbf{Por Características}:
            \begin{itemize}
                \item \textbf{Residuos Inertes}: No sufren transformaciones significativas y no son peligrosos (ej. escombros).
                \item \textbf{Residuos Peligrosos}: Presentan una o varias de las 15 características de peligrosidad (códigos \textbf{HP1} a \textbf{HP15}), como HP1 (Explosivo), HP3 (Inflamable), HP6 (Toxicidad aguda), HP7 (Carcinógeno) o HP14 (Ecotóxico).
            \end{itemize}
    \end{itemize}
    \item \textbf{Identificación}: Se utiliza la \textbf{Lista Europea de Residuos (LER)} para asignar un código de 6 cifras. Los residuos peligrosos se marcan con un asterisco (*). Los peligros se señalizan con los pictogramas del Sistema Globalmente Armonizado (SGA o GHS).
\end{itemize}

\subsection*{Política y Tratamiento de Residuos}
\begin{itemize}
    \item \textbf{Jerarquía de Residuos}: Principio rector de la política de residuos, que establece el siguiente orden de prioridad:
    \begin{enumerate}
        \item \textbf{Prevención}: Reducir la generación de residuos.
        \item \textbf{Preparación para la reutilización}: Limpiar o reparar productos para volver a usarlos.
        \item \textbf{Reciclado}: Transformar los residuos en nuevos materiales.
        \item \textbf{Otro tipo de valorización}: Incluye la \textbf{valorización energética} (incineración con recuperación de energía).
        \item \textbf{Eliminación}: La opción menos deseable (ej. vertedero).
    \end{enumerate}
    \item \textbf{Tratamientos de Residuos}:
    \begin{itemize}
        \item \textbf{Químicos}: Neutralización, oxidación-reducción, precipitación, estabilización-solidificación.
        \item \textbf{Térmicos}: \textbf{Incineración} (con exceso de oxígeno), \textbf{Gasificación} (con oxígeno limitado para producir gas de síntesis) y \textbf{Pirólisis} (descomposición térmica en ausencia de oxígeno).
        \item \textbf{Biológicos}: \textbf{Compostaje} (proceso aeróbico para materia orgánica sólida que produce compost) y \textbf{Biometanización} (proceso anaeróbico que produce biogás).
    \end{itemize}
\end{itemize}

\subsection*{Instalaciones para el Tratamiento de Residuos}
\begin{itemize}
    \item \textbf{Plantas de Compostaje}: Instalaciones que gestionan la descomposición controlada de residuos orgánicos.
    \item \textbf{Vertederos}: Instalaciones para la eliminación final de residuos. Deben cumplir estrictas condiciones de diseño para prevenir la contaminación, incluyendo la impermeabilización del fondo, la recogida y tratamiento de lixiviados y la captura de gases.
    \item \textbf{Incineradoras}: Plantas que queman residuos a altas temperaturas para reducir su volumen y, a menudo, recuperar energía. Deben contar con sistemas avanzados de limpieza de gases de combustión.
\end{itemize}

\subsection*{Gestión e Impacto Ambiental}
\begin{itemize}
    \item \textbf{Impacto Ambiental}: Es la alteración (positiva o negativa) que se produce sobre el medioambiente por la ejecución de un proyecto. Se clasifica según su recuperación (reversible/irreversible), duración (temporal/permanente) e interrelación (simple/acumulativo/sinérgico).
    \item \textbf{Evaluación Ambiental (Ley 21/2013)}: Procedimiento administrativo para analizar los posibles impactos significativos de planes, programas y proyectos.
    \begin{itemize}
        \item \textbf{Evaluación Ambiental Estratégica (EAE)}: Para planes y programas.
        \item \textbf{Evaluación de Impacto Ambiental (EIA)}: Para proyectos. Puede ser \textbf{ordinaria} (proyectos del Anexo I de la ley) o \textbf{simplificada} (proyectos del Anexo II).
    \end{itemize}
\end{itemize}
\newpage

%%%%%%%%%%%%%%%%%%%%%%%%%%%%%%%%%%%%%%%%%%%%%%%%%%%%%%%%%%%%%%%%%%%%%%%%%%%%%%%%
% UNIDAD 6: ENERGÍAS NO RENOVABLES
%%%%%%%%%%%%%%%%%%%%%%%%%%%%%%%%%%%%%%%%%%%%%%%%%%%%%%%%%%%%%%%%%%%%%%%%%%%%%%%%

% --- Página del Esquema Visual ---
\newgeometry{margin=0pt}
\thispagestyle{empty}
\begin{figure}[p]
    \centering
    \includegraphics[width=\paperwidth, height=\paperheight, keepaspectratio]{MA6.png}
\end{figure}
\restoregeometry
\newpage

% --- Resumen de Texto Detallado ---
\section*{Unidad 6: Fuentes de energía. Energías no renovables}

\subsection*{Introducción y Clasificación}
\begin{itemize}
    \item \textbf{Consumo y Desarrollo}: El consumo de energía per cápita es uno de los indicadores más precisos para evaluar el grado de desarrollo económico de una sociedad.
    \item \textbf{Clasificación de Fuentes de Energía}:
    \begin{itemize}
        \item \textbf{No Renovables}: Recursos finitos y agotables (petróleo, gas natural, carbón, nuclear).
        \item \textbf{Renovables}: Fuentes sostenibles cuyo recurso no se agota con su uso (solar, eólica, hidráulica, etc.).
        \item \textbf{Primarias}: Se obtienen directamente de la naturaleza (sol, viento, crudo de petróleo).
        \item \textbf{Secundarias}: Resultan de la transformación de las primarias (electricidad, gasolina).
    \end{itemize}
    \item \textbf{Vectores Energéticos}: Sustancias o sistemas que permiten almacenar y transportar energía de forma controlada, como la \textbf{electricidad} y el \textbf{hidrógeno}.
\end{itemize}

\subsection*{Petróleo}
\begin{itemize}
    \item \textbf{Formación}: A partir de la descomposición de materia orgánica marina (plancton) durante millones de años bajo condiciones de alta presión y temperatura.
    \item \textbf{Clasificación}:
        \begin{itemize}
            \item Por su \textbf{densidad API}: Medida de su densidad relativa al agua. Se clasifican en superligero, ligero, medio, pesado y extrapesado.
            \item Por \textbf{crudos de referencia} del mercado: \textbf{Brent} (referencia en Europa), \textbf{West Texas Intermediate (WTI)} (referencia en Norteamérica) y \textbf{Dubái/Omán} (referencia en Asia).
        \end{itemize}
    \item \textbf{Proceso de Refino}: Conjunto de operaciones para convertir el crudo en productos útiles.
    \begin{itemize}
        \item \textbf{Destilación}: Proceso físico que separa los componentes del crudo en una torre de destilación según sus puntos de ebullición, obteniendo fracciones como naftas, queroseno, gasóleo y residuo. Se realiza a presión atmosférica y al vacío.
        \item \textbf{Conversión}: Procesos químicos como el \textbf{craqueo catalítico} que rompen las moléculas grandes de hidrocarburos en otras más pequeñas y valiosas (como la gasolina).
        \item \textbf{Tratamiento}: Procesos como la \textbf{hidrodesulfuración} que eliminan impurezas (principalmente azufre) para mejorar la calidad del combustible y reducir la contaminación.
    \end{itemize}
\end{itemize}

\subsection*{Carbón y Gas Natural}
\begin{itemize}
    \item \textbf{Carbón}: Recurso formado por la descomposición de materia vegetal terrestre (proceso de carbonización). Se clasifica por su antigüedad y contenido en carbono en: \textbf{Turba} (más joven), \textbf{Lignito}, \textbf{Hulla} y \textbf{Antracita} (más antiguo y con mayor poder calorífico).
    \item \textbf{Gas Natural}: Mezcla de hidrocarburos gaseosos, principalmente \textbf{metano ($CH_4$)}. Se extrae de yacimientos y se procesa para eliminar impurezas (agua, $H_2S$, $CO_2$). Para su transporte en barco se enfría a -162°C para convertirlo en \textbf{Gas Natural Licuado (GNL)}, reduciendo su volumen 600 veces.
\end{itemize}

\subsection*{Energía Nuclear}
\begin{itemize}
    \item \textbf{Fundamentos}:
    \begin{itemize}
        \item \textbf{Fisión Nuclear}: Proceso utilizado en las centrales actuales. Consiste en la \textbf{división} del núcleo de un átomo pesado (como el \textbf{Uranio-235}) al ser bombardeado por un neutrón, liberando una gran cantidad de energía y más neutrones.
        \item \textbf{Fusión Nuclear}: Proceso que ocurre en las estrellas. Consiste en la \textbf{unión} de dos núcleos ligeros (isótopos de hidrógeno) para formar uno más pesado, liberando una cantidad de energía inmensa. Aún se encuentra en fase experimental en la Tierra (proyecto ITER).
    \end{itemize}
    \item \textbf{Reacción en Cadena y Control}: Los neutrones liberados en una fisión pueden causar nuevas fisiones. Para mantener una reacción controlada, se utilizan:
    \begin{itemize}
        \item \textbf{Moderador} (agua o grafito): Frena los neutrones para hacerlos más efectivos.
        \item \textbf{Barras de control} (boro o cadmio): Absorben el exceso de neutrones para regular o detener la reacción.
    \end{itemize}
    \item \textbf{Centrales Nucleares}: Utilizan el calor de la fisión para generar vapor que mueve una turbina (ciclo Rankine).
    \begin{itemize}
        \item \textbf{PWR (Reactor de Agua a Presión)}: El más común. Utiliza un circuito primario de agua a alta presión que no hierve y transfiere el calor a un circuito secundario donde se genera el vapor.
        \item \textbf{BWR (Reactor de Agua en Ebullición)}: Diseño más simple con un único circuito. El agua hierve directamente en el núcleo del reactor y el vapor generado va directo a la turbina.
    \end{itemize}
\end{itemize}
\newpage

%%%%%%%%%%%%%%%%%%%%%%%%%%%%%%%%%%%%%%%%%%%%%%%%%%%%%%%%%%%%%%%%%%%%%%%%%%%%%%%%
% UNIDAD 7: ENERGÍAS RENOVABLES
%%%%%%%%%%%%%%%%%%%%%%%%%%%%%%%%%%%%%%%%%%%%%%%%%%%%%%%%%%%%%%%%%%%%%%%%%%%%%%%%

% --- Página del Esquema Visual ---
\newgeometry{margin=0pt}
\thispagestyle{empty}
\begin{figure}[p]
    \centering
    \includegraphics[width=\paperwidth, height=\paperheight, keepaspectratio]{MA7.png}
\end{figure}
\restoregeometry
\newpage

% --- Resumen de Texto Detallado ---
\section*{Unidad 7: Fuentes de energía. Energías renovables}
\subsection*{Introducción a las Energías Renovables}
\begin{itemize}
    \item \textbf{Definición (Directiva UE 2018/2001)}: "La energía procedente de fuentes no fósiles, como la energía eólica, la energía solar (tanto térmica como fotovoltaica), la energía geotérmica, la energía ambiental, la energía mareomotriz, la energía undimotriz y otras formas de energía oceánica, así como la energía hidráulica, la biomasa, los gases de vertedero, los gases de plantas de depuración y el biogás".
    \item \textbf{Contexto Histórico}: El uso de renovables es antiguo (molinos de viento, norias), pero fue relegado por los combustibles fósiles. Las crisis energéticas de los 70 y la conciencia sobre el cambio climático han impulsado su desarrollo exponencial en las últimas décadas.
    \item \textbf{Objetivos en España (Ley 7/2021)}: Para 2030, se pretende que el 74\% de la electricidad provenga de fuentes renovables.
\end{itemize}

\subsection*{Energía Solar}
\begin{itemize}
    \item \textbf{Solar Fotovoltaica (FV)}: Convierte la luz solar directamente en electricidad mediante el efecto fotovoltaico en células de silicio. Los paneles pueden ser de silicio \textbf{monocristalino} (más eficientes, 18-22\%) o \textbf{policristalino} (más económicos, 15-17\%).
    \item \textbf{Solar Térmica}: Aprovecha el calor del sol.
        \begin{itemize}
            \item \textbf{Baja Temperatura (<100 °C)}: Se usa para calentar agua sanitaria (ACS) y climatizar piscinas mediante captadores solares.
            \item \textbf{Alta Temperatura (CSP - Concentrated Solar Power)}: Usa espejos para concentrar la radiación solar, calentar un fluido a altas temperaturas, generar vapor y mover una turbina para producir electricidad.
        \end{itemize}
\end{itemize}

\subsection*{Energía Eólica}
\begin{itemize}
    \item \textbf{Funcionamiento}: Convierte la energía cinética del viento en electricidad mediante aerogeneradores. Las palas del rotor capturan la energía del viento y hacen girar un generador.
    \item \textbf{Componentes Principales}: Un aerogenerador consta de cimientos, una torre, una \textbf{góndola} (que aloja el multiplicador y el generador) y un \textbf{rotor} (formado por las palas y el buje).
    \item \textbf{Obtención de Energía}: La potencia disponible en el viento es proporcional al área barrida por las palas y, de forma cúbica, a la velocidad del viento ($P_v = \frac{1}{2}\rho A V^3$).
    \item \textbf{Límite de Betz}: Principio físico que establece que un aerogenerador no puede extraer más del 59,26\% de la energía cinética del viento.
    \item \textbf{Eólica Marina}: Instalación de aerogeneradores en el mar, donde los vientos son más fuertes y constantes, aunque su construcción y mantenimiento son más costosos.
\end{itemize}

\subsection*{Energía Hidráulica}
\begin{itemize}
    \item \textbf{Funcionamiento}: Aprovecha la energía potencial del agua almacenada a cierta altura en un embalse. Al liberarla, la energía potencial se convierte en cinética, que mueve una turbina acoplada a un generador.
    \item \textbf{Tipos de Centrales}:
        \begin{itemize}
            \item \textbf{De Embalse}: El tipo más común. Una presa retiene grandes volúmenes de agua.
            \item \textbf{De Agua Fluyente (o Pasantes)}: Desvían parte del caudal de un río, sin un gran embalse.
            \item \textbf{Reversibles (o de Bombeo)}: Funcionan como una "batería". Usan energía sobrante de la red (en horas de baja demanda) para bombear agua a un embalse superior, y la liberan para generar electricidad en horas punta.
        \end{itemize}
    \item \textbf{Turbinas Hidráulicas}: La elección depende de la altura del salto y del caudal.
    \begin{itemize}
        \item \textbf{Turbina Pelton}: Para saltos muy altos y caudales bajos.
        \item \textbf{Turbina Francis}: Para saltos y caudales medios.
        \item \textbf{Turbina Kaplan}: Para saltos bajos y grandes caudales.
    \end{itemize}
\end{itemize}

\subsection*{Biomasa}
\begin{itemize}
    \item \textbf{Definición}: Fracción biodegradable de productos, residuos y desechos de origen biológico.
    \item \textbf{Tipos}: \textbf{Natural} (residuos forestales espontáneos), \textbf{residual} (subproductos de agricultura, ganadería, industria o residuos urbanos) y de \textbf{cultivos energéticos}.
    \item \textbf{Métodos de Aprovechamiento}:
    \begin{itemize}
        \item \textbf{Combustión Directa}: Quema para generar calor y/o electricidad.
        \item \textbf{Procesos Termoquímicos}: Gasificación (produce gas de síntesis) y pirólisis (produce biogás, bioaceite y carbón vegetal).
        \item \textbf{Procesos Bioquímicos}: Fermentación alcohólica (produce bioetanol) y digestión anaerobia (produce biogás).
    \end{itemize}
\end{itemize}

\subsection*{Energía Geotérmica}
\begin{itemize}
    \item \textbf{Definición}: Energía almacenada en forma de calor bajo la superficie terrestre.
    \item \textbf{Yacimientos}: Se clasifican por su temperatura (entalpía). Los de \textbf{alta entalpía} (>150 °C) se usan para generar electricidad. Los de \textbf{baja entalpía} (<100 °C) se destinan a usos directos como calefacción y climatización.
    \item \textbf{Sistemas de Baja Entalpía}: Son los más comunes en España. Utilizan una \textbf{bomba de calor geotérmica} que intercambia calor con el subsuelo a través de captadores (horizontales o verticales) para climatizar edificios.
\end{itemize}

\subsection*{Energía Mareomotriz}
\begin{itemize}
    \item \textbf{Definición}: Aprovecha el movimiento cíclico y predecible de las mareas. No debe confundirse con la \textbf{energía undimotriz}, que aprovecha el movimiento de las olas (causado por el viento).
    \item \textbf{Tecnologías}: \textbf{Presas de marea} (similares a una central hidroeléctrica en un estuario) y \textbf{generadores de corrientes de marea} (turbinas submarinas).
\end{itemize}

\subsection*{Mix Energético y Regulación de la Red}
\begin{itemize}
    \item \textbf{Mix Energético}: Es la combinación de las diferentes fuentes (renovables y no renovables) que un país utiliza para generar electricidad.
    \item \textbf{Regulación de la Red Eléctrica}: Como la electricidad no se puede almacenar a gran escala, la generación debe igualar a la demanda en todo momento. Esto se logra mediante:
    \begin{itemize}
        \item \textbf{Regulación Primaria}: Respuesta automática e inmediata (segundos) de las centrales para estabilizar la frecuencia de la red (50 Hz).
        \item \textbf{Regulación Secundaria}: Ajuste centralizado y automático (minutos) para corregir desviaciones.
        \item \textbf{Regulación Terciaria}: Ajustes manuales a más largo plazo para optimizar las reservas de potencia.
    \end{itemize}
\end{itemize}
\newpage

%%%%%%%%%%%%%%%%%%%%%%%%%%%%%%%%%%%%%%%%%%%%%%%%%%%%%%%%%%%%%%%%%%%%%%%%%%%%%%%%
% UNIDAD 8: COMBUSTIBLES Y COMBUSTIÓN
%%%%%%%%%%%%%%%%%%%%%%%%%%%%%%%%%%%%%%%%%%%%%%%%%%%%%%%%%%%%%%%%%%%%%%%%%%%%%%%%

% --- Página del Esquema Visual ---
\newgeometry{margin=0pt}
\thispagestyle{empty}
\begin{figure}[p]
    \centering
    \includegraphics[width=\paperwidth, height=\paperheight, keepaspectratio]{MA8.png}
\end{figure}
\restoregeometry
\newpage

% --- Resumen de Texto Detallado ---
\section*{Unidad 8: Combustibles y combustión}

\subsection*{Conceptos Fundamentales}
\begin{itemize}
    \item \textbf{Combustible}: Cualquier sustancia capaz de liberar energía al oxidarse rápidamente en una reacción exotérmica.
    \item \textbf{Comburente}: Sustancia que reacciona con el combustible, generalmente el oxígeno ($O_2$) del aire.
    \item \textbf{Combustión}: Reacción entre combustible y comburente.
    \item \textbf{Poder Calorífico}: Cantidad de energía liberada por unidad de masa o volumen del combustible.
    \begin{itemize}
        \item \textbf{Poder Calorífico Superior (PCS)}: Considera que el agua producida en la combustión se condensa, liberando su calor latente. Es la energía total máxima.
        \item \textbf{Poder Calorífico Inferior (PCI)}: Considera que el agua producida permanece en estado de vapor. Es el valor de energía útil en la mayoría de aplicaciones prácticas (motores, turbinas).
    \end{itemize}
\end{itemize}

\subsection*{Tipos de Combustibles}
\begin{itemize}
    \item \textbf{Gases Combustibles}: Se clasifican según el \textbf{Índice de Wobbe} ($W_s = \frac{PCS}{\sqrt{d_r}}$), que mide la intercambiabilidad energética de los gases.
    \begin{itemize}
        \item \textbf{Familia 1}: Gas ciudad (manufacturado).
        \item \textbf{Familia 2}: Gas natural (principalmente metano).
        \item \textbf{Familia 3}: Gases Licuados del Petróleo (GLP), como butano y propano.
    \end{itemize}
    \item \textbf{Líquidos Combustibles}:
    \begin{itemize}
        \item \textbf{Gasolina}: Para motores de encendido por chispa (ciclo Otto). Su propiedad antidetonante se mide con el \textbf{índice de octano}. Se establece una escala de 0 (n-heptano) a 100 (iso-octano).
        \item \textbf{Gasóleo (Diésel)}: Para motores de encendido por compresión (ciclo Diésel). Su calidad de ignición se mide con el \textbf{índice de cetano}, usando una escala de 0 (1-metilnaftaleno) a 100 (cetano/n-hexadecano).
        \item \textbf{Queroseno}: Principalmente para aviación.
    \end{itemize}
    \item \textbf{Biocombustibles}:
    \begin{itemize}
        \item \textbf{Bioetanol}: Alcohol producido por \textbf{fermentación} de azúcares (de maíz, caña de azúcar). Se mezcla con gasolina.
        \item \textbf{Biodiésel}: Ésteres metílicos producidos por \textbf{transesterificación} de aceites vegetales o grasas animales. Se mezcla con diésel.
    \end{itemize}
    \item \textbf{Hidrógeno ($H_2$)}: Un vector energético. Se clasifica por "colores" según su producción:
    \begin{itemize}
        \item \textbf{Hidrógeno Verde}: El único 100\% limpio. Producido por \textbf{electrólisis} del agua usando electricidad renovable.
        \item \textbf{Hidrógeno Gris}: El más común. Producido por \textbf{reformado de gas natural} con vapor, emitiendo $CO_2$.
        \item \textbf{Hidrógeno Azul}: Igual que el gris, pero el $CO_2$ es capturado (tecnología CAC).
        \item \textbf{Hidrógeno Negro/Marrón}: Producido por gasificación de carbón, muy contaminante.
    \end{itemize}
\end{itemize}

\subsection*{Centrales de Combustión}
\begin{itemize}
    \item \textbf{Centrales Térmicas Convencionales}: Queman un combustible (carbón, fueloil) para generar vapor que mueve una turbina. Funcionan según el \textbf{ciclo Rankine} (bomba $\rightarrow$ caldera $\rightarrow$ turbina $\rightarrow$ condensador).
    \item \textbf{Centrales de Ciclo Combinado}: Son más eficientes ($\approx$60\%) porque combinan dos ciclos:
    \begin{enumerate}
        \item Un \textbf{ciclo Brayton} (turbina de gas): Se quema gas natural, y los gases de combustión calientes mueven una turbina de gas.
        \item Un \textbf{ciclo Rankine} (turbina de vapor): Los gases de escape calientes de la turbina de gas se usan en una caldera de recuperación para generar vapor, que mueve una segunda turbina de vapor.
    \end{enumerate}
\end{itemize}

\subsection*{Principios de la Combustión}
\begin{itemize}
    \item \textbf{Estequiometría y Dosado}: La estequiometría es el cálculo de la proporción exacta de aire (oxígeno) y combustible para una combustión completa, donde todos los reactivos se consumen. La relación aire-combustible ($A/C$) puede ser molar o másica.
    \item \textbf{Coeficiente de Exceso de Aire (Factor $\lambda$)}: Es la relación entre la cantidad real de aire y la cantidad estequiométrica. $$ \lambda = \frac{\text{Cantidad real de aire}}{\text{Cantidad estequiométrica de aire}} $$
    \begin{itemize}
        \item \textbf{$\lambda < 1$}: **Mezcla rica** (defecto de aire). Genera combustión incompleta, produciendo CO y hollín.
        \item \textbf{$\lambda = 1$}: **Mezcla estequiométrica**.
        \item \textbf{$\lambda > 1$}: **Mezcla pobre** (exceso de aire). Asegura la combustión completa, aunque puede reducir la eficiencia térmica.
    \end{itemize}
\end{itemize}
\newpage

%%%%%%%%%%%%%%%%%%%%%%%%%%%%%%%%%%%%%%%%%%%%%%%%%%%%%%%%%%%%%%%%%%%%%%%%%%%%%%%%
% UNIDAD 9: DISPONIBILIDAD SOLAR
%%%%%%%%%%%%%%%%%%%%%%%%%%%%%%%%%%%%%%%%%%%%%%%%%%%%%%%%%%%%%%%%%%%%%%%%%%%%%%%%

% --- Página del Esquema Visual ---
\newgeometry{margin=0pt}
\thispagestyle{empty}
\begin{figure}[p]
    \centering
    \includegraphics[width=\paperwidth, height=\paperheight, keepaspectratio]{MA9.png}
\end{figure}
\restoregeometry
\newpage

% --- Resumen de Texto Detallado ---
\section*{Unidad 9: Disponibilidad solar}

\subsection*{El Sol y la Radiación Solar}
\begin{itemize}
    \item \textbf{Fuente de Energía}: El Sol es una estrella que genera energía mediante la \textbf{fusión nuclear} de hidrógeno para formar helio.
    \item \textbf{Movimiento Terrestre y Estaciones}:
    \begin{itemize}
        \item La órbita de la Tierra es elíptica, con un punto más cercano al Sol (\textbf{perihelio}, $\approx$3 de enero) y uno más lejano (\textbf{afelio}, $\approx$3 de julio).
        \item Las estaciones del año se deben a la \textbf{inclinación del eje de rotación de la Tierra (23,45º)} respecto al plano de su órbita, no a la distancia al Sol. Esta inclinación hace que los rayos solares incidan de forma más directa o indirecta sobre cada hemisferio según la época.
    \end{itemize}
    \item \textbf{Constante Solar (Irradiancia Solar Total)}: Es la cantidad de energía solar que llega a la parte superior de la atmósfera sobre una superficie perpendicular a los rayos. Su valor medio es de \textbf{1361 W/m²}. La intensidad disminuye con el cuadrado de la distancia (\textbf{ley inversa del cuadrado}).
\end{itemize}

\subsection*{Atenuación Solar en la Atmósfera}
\begin{itemize}
    \item La radiación solar se debilita al atravesar la atmósfera debido a tres fenómenos:
    \begin{itemize}
        \item \textbf{Reflexión}: Parte de la radiación es reflejada de vuelta al espacio. La fracción de radiación reflejada por un planeta o superficie se denomina \textbf{albedo}. El albedo medio de la Tierra es de 0,3.
        \item \textbf{Absorción}: Gases como el ozono ($O_3$), el vapor de agua ($H_2O$) y el dióxido de carbono ($CO_2$) absorben selectivamente ciertas longitudes de onda, transformando la energía radiante en calor.
        \item \textbf{Dispersión}: La radiación cambia de dirección al chocar con partículas.
            \begin{itemize}
                \item \textbf{Dispersión de Rayleigh}: Ocurre con partículas más pequeñas que la longitud de onda de la luz (moléculas de aire). Dispersa más eficientemente las longitudes de onda cortas (azules y violetas), lo que explica por qué \textbf{el cielo es azul}.
                \item \textbf{Dispersión de Mie}: Ocurre con partículas de tamaño similar o mayor a la longitud de onda (gotas de agua, polvo). Dispersa todas las longitudes de onda por igual, por lo que \textbf{las nubes y la niebla son blancas}.
            \end{itemize}
    \end{itemize}
    \item \textbf{Masa de Aire (AM)}: Es la relación entre la longitud del camino que la luz recorre a través de la atmósfera y la longitud del camino si el Sol estuviera en el cénit (vertical). Se calcula como $AM = 1 / \sin(\beta)$, donde $\beta$ es el ángulo de elevación solar. Un valor de AM=1,5 se toma como referencia para condiciones estándar.
\end{itemize}

\subsection*{Medición de la Radiación Solar}
\begin{itemize}
    \item \textbf{Instrumentos de Medida}:
    \begin{itemize}
        \item \textbf{Piranómetro}: Mide la radiación solar \textbf{global} (directa + difusa) sobre una superficie horizontal.
        \item \textbf{Pirheliómetro}: Mide exclusivamente la radiación solar \textbf{directa}. Requiere un sistema de seguimiento solar para apuntar siempre al Sol.
    \end{itemize}
    \item \textbf{Tipos de Irradiancia}:
    \begin{itemize}
        \item \textbf{DNI (Irradiancia Normal Directa)}: Radiación directa que incide sobre una superficie perpendicular a los rayos del Sol.
        \item \textbf{DHI (Irradiancia Horizontal Difusa)}: Radiación que ha sido dispersada por la atmósfera y llega a una superficie horizontal.
        \item \textbf{GHI (Irradiancia Horizontal Global)}: Es la radiación total que llega a una superficie horizontal ($GHI = DHI + DNI \cdot \cos(\alpha)$, donde $\alpha$ es el ángulo cenital).
    \end{itemize}
\end{itemize}

\subsection*{Balance Térmico Global}
\begin{itemize}
    \item Describe el equilibrio entre la energía entrante del Sol y la energía saliente de la Tierra. La radiación solar incidente media en la Tierra es de unos \textbf{340,4 W/m²}.
    \item Del total, unos \textbf{99,9 W/m²} son reflejados (albedo) y \textbf{240,5 W/m²} son absorbidos por el sistema Tierra-atmósfera.
    \item La superficie, a su vez, emite radiación infrarroja, que es en gran parte absorbida y reemitida hacia la Tierra por los GEI (efecto invernadero).
    \item Actualmente, existe un pequeño desequilibrio: la Tierra absorbe \textbf{0,6 W/m²} más de lo que emite, lo que provoca el calentamiento global.
\end{itemize}

\subsection*{Inclinación Óptima para Captación Solar}
\begin{itemize}
    \item Es el ángulo de inclinación de un panel solar respecto a la horizontal que maximiza la energía recibida.
    \item \textbf{Factores Clave}:
    \begin{itemize}
        \item \textbf{Latitud ($\phi$)}: Para maximizar la captación anual, la inclinación óptima ($\beta$) es aproximadamente igual a la latitud del lugar ($\beta \approx \phi$).
        \item \textbf{Época del año}: La inclinación óptima varía con la estación. Para maximizar en invierno, $\beta \approx \phi + 15^\circ$; para verano, $\beta \approx \phi - 15^\circ$.
        \item \textbf{Declinación Solar ($\delta$)}: Es el ángulo entre el plano ecuatorial y los rayos del Sol, que varía entre +23,45° (solsticio de verano) y -23,45° (solsticio de invierno). Una fórmula para calcularla es: $\delta = 23,45^\circ \cdot \sin\left(\frac{360}{365} \cdot (N+284)\right)$, donde N es el día del año.
        \item \textbf{Cálculo para un día específico}: La inclinación óptima para un día concreto es $\beta = \phi - \delta$.
    \end{itemize}
\end{itemize}
\newpage

%%%%%%%%%%%%%%%%%%%%%%%%%%%%%%%%%%%%%%%%%%%%%%%%%%%%%%%%%%%%%%%%%%%%%%%%%%%%%%%%
% UNIDAD 10: ENERGÍA SOLAR TÉRMICA Y FOTOVOLTAICA
%%%%%%%%%%%%%%%%%%%%%%%%%%%%%%%%%%%%%%%%%%%%%%%%%%%%%%%%%%%%%%%%%%%%%%%%%%%%%%%%

% --- Página del Esquema Visual ---
\newgeometry{margin=0pt}
\thispagestyle{empty}
\begin{figure}[p]
    \centering
    \includegraphics[width=\paperwidth, height=\paperheight, keepaspectratio]{MA10.png}
\end{figure}
\restoregeometry
\newpage

% --- Resumen de Texto Detallado ---
\section*{Unidad 10: Energía solar térmica y fotovoltaica}

\subsection*{Energía Solar Térmica de Alta Temperatura (CSP)}
\begin{itemize}
    \item \textbf{Definición}: Tecnología que utiliza espejos o lentes para concentrar la radiación solar y generar calor a temperaturas elevadas (superiores a 400 °C), que se usa para producir electricidad mediante un ciclo de vapor (generalmente Rankine).
    \item \textbf{Tecnologías Principales}:
    \begin{itemize}
        \item \textbf{Torre Central}: Un campo de espejos móviles (\textbf{helióstatos}) concentra la luz solar en un receptor situado en lo alto de una torre. El fluido calentado (a menudo sales fundidas) puede almacenarse para generar electricidad incluso sin sol.
        \item \textbf{Colectores Cilindro-Parabólicos (CCP)}: Espejos curvos con forma de parábola que concentran la luz en un tubo receptor por el que circula un fluido térmico.
        \item \textbf{Tecnología Fresnel}: Utiliza reflectores planos y largos que giran para enfocar la luz en un receptor fijo.
        \item \textbf{Tecnología de Disco Stirling}: Un disco parabólico concentra la luz en un motor Stirling para generar electricidad de forma directa.
    \end{itemize}
\end{itemize}

\subsection*{Energía Solar Térmica de Baja Temperatura}
\begin{itemize}
    \item \textbf{Definición}: Aprovecha el calor del sol para aplicaciones que requieren temperaturas por debajo de los 100 °C, principalmente Agua Caliente Sanitaria (ACS) y calefacción.
    \item \textbf{Tipos de Instalaciones}:
    \begin{itemize}
        \item \textbf{Circulación Natural (Termosifón)}: El fluido caloportador se mueve por convección natural (el agua caliente sube). El depósito de acumulación debe estar situado por encima de los captadores.
        \item \textbf{Circulación Forzada}: Utiliza una bomba para mover el fluido, lo que ofrece mayor flexibilidad en el diseño.
    \end{itemize}
    \item \textbf{Componentes Principales}:
    \begin{itemize}
        \item \textbf{Captadores Solares}: Absorben la radiación. Pueden ser \textbf{captadores planos} o \textbf{de tubos de vacío} (más eficientes, especialmente en climas fríos).
        \item \textbf{Acumulador}: Tanque aislado que almacena el agua caliente.
    \end{itemize}
    \item \textbf{Balance Energético del Captador}: El rendimiento ($\eta$) de un captador depende de su rendimiento óptico ($\eta_0$) y de sus pérdidas térmicas, que aumentan con la diferencia de temperatura entre el captador y el ambiente. La ecuación de Hottel-Whillier-Bliss describe este comportamiento: $$ \eta = F_R(\tau\alpha) - F_R U_L \frac{(t_e - t_a)}{I_0} $$
\end{itemize}

\subsection*{Energía Solar Fotovoltaica (PV)}
\begin{itemize}
    \item \textbf{Principio de Funcionamiento (Efecto Fotovoltaico)}:
    \begin{enumerate}
        \item Se utiliza un material semiconductor (generalmente \textbf{silicio}) dopado para crear una capa tipo N (exceso de electrones) y una capa tipo P (exceso de "huecos").
        \item La unión de ambas capas crea un campo eléctrico permanente en la \textbf{unión P-N}.
        \item Cuando los fotones de la luz solar inciden en la célula, su energía libera electrones, creando pares electrón-hueco.
        \item El campo eléctrico de la unión P-N separa estos pares, generando una diferencia de potencial y, si se conecta a un circuito externo, una corriente eléctrica en \textbf{corriente continua (CC)}.
    \end{enumerate}
    \item \textbf{Tipos de Células Solares}:
    \begin{itemize}
        \item \textbf{Monocristalino}: Hechas de un único cristal de silicio. Son las más eficientes (18-22\%) y de color negro uniforme.
        \item \textbf{Policristalino}: Hechas de múltiples cristales de silicio. Ligeramente menos eficientes (15-17\%), más económicas y de color azul con vetas visibles.
        \item \textbf{Capa Fina (Thin-film)}: Ligeras y flexibles, pero con menor eficiencia (7-18\%).
        \item \textbf{Multiunión}: Muy alta eficiencia (>40\%), pero muy costosas. Se usan en aplicaciones espaciales.
    \end{itemize}
    \item \textbf{Parámetros y Curva I-V}:
    \begin{itemize}
        \item La \textbf{curva I-V} de un panel muestra la relación entre la intensidad (I) y el voltaje (V) que genera. Puntos clave son la \textbf{Intensidad de Cortocircuito ($I_{sc}$)} y la \textbf{Tensión de Circuito Abierto ($V_{oc}$)}.
        \item El \textbf{Punto de Máxima Potencia (MPP)} es el punto de la curva donde la potencia ($P = V \cdot I$) es máxima.
        \item El rendimiento de un panel disminuye con el aumento de la \textbf{temperatura} y aumenta con la \textbf{irradiancia}.
    \end{itemize}
    \item \textbf{Tipos de Instalación Fotovoltaica}:
    \begin{itemize}
        \item \textbf{Aisladas (Off-grid)}: Son autónomas y no están conectadas a la red eléctrica. Requieren \textbf{baterías} para almacenar energía y un \textbf{regulador de carga}.
        \item \textbf{Conectadas a Red (On-grid)}: Operan en paralelo con la red eléctrica. Permiten el \textbf{autoconsumo} y verter los excedentes a la red. El componente clave es el \textbf{inversor}, que convierte la CC en Corriente Alterna (CA) y la sincroniza con la red.
    \end{itemize}
\end{itemize}

\end{document}