\documentclass[a4paper,12pt]{article}
\usepackage[utf8]{inputenc}
\usepackage{graphicx}
\usepackage{amsmath}
\usepackage{hyperref}

\begin{document}
\section{Procesos de Fundición y Moldeo}

\subsection{Fluidez y Contracciones en Fundición}

\textbf{Fluidez}: Es la capacidad de un metal fundido para llenar completamente el molde y reproducir sus detalles. Depende de factores como:
\begin{itemize}
    \item Viscosidad del metal.
    \item Tensión superficial.
    \item Contenido de impurezas.
    \item Forma de solidificación de la aleación.
    \item Temperatura de colada.
\end{itemize}

Se puede medir utilizando moldes de prueba en forma de estrella o espiral, observando hasta qué punto el metal fundido es capaz de rellenar estas formas.

\textbf{Contracciones}: Ocurren cuando el metal se enfría y solidifica, generando disminuciones de volumen en diferentes fases:
\begin{itemize}
    \item \textbf{Contracción en estado líquido}: Reducción de volumen del metal antes de iniciar la solidificación.
    \item \textbf{Contracción de solidificación}: Cambio de volumen al pasar de líquido a sólido.
    \item \textbf{Contracción en estado sólido}: Reducción del volumen desde la solidificación hasta la temperatura ambiente.
\end{itemize}
Estas contracciones pueden generar defectos como rechupes, los cuales se minimizan con el uso de mazarotas y canales de alimentación adecuados.

\subsection{Diseño y Elaboración de Modelos y Noyos para Moldeo por Colada}

Los modelos y noyos son esenciales en el proceso de moldeo:

\textbf{Modelos:} Son representaciones geométricas de la pieza a fundir, con ciertas modificaciones para compensar la contracción del metal y facilitar la fabricación. Se pueden fabricar en:
\begin{itemize}
    \item Madera (fácil de trabajar y económica).
    \item Fundición (mayor resistencia a la abrasión).
    \item Aluminio (ligero y duradero).
    \item Polímeros (resistentes y reutilizables).
\end{itemize}

\textbf{Noyos o machos:} Se utilizan para generar cavidades internas en las piezas fundidas. Se fabrican con arena compactada y endurecida con aglutinantes, y pueden requerir refuerzos metálicos para mayor estabilidad.

\subsection{Sistema de Colada para Moldeo por Arena}

El sistema de colada es el conjunto de canales y cavidades que permiten la entrada y distribución del metal fundido en el molde. Sus partes principales incluyen:

\begin{itemize}
    \item \textbf{Copa de vaciado}: Punto donde se vierte el metal fundido en el molde.
    \item \textbf{Bebedero}: Conducto vertical que transporta el metal hacia el sistema de alimentación.
    \item \textbf{Canales de alimentación}: Distribuyen el metal dentro del molde.
    \item \textbf{Compuertas}: Puntos de entrada del metal en la cavidad del molde.
    \item \textbf{Mazarotas}: Reservas de metal que compensan la contracción del material durante la solidificación.
    \item \textbf{Respiraderos}: Pequeñas aberturas que permiten la salida de gases atrapados en el molde.
\end{itemize}

\subsection{Moldeo en Cáscara}

El \textbf{moldeo en cáscara} es un proceso desarrollado por Croning que utiliza arena fina aglutinada con resinas poliméricas. El proceso consiste en:
\begin{enumerate}
    \item Se calienta un modelo metálico a 250-300°C.
    \item Se recubre con un agente antiadherente.
    \item Se aplica la mezcla de arena y resina, la cual endurece por polimerización.
    \item Se retira la cáscara endurecida y se ensambla para formar el molde final.
    \item Se refuerza con arena de cuarzo o granalla de acero.
\end{enumerate}

Ventajas:
\begin{itemize}
    \item Mejor acabado superficial.
    \item Tolerancias más estrechas.
    \item Menor cantidad de arena requerida.
\end{itemize}

\subsection{Fundición a la Cera Perdida}

El proceso de \textbf{fundición a la cera perdida} es uno de los más antiguos y permite fabricar piezas con alto nivel de detalle. Sus etapas son:

\begin{enumerate}
    \item Fabricación del modelo de cera y el sistema de alimentación.
    \item Recubrimiento del modelo con material refractario.
    \item Eliminación de la cera por calentamiento.
    \item Cocción del molde para aumentar su resistencia.
    \item Colada del metal fundido dentro del molde.
    \item Rotura del molde para extraer la pieza.
\end{enumerate}

Ventajas:
\begin{itemize}
    \item Alta precisión dimensional.
    \item Posibilidad de fabricar geometrías complejas.
    \item Buen acabado superficial.
\end{itemize}

\subsection{Fundición con Poliestireno Expandido}

También conocido como \textbf{Lost Foam}, este método emplea modelos de poliestireno que se evaporan cuando se vierte el metal. El proceso incluye:

\begin{enumerate}
    \item Creación del modelo de poliestireno mediante moldeo por inyección.
    \item Recubrimiento con material refractario.
    \item Compactación en un molde con arena refractaria.
    \item Colada del metal, que vaporiza el poliestireno y ocupa su lugar.
    \item Eliminación de residuos y acabado final.
\end{enumerate}

Ventajas:
\begin{itemize}
    \item Menor coste en modelos (al ser desechables).
    \item Facilidad de automatización.
    \item Buena precisión y acabado superficial.
\end{itemize}

\subsection{Procesos de Fundición en Molde Permanente}

La fundición en molde permanente utiliza moldes reutilizables de metal o grafito. Existen varios métodos:

\textbf{1. Fundición por gravedad}
\begin{itemize}
    \item Se vierte el metal por gravedad en un molde metálico.
    \item Aplicado a aluminio, zinc y cobre.
    \item Genera piezas con buena resistencia mecánica y acabado.
\end{itemize}

\textbf{2. Fundición a presión}
\begin{itemize}
    \item El metal fundido es inyectado a alta presión en el molde.
    \item Se usa en la producción de piezas de aluminio y magnesio.
    \item Ideal para fabricar grandes volúmenes de piezas de alta precisión.
\end{itemize}

\textbf{3. Fundición centrífuga}
\begin{itemize}
    \item El metal es vertido en un molde giratorio.
    \item Se usa en la fabricación de tubos y cilindros sin soldadura.
\end{itemize}

\textbf{4. Fundición estampada (Squeeze Casting)}
\begin{itemize}
    \item Se vierte el metal en una matriz precalentada y se aplica presión para mejorar la calidad de la pieza.
    \item Permite fabricar piezas con menor porosidad y mejor resistencia mecánica.
\end{itemize}

\section{Procesos de Conformado de Metales}

\subsection{Proceso de Extrusión de Metales}

La \textbf{extrusión} es un proceso de conformado en el que un material metálico es forzado a fluir a través de una matriz para obtener un perfil con una sección transversal específica. Se emplea para fabricar tubos, barras y perfiles con formas complejas.

\textbf{Tipos de Extrusión}:
\begin{itemize}
    \item \textbf{Extrusión en caliente}: Se realiza a temperaturas elevadas para mejorar la ductilidad del material y reducir la resistencia a la deformación.
    \item \textbf{Extrusión en frío}: Se realiza a temperatura ambiente o ligeramente elevada, generando piezas con mayor resistencia mecánica.
\end{itemize}

\textbf{Métodos de Extrusión}:
\begin{itemize}
    \item \textbf{Extrusión directa}: El material se empuja en la misma dirección del movimiento del émbolo.
    \item \textbf{Extrusión indirecta}: El material fluye en dirección opuesta al desplazamiento del émbolo, reduciendo la fricción y mejorando la eficiencia del proceso.
    \item \textbf{Extrusión hidrostática}: Se emplea un fluido a presión para empujar el material a través de la matriz, minimizando la fricción.
\end{itemize}

\subsection{Métodos de Laminación para la Obtención de Piezas Metálicas}

La \textbf{laminación} es un proceso de conformado en el que un material se deforma plásticamente al pasar entre rodillos giratorios. Se usa para fabricar chapas, láminas, barras y perfiles metálicos.

\textbf{Tipos de Laminación}:
\begin{itemize}
    \item \textbf{Laminado en caliente}: Se realiza por encima de la temperatura de recristalización del metal, facilitando la deformación.
    \item \textbf{Laminado en frío}: Se realiza por debajo de la temperatura de recristalización, mejorando el acabado superficial y las propiedades mecánicas.
\end{itemize}

\textbf{Configuraciones de Laminadores}:
\begin{itemize}
    \item \textbf{Laminador dúo}: Dos rodillos giran en sentidos opuestos.
    \item \textbf{Laminador trío}: Tres rodillos permiten múltiples pasadas sin inversión de giro.
    \item \textbf{Laminador Sendzimir}: Usa múltiples rodillos de apoyo para reducir el espesor del material.
\end{itemize}

\subsection{Proceso de Embutición}

La \textbf{embutición} es un proceso de conformado de chapa en el que una lámina metálica se transforma en una cavidad mediante la acción de un punzón sobre una matriz.

\textbf{Tipos Especiales de Embutición}:
\begin{itemize}
    \item \textbf{Embutición con elastómeros}: Se usa una matriz de material elastomérico para conformar la chapa.
    \item \textbf{Repujado}: Se presiona la chapa contra un mandril mediante un rodillo.
    \item \textbf{Embutición por estirado}: Se sujeta la chapa y se estira hasta darle forma.
\end{itemize}

\subsection{Proceso de Doblado para la Obtención de Piezas Metálicas}

El \textbf{doblado} consiste en deformar una chapa metálica aplicando un esfuerzo mecánico para lograr una curvatura deseada.

\textbf{Métodos de Doblado}:
\begin{itemize}
    \item \textbf{Doblado en V}: Se utiliza un punzón en forma de V que presiona la chapa sobre una matriz con el mismo perfil.
    \item \textbf{Doblado en voladizo}: Se emplea un aprieta-chapa para sujetar la pieza mientras se aplica la fuerza de doblado.
    \item \textbf{Doblado con rodillos}: Se pasa la chapa a través de una serie de rodillos hasta obtener la curvatura deseada.
\end{itemize}

El proceso de doblado puede generar recuperación elástica, lo que hace que la pieza tienda a volver a su forma original. Para evitarlo, se aplican métodos como el \textbf{sobredoblado} o el \textbf{doblado a fondo}.
\section{Procesos de Conformado de Polímeros}

\subsection{Funcionamiento de las Extrusoras para la Fabricación de Polímeros}

La \textbf{extrusión} de polímeros es un proceso continuo en el que un material termoplástico se funde y se empuja a través de una matriz para formar productos con secciones constantes, como tubos, películas y perfiles.

\textbf{Partes de una extrusora}:
\begin{itemize}
    \item \textbf{Tolva de alimentación}: Lugar donde se introduce la granza o polvo del polímero.
    \item \textbf{Cilindro calefactor}: Contiene resistencias eléctricas que funden el material.
    \item \textbf{Husillo (tornillo)}: Transporta y mezcla el polímero mientras lo empuja hacia la boquilla.
    \item \textbf{Plato rompedor}: Uniformiza el flujo y retiene partículas contaminantes.
    \item \textbf{Boquilla o matriz}: Define la forma final del producto extruido.
\end{itemize}

Existen dos tipos principales de extrusoras:
\begin{itemize}
    \item \textbf{Extrusoras de husillo único}: Usadas en la mayoría de los procesos industriales.
    \item \textbf{Extrusoras de husillo doble}: Mejoran la mezcla y la desgasificación del material.
\end{itemize}

\subsection{Moldeo por Inyección de Polímeros}

El \textbf{moldeo por inyección} es uno de los métodos más utilizados para fabricar piezas plásticas complejas. El proceso incluye las siguientes etapas:

\begin{enumerate}
    \item Calentamiento del polímero en una extrusora.
    \item Inyección del material en el molde a alta presión.
    \item Enfriamiento del polímero hasta solidificarse.
    \item Apertura del molde y expulsión de la pieza terminada.
\end{enumerate}

\textbf{Parámetros a controlar}:
\begin{itemize}
    \item Temperatura de inyección.
    \item Presión del cilindro inyector.
    \item Tiempo de enfriamiento del molde.
    \item Velocidad de llenado del molde.
\end{itemize}

\textbf{Ventajas}:
\begin{itemize}
    \item Alta precisión en la fabricación de piezas.
    \item Producción en masa con tiempos de ciclo cortos.
    \item Posibilidad de fabricar piezas con geometrías complejas.
\end{itemize}

\textbf{Desventajas}:
\begin{itemize}
    \item Coste elevado de los moldes.
    \item No es rentable para pequeñas producciones.
\end{itemize}

\subsection{Moldeo por Soplado para Fabricación de Envases}

El \textbf{moldeo por soplado} es un proceso utilizado para fabricar piezas huecas, como botellas y envases plásticos. Existen varias variantes:

\begin{itemize}
    \item \textbf{Moldeo por extrusión-soplado}: Se extruye un tubo de polímero caliente (parison), que luego es soplado contra las paredes del molde.
    \item \textbf{Moldeo por inyección-soplado}: Se moldea una preforma mediante inyección, que luego se calienta y se sopla en el molde final.
    \item \textbf{Moldeo por inyección-estirado-soplado}: Similar al anterior, pero con un estiramiento adicional para mejorar la orientación molecular.
\end{itemize}

\subsection{Soplado de Films}

El \textbf{soplado de films} se utiliza para fabricar películas plásticas de pequeño espesor. En este proceso, el polímero fundido es extruido a través de una matriz circular para formar un tubo, que luego se infla con aire y se estira para reducir su espesor.

\subsection{Termoconformado de Polímeros}

El \textbf{termoconformado} es un proceso en el que una lámina de plástico se calienta y se adapta a un molde. Se divide en:

\textbf{Métodos de un paso}:
\begin{itemize}
    \item \textbf{Termoconformado por envoltura}: La lámina se adapta a un molde macho.
    \item \textbf{Termoconformado con vacío}: La lámina se succiona hacia un molde hembra mediante vacío.
    \item \textbf{Termoconformado con presión}: Similar al anterior, pero con aire a presión en lugar de vacío.
\end{itemize}

\textbf{Métodos multipaso}:
\begin{itemize}
    \item \textbf{Termoconformado por soplado y molde macho}: La lámina se infla antes de ser presionada contra el molde.
    \item \textbf{Termoconformado por vacío con ariete}: Se usa un empujador mecánico para estirar la lámina antes de aplicar el vacío.
\end{itemize}

\subsection{Grado de Estiramiento en Termoconformado}

El \textbf{grado de estiramiento} mide cuánto se ha deformado la lámina durante el termoconformado:

\begin{itemize}
    \item \textbf{Grado de estiramiento lineal (RL)}: Relación entre la longitud final y la inicial de la lámina.
    \item \textbf{Grado de estiramiento superficial (Ra)}: Relación entre la superficie final y la inicial.
\end{itemize}

Estos parámetros dependen de:
\begin{itemize}
    \item Temperatura del material.
    \item Presión aplicada.
    \item Propiedades del polímero.
\end{itemize}

\section{Fabricación y Procesado del Vidrio}

\subsection{Funcionamiento del Horno de Solera para la Fabricación de Vidrio}

El \textbf{horno de solera} es un tipo de horno utilizado en la fabricación de vidrio en continuo. Su diseño permite la producción masiva de vidrio con un flujo constante de material fundido. 

\textbf{Principales Zonas del Horno de Solera}:
\begin{itemize}
    \item \textbf{Zona de fusión}: Se introducen las materias primas y se funden a temperaturas superiores a 1500 °C.
    \item \textbf{Zona de afino}: Se eliminan burbujas y defectos del vidrio fundido.
    \item \textbf{Zona de trabajo}: Se regula la temperatura para obtener la viscosidad adecuada antes del conformado.
\end{itemize}

El calor es generado por quemadores de gas o fueloil, y el aire de combustión se precalienta mediante regeneradores. El vidrio fundido avanza a lo largo del horno hasta ser dosificado y conformado según el producto deseado.

\subsection{Métodos de Fabricación de Envases de Vidrio por Soplado}

Existen dos métodos principales para la fabricación de envases de vidrio mediante soplado:

\begin{itemize}
    \item \textbf{Prensado-soplado}: Se forma una preforma mediante prensado en un molde inicial, y luego se expande mediante aire a presión en un molde final. Se utiliza para envases de boca ancha.
    \item \textbf{Soplado-soplado}: La preforma se obtiene soplando aire en el molde inicial, y luego se transfiere al molde final donde se sopla nuevamente hasta alcanzar la forma definitiva. Se usa en la fabricación de botellas.
\end{itemize}

En ambos procesos, la presión del aire ayuda a distribuir el vidrio fundido uniformemente dentro del molde, garantizando paredes uniformes y resistencia mecánica adecuada.

\subsection{Fabricación de Fibra de Vidrio, Lana de Vidrio y Lana de Roca}

\textbf{Fibra de Vidrio}:
Se obtiene mediante la extrusión de vidrio fundido a través de una hilera con múltiples orificios. Los filamentos resultantes son estirados y recubiertos con polímeros para mejorar su resistencia mecánica y facilitar su uso en materiales compuestos.

\textbf{Lana de Vidrio}:
Se fabrica haciendo pasar el vidrio fundido por un sistema de centrifugado, en el que el material se transforma en fibras delgadas mediante la acción de la fuerza centrífuga y un flujo de aire a alta velocidad. Luego, se aplica un ligante adhesivo y se compacta en mantas o paneles aislantes.

\textbf{Lana de Roca}:
Su proceso es similar al de la lana de vidrio, pero utiliza minerales volcánicos en lugar de vidrio. Se calientan a altas temperaturas y se someten a centrifugado para formar fibras resistentes al fuego y al calor.

\subsection{Métodos de Fabricación de Vidrio Plano}

El vidrio plano se produce mediante los siguientes métodos:

\begin{itemize}
    \item \textbf{Proceso de estirado}: Se extrae una lámina de vidrio fundido de un baño de fusión y se enfría progresivamente mientras pasa por rodillos. Este método fue ampliamente utilizado antes de la introducción del vidrio flotado.
    \item \textbf{Proceso de flotación}: Se vierte el vidrio fundido sobre un baño de estaño líquido, donde se expande de manera uniforme, formando una superficie lisa y de espesor constante. Es el método más utilizado actualmente.
\end{itemize}

El vidrio flotado es enfriado gradualmente para eliminar tensiones internas y obtener un producto de alta calidad óptica y mecánica.

\subsection{Templado y Revenido en Vidrios}

El \textbf{templado} es un tratamiento térmico que aumenta la resistencia del vidrio. Se calienta el vidrio hasta su temperatura de reblandecimiento y luego se enfría rápidamente con aire. Este proceso genera tensiones internas que mejoran su resistencia mecánica y térmica.

El \textbf{revenido} es un tratamiento térmico que reduce las tensiones internas del vidrio después del templado, evitando fracturas espontáneas.

El \textbf{templado químico} consiste en sumergir el vidrio en un baño de sales fundidas, donde los iones superficiales son reemplazados por otros de mayor tamaño, generando compresión en la superficie y aumentando su resistencia.

\section{Procesos de Fabricación de Piezas Cerámicas}

\subsection{Moldeo por Inyección de Polvos para la Fabricación de Piezas Cerámicas}

El \textbf{moldeo por inyección de polvos} es un proceso utilizado para fabricar piezas cerámicas de alta precisión y geometrías complejas. Consiste en mezclar los polvos cerámicos con un ligante polimérico, formando una pasta que puede ser inyectada en un molde.

\textbf{Etapas del proceso}:
\begin{enumerate}
    \item \textbf{Preparación de la mezcla}: Se combinan polvos cerámicos con un ligante orgánico y aditivos lubricantes.
    \item \textbf{Granulación}: Se obtiene una granza para facilitar la alimentación de la máquina de inyección.
    \item \textbf{Moldeo por inyección}: Se inyecta la mezcla en un molde mediante presión controlada.
    \item \textbf{Eliminación del ligante}: Se eliminan los componentes orgánicos mediante calor o solventes.
    \item \textbf{Sinterización}: Se somete la pieza a altas temperaturas para consolidar la estructura cerámica.
\end{enumerate}

Este proceso permite obtener piezas con tolerancias muy ajustadas y excelente acabado superficial.

\subsection{Compactación Isostática en Frío}

La \textbf{compactación isostática en frío} (CIP, Cold Isostatic Pressing) es una técnica que utiliza presión hidrostática para compactar polvos cerámicos dentro de un molde flexible.

\textbf{Características del proceso}:
\begin{itemize}
    \item Se emplea un molde de caucho o plástico relleno con polvo cerámico.
    \item La presión se aplica de manera uniforme en todas direcciones dentro de un recipiente lleno de fluido a alta presión.
    \item Se logran piezas con una densidad uniforme y mínima porosidad.
\end{itemize}

Es un método eficiente para obtener piezas cerámicas con formas complejas antes del proceso de sinterización.

\subsection{Compactación Uniaxial en Caliente}

La \textbf{compactación uniaxial en caliente} (Hot Pressing) combina presión y temperatura para aumentar la densificación de los polvos cerámicos durante el sinterizado.

\textbf{Ventajas del proceso}:
\begin{itemize}
    \item Se reduce el tiempo de sinterización.
    \item Se obtiene una menor porosidad residual en la pieza final.
    \item Se pueden mejorar las propiedades mecánicas de los materiales cerámicos.
\end{itemize}

Este método se emplea en la fabricación de componentes de alta resistencia térmica y mecánica.

\subsection{Compactación Isostática en Caliente}

El \textbf{prensado isostático en caliente} (HIP, Hot Isostatic Pressing) es un proceso en el que la compactación y la sinterización ocurren simultáneamente en un ambiente de alta presión y temperatura.

\textbf{Etapas del proceso}:
\begin{enumerate}
    \item La pieza preformada se coloca en un horno de alta presión.
    \item Se aplica presión con gas inerte (helio o argón) a temperaturas de hasta 2300 K.
    \item La presión homogénea permite eliminar prácticamente toda la porosidad de la pieza.
\end{enumerate}

Este proceso es utilizado para la fabricación de componentes cerámicos de alto rendimiento en industrias como la aeroespacial y biomédica.

\subsection{Fabricación de Ladrillos y Baldosas}

El proceso de fabricación de \textbf{ladrillos y baldosas} se basa en la transformación de arcillas naturales en productos cerámicos mediante secado y cocción.

\textbf{Fases del proceso}:
\begin{enumerate}
    \item \textbf{Extracción y preparación de la arcilla}: Se selecciona la materia prima y se somete a trituración y mezclado.
    \item \textbf{Conformado}: Se moldean las piezas mediante extrusión o prensado.
    \item \textbf{Secado}: Se elimina la humedad controladamente para evitar defectos.
    \item \textbf{Cocción}: Se somete a altas temperaturas para obtener la resistencia final.
    \item \textbf{Enfriamiento y acabado}: Se aplican tratamientos superficiales si es necesario.
\end{enumerate}

Los ladrillos y baldosas cerámicas se emplean ampliamente en construcción debido a su durabilidad y propiedades térmicas.

\section{Procesos de Fabricación de Piezas Cerámicas}

\subsection{Moldeo por Inyección de Polvos}

El \textbf{moldeo por inyección de polvos} es un proceso empleado para la fabricación de piezas cerámicas con geometrías complejas y alta precisión dimensional. El proceso sigue las siguientes etapas:

\begin{enumerate}
    \item \textbf{Preparación de la mezcla (feedstock)}: Se combinan polvos cerámicos con un ligante polimérico y lubricantes para mejorar la fluidez.
    \item \textbf{Granulación}: La mezcla se transforma en pequeños gránulos para facilitar su alimentación en la máquina de inyección.
    \item \textbf{Inyección}: La mezcla se calienta y se inyecta en un molde con la forma deseada.
    \item \textbf{Eliminación del ligante}: Se eliminan los polímeros mediante procesos térmicos o disolventes.
    \item \textbf{Sinterización}: Se somete la pieza a altas temperaturas para densificarla y mejorar sus propiedades mecánicas.
\end{enumerate}

Las principales ventajas del proceso incluyen:
\begin{itemize}
    \item Alta precisión dimensional.
    \item Producción en serie con costos reducidos.
    \item Posibilidad de fabricar piezas con cavidades internas y formas complejas.
\end{itemize}

\subsection{Compactación Isostática en Frío}

La \textbf{compactación isostática en frío} es un método en el que los polvos cerámicos se introducen en un molde flexible que es sometido a una presión uniforme en todas las direcciones mediante un fluido presurizado.

Características del proceso:
\begin{itemize}
    \item Se logra una distribución homogénea de la densidad.
    \item Se pueden obtener piezas con formas complejas.
    \item Las presiones de operación pueden alcanzar los 2000 bar.
\end{itemize}

Desventajas:
\begin{itemize}
    \item Baja productividad en comparación con otros métodos.
    \item Precisión dimensional inferior a otros procesos como la compactación uniaxial.
\end{itemize}

\subsection{Compactación Uniaxial en Caliente}

El \textbf{prensado uniaxial en caliente} es una técnica similar a la compactación en frío, pero realizada a altas temperaturas. También se denomina \textbf{sinterizado a presión} y permite mejorar la densificación de la pieza.

Ventajas del proceso:
\begin{itemize}
    \item Reduce el tiempo y la temperatura de sinterización.
    \item Minimiza la porosidad residual.
    \item Mejora las propiedades mecánicas finales de la pieza.
\end{itemize}

El material más comúnmente utilizado para moldes y punzones es el grafito debido a su estabilidad térmica y resistencia mecánica.

\subsection{Compactación Isostática en Caliente}

El \textbf{prensado isostático en caliente} (HIP) es una de las técnicas más avanzadas para la fabricación de piezas cerámicas de alta resistencia. Se realiza en un autoclave presurizado donde las piezas preformadas se calientan a temperaturas de hasta 2300 K bajo presiones de hasta 1 Mbar.

Características principales:
\begin{itemize}
    \item Se logra una densificación completa con porosidad prácticamente nula.
    \item Se utiliza un gas inerte como helio o argón para la presurización.
    \item Se emplean recubrimientos de vidrio en la pieza para transmitir la presión de manera uniforme.
\end{itemize}

\subsection{Fabricación de Ladrillos y Baldosas}

El proceso de fabricación de \textbf{ladrillos y baldosas} sigue las siguientes etapas:

\begin{enumerate}
    \item \textbf{Extracción de arcilla}: Se obtiene la materia prima de canteras o depósitos naturales.
    \item \textbf{Preparación}: La arcilla se tritura, muele y tamiza para eliminar impurezas.
    \item \textbf{Mezclado y amasado}: Se agregan aditivos para mejorar la plasticidad y regularidad.
    \item \textbf{Moldeo}: Se conforman los ladrillos mediante extrusión o prensado.
    \item \textbf{Secado}: Se eliminan los restos de humedad para evitar defectos en la cocción.
    \item \textbf{Cocción}: Se someten a altas temperaturas en hornos circulares o de túnel.
\end{enumerate}

Dependiendo del tipo de ladrillo, se pueden aplicar barnices o esmaltes para mejorar la estética y la resistencia al desgaste.

\section{Fibras y Materiales Compuestos}

\subsection{Fabricación de las Fibras de Carbono}

Las fibras de carbono se obtienen a partir de precursores poliméricos mediante un proceso de tratamiento térmico que elimina todos los elementos excepto el carbono. Los precursores principales son:

\begin{itemize}
    \item \textbf{Poliacrilonitrilo (PAN)}: Es el más utilizado, con buenas propiedades mecánicas y coste moderado.
    \item \textbf{Rayón}: Cada vez menos empleado, ya que requiere temperaturas de grafitización muy elevadas.
    \item \textbf{Alquitrán o brea de petróleo}: Da lugar a fibras de menor coste, aunque sus propiedades varían según el tratamiento térmico.
\end{itemize}

\textbf{Etapas del proceso de fabricación}:
\begin{enumerate}
    \item \textbf{Estabilización}: Se calientan las fibras precursoras a temperaturas de hasta 400°C en atmósfera de oxígeno, para hacerlas infusibles.
    \item \textbf{Carbonización}: Se somete el material a temperaturas entre 800 y 1700°C en atmósfera inerte, eliminando elementos no carbonosos.
    \item \textbf{Grafitización}: A temperaturas superiores a 2000°C, se mejora la cristalinidad y se obtienen fibras de alto módulo.
    \item \textbf{Tratamiento superficial}: Se oxida la superficie de la fibra para mejorar la adhesión con la matriz en materiales compuestos.
    \item \textbf{Ensimaje y bobinado}: Se aplican agentes protectores y se enrollan en bobinas para su posterior uso.
\end{enumerate}

\subsection{Arquitectura del Refuerzo en Materiales Compuestos}

Las fibras de refuerzo pueden disponerse de diferentes formas según las necesidades mecánicas de la pieza:

\begin{itemize}
    \item \textbf{Haces de fibras o roving}: Grupos de filamentos sin torsión, utilizados en procesos de proyección o pultrusión.
    \item \textbf{Cintas unidireccionales}: Fibras alineadas en una única dirección, ideales para piezas con alta exigencia mecánica.
    \item \textbf{Telas}: Fibras entrelazadas en diferentes patrones, como el tafetán o el sarga, mejorando la resistencia en múltiples direcciones.
    \item \textbf{Preformas tridimensionales}: Estructuras volumétricas con fibras dispuestas en tres dimensiones, usadas en aplicaciones de alta responsabilidad.
\end{itemize}

\subsection{Procesos de Conformado de Materiales Compuestos}

Los métodos de fabricación de materiales compuestos pueden clasificarse en procesos de molde abierto y molde cerrado.

\subsubsection{Procesos en Molde Abierto}

\textbf{1. Deposición por Proyección}
\begin{itemize}
    \item Se proyectan simultáneamente fibras cortas y resina sobre un molde mediante una pistola de proyección.
    \item Se deja curar la pieza antes del desmoldeo.
    \item Método rápido y económico, pero con menor control de la orientación de las fibras.
\end{itemize}

\textbf{2. Impregnación Manual con Resina Líquida}
\begin{itemize}
    \item Se aplica la resina manualmente sobre una preforma de fibra con ayuda de un rodillo.
    \item Método utilizado en la industria náutica y reparaciones.
    \item Puede complementarse con curado en bolsa de vacío para mejorar la relación refuerzo-matriz.
\end{itemize}

\textbf{3. Arrollamiento de Filamentos}
\begin{itemize}
    \item Se enrollan fibras impregnadas en resina sobre un mandril giratorio.
    \item Se emplea en la fabricación de tanques de presión y tubos de alta resistencia.
\end{itemize}

\subsubsection{Procesos en Molde Cerrado}

\textbf{4. Pultrusión}
\begin{itemize}
    \item Fibras continuas se impregnan en resina y se arrastran a través de una matriz caliente que da la forma final.
    \item Se utiliza para fabricar perfiles estructurales de alta resistencia.
\end{itemize}

\textbf{5. Moldeo por Transferencia de Resina (RTM)}
\begin{itemize}
    \item La resina se inyecta a presión dentro de un molde cerrado que contiene el refuerzo seco.
    \item Se logran piezas con buen acabado y control preciso del espesor.
\end{itemize}

\textbf{6. Consolidación de Prepregs}
\begin{itemize}
    \item Se apilan láminas de fibras preimpregnadas con resina y se someten a presión y temperatura en un autoclave.
    \item Se usa en la industria aeroespacial y automotriz por su alto rendimiento.
\end{itemize}

\section{Procesos de Soldadura}

\subsection{Soldadura Oxiacetilénica}

La \textbf{soldadura oxiacetilénica} es un proceso que utiliza la combustión de una mezcla de \textbf{oxígeno} y \textbf{acetileno} para generar una llama de alta temperatura capaz de fundir los metales y permitir su unión.

\textbf{Equipo necesario}:
\begin{itemize}
    \item \textbf{Soplete de soldadura}: Mezcla y regula el flujo de oxígeno y acetileno.
    \item \textbf{Bombonas de gas}: Contienen oxígeno y acetileno.
    \item \textbf{Reguladores de presión}: Controlan la salida de gas desde las bombonas.
    \item \textbf{Mangueras}: Transportan los gases hasta el soplete.
    \item \textbf{Varilla de aporte}: Metal adicional que se funde para reforzar la unión.
\end{itemize}

\textbf{Tipos de llama}:
\begin{itemize}
    \item \textbf{Llama neutra}: Se logra con una proporción equilibrada de oxígeno y acetileno. Es la más utilizada en soldadura.
    \item \textbf{Llama oxidante}: Contiene un exceso de oxígeno, lo que puede provocar la oxidación del metal.
    \item \textbf{Llama carburante}: Contiene un exceso de acetileno, lo que genera depósitos de carbono en la soldadura.
\end{itemize}

\textbf{Aplicaciones}:
\begin{itemize}
    \item Soldadura de láminas delgadas de acero y metales no ferrosos.
    \item Reparaciones y mantenimiento industrial.
    \item Corte de metales (oxicorte).
\end{itemize}

\subsection{Soldadura TIG (Tungsten Inert Gas)}

La \textbf{soldadura TIG} es un proceso de soldadura por arco eléctrico en el que se emplea un \textbf{electrodo no consumible de tungsteno} y un \textbf{gas inerte} para proteger la soldadura de la contaminación atmosférica.

\textbf{Equipo necesario}:
\begin{itemize}
    \item \textbf{Fuente de energía}: Suministra corriente continua o alterna.
    \item \textbf{Antorcha con electrodo de tungsteno}: Mantiene el arco eléctrico estable.
    \item \textbf{Gas protector}: Generalmente argón o helio.
    \item \textbf{Material de aporte}: Se usa cuando es necesario reforzar la soldadura.
\end{itemize}

\textbf{Características}:
\begin{itemize}
    \item Produce cordones de soldadura de alta calidad y precisión.
    \item Permite soldar aceros inoxidables, aluminio, magnesio y aleaciones de titanio.
    \item Es adecuado para soldadura en todas las posiciones.
    \item Es más lento en comparación con otros procesos de soldadura por arco.
\end{itemize}

\textbf{Aplicaciones}:
\begin{itemize}
    \item Industria aeroespacial y automotriz.
    \item Soldadura de tuberías y estructuras de aluminio.
    \item Fabricación de equipos médicos y electrónicos.
\end{itemize}

\subsection{Soldadura MIG y MAG (Metal Inert Gas / Metal Active Gas)}

La \textbf{soldadura MIG/MAG} es un proceso de soldadura por arco en el que un \textbf{electrodo consumible en forma de hilo continuo} se funde mediante el calor generado por un arco eléctrico. Se protege con un gas inerte (MIG) o un gas activo (MAG).

\textbf{Diferencias entre MIG y MAG}:
\begin{itemize}
    \item \textbf{MIG (Metal Inert Gas)}: Utiliza gases inertes como el argón o el helio. Se emplea en aluminio y aceros inoxidables.
    \item \textbf{MAG (Metal Active Gas)}: Usa gases activos como el dióxido de carbono (CO2) o mezclas con oxígeno. Se utiliza principalmente en aceros al carbono.
\end{itemize}

\textbf{Equipo necesario}:
\begin{itemize}
    \item \textbf{Fuente de energía}: Generalmente de corriente continua.
    \item \textbf{Pistola de soldadura}: Alimenta el alambre y suministra el gas protector.
    \item \textbf{Alambre electrodo}: Funciona como material de aporte.
    \item \textbf{Gas de protección}: Evita la contaminación del cordón de soldadura.
\end{itemize}

\textbf{Tipos de transferencia del metal de aporte}:
\begin{itemize}
    \item \textbf{Transferencia por cortocircuito}: Se produce cuando el electrodo toca el metal base y genera pequeños depósitos de material.
    \item \textbf{Transferencia globular}: Se forman gotas grandes de metal que caen por gravedad en la zona de soldadura.
    \item \textbf{Transferencia spray}: Se generan pequeñas gotas que son transportadas por el arco hasta la zona de fusión.
    \item \textbf{Transferencia pulsada}: Se controla la corriente para formar gotas uniformes de metal.
\end{itemize}

\textbf{Ventajas}:
\begin{itemize}
    \item Alta velocidad de soldadura.
    \item Se puede automatizar fácilmente.
    \item Permite trabajar con una amplia variedad de materiales.
\end{itemize}

\textbf{Aplicaciones}:
\begin{itemize}
    \item Fabricación de estructuras metálicas.
    \item Industria automotriz y naval.
    \item Reparación y mantenimiento industrial.
\end{itemize}
\section{Adhesivos y Uniones Adhesivas}

\subsection{Las Cinco Familias de Adhesivos}

Los adhesivos se pueden clasificar en cinco familias principales según su composición y método de curado:

\begin{itemize}
    \item \textbf{Epoxis}: Adhesivos estructurales de alta resistencia y durabilidad. Se utilizan en la industria aeroespacial, automotriz y en aplicaciones estructurales. Requieren un proceso de curado térmico o químico.
    
    \item \textbf{Acrílicos}: Adhesivos de curado rápido y alta tenacidad. Pueden ser monocomponentes o bicomponentes y se emplean en la industria automotriz, electrónica y médica.
    
    \item \textbf{Anaeróbicos}: Se endurecen en ausencia de oxígeno y en contacto con superficies metálicas. Son empleados para fijación de roscas, sellado de juntas y retención de piezas mecánicas.
    
    \item \textbf{Cianoacrilatos}: Conocidos como adhesivos instantáneos, se curan rápidamente por la humedad del ambiente. Son utilizados en aplicaciones médicas, electrónicas y en la industria del calzado.
    
    \item \textbf{Adhesivos elásticos}: Incluyen siliconas y poliuretanos, caracterizados por su flexibilidad y resistencia a la intemperie. Se usan en construcción, automoción y sellado de ventanas.
\end{itemize}

\subsection{Tipos de Solicitaciones en Uniones Adhesivas}

Las uniones adhesivas pueden estar sometidas a diferentes tipos de esfuerzos mecánicos, que influyen en su durabilidad y desempeño. Los principales son:

\begin{itemize}
    \item \textbf{Tracción}: La fuerza actúa perpendicularmente al plano de la unión, distribuyéndose de manera uniforme.
    
    \item \textbf{Compresión}: Similar a la tracción, pero en dirección opuesta, generando una presión sobre la unión adhesiva.
    
    \item \textbf{Cortadura o cizalla}: Las fuerzas son paralelas al plano de la unión, distribuyéndose de manera uniforme sobre toda la superficie adhesiva.
    
    \item \textbf{Pelado}: Se genera cuando la carga se aplica en un extremo de la unión, causando una separación progresiva.
    
    \item \textbf{Desgarro}: Similar al pelado, pero aplicado en un solo punto, concentrando la carga en un área reducida.
\end{itemize}

\textbf{Esquema de los Tipos de Solicitaciones}:
% [DIBUJO REQUERIDO]
\begin{center}
    \begin{tabular}{|c|c|}
        \hline
        \textbf{Tipo de Esfuerzo} & \textbf{Descripción} \\
        \hline
        Tracción & Fuerza perpendicular al plano de la unión, distribuida uniformemente. \\
        \hline
        Compresión & Similar a la tracción, pero en dirección opuesta. \\
        \hline
        Cortadura & Fuerza paralela a la unión adhesiva. \\
        \hline
        Pelado & Separación progresiva en un extremo de la unión. \\
        \hline
        Desgarro & Concentración de fuerza en un solo punto. \\
        \hline
    \end{tabular}
\end{center}

\section{Procesos de Obtención de Recubrimientos}

\subsection{Implantación Iónica}

La \textbf{implantación iónica} es un proceso que modifica las propiedades superficiales de un material al incrustar átomos de un elemento en su estructura cristalina mediante un haz de iones de alta energía. A diferencia de otros procesos de difusión, este método se utiliza cuando no es posible elevar la temperatura del material hasta los niveles requeridos para la difusión térmica.

\textbf{Proceso:}
\begin{enumerate}
    \item Se genera un haz de iones en una \textbf{fuente de iones}.
    \item Los iones seleccionados se aceleran en un campo eléctrico dentro de un \textbf{acelerador de partículas}.
    \item Los iones impactan sobre la superficie del material, incrustándose en la red cristalina y alterando sus propiedades.
    \item Se puede aplicar un \textbf{recocido térmico} posterior para reparar defectos en la estructura del material.
\end{enumerate}

\textbf{Ventajas:}
\begin{itemize}
    \item Se puede controlar la profundidad de penetración de los iones.
    \item No requiere elevadas temperaturas de procesamiento.
    \item Mejora la resistencia al desgaste y la dureza superficial.
\end{itemize}

\textbf{Aplicaciones:}
\begin{itemize}
    \item Modificación superficial de herramientas y componentes metálicos.
    \item Aumento de la resistencia a la corrosión en materiales estructurales.
    \item Fabricación de dispositivos semiconductores.
\end{itemize}

\subsection{Deposición Física en Fase Vapor (PVD)}

La \textbf{deposición física en fase vapor (PVD, Physical Vapor Deposition)} es un proceso en el cual un material sólido es vaporizado en una atmósfera de vacío y luego depositado sobre una superficie para formar un recubrimiento delgado.

\textbf{Proceso:}
\begin{enumerate}
    \item Se coloca el material a depositar (\textbf{blanco}) dentro de una cámara de vacío.
    \item Se aplica un método de vaporización como:
    \begin{itemize}
        \item \textbf{Evaporación térmica}: El material se calienta hasta la sublimación y se transporta en fase gaseosa.
        \item \textbf{Pulverización catódica (sputtering)}: Un gas inerte (como argón) ionizado impacta sobre el blanco, desprendiendo átomos que se depositan sobre el sustrato.
    \end{itemize}
    \item Se pueden añadir gases reactivos para formar compuestos como nitruros o carburos.
    \item Los átomos vaporizados se condensan sobre la superficie del sustrato, formando una película delgada uniforme.
\end{enumerate}

\textbf{Ventajas:}
\begin{itemize}
    \item Recubrimientos con alta adherencia y resistencia mecánica.
    \item Espesores de capa controlados a escala nanométrica.
    \item Puede aplicarse a herramientas de corte, componentes ópticos y dispositivos electrónicos.
\end{itemize}

\subsection{Deposición Química en Fase Vapor (CVD)}

La \textbf{deposición química en fase vapor (CVD, Chemical Vapor Deposition)} es un método de deposición en el que los reactivos químicos en fase gaseosa reaccionan sobre la superficie de un sustrato para formar un recubrimiento sólido.

\textbf{Proceso:}
\begin{enumerate}
    \item Se introduce un gas precursor en una cámara de reacción a alta temperatura.
    \item Los gases reaccionan químicamente en la superficie del sustrato, formando un recubrimiento sólido.
    \item Se eliminan los productos secundarios gaseosos mediante un sistema de vacío.
    \item En algunos casos, se emplea un plasma para activar las reacciones químicas (\textbf{Plasma-Enhanced CVD, PECVD}).
\end{enumerate}

\textbf{Ventajas:}
\begin{itemize}
    \item Permite obtener recubrimientos homogéneos en geometrías complejas.
    \item Se pueden producir películas de alta pureza y baja porosidad.
    \item Se pueden formar recubrimientos de nitruros, carburos y óxidos con excelentes propiedades mecánicas y térmicas.
\end{itemize}

\textbf{Aplicaciones:}
\begin{itemize}
    \item Revestimientos de herramientas de corte (TiN, Al2O3).
    \item Protección térmica en componentes aeroespaciales.
    \item Fabricación de semiconductores y circuitos integrados.
\end{itemize}

\end{document}
