\documentclass{article}
\usepackage{lipsum}
\usepackage[backend=biber]{biblatex}
\addbibresource{bib.bib}
\usepackage{authoraftertitle}
\usepackage[top=2cm,bottom=1.5cm,left=1.5cm, right=3cm,includeheadfoot]{geometry}
\usepackage{graphicx}
\usepackage{fancyhdr}
\usepackage[spanish]{babel}
\usepackage{mathtools}
\usepackage{nicefrac}
\usepackage{csquotes}
\usepackage{amssymb}
\usepackage{fancybox, graphicx}
\usepackage{array}
\usepackage{hhline}
\usepackage{hyperref}
\usepackage{tikz}
\usepackage{amsmath}
\usepackage{wrapfig}
\usepackage{float}
\usepackage{amsmath}
\usepackage{esint}
\usepackage{caption}
\usepackage{esvect}
\usepackage{url}
\usepackage{siunitx}
\usepackage{commath}
\usepackage{lastpage} 

\title{Enunciado de Ejercicio}
\author{}
\date{}

\begin{document}



\section*{ENUNCIADO 2}

\subsection*{Diagrama de actividades y camino crítico}

\includegraphics[height=9cm]{network.PNG}

\subsection*{Cáclculo del camino crítico analíticamente y desviaciones máximas}

\includegraphics[height=9cm]{caminocriticoanal.png}

\subsection*{Un cambio de legislación implica la obligatoriedad de dejar transcurrir un periodo de 3 días entre K y L. 
Indicar cómo se ve afectado el proyecto ante este cambio en las condiciones.}
El proyecto se va a atrasar 3 días.

\subsection{Por un accidente en la actividad D, tardará 3 días más de lo previsto. Indicar cómo se ve afectado el 
proyecto ante este cambio. }
No cambia en nada puesto que la actividad D cuenta con 12 días de holgura.

\subsection*{Demostrar gráficamente que si la última actividad del camino crítico se retrasa 1 día más de lo obtenido
en el apartado 4, el camino critico cambia en tareas y/o duración.}

... ?

\subsection*{A partir de las actividades identificadas en la tabla del enunciado, se pide elaborar  con OpenProj (o
similares) un EDT lo más detallado posible }
\includegraphics[height=9cm]{EDT.PNG}


\end{document}
