\documentclass{article}
\usepackage{lipsum}
\usepackage[backend=biber]{biblatex}
\addbibresource{bib.bib}
\usepackage{authoraftertitle}
\usepackage[top=2cm,bottom=1.5cm,left=1.5cm, right=3cm,includeheadfoot]{geometry}
\usepackage{graphicx}
\usepackage{fancyhdr}
\usepackage[spanish]{babel}
\usepackage{mathtools}
\usepackage{nicefrac}
\usepackage{csquotes}
\usepackage{amssymb}
\usepackage{fancybox, graphicx}
\usepackage{array}
\usepackage{hhline}
\usepackage{hyperref}
\usepackage{tikz}
\usepackage{amsmath}
\usepackage{wrapfig}
\usepackage{float}
\usepackage{amsmath}
\usepackage{esint}
\usepackage{caption}
\usepackage{esvect}
\usepackage{url}
\usepackage{siunitx}
\usepackage{commath}
\usepackage{lastpage} 
\geometry{a4paper, margin=1in}
\title{Documento de alcance del proyecto}
\author{Consulting S.L.}
\date{\today}

\begin{document}

\maketitle

\begin{titlepage}
    \centering
    \vspace*{1.5cm}
    
    \Huge
    \textbf{Documento de alcance del proyecto}
    
    \vspace{0.5cm}
    \LARGE
    Encuesta de ambiente laboral en Client S.A.
    
    \vspace{1.5cm}
    
    \textbf{Consulting S.L.}\\
    
    \vfill
    
    \Large
    \textbf{Fecha:} \today
    
    \vspace{1cm}
\end{titlepage}

\section*{Título del proyecto}
Encuesta de ambiente laboral en Client S.A.

\section*{Justificación del proyecto}
Client S.A. ha identificado la necesidad de evaluar el ambiente laboral actual en su organización como base para mejorar la satisfacción y productividad de sus empleados. Los estudios realizados sugieren que el conocimiento sobre el clima laboral interno facilita la implementación de estrategias efectivas en áreas clave como la comunicación interna, la cultura organizacional, el bienestar y la satisfacción laboral. En respuesta a esta necesidad, Client S.A. ha encargado a Consulting S.L. la realización de una encuesta que permita recoger y analizar las opiniones de los empleados, generando datos relevantes que servirán como base para posibles acciones de mejora.

La justificación de este proyecto se basa en los beneficios esperados para la organización, entre ellos:
\begin{itemize}
    \item Identificación de áreas de oportunidad para mejorar el ambiente laboral y reducir el índice de rotación de empleados.
    \item Incremento en la satisfacción y motivación de los empleados al sentirse escuchados y considerados en decisiones clave.
    \item Mejora en la eficiencia y productividad mediante la implementación de acciones enfocadas en fortalecer el clima organizacional.
\end{itemize}
Este proyecto es una pieza clave en la estrategia de recursos humanos de Client S.A. y responde a la creciente competencia en el mercado laboral, donde las empresas con un buen ambiente de trabajo logran retener y atraer el mejor talento.

\section*{Objetivos del proyecto}
El objetivo general del proyecto es realizar un análisis exhaustivo del ambiente laboral en Client S.A. mediante una encuesta digital, recogiendo datos representativos y fiables de la percepción de los empleados sobre diversas dimensiones del entorno laboral. Se pretende que el proyecto contribuya a la toma de decisiones informadas en torno a la gestión del talento humano.

\subsection*{Objetivos específicos}
\begin{itemize}
    \item Diseñar una encuesta estructurada de hasta 12 preguntas enfocadas en la satisfacción laboral, condiciones de trabajo, comunicación interna y oportunidades de desarrollo.
    \item Asegurar una participación mínima del 70\% de los empleados de Client S.A.
    \item Analizar los resultados de la encuesta utilizando métodos cuantitativos y cualitativos, identificando áreas de mejora.
    \item Generar un informe final con conclusiones claras y recomendaciones accionables para optimizar el ambiente laboral.
\end{itemize}

\section*{Alcance del proyecto}
El alcance del proyecto cubre las actividades necesarias para planificar, ejecutar y entregar un análisis exhaustivo del ambiente laboral. Las actividades específicas incluyen:

\subsection*{Actividades incluidas en el alcance}
\begin{itemize}
    \item \textbf{Diseño de la encuesta:} Desarrollo de una encuesta en formato digital, garantizando que sea fácil de completar y comprendiendo un máximo de 12 preguntas de opción múltiple y escala de satisfacción. La encuesta estará diseñada para evaluar áreas clave como satisfacción general, comunicación con superiores, balance entre vida personal y trabajo, ambiente físico y oportunidades de desarrollo profesional.
    \item \textbf{Distribución de la encuesta:} Envío de la encuesta a través del correo corporativo, asegurando que todos los empleados de Client S.A. reciban la invitación y motivación necesaria para completarla. Se enviarán recordatorios semanales para maximizar la tasa de respuesta.
    \item \textbf{Recopilación de datos:} Los datos serán recopilados a través de una plataforma digital que garantice el anonimato y la seguridad de las respuestas. La recopilación se llevará a cabo durante un período de 10 días hábiles, al final del cual se evaluará la tasa de respuesta y se decidirá si es necesario un segundo recordatorio.
    \item \textbf{Análisis de resultados:} Los datos recolectados serán procesados utilizando herramientas de análisis estadístico como SPSS o Excel. Se realizarán gráficos y tablas para ilustrar las respuestas en función de categorías como departamentos, antigüedad y nivel jerárquico. También se realizarán análisis de texto para identificar patrones cualitativos en comentarios abiertos, en caso de incluirse.
    \item \textbf{Informe final:} Elaboración de un informe completo que incluya hallazgos clave, análisis detallado y recomendaciones prácticas para mejorar el ambiente laboral. El informe final se entregará al cliente en formato digital e impreso, incluyendo secciones detalladas para cada área evaluada en la encuesta.
\end{itemize}

\subsection*{Actividades excluidas del alcance}
El proyecto no incluye las siguientes actividades:
\begin{itemize}
    \item Entrevistas individuales o análisis psicológico de las respuestas.
    \item Recolección de datos a través de métodos alternativos a la encuesta digital, como entrevistas cara a cara o grupos focales.
    \item Implementación de las recomendaciones derivadas de la encuesta.
    \item Análisis continuo o encuestas de seguimiento tras la finalización de este estudio.
\end{itemize}
Estas exclusiones permiten un enfoque centrado y eficiente en el desarrollo del proyecto, limitando las actividades a aquellas que son esenciales para el cumplimiento de los objetivos.

\section*{Requisitos del proyecto}
Para asegurar la correcta realización del proyecto, se deben cumplir los siguientes requisitos:

\subsection*{Requisitos técnicos}
\begin{itemize}
    \item La encuesta debe contener un máximo de 12 preguntas, ser fácil de entender y completarse en menos de 5 minutos.
    \item La distribución y recolección de respuestas deben garantizar la privacidad y el anonimato de los empleados.
    \item La plataforma de encuesta debe contar con mecanismos de seguridad para proteger los datos recolectados.
\end{itemize}

\subsection*{Requisitos de gestión}
\begin{itemize}
    \item Es necesaria la colaboración del departamento de Recursos Humanos de Client S.A. para proporcionar la base de datos de los empleados.
    \item El equipo del proyecto deberá coordinarse con el cliente para validar el diseño de la encuesta antes de su distribución.
    \item Es indispensable mantener una comunicación fluida entre el equipo del proyecto y el cliente, con reportes semanales de avance.
\end{itemize}

\section*{Cronograma del proyecto}
El proyecto tiene una duración total de 4 semanas, distribuidas en las siguientes fases:

\begin{itemize}
    \item \textbf{Semana 1:} Diseño y validación de la encuesta (5 días). Incluye el desarrollo de preguntas, revisión y aprobación por parte de Client S.A.
    \item \textbf{Semana 2:} Envío de la encuesta y seguimiento de respuestas (5 días). El envío inicial se realizará el primer día y se enviarán recordatorios de participación.
    \item \textbf{Semana 3:} Análisis de los datos recopilados (5 días). Durante esta semana, se procesarán los datos utilizando herramientas estadísticas y se comenzará a estructurar el informe final.
    \item \textbf{Semana 4:} Redacción y revisión del informe final (5 días). Se documentarán hallazgos, análisis y recomendaciones, y se entregará una versión preliminar para revisión del cliente.
\end{itemize}

\section*{Recursos necesarios}
\subsection*{Recursos humanos}
\begin{itemize}
    \item \textbf{Director del proyecto:} Responsable de la planificación y ejecución general del proyecto, asegurando que se cumplan los plazos y objetivos.
    \item \textbf{Analista de encuestas:} Encargado del diseño de la encuesta, el procesamiento y el análisis de los datos.
    \item \textbf{Coordinador de comunicación:} Responsable de la comunicación con el cliente y del seguimiento de la tasa de respuesta de la encuesta.
\end{itemize}

\subsection*{Recursos materiales}
\begin{itemize}
    \item \textbf{Software de encuesta digital:} Plataforma segura para diseñar y enviar la encuesta, así como para almacenar las respuestas.
    \item \textbf{Herramientas de análisis de datos:} Programas como Excel o SPSS para el procesamiento y análisis estadístico de los resultados.
    \item \textbf{Equipos de computación:} Ordenadores con acceso a internet y al software necesario para ejecutar el proyecto.
\end{itemize}

\section*{Roles y responsabilidades}
\begin{itemize}
    \item \textbf{Director del proyecto:} Supervisará la ejecución del proyecto, reportando al cliente y asegurando la disponibilidad de los recursos necesarios. También tendrá la responsabilidad de coordinar las reuniones de seguimiento semanal.
    \item \textbf{Analista de encuestas:} Diseñará la encuesta, analizará los datos y generará gráficos y tablas para el informe final.
    \item \textbf{Coordinador de comunicación:} Mantendrá al cliente informado sobre el progreso del proyecto y gestionará los recordatorios para maximizar la tasa de respuesta.
\end{itemize}

\section*{Indicadores de éxito del proyecto}
El éxito del proyecto se medirá mediante los siguientes indicadores:
\begin{itemize}
    \item \textbf{Tasa de respuesta:} Lograr una participación de al menos el 70\% de los empleados de Client S.A.
    \item \textbf{Cumplimiento del cronograma:} Completar todas las fases del proyecto en el plazo de 4 semanas.
    \item \textbf{Calidad del informe final:} El cliente debe expresar satisfacción con el análisis y las recomendaciones presentadas, valorando su utilidad para el desarrollo de políticas de mejora del ambiente laboral.
\end{itemize}

\section*{Riesgos y suposiciones}
\subsection*{Riesgos potenciales}
\begin{itemize}
    \item \textbf{Baja tasa de respuesta:} Existe el riesgo de que los empleados no participen en la encuesta, lo cual podría afectar la representatividad de los resultados. Para mitigar esto, se enviarán recordatorios semanales y se incentivará la participación.
    \item \textbf{Retrasos en el análisis de datos:} Problemas técnicos con el software de análisis o falta de recursos pueden retrasar la fase de análisis. Esto se reducirá al contar con un respaldo de herramientas y un plan de contingencia en caso de que surjan problemas técnicos.
\end{itemize}

\subsection*{Suposiciones}
\begin{itemize}
    \item Se asume que todos los empleados tienen acceso a un correo electrónico corporativo.
    \item Se espera que el departamento de Recursos Humanos de Client S.A. colabore activamente en el proceso de distribución y seguimiento de la encuesta.
\end{itemize}

\section*{Conclusión}
El presente documento de alcance establece una guía clara y detallada para la realización de una encuesta de ambiente laboral en Client S.A., definiendo las actividades, recursos y criterios de éxito necesarios para completar el proyecto en el plazo acordado. Consulting S.L. se compromete a llevar a cabo este proyecto con los más altos estándares de calidad y a mantener una comunicación constante con el cliente para asegurar la satisfacción y utilidad de los resultados obtenidos.

\end{document}
