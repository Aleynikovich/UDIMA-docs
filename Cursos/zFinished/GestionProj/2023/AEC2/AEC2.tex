\documentclass{article}
\usepackage[utf8]{inputenc}
\usepackage{graphicx}
\usepackage{geometry}
\usepackage{enumitem}
\usepackage{comment}
\usepackage{tikz}
\usepackage{amsmath}
\usepackage{grffile}
\usepackage{pdflscape}
\usetikzlibrary{positioning}
\usetikzlibrary{arrows.meta}

\geometry{a4paper, margin=1in}

\title{Caso Práctico I: Procesos en la dirección de proyectos}
\author{Alexander Kalis}
\date{\today}

\begin{document}


\begin{titlepage}

   \begin{center}

   
       %\huge{\bfseries {\MyTitle}}\\
       \line(2,0){200}\\
       [0.75cm]
       \textsc{\LARGE Gestión de Proyectos, AEC2}\\
       [0.75cm]
       \line(2,0){200}\\
       [2cm]
       \includegraphics[height=15cm]{portproj.png}\\
       [3cm]

   \end{center}

   \begin{flushright}

       Autores: Manuel Rubio, Carmelo Garcia, Alexander Kalis\\
       Profesor: Dr. Juan Luis Rubio Sánchez\\
       Curso: Ingeniería de Organización Industrial\\
       UDIMA         

   \end{flushright}
   
\end{titlepage}

\tableofcontents

\newpage

\section{Enunciado 1}

Un proyecto consta de 8 subproyectos S1,S2,S3,S4,S5,S6,S7,S8. Cada subproyecto se estima que durará 
2 meses y tendrán un coste individual de 10.000 € Los subprpyectos están enlazados secuencialmente:
Al finalizar el mes 12 la situación es la siguiente:


\begin{center}
    \includegraphics[width=0.9\textwidth]{e1p1.png}\\
\end{center}


\subsection{Calcular PV, EV, AC, BAC, CV, CPI, SV, SPI, TCPI, EAC, ETC para cada periodo.}

\textbf{Datos de los Subproyectos:}

\begin{itemize}
    \item BAC (Presupuesto a la Finalización) = 80,000 € (10,000 € por cada uno de los 8 subproyectos)
    \item PV (Valor Planificado) para cada subproyecto = 10,000 €
    \item AC (Coste Real) y EV (Valor Ganado) son dados para cada subproyecto.
\end{itemize}

\textbf{Fórmulas Aplicadas:}
\begin{align*}
    SV &= EV - PV \\
    CV &= EV - AC \\
    SPI &= \frac{EV}{PV} \\
    CPI &= \frac{EV}{AC} \\
    EAC &= AC + (BAC - EV) \\
    TCPI &= \frac{BAC - EV}{BAC - AC}
\end{align*}


% Subproyecto S1
\subsubsection{Subproyecto S1}
AC = 10,000 €, \, EV = 10,000 € \\
SV = 0 €, \, CV = 0 €, \, SPI = 1, \, CPI = 1 \\
EAC = 80,000 €, \, TCPI = 0.875

% Subproyecto S2
\subsubsection{Subproyecto S2}
AC = 11,000 €, \, EV = 10,000 € \\
SV = 0 €, \, CV = -1,000 €, \, SPI = 1, \, CPI = 0.91 \\
EAC = 81,000 €, \, TCPI = 0.875

% Subproyecto S3
\subsubsection{Subproyecto S3}
AC = 12,000 €, \, EV = 10,000 € \\
SV = 0 €, \, CV = -2,000 €, \, SPI = 1, \, CPI = 0.83 \\
EAC = 82,000 €, \, TCPI = 0.875

% Subproyecto S4
\subsubsection{Subproyecto S4}
AC = 11,000 €, \, EV = 10,000 € \\
SV = 0 €, \, CV = -1,000 €, \, SPI = 1, \, CPI = 0.91 \\
EAC = 81,000 €, \, TCPI = 0.875

% Subproyecto S5
\subsubsection{Subproyecto S5}
AC = 12,000 €, \, EV = 10,000 € \\
SV = 0 €, \, CV = -2,000 €, \, SPI = 1, \, CPI = 0.83 \\
EAC = 82,000 €, \, TCPI = 0.875

% Subproyecto S6
\subsubsection{Subproyecto S6}
AC = 6,000 €, \, EV = 5,000 € \\
SV = -5,000 €, \, CV = -1,000 €, \, SPI = 0.5, \, CPI = 0.83 \\
EAC = 76,000 €, \, TCPI = 1.25

% Subproyecto S7
\subsubsection{Subproyecto S7}
AC = 5,000 €, \, EV = 5,000 € \\
SV = -5,000 €, \, CV = 0 €, \, SPI = 0.5, \, CPI = 1 \\
EAC = 75,000 €, \, TCPI = 1.25



\subsection{Estimación del Tiempo de Finalización y Costo Total del Proyecto}

Dado que se asume que las tareas 6, 7 y 8 se completarán a un ritmo y con un nivel de gasto similar al actual, utilizaremos los índices de desempeño de tiempos (SPI) y costes (CPI) promedio para estas estimaciones.

\subsubsection{Cálculo del SPI Promedio}
El SPI promedio se calcula como el promedio de los SPI de los subproyectos completados (S1 a S7):
\begin{align*}
    SPI_{promedio} &= \frac{1 + 1 + 1 + 1 + 1 + 0.5 + 0.5}{7} 
    = \frac{6}{7} 
    = 0.857
\end{align*}

\subsubsection*{Cálculo del CPI Promedio}
De manera similar, el CPI promedio se calcula como el promedio de los CPI de los subproyectos completados (S1 a S7):
\begin{align*}
    CPI_{promedio} &= \frac{1 + 0.91 + 0.83 + 0.91 + 0.83 + 0.83 + 1}{7} 
    = \frac{6.32}{7} 
    = 0.903
\end{align*}

\subsubsection*{Estimación del Tiempo de Finalización del Proyecto}
Usando el SPI promedio, el tiempo estimado de finalización (TEFP) es:
\begin{align*}
    TEFP &= \frac{Duraci\acute{o}n\,Original\,del\,Proyecto}{SPI_{promedio}} 
    = \frac{16\,meses}{0.857} 
    = 18.67\,meses
\end{align*}

\subsubsection*{Estimación del Costo Total del Proyecto}
Utilizando el CPI promedio, el costo total estimado (CTEP) es:
\begin{align*}
    CTEP &= \frac{BAC}{CPI_{promedio}} 
    = \frac{80000\,€}{0.903} 
    = 88594.68 €
\end{align*}


\subsection{Reflexión sobre la Actuación del Jefe de Proyecto y Propuestas de Acción}

\subsubsection{Análisis del Desempeño del Proyecto hasta el Mes 12}
\begin{itemize}
    \item \textbf{Desviaciones en Coste y Tiempo (CV y SV):} Hemos observado que algunos subproyectos han incurrido en sobrecostes y retrasos. Esto puede sugerir que la planificación inicial fue demasiado optimista o que surgieron problemas inesperados.
    \item \textbf{Índices de Eficiencia (CPI y SPI):} Los valores de CPI menores que 1 en varios subproyectos indican sobrecostes, mientras que los valores de SPI menores que 1 reflejan retrasos en algunos subproyectos.
    \item \textbf{Estimaciones Futuras (EAC y TCPI):} Las estimaciones indican que el proyecto podría terminar con un sobrecoste y retraso si se mantiene el ritmo y nivel de gasto actual.
\end{itemize}

\subsubsection{Propuestas de Acción a partir del Mes 13}
Como jefe de proyecto, sería esencial tomar medidas correctivas inmediatas:
\begin{enumerate}
    \item \textbf{Revisión del Plan de Proyecto:} Evaluar y ajustar la planificación de los subproyectos restantes para alinearlos mejor con la realidad del proyecto.
    \item \textbf{Control de Costes:} Implementar un control más estricto de los costes, identificando y mitigando las causas de los sobrecostes.
    \item \textbf{Gestión de Riesgos:} Revisar y actualizar el plan de gestión de riesgos, prestando especial atención a los riesgos que ya han impactado el proyecto.
    \item \textbf{Comunicación con el Equipo y Stakeholders:} Mantener una comunicación transparente y regular con el equipo y los stakeholders sobre los cambios y el progreso del proyecto.
    \item \textbf{Seguimiento Continuo:} Realizar un seguimiento continuo de los indicadores de desempeño del proyecto (CPI y SPI) para asegurar que las acciones correctivas están teniendo el efecto deseado.
\end{enumerate}



\section{Enunciado 2}

En el ejercicio anterior se sabe que existe una ameza A6 cuya probabilidad de ocurrencia es alta, que
puede afectar al subproyecto S6. Dado que el subproyecto S6 pertenece al cc se le otorga un valor muy
alto para la conclusión exitosa del proyecto. Los subproyectos S7 y S8 se valoran como ‘medio’ de cara a
la conclusión con éxito del proyecto, ya que aún siendo del cc, existen alternativas en caso de que
presentaran problemas o retrasos.
En caso de que se materializara A6, se sabe que la actividad quedaría completamente paralizada, por lo
que el jefe de proyecto decide tomar medidas para que en caso de ocurrencia, el impacto quede reducido a
un valor muy bajo.
Se pide estudiar qué acciones se deberían tomar a partir de los valores RAR y RRR finales, sabiendo que
la tolerancia al riesgo de la organización se situa en un valor muy alto.


\subsection{Contexto del riesgo}
El subproyecto S6 se enfrenta a la amenaza A6, con una alta probabilidad de ocurrencia. Dado su valor crítico para el éxito del proyecto, es imperativo gestionar este riesgo de manera efectiva.

\subsubsection{Evaluación del riesgo antes de la respuesta (RAR)}
\begin{itemize}
    \item \textbf{Probabilidad de ocurrencia:} Alta.
    \item \textbf{Impacto en caso de ocurrencia:} Paralización completa del subproyecto S6.
    \item \textbf{Valoración del riesgo (RAR):} Muy alto debido a la importancia crítica del subproyecto S6 para el éxito del proyecto.
\end{itemize}

\subsubsection{Estrategias de mitigación y gestión del riesgo}
Dado que la organización tiene una alta tolerancia al riesgo, el jefe de proyecto decide tomar medidas para reducir el impacto a un valor muy bajo.

\begin{enumerate}
    \item \textbf{Desarrollo de un plan de contingencia:} Creación de un plan detallado para acciones inmediatas en caso de que A6 se materialice.
    \item \textbf{Refuerzo de los controles del proyecto:} Aumentar la supervisión y el control en las fases críticas del subproyecto S6 para detectar tempranamente cualquier señal de la amenaza A6.
    \item \textbf{Formación y preparación del equipo:} Capacitar al equipo sobre cómo actuar eficazmente en caso de que la amenaza se materialice.
    \item \textbf{Asignación de recursos adicionales:} Prever recursos adicionales (humanos, técnicos, financieros) para abordar rápidamente la amenaza si ocurre.
\end{enumerate}

\subsubsection{Evaluación del riesgo restante residual (RRR)}
Tras implementar las medidas de mitigación, el RRR se espera que sea:
\begin{itemize}
    \item \textbf{Probabilidad de ocurrencia:} Sigue siendo alta.
    \item \textbf{Impacto en caso de ocurrencia:} Reducido a un valor muy bajo.
    \item \textbf{Valoración del riesgo (RRR):} Bajo, en línea con la alta tolerancia al riesgo de la organización.
\end{itemize}



\end{document}