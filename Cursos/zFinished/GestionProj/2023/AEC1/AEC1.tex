\documentclass{article}
\usepackage[utf8]{inputenc}
\usepackage{graphicx}
\usepackage{geometry}
\usepackage{enumitem}
\usepackage{comment}
\usepackage{tikz}
\usepackage{amsmath}
\usepackage{grffile}
\usepackage{pdflscape}
\usetikzlibrary{positioning}
\usetikzlibrary{arrows.meta}

\geometry{a4paper, margin=1in}

\title{Caso Práctico I: Procesos en la dirección de proyectos}
\author{Alexander Kalis}
\date{\today}

\begin{document}


\begin{titlepage}

   \begin{center}

   
       %\huge{\bfseries {\MyTitle}}\\
       \line(2,0){200}\\
       [0.75cm]
       \textsc{\LARGE Gestión de Proyectos}\\
       [0.75cm]
       \line(2,0){200}\\
       [2cm]
       \includegraphics[height=15cm]{E:/KUKADisk/OneDrive - KUKA AG/UDIMA/GestionProj/portproj.png}\\
       [3cm]

   \end{center}

   \begin{flushright}

       Autores: Manuel Rubio, Carmelo Garcia, Alexander Kalis\\
       Profesor: Dr. Juan Luis Rubio Sánchez\\
       Curso: Ingeniería de Organización Industrial\\
       UDIMA         

   \end{flushright}
   
\end{titlepage}



\section{Enunciado 1}
\subsection{Introducción}
En cualquier proyecto la definición y aplicación de los diferentes procesos con los que se gestionará cada
fase del proyecto es una actividad de la que depende el éxito o fracaso del mismo. La correcta definición
del alcance así como la correcta estimación de tiempos y planificación de los recursos determina las
herramientas para facilitar el seguimiento y control del proyecto, cuestión esta por la que dicha actividad es
vital en la vida del mismo.

\subsection{Objetivo del Caso Práctico}
El objetivo del caso práctico es la aplicación de las técnicas de planificación vistas en teoría a un caso real
y la familiarización con herramientas de software que faciliten estas tareas.

\subsection{Enunciado}
La empresa Suelos S.L. ha recibido en encargo de poner tarima flotante en un piso que tiene 120m²,
distribuidos en 3 habitaciones (10m² cada una), un salón (25 m²), pasillos y hall (12 m²) y el resto son
cocina, baños y terraza que no llevan tarima. Se pide acabar el proyecto en un plazo de 10 días,
incluyendo la colocación de rodapié, limpieza de la obra y eliminación de residuos.

\subsubsection{Se pide}
\begin{enumerate}
    \item Elaborar un esbozo del documento de Alcance del Proyecto que debe contener:
    \begin{itemize}
        \item Documento de requisitos
        \item Matriz de requisitos
        \item Entregables o hitos
        \item EDT, con al menos 8 actividades.
        \item Criterios de aceptación
    \end{itemize}
    En caso de ser necesario se pueden incorporar cuantas hipótesis sean precisas indicándolas claramente.
    
    \item Generar un posible diagrama de actividades y otro de Gantt mediante alguna herramienta software.
\end{enumerate}

\subsection{Documento de Alcance del Proyecto}

\subsubsection{Descripción del Proyecto}
\begin{itemize}
    \item El proyecto consiste en la instalación de tarima flotante en un piso de 120m².
    \item El trabajo incluye la colocación de rodapié, limpieza de la obra y eliminación de residuos.
    \item El plazo para completar el proyecto es de 10 días.
\end{itemize}

\subsubsection{Documento de Requisitos}
\begin{itemize}
    \item Se requiere la instalación de tarima flotante en todas las áreas del piso, excepto la cocina, baños y terraza.
    \item La tarima debe ser de alta calidad y cumplir con las normativas locales.
    \item Los rodapiés deben ser instalados en todas las áreas con tarima.
    \item La obra debe estar completamente limpia al finalizar.
    \item Todos los residuos generados durante la instalación deben ser eliminados adecuadamente.
\end{itemize}

\subsubsection{Matriz de Requisitos}

\begin{tabular}{|p{6cm}|c|}
\hline
Requisito & Cumplimiento \\
\hline
Instalación de tarima en áreas designadas & Sí \\
Uso de tarima de alta calidad & Sí \\
Instalación de rodapié en todas las áreas & Sí \\
Limpieza completa de la obra al finalizar & Sí \\
Eliminación adecuada de residuos & Sí \\
\hline
\end{tabular}

\subsubsection{Entregables o Hitos}
\begin{itemize}
    \item Hito 1: Instalación de tarima completada.
    \item Hito 2: Instalación de rodapié completada.
    \item Hito 3: Limpieza de la obra realizada.
    \item Hito 4: Eliminación de residuos completada.
    \item Hito 5: Finalización del proyecto.
\end{itemize}

\subsubsection{EDT (Estructura de Desglose de Trabajo) con al menos 8 actividades}

\begin{figure}[h]
   \centering
   \includegraphics[width=0.9\textwidth]{E:/KUKADisk/OneDrive - KUKA AG/UDIMA/GestionProj/c1p1.PNG}
 \end{figure}

 \begin{figure}[h]
   \centering
   \includegraphics[width=0.9\textwidth]{E:/KUKADisk/OneDrive - KUKA AG/UDIMA/GestionProj/c1p2.PNG}
 \end{figure}

 \begin{figure}[h]
   \centering
   \includegraphics[width=0.9\textwidth]{E:/KUKADisk/OneDrive - KUKA AG/UDIMA/GestionProj/c1p3.PNG}
 \end{figure}

\subsubsection{Criterios de Aceptación}
\begin{itemize}
    \item Todas las áreas designadas tienen tarima instalada de manera adecuada.
    \item Los rodapiés están instalados correctamente en todas las áreas.
    \item La obra está limpia y libre de residuos.
    \item El proyecto se ha completado en un plazo máximo de 10 días.
\end{itemize}


\subsection{Diagrama de Actividades}

\begin{figure}[h]
   \centering
   \includegraphics[width=0.9\textwidth]{E:/KUKADisk/OneDrive - KUKA AG/UDIMA/GestionProj/ganntc1.PNG}
   \caption{Diagrama de Actividades}
 \end{figure}

\newpage




\section{Enunciado 2}

\subsection{Diagrama de Actividades}

\begin{figure}[h]
   \centering
   \includegraphics[width=0.9\textwidth]{E:/KUKADisk/OneDrive - KUKA AG/UDIMA/GestionProj/c2p1.PNG}
   \caption{Diagrama de Actividades}
 \end{figure}

\subsection{Cálculo del Camino Crítico}

\begin{figure}[h]
   \centering
   \includegraphics[width=0.9\textwidth]{E:/KUKADisk/OneDrive - KUKA AG/UDIMA/GestionProj/c2p2.PNG}
   \caption{Cálculo del Camino Crítico}
 \end{figure}

 \newpage
\subsection{Camino Crítico en OpenProj}

\begin{figure}[h]
   \centering
   \includegraphics[height=7cm]{E:/KUKADisk/OneDrive - KUKA AG/UDIMA/GestionProj/c2p3.PNG}
   \caption{Camino Crítico en OpenProj}
 \end{figure}

\subsection{Cálculo de Desviación Admisible}

\begin{figure}[h]
   \centering
   \includegraphics[width=0.9\textwidth]{E:/KUKADisk/OneDrive - KUKA AG/UDIMA/GestionProj/c2p4.PNG}
   \caption{Cálculo de Desviación Admisible}
 \end{figure}

 Las tareas A, D, J y L son tareas críticas y no pueden retrasarse ninguna semana sin alargar el periodo total de las 14 semanas para terminar el proyecto.\\
La tarea B se puede retrasar 1 semana, lo que supone el 25\% sobre el comienzo más tardío.\\
La tarea C se puede retrasar 7 semanas sin afectar a la duración del proyecto, lo que supone el 100\% sobre su periodo de comienzo más tardío.\\
La tarea E se puede retrasar 4 semanas sin afectar a la duración del proyecto, lo que supone el 57,14\% sobre su periodo de comienzo más tardío.\\
La tarea F se puede retrasar 1 semana sin afectar a la duración del proyecto, lo que supone el 11,11\% sobre su periodo de comienzo más tardío.\\
La tarea G se puede retrasar 2 semanas sin afectar a la duración del proyecto, lo que supone el 25\% sobre su periodo de comienzo más tardío.\\
La tarea H se puede retrasar 2 semanas sin afectar a la duración del proyecto, lo que supone el 25\% sobre su periodo de comienzo más tardío.\\
La tarea I se puede retrasar 4 semanas sin afectar a la duración del proyecto, lo que supone el 40\% sobre su periodo de comienzo más tardío.\\
La tarea K se puede retrasar 1 semana sin afectar a la duración del proyecto, lo que supone el 10\% sobre su periodo de comienzo más tardío.\\

\subsection{Impacto del Cambio de Legislación}
El proyecto no se vería afectado ya que el comienzo más temprano de la actividad L viene marcado por la actividad más tardía entre J y K, que en este caso es J con 11 semanas, mientras que la actividad K parte de 10 semanas y, por tanto, retrasándose una semana partiría de 11 semanas la actividad L partiría de 11 semanas, que es de donde parte ahora mismo al termina de la actividad J.

\subsection{Impacto del Accidente en la Actividad D}
El proyecto se alargaría una semana más pasando de 14 a 15 ya que la actividad D es crítica.

\begin{figure}[h]
   \centering
   \includegraphics[width=0.9\textwidth]{E:/KUKADisk/OneDrive - KUKA AG/UDIMA/GestionProj/c2p6.PNG}
   \caption{Impacto del Accidente en la Actividad D}
 \end{figure}

 
\begin{figure}[h]
   \centering
   \includegraphics[width=0.9\textwidth]{E:/KUKADisk/OneDrive - KUKA AG/UDIMA/GestionProj/c2p62.PNG}
   \caption{Impacto del Accidente en la Actividad D}
 \end{figure}

 \begin{figure}[h]
   \centering
   \includegraphics[width=0.9\textwidth]{E:/KUKADisk/OneDrive - KUKA AG/UDIMA/GestionProj/c2p63.PNG}
   \caption{Impacto del Accidente en la Actividad D}
 \end{figure}

 Pero las holguras del resto de tareas no criticas aumentarían también en una semana tal como quedan en el siguiente detalle.

 \begin{figure}[h]
   \centering
   \includegraphics[width=0.9\textwidth]{E:/KUKADisk/OneDrive - KUKA AG/UDIMA/GestionProj/c2p64.PNG}

 \end{figure}

\subsection{Demostrar gráficamente que si la última actividad del camino crítico se retrasa 1 semana más de lo obtenido en el apartado 4, el camino critico cambia en tareas y/o duración}

Si aumentamos el periodo de la actividad L (última del camino crítico visto A,D,J,L), y pasa de 3 a 4 semanas, la duración total de las tareas pasan de 14 a 15 semanas, pero no así la ruta crítica que seguiría siendo la misma (A,D,L,J)

\begin{figure}[h]
   \centering
   \includegraphics[width=0.9\textwidth]{E:/KUKADisk/OneDrive - KUKA AG/UDIMA/GestionProj/c2p7.PNG}

 \end{figure}


 \begin{figure}[h]
   \centering
   \includegraphics[width=0.9\textwidth]{E:/KUKADisk/OneDrive - KUKA AG/UDIMA/GestionProj/c2p71.PNG}

 \end{figure}



\subsection{Elaboración del EDT Detallado}

Como tareas dentro de cada actividad hemos considerado las siguientes


\begin{figure}[h]
   \centering
   \includegraphics[width=0.9\textwidth]{E:/KUKADisk/OneDrive - KUKA AG/UDIMA/GestionProj/c2p8.PNG}
 \end{figure}

 \begin{figure}[h]
   \centering
   \includegraphics[width=0.9\textwidth]{E:/KUKADisk/OneDrive - KUKA AG/UDIMA/GestionProj/c2p81.PNG}
 \end{figure}

 \begin{figure}[h]
   \centering
   \includegraphics[width=0.9\textwidth]{E:/KUKADisk/OneDrive - KUKA AG/UDIMA/GestionProj/c2p82.PNG}
 \end{figure}


\end{document}