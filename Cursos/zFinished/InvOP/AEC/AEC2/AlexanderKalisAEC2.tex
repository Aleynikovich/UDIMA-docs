\documentclass{article}
\usepackage[utf8x]{inputenc}
\usepackage{lmodern,textcomp}
\usepackage{lipsum}
\usepackage{authoraftertitle}
\usepackage[top=2cm,bottom=1.5cm,left=1.5cm, right=3cm,includeheadfoot]{geometry}
\usepackage{graphicx}
\usepackage{fancyhdr}
%\usepackage[spanish]{babel}
\usepackage{mathtools}
\usepackage{nicefrac}
\usepackage{csquotes}
\usepackage{amssymb}
\usepackage{fancybox, graphicx}
\usepackage{array}
\usepackage{hhline}
\usepackage{hyperref}
\usepackage{tikz}
\usepackage{amsmath}
\usepackage{wrapfig}
\usepackage{float}
\usepackage{amsmath}
\usepackage{esint}
\usepackage{caption}
\usepackage[]{ucs}
\usepackage{esvect}
\usepackage{siunitx}
\usepackage{commath}
\newcommand{\ihat}{\textbf{\^\i}}
\newcommand{\jhat}{\textbf{\^\j}}
%Header & Footer

\pagestyle{fancy}
%\fancyhead[LE]{\MyTitle}
\fancyhead[LO]{Investigación Operativa}
\fancyhead[RO]{Actividad de Evaluación Contínua 2}
%\fancyhead[RE]{\leftmark}
\fancyfoot[L]{\raisebox{-1cm}{\includegraphics[height=1.5cm]{D:/KUKADisk/OneDrive - KUKA AG/UDIMA/DocumentGraphics/LOGOUDIMA.jpg}}}
\fancyfoot[R]{Corregido:\\ Dr. Fco. David de la Peña Esteban}
%\fancyfoot[RO]{07/12/2018}


%Vars
\author{Alexander Sebastian Kalis}
\title{Actividad de Evaluación Contínua 2}


%DOC


\begin{document}

\begin{titlepage}

    \begin{center}

        \line(1,0){300}\\
        [0.2in]
        \huge{\bfseries {\MyTitle}}\\
        [1mm]
        \line(2,0){200}\\
        [0.75cm]
        \textsc{\LARGE Investigación Operativa}\\
        [2cm]
        \includegraphics[height=10cm]{D:/KUKADisk/OneDrive - KUKA AG/UDIMA/InvOP/AEC/AEC1/img/portada.png}\\
        [3cm]

    \end{center}

    \begin{flushright}

        Autor: {\MyAuthor}\\
        Profesor: Dr. Fco. David de la Peña Esteban\\
        Ingeniería de Organización Industrial\\
        UDIMA         

    \end{flushright}
    
\end{titlepage}

\newpage

\section*{Caso 1}

Se tienen cinco recursos (A, B, C, D, E) para realizar cuatro actividades (1, 2, 3, 4). En
la siguiente tabla están los tiempos estimados de realización de cada una de las
actividades por los recursos. Existe la restricción de que el Recurso B no puede
realizar la Actividad 3.

\begin{center}
    \includegraphics[width=.5\textwidth]{D:/KUKADisk/OneDrive - KUKA AG/UDIMA/InvOP/AEC/AEC2/imgs/c1e1.PNG}\\
\end{center}

a-Identificar el tipo de problema por sus características y especificar cuál es la tećnica
de resolución que vais a emplear.\\

Se trata de un problema de asignación imposible por lo cual realizamos una tarea ficticia.\\


b-Especificar qué tareas hace cada recurso si se quiere minimizar el tiempo total.
Calcular dicho tiempo.\\

\begin{center}
    \includegraphics[width=.8\textwidth]{D:/KUKADisk/OneDrive - KUKA AG/UDIMA/InvOP/AEC/AEC2/imgs/c1r1.PNG}\\
\end{center}

\begin{center}
    \includegraphics[width=.8\textwidth]{D:/KUKADisk/OneDrive - KUKA AG/UDIMA/InvOP/AEC/AEC2/imgs/c1r2.PNG}\\
\end{center}

Calulando el tiempo obtenemos:

\[
    Tiempo = 22+42+70+56+0=190s
\]

\newpage

\section*{Caso 2}

Una empresa que se dedica a la fabricación y distribución de productos, tiene 2
fábricas (O1 y O2) que deben dar servicio a 3 localidades (D1, D2 y D3). La demanda
prevista que como mínimo hay que cubrir en cada una de las localidades es la
siguiente para la semana que viene:

\begin{center}
    \includegraphics[width=.5\textwidth]{D:/KUKADisk/OneDrive - KUKA AG/UDIMA/InvOP/AEC/AEC2/imgs/c2e1.PNG}\\
\end{center}

Los costes unitarios (en euros) de transporte desde cada una de las fábricas a las
localidades son los siguientes:

\begin{center}
    \includegraphics[width=.5\textwidth]{D:/KUKADisk/OneDrive - KUKA AG/UDIMA/InvOP/AEC/AEC2/imgs/c2e2.PNG}\\
\end{center}

Está la restricción de que no puede haber transporte desde O1 a D2.
Las capacidades de producción de las fábricas para la semana que viene son de 550
unidades para O1, y de 375 unidades para O2.


a-Identificar el tipo de problema por sus características y especificar cuál es la tećnica
de resolución que vais a emplear.\\

En este caso estamos ante un problema de transporte imposible en el que nos encontramos con más demanda que oferta. Generaremos entronces un origen ficticio y aplicaremos 
el método de costes mínimos para ir rellenando la tabla.


b-Cuantificar las unidades de producto que deben ir, la semana que viene, desde cada
una de las fábricas a cada de las localidades a las que se da servicio, buscando
minimizar los costes totales.\\

Podemos observar como queda la tabla paso a paso al aplicar el método de costes mínimos:

Con los recursos de O1 cumplimos toda la demanda de D1 y nos quedan 150 recursos sobrantes que aplicaremos a D3 ya que D2 no es un destino posible desde O1:
\begin{center}
    \includegraphics[width=.9\textwidth]{D:/KUKADisk/OneDrive - KUKA AG/UDIMA/InvOP/AEC/AEC2/imgs/c2r1.PNG}\\
\end{center}

Luego utilizamos los recursos de O2 para terminar de cumplir con la demanda de D3 (125) ya que es el coste mínimo. Con los recursos sobrantes sólo podemos asignarlos a D2 ya que en D1
ya no hay demanda pendiente:

\begin{center}
    \includegraphics[width=.9\textwidth]{D:/KUKADisk/OneDrive - KUKA AG/UDIMA/InvOP/AEC/AEC2/imgs/c2r2.PNG}\\
\end{center}

Finalmente nos quedan 75 recursos para la demanda de 75 en D2:

\begin{center}
    \includegraphics[width=.9\textwidth]{D:/KUKADisk/OneDrive - KUKA AG/UDIMA/InvOP/AEC/AEC2/imgs/c2r3.PNG}\\
\end{center}




\section*{Caso 3}
Una empresa que se dedica a la fabricación y distribución de productos, tiene 3
fabricas (O1, O2, O3) que deben dar servicio a 2 localidades (D1 y D2). La demandar
prevista que como mínimo hay que cubrir en cada una de las localidades es la
siguiente para la semana que viene:

\begin{center}
    \includegraphics[width=.5\textwidth]{D:/KUKADisk/OneDrive - KUKA AG/UDIMA/InvOP/AEC/AEC2/imgs/c3e1.PNG}\\
\end{center}

Los costes unitarios (en euros) de transporte desde cada una de las fábricas a las
localidades son los siguientes:


\begin{center}
    \includegraphics[width=.5\textwidth]{D:/KUKADisk/OneDrive - KUKA AG/UDIMA/InvOP/AEC/AEC2/imgs/c3e2.PNG}\\
\end{center}

Está la restricción de que no puede haber transporte desde O1 a D2.
Las capacidades de producción de las fábricas para la semana que viene son de 700
unidades para O1, de 400 unidades para O2 y de 375 unidades para O3.

a-Identificar el tipo de problema por sus características y especificar cuál es la tećnica
de resolución que vais a emplear.\\

Tenemos el mismo caso que en el anterior, pero ahora hay más oferta que demanda. Sabiendo esto, generamos un destino ficticio para equilibrarlo. Por otro lado,
optimizaremos los costes con el método de costes mínimos.



b-Cuantificar las unidades de producto que deben ir, la semana que viene, desde cada
una de las fábricas a cada de las localidades a las que se da servicio, buscando
minimizar los costes totales.\\

Volvemos a realizar la tabla paso a paso utilizando el método de costes mínimos:

Primero adjudicamos todos los recursos posibles de O3 a D2 y por tanto 0 a D1 y DF:


\begin{center}
    \includegraphics[width=.9\textwidth]{D:/KUKADisk/OneDrive - KUKA AG/UDIMA/InvOP/AEC/AEC2/imgs/c3r1.PNG}\\
\end{center}

Para asegurar la demanda de D2, utilizamos 375 recursos de O2:

\begin{center}
    \includegraphics[width=.9\textwidth]{D:/KUKADisk/OneDrive - KUKA AG/UDIMA/InvOP/AEC/AEC2/imgs/c3r2.PNG}\\
\end{center}

Ahora el siguiente coste mínimo que podemos aplicar es O1:D1 ya que en O3 no nos quedan recursos:

\begin{center}
    \includegraphics[width=.9\textwidth]{D:/KUKADisk/OneDrive - KUKA AG/UDIMA/InvOP/AEC/AEC2/imgs/c3r3.PNG}\\
\end{center}

Finalmente nos queda por cubrir con los recursos sobrantes el destino ficticio:

\begin{center}
    \includegraphics[width=.9\textwidth]{D:/KUKADisk/OneDrive - KUKA AG/UDIMA/InvOP/AEC/AEC2/imgs/c3r4.PNG}\\
\end{center}




\section*{Caso 4}

Dada la siguiente red de flujo, se quiere saber cuál será el flujo máximo que puede ir
desde el origen O hasta el destino T, y cuáles van a ser las capacidades que se van a
utilizar de cada uno de los arcos:

\begin{center}
    \includegraphics[width=.7\textwidth]{D:/KUKADisk/OneDrive - KUKA AG/UDIMA/InvOP/AEC/AEC2/imgs/c4e1.PNG}\\
\end{center}

a-Identificar el tipo de problema por sus características y especificar cuál es la tećnica
de resolución que vais a emplear.

Se trata de un problema de Flujo máximo entre dos nodos. Para resolverlo, preparamos el grafo, obtenemos la trayectoria de aumento, determinamos el flujo de la misma, actualizamos
el grafo y repetimos hasta que no existan más trayectorias de aumento.

b-Identificar las trayectorias de aumento y el flujo de cada una.

c-Calcular el flujo máximo total.


d-Especificar la capacidad residual y utilizada de cada arco

\begin{center}
    \includegraphics[width=.7\textwidth]{D:/KUKADisk/OneDrive - KUKA AG/UDIMA/InvOP/AEC/AEC2/imgs/c4r1.PNG}\\
\end{center}




TA1:

\begin{center}
    \includegraphics[width=.7\textwidth]{D:/KUKADisk/OneDrive - KUKA AG/UDIMA/InvOP/AEC/AEC2/imgs/c4r2.PNG}\\
\end{center}

\begin{center}
    \includegraphics[width=.7\textwidth]{D:/KUKADisk/OneDrive - KUKA AG/UDIMA/InvOP/AEC/AEC2/imgs/c4r3.PNG}\\
\end{center}

TA2:

\begin{center}
    \includegraphics[width=.7\textwidth]{D:/KUKADisk/OneDrive - KUKA AG/UDIMA/InvOP/AEC/AEC2/imgs/c4r11.PNG}\\
\end{center}

\begin{center}
    \includegraphics[width=.7\textwidth]{D:/KUKADisk/OneDrive - KUKA AG/UDIMA/InvOP/AEC/AEC2/imgs/c2r22.PNG}\\
\end{center}

TA3:

\begin{center}
    \includegraphics[width=.7\textwidth]{D:/KUKADisk/OneDrive - KUKA AG/UDIMA/InvOP/AEC/AEC2/imgs/c4r31.PNG}\\
\end{center}

\begin{center}
    \includegraphics[width=.7\textwidth]{D:/KUKADisk/OneDrive - KUKA AG/UDIMA/InvOP/AEC/AEC2/imgs/c4r32.PNG}\\
\end{center}

Por lo tanto si sumamos todos los flujos obtenemos:

\[
    Fmax = 18
\]







\end{document}