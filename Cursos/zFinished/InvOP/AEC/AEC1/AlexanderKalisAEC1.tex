\documentclass{article}
\usepackage[utf8x]{inputenc}
\usepackage{lmodern,textcomp}
\usepackage{lipsum}
\usepackage{authoraftertitle}
\usepackage[top=2cm,bottom=1.5cm,left=1.5cm, right=3cm,includeheadfoot]{geometry}
\usepackage{graphicx}
\usepackage{fancyhdr}
%\usepackage[spanish]{babel}
\usepackage{mathtools}
\usepackage{nicefrac}
\usepackage{csquotes}
\usepackage{amssymb}
\usepackage{fancybox, graphicx}
\usepackage{array}
\usepackage{hhline}
\usepackage{hyperref}
\usepackage{tikz}
\usepackage{amsmath}
\usepackage{wrapfig}
\usepackage{float}
\usepackage{amsmath}
\usepackage{esint}
\usepackage{caption}
\usepackage{esvect}
\usepackage{siunitx}
\usepackage{commath}
\newcommand{\ihat}{\textbf{\^\i}}
\newcommand{\jhat}{\textbf{\^\j}}
%Header & Footer

\pagestyle{fancy}
%\fancyhead[LE]{\MyTitle}
\fancyhead[LO]{Investigación Operativa}
\fancyhead[RO]{Actividad de Evaluación Contínua 1}
%\fancyhead[RE]{\leftmark}
\fancyfoot[L]{\raisebox{-1cm}{\includegraphics[height=1.5cm]{D:/KUKADisk/OneDrive - KUKA AG/UDIMA/DocumentGraphics/LOGOUDIMA.jpg}}}
\fancyfoot[R]{Corregido:\\ Dr. Fco. David de la Peña Esteban}
%\fancyfoot[RO]{07/12/2018}


%Vars
\author{Alexander Sebastian Kalis}
\title{Actividad de Evaluación Contínua 1}


%DOC


\begin{document}

\begin{titlepage}

    \begin{center}

        \line(1,0){300}\\
        [0.2in]
        \huge{\bfseries {\MyTitle}}\\
        [1mm]
        \line(2,0){200}\\
        [0.75cm]
        \textsc{\LARGE Investigación Operativa}\\
        [2cm]
        \includegraphics[height=10cm]{D:/KUKADisk/OneDrive - KUKA AG/UDIMA/InvOP/AEC/AEC1/img/portada.png}\\
        [3cm]

    \end{center}

    \begin{flushright}

        Autor: {\MyAuthor}\\
        Profesor: Dr. Fco. David de la Peña Esteban\\
        Ingeniería de Organización Industrial\\
        UDIMA         

    \end{flushright}
    
\end{titlepage}

\newpage

\section*{Caso 1. Formulación de problema de programación lineal }

Una compañía tiene 2 edificios, cada uno con un horario distinto:

-Edificio 1: Se quiere que este edificio esté abierto de 10:00 a 20:00.

-Edificio 2: Se quiere que este edificio esté abierto de 10:00 a 16:00.

El departamento de seguridad establece turnos de vigilancia que empiezan cada 2 
horas. Los vigilantes necesarios por rangos horarios en cada edificio son:\\

\begin{center}
    \includegraphics[height=3cm]{D:/KUKADisk/OneDrive - KUKA AG/UDIMA/InvOP/AEC/AEC1/img/c1e1.PNG}\\
\end{center}

Se contrata a los vigilantes para hacer turnos de 4 horas, 6 horas o de 8 horas. El 
coste asociado por hora para los que hacen el turno de 4 horas es de 18€. El coste 
asociado por hora para los que hacen el turno de 6 horas es de 15€. El coste asociado
por hora para los que hacen el turno de 8 horas es de 14€. Un vigilante puede 
empezar el turno en un edificio y acabarlo en el otro.


\subsection*{Variables}
La cantidad de vigilantes se identificarán según su cantidad de horas: $vxi=8, vyi=6, vzi=4$. Se representan los turnos posibles en la tabla:

\begin{center}
    \includegraphics[height=3cm]{D:/KUKADisk/OneDrive - KUKA AG/UDIMA/InvOP/AEC/AEC1/img/turnos.PNG}\\
\end{center}

\subsection*{Función objetivo}

Tomando en cuenta el coste total de cada vigilante según sus horas:

\[
    min(Z) = 112(vx1 + vx2)  + 90(vy1 + vy2 + vy3)  + 72(vz1 + vz2 + vz3 + vz4)
\]

\subsection*{Restricciones}

{\setlength{\parindent}{0cm}
    $vx1+vy1+vz1 \geq 13$ \\
    $vx1+vx2+vy1+vy2+vz1+vz2 \geq 23$\\
    $vx1+vx2+vy1+vy2+vy3+vz2+vz3 \geq 19$\\
    $vx2+vx2+vy2+vy3+vz3+vz4 \geq 8$\\
    $vx2+vy3+vz4 \geq 5$\\
    $vxi \geq 0, i=1,2$ y enteros.\\
    $vyi \geq 0, i=1,2,3$ y enteros.\\
    $vzi \geq 0, i=1,...,4$ y enteros.
}

 




\section*{Caso 2. Método gráfico}

F.O.: $Max / Min (120 x1 + 200 x2)$\\



{\setlength{\parindent}{0cm}
Restricciones:\\

$x_1 + x_2 = 65$\\
$x_1 \geq 23$ \\ 
$x_2 \geq 20$ \\
$60 x_1 + 24 x_2 \leq 3000$\\

Resolver por el método gráfico, tanto para el caso de maximizar como de
minimizar la función objetivo. \\
}

Graficamos las funciones en Geogebra:


\begin{center}
    \includegraphics[width=0.9\textwidth]{D:/KUKADisk/OneDrive - KUKA AG/UDIMA/InvOP/AEC/AEC1/img/c2r1.PNG}\\
\end{center}

En este caso, nos encontramos con una región factible acotada por una recta.
El máximo se encuentra en el punto A(23,42) con un valor de $11160$ y el mínimo en el punto B(40,25) con un valor de $9800$.


\newpage

\section*{Caso 3. Método gráfico}

F.O.: $Max / Min (0.2 x_3 + 0.5 x_4)$\\

{\setlength{\parindent}{0cm}
Restricciones:\\

$0.1 x_3 + 0.6 x_4 \leq 2000$\\
$x_3 + x_4 \leq 6000$\\
$x_3 \leq 4000$\\
$x_3 \geq 0$\\
$x_4 \geq 0$\\

Resolver por el método gráfico, tanto para el caso de maximizar como de
minimizar la función objetivo. 
}

De la misma forma graficamos con Geogebra:

\begin{center}
    \includegraphics[width=0.9\textwidth]{D:/KUKADisk/OneDrive - KUKA AG/UDIMA/InvOP/AEC/AEC1/img/c3r1.PNG}\\
\end{center}

Aquí observamos una región factible acotada en la que el máximo se encuentra en el punto B(3200,2800) con valor 1720 y el mínimo en el punto D(4000,0) con valor 800.













\end{document}