\documentclass[a4paper,12pt]{article}

\usepackage[utf8]{inputenc}
\usepackage{amsmath, amssymb, amsthm}
\usepackage{geometry}
\geometry{left=2.5cm, right=2.5cm, top=2.5cm, bottom=2.5cm}
\usepackage{graphicx}

\title{Guía de Probabilidad y Estadística}
\author{Preparado por ChatGPT}
\date{\today}

\begin{document}

\maketitle

\section{Introducción}
Este documento es una guía para comprender los conceptos fundamentales de probabilidad y estadística, aplicada a los ejercicios de la actividad de evaluación continua.

\section{Conceptos Básicos}

\subsection{Probabilidad}
La probabilidad mide qué tan probable es que ocurra un evento. Se expresa como un número entre 0 y 1:
\begin{itemize}
    \item Si un evento es imposible, su probabilidad es 0.
    \item Si un evento es seguro, su probabilidad es 1.
    \item Si hay un 50\% de posibilidad, su probabilidad es 0.5.
\end{itemize}
Ejemplo: Si en una bolsa hay 3 pelotas rojas y 7 azules (10 en total), la probabilidad de sacar una roja es:
\[
P(A) = \frac{\text{Número de pelotas rojas}}{\text{Total de pelotas}} = \frac{3}{10} = 0.3
\]

\subsection{Probabilidad de Unión}
La probabilidad de que ocurra al menos uno de dos eventos $A$ o $B$ se calcula con:
\[
P(A \cup B) = P(A) + P(B) - P(A \cap B)
\]
Donde:
\begin{itemize}
    \item $P(A)$: Probabilidad de que ocurra el evento A.
    \item $P(B)$: Probabilidad de que ocurra el evento B.
    \item $P(A \cap B)$: Probabilidad de que ocurran ambos eventos a la vez.
\end{itemize}

Ejemplo: Si el 50\% de los estudiantes tienen beca y el 30\% estudia, pero el 10\% tiene beca y estudia:
\[
P(B \cup E) = 0.5 + 0.3 - 0.1 = 0.7
\]

\subsection{Probabilidad Condicional}
La probabilidad de que ocurra $A$ dado que ocurrió $B$ se calcula con:
\[
P(A | B) = \frac{P(A \cap B)}{P(B)}
\]
Ejemplo: Si el 30\% de los estudiantes tienen beca y estudian, pero solo el 50\% tienen beca, la probabilidad de que estudie sabiendo que tiene beca es:
\[
P(E | B) = \frac{P(B \cap E)}{P(B)} = \frac{0.1}{0.5} = 0.2
\]

\subsection{Teorema de Bayes}
Se usa para invertir probabilidades condicionales:
\[
P(A | B) = \frac{P(B | A) P(A)}{P(B)}
\]

Ejemplo en una urna:
Si hay 7 bolas blancas y 12 negras, y sacamos una bola y la reemplazamos con dos del otro color, queremos saber la probabilidad de que la primera bola haya sido negra dado que la segunda fue blanca:
\[
P(B_1 | W_2) = \frac{P(W_2 | B_1) P(B_1)}{P(W_2)}
\]

\subsection{Independencia de Eventos}
Dos eventos son independientes si:
\[
P(A \cap B) = P(A) P(B)
\]
Si esto no se cumple, los eventos son dependientes.

\section{Ejercicios Resueltos}

\subsection{Ejercicio 1}
Datos:
\begin{itemize}
    \item $P(B) = 0.5$
    \item $P(E) = 0.3$
    \item $P(B \cap E) = 0.1$
\end{itemize}

Calculamos $P(B \cup E)$:
\[
P(B \cup E) = P(B) + P(E) - P(B \cap E) = 0.5 + 0.3 - 0.1 = 0.7
\]

\subsection{Ejercicio 2}
Se extrae una bola de una urna y se reemplaza con dos del otro color.

\textbf{Paso 1: Posibilidades}
\begin{itemize}
    \item Si la primera bola es blanca, la segunda extracción se hace con 6 blancas y 14 negras.
    \item Si la primera bola es negra, la segunda extracción se hace con 9 blancas y 10 negras.
\end{itemize}

\textbf{Paso 2: Aplicar probabilidad total}
\[
P(W_2) = P(W_2 | W_1) P(W_1) + P(W_2 | B_1) P(B_1)
\]

\section{Conclusión}
Con esta guía, puedes resolver problemas de probabilidad paso a paso, aplicando las fórmulas clave y verificando la coherencia de los resultados.

\end{document}
