\documentclass{article}
\usepackage{lipsum}
\usepackage{authoraftertitle}
\usepackage[top=2cm,bottom=1.5cm,left=1.5cm, right=3cm,includeheadfoot]{geometry}
\usepackage{graphicx}
\usepackage[parfill]{parskip}
\usepackage{fancyhdr}
\usepackage[spanish]{babel}
\usepackage{mathtools}
\usepackage{csquotes}
\usepackage{amssymb}
\usepackage[shortlabels]{enumitem}
\usepackage{fancybox, graphicx}
\usepackage[figuresleft]{rotating}
\usepackage{array}
\usepackage{hhline}
\usepackage{subfigure}
\usepackage{lscape}
\usepackage{gensymb}
\usepackage{hyperref}
\usepackage{tikz}
\usepackage{amsmath}
\usepackage{wrapfig}
\usepackage{float}
\usepackage{amsmath} 
\usepackage{caption}
\usepackage{esvect}
\usepackage{siunitx}
\usepackage{commath}
%Header & Footer

\pagestyle{fancy}
%\fancyhead[LE]{}
\fancyhead[L]{Bases de Datos}
\fancyhead[R]{Diseño Lógico de bases de datos}
%\fancyhead[RO]{\leftmark}
%\fancyhead[RE]{\leftmark}
\fancyfoot[L]{\raisebox{-1cm}{\includegraphics[height=1.5cm]{D:/KUKADisk/UDIMA/DocumentGraphics/LOGOUDIMA.jpg}}}
\fancyfoot[R]{\small Universidad a Distancia de Madrid}
%\fancyfoot[RO]{07/12/2018}        +

p`ñ´´´´´´´´´´´´´´´´´´´´´´´´´

%Vars
\author{Alexander Sebastian Kalis}
\title{Actividad 11. AEC: Caso práctico IV: Consultas SQL}


%DOC


\begin{document}

\begin{titlepage}

    \begin{center}

        \line(1,0){300}\\
        [0.2in]
        \huge{\bfseries {\MyTitle}}\\
        [1mm]
        \line(2,0){200}\\
        [0.75cm]
        \textsc{\LARGE Bases de Datos}\\
        [2cm]
            \includegraphics[width=.9\textwidth]{D:/KUKADisk/UDIMA/BBDD/portada.png}\\    
        [3cm]

    \end{center}

    \begin{flushright}

        Líder: Alexander Sebastian Kalis\\
        José María Quintanilla Alonso\\
        Raúl Alonso Crespo\\
        %Profesora: Dra. Isabel Cristina Gil García\\
        Ingeniería de Organización Industrial\\
        %UDIMA\\
        \today        

    \end{flushright}
    
\end{titlepage}

\tableofcontents % \thispagestyle{empty}
\newpage

\section{Modelo ER base corregido (CP I)}

\begin{figure}[!htb]
    \centering
        \includegraphics[angle=90,scale=.85]{D:/KUKADisk/UDIMA/BBDD/AEC2/ERec.png}\\
\end{figure}

\newpage

\section{Modelo lógico corregido (CP II)}
\begin{figure}[htb!]
  \centering
  {\includegraphics[height=19cm]{D:/KUKADisk/UDIMA/BBDD/AEC2/investigadorrealizaarticulo.png}}
  \caption{Diseño lógico final}
  \label{}
\end{figure}

\newpage

\section{Correcciones de inserts y scripts (CP III)}

\subsection{script.sql}

En la creación de la tabla `journal' se ha cambiado el tipo de atributo de `Ultimo\_Indice\_Impacto' de INT a FLOAT para permitir valores decimales y 
se ha ampliado el campo de caracteres disponibles a 3000 del atributo `Instrucciones\_Envio'.\\

\begin{figure}[htb!]
  \centering
  {\includegraphics[height=6cm]{D:/KUKADisk/UDIMA/BBDD/AEC4/cambiosjournal.png}}
  \caption{Correciones en tabla journal}
  \label{}
\end{figure}

En la creación de la tabla `articulo\_investigacion', se ha ampliado los caracteres disponibles del atributo `Resumen'.\\


\begin{figure}[htb!]
  \centering
  {\includegraphics[height=7.5cm]{D:/KUKADisk/UDIMA/BBDD/AEC4/cambiosarticulo.png}}
  \caption{Correciones en tabla artículo investigación}
  \label{}
\end{figure}

\newpage

\subsection{Inserts.sql}

Debido al cambio de la tabla `journal', se han cambiado también los datos introducidos en la columna `Ultimo\_Indice\_Impacto' a valores decimales.

\begin{figure}[htb!]
  \centering
  {\includegraphics[width=\textwidth]{D:/KUKADisk/UDIMA/BBDD/AEC4/cambiosinsert.png}}
  \caption{Correciones en la inserción de datos de journal}
  \label{}
\end{figure}

También se han realizado pequeñas modificaciones y adiciones en los valores de los atributos para hacer que los resultados
de las consultas muestren almenos un resultado que coincida con todas las condiciones requeridas.

\section{Consultas SQL (CP IV)}

\subsection{Consulta 1}

Listado de editores de la editorial, donde aparecerá el nombre del journal en el que presta sus servicios.

\begin{figure}[htb!]
  \centering
  {\includegraphics[height=7cm]{D:/KUKADisk/UDIMA/BBDD/AEC4/ScConsultas/consulta1.png}}
  \caption{Consulta 1}
  \label{}
\end{figure}

\textbf{Descripción}

En este caso seleccionamos el nombre completo de la tabla editor y el nombre del journal. Realizamos una unión con la tabla journal
bajo la condición de que coincida el ISSN del journal gestionado del editor con el ISSN del journal, para que no aparezcan otras combinaciones editor-journal
distintas.


\subsection{Consulta 2}

Listado de los journals de la editorial, ordenados por índice de impacto (aparecerán primero los de mayor índice de impacto). 

\begin{figure}[htb!]
  \centering
  {\includegraphics[height=7cm]{D:/KUKADisk/UDIMA/BBDD/AEC4/ScConsultas/consulta2.png}}
  \caption{Consulta 2}
  \label{}
\end{figure}

\textbf{Descripción}

En esta consulta seleccionamos el nombre y el índice de impacto de la tabla journal. Finalmente los ordenamos por el atributo índice impacto
de mayor a menor añadiendo el keyword `desc'.

\subsection{Consulta 3}

Listado de los journals de la editorial,
junto con el número (cantidad) de editores que gestionan cada uno. Se ordenarán los journals de menor a mayor número de editores. 

\begin{figure}[htb!]
  \centering
  {\includegraphics[height=7cm]{D:/KUKADisk/UDIMA/BBDD/AEC4/ScConsultas/consulta3.png}}
  \caption{Consulta 3}
  \label{}
\end{figure}

\newpage 

\textbf{Descripción}


Empezamos seleccionando el nombre de la tabla journal y con la función count() el recuento de las filas que contienen valor en el atributo 
editor.ISSN\_Journal\_Gestionado. Este recuento lo guardamos bajo alias `Gestores'.

Para que ese recuento sea posible, unimos la tabla journal con editor con la condición de relación de ISSN mútuo.

Agrupamos los resultados de conteo de gestores por nombre de cada journal y finalmente utilizamos los resultados guardados en el alias `Gestores' 
para ordenar de menor a mayor con el uso del keyword `asc'.

\subsection{Consulta 4}

Listado de los investigadores que hayan realizado algún artículo de más de
35 páginas publicado en algún número del journal “Information Sciences”, y
cuyo editor responsable (del número) pertenezca a la misma institución que el investigador. 

\begin{figure}[htb!]
  \centering
  {\includegraphics[height=7.5cm]{D:/KUKADisk/UDIMA/BBDD/AEC4/ScConsultas/consulta4.png}}
  \caption{Consulta 4}
  \label{}
\end{figure}

\textbf{Descripción}

En esta consulta algo más compleja, seleccionamos (distintos) los nombres completos de los investigadores.

Empezamos realizando las uniones de las tablas que nos van a ser necesarias para aplicar todos los filtros. Se puede observar que se crean dos
bloques. El primer bloque son las uniones que tienen como condición relaciones de mutualidad. El segundo bloque son los filtros específicos de
atributos de esas tablas.

En el bloque de uniones JOIN, aplicamos directamente el filtro que condiciona que solo deben aparecer los investigadores cuyo artículo pertenezca a un
journal que tiene como editor responsable a un miembro de su institución.

En el bloque de condiciones WHERE, realizamos una resta de pagina final e inicial para obtener el número de páginas del artículo y condicionamos que 
solo aparezcan los investigadores cuyo artículo supera las 35 páginas.

Por último condicionamos que sólo aparezcan los nombres de investigadores cuyo artículo sea publicado en el journal con ISSN `5088-1245', que
corresponde al journal “Information Sciences”.

\subsection{Consulta 5}

Listado de artículos open access con 4 o más autores, 
contenidos en números de journals con periodicidad trimestral gestionados por menos de 10 editores. 

\begin{figure}[htb!]
  \centering
  {\includegraphics[height=12cm]{D:/KUKADisk/UDIMA/BBDD/AEC4/ScConsultas/consulta5.png}}
  \caption{Consulta 5}
  \label{}
\end{figure}

\textbf{Descripción}

Similarmente a la consulta 4, empezamos por unir tablas con relación de mutualidad. 

Posteriormente aplicamos filtros condicionales donde especificamos que el artículo debe ser open access y que debe estar en un journal con periodicidad
trimestral.

Para mostrar solamente aquellos artículos que tengan 4 o más editores, se ha creado una subconsulta la cual nos devuelve una lista de IDs de artículos;

Esta sublista está condicionada con el uso del keyword `having', haciendo que sólo se muestre en la sublista aquellas IDs (agrupadas por ID) de artículos que se repiten 4 o más
veces (utilizando la función count()) en la tabla `realizar' ya que eso significa que tienen 4 o más autores. Por último, mediante el keyword `IN', obligamos a que
la consulta principal nos muestre solamente los artículos cuyo ID aparezca en esta sublista.

De la misma forma se ha realizado la condición de que solamente se muestren los artículos cuyo journal tiene menos de 10 editores.

\newpage

\subsection{Consulta 6}

Sueldo medio de los editores, jefes de algún journal con algún número especial publicado en 2018. 

\begin{figure}[htb!]
  \centering
  {\includegraphics[height=12cm]{D:/KUKADisk/UDIMA/BBDD/AEC4/ScConsultas/consulta6.png}}
  \caption{Consulta 6}
  \label{}
\end{figure}

\textbf{Descripción}

En esta consulta seleccionamos la media de los sueldos de la tabla editor mediante el uso de la función avg() y lo guardamos bajo alias `Sueldo medio'

Para los filtros, hacemos una subconsulta dentro de otra subconsulta:

En la primera subconsulta, especificamos que el pasaporte del editor (que debe ser jefe de un journal) estará contenido en (IN) la lista
de pasaportes que son jefes de la tabla journal.

Sin embargo, con el uso de una subconsulta dentro de esa subconsulta, hacemos que sólo aparezcan los pasaportes de journals los cuales contienen algun número que
sea de tipo especial y además su fecha de publicación esté comprendida entre 2018-01-01 y 2019-01-01 para que sea del 2018.

\newpage

\subsection{Consulta 7}

Número medio de artículos por número (se pide un único valor).

\begin{figure}[htb!]
  \centering
  {\includegraphics[height=8cm]{D:/KUKADisk/UDIMA/BBDD/AEC4/ScConsultas/consulta7.png}}
  \caption{Consulta 7}
  \label{}
\end{figure}

\textbf{Descripción}

En esta consulta se utiliza el concepto de tabla derivada, es decir, una tabla volátil que se inicializa en el runtime de la consulta.

Lo que hacemos es seleccionar la media con avg() de los números que contiene cada artículo y se muestra bajo alias `Media articulos por numero'.

Para obtener los números por artículo creamos una tabla derivada en la cual seleccionamos el recuento (con count()) de los identificadores del número
que contiene cada artículo y lo guardamos bajo alias `articulosPorNumero'. Agrupamos por ese identificador para que cada número tengo su cantidad de artículos.

Finalmente el lenguaje nos obliga a guardar esta tabla derivada bajo un nombre. En este caso se ha guardado bajo el nombre `tablaArticulosPorNumero'.


\end{document}