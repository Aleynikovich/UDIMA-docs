\documentclass{article}
\usepackage{lipsum}
\usepackage[backend=biber]{biblatex}
\usepackage{authoraftertitle}
\usepackage[top=2cm,bottom=1.5cm,left=1.5cm, right=3cm,includeheadfoot]{geometry}
\usepackage{graphicx}
\usepackage{fancyhdr}
\usepackage[spanish]{babel}
\usepackage{mathtools}
\usepackage{csquotes}
\usepackage{amssymb}
\usepackage{fancybox, graphicx}
\usepackage{array}
\usepackage{hhline}
\usepackage{hyperref}
\usepackage{tikz}
\usepackage{amsmath}
\usepackage{wrapfig}
\usepackage{float}
\usepackage{siunitx}
\usepackage{amsmath}
\usepackage{caption}
\usepackage{esvect}
\usepackage{siunitx}
\usepackage{commath}
\newcommand{\ihat}{\textbf{\^\i}}
\newcommand{\jhat}{\textbf{\^\j}}
%Header & Footer

\pagestyle{fancy}
%\fancyhead[LE]{\MyTitle}
\fancyhead[LO]{Química}
%\fancyhead[RO]{\leftmark}
%\fancyhead[RE]{\leftmark}
\fancyfoot[L]{\raisebox{-1cm}{\includegraphics[height=1.5cm]{D:/KUKADisk/UDIMA/DocumentGraphics/LOGOUDIMA.jpg}}}
\fancyfoot[R]{Corregido:\\ Dr. Lucas Castro Martínez}
%\fancyfoot[RO]{07/12/2018}


%Vars
\author{Alexander Sebastian Kalis}
\title{AEC 3 - Unidades 5 y 6}


%DOC


\begin{document}

\begin{titlepage}

    \begin{center}

        \line(1,0){300}\\
        [0.2in]
        \huge{\bfseries {\MyTitle}}\\
        [1mm]
        \line(2,0){200}\\
        [0.75cm]
        \textsc{\LARGE Química}\\
        [2cm]
        \includegraphics[height=10cm]{D:/KUKADisk/UDIMA/Quimica/Practicas/portada.jpg}\\
        [3cm]

    \end{center}

    \begin{flushright}

        Autor: {\MyAuthor}\\
        Profesor: Dr. Lucas Castro Martínez\\
        Curso: Ingeniería de Organización Industrial\\
        UDIMA         

    \end{flushright}
    
\end{titlepage}


\section*{Problema 1}

En un dispositivo con un émbolo se introducen $0.854 \ mol$ de gas neón a una presión 
constante de $105 \ Pa$, se calienta aportando $53 \ J$ en forma de calor, lo que produce un aumento de 
temperatura de $3,00 \si[]{\degree} \ C$. Calcule:\\

a) El incremento de volumen que se produce en el gas.

b) El trabajo realizado por el gas, ¿es de expansión o de contracción? ¿positivo o negativo?

c) La variación de entalpía y de energía interna del gas. ¿son positivas o negativas?

\subsection*{Apartado a}

Calculamos el incremento de volumen mediante la ecuación general de los gases ideales:

\[
    \Delta V=\cfrac[]{nR\Delta T}{p}
\]

Ya que $\Delta K = \Delta \si[]{\degree} \ C$, sustituyendo los datos del enunciado obtenemos:

\[
    \Delta V=\cfrac[]{0.854 \cdot 8.31 \cdot 3}{10^5}=\num{2.13e-4} \ m^3
\]

\subsection*{Apartado b}

Podemos calcular el trabajo realizado con la relación presión-volumen. Sabiendo que la presión es constante:

\[
    W =  P\Delta V
\]
\[
    W= 10^5 \cdot \num{2.13e-4} = 21.3 J
\]

Podemos decir que será un trabajo de expansión pues el volumen incrementa y es positivo.


\subsection*{Apartado c}

Calculamos la variación interna:

\[
    \Delta E = q - p \Delta V=
    53-10^5 \cdot \num{2.13e-4}=
    31.71 \ J
\]  

Y la entalpía:

\[
    \Delta H = q_p=53 \ J
\]

Entonces la entalpía y el incremento de energía es positivo pues el gas se calienta y aporta calor.


\newpage

\section*{Problema 2}

Calcúlese la entalpía estándar de formación del acetileno (etino), $C_2H_2$ (g), a partir de las
entalpías estándar de combustión del C (grafito), el $H_2$ (g) y el $C_2H_2$ (g). Para ello aplique la ley de Hess,
la suma algebraica de las correspondientes ecuaciones termoquímicas. \\

%\includegraphics[height=2cm]{D:/KUKADisk/UDIMA/Quimica/Practicas/AEC3/p2e1.PNG}\\

\[
    \begin{matrix}
        \Delta H^0_c \left[C(grafito)\right]        =-286 kJ \cdot mol^{-1}\\
        \Delta H^0_c \left[H_2(g)\right]            = -393 \cdot mol^{-1}\\
        \Delta H^0_c \left[C_2H_2(g)\right]         =-1300 \cdot mol^{-1}
    \end{matrix}
\]

(CUIDADO los datos de entalpías de combustión son por mol de compuesto que reacciona).\\

Recuerda que una combustión es la reacción con el oxígeno para dar CO2, H2O o ambos).\\\\

Tenemos 3 ecuaciones cuyos incrementos de entalpía estandar son conocidos:

\[
    \begin{matrix}
      A &  H_2 (g) + \frac{1}{2} O_2 (g) \rightarrow H_2O (l) & \Delta H^0_c=-286 kJ \cdot mol^{-1}\\
      B &  C(s)+O_2(g) \rightarrow CO_2(g) & \Delta H^0_c= -393 kJ \cdot mol^{-1}\\
      C &  C_2H_2(g) + \frac{5}{2}O_2 \rightarrow 2CO_2(g)+H_2O(l) & \Delta H^0_c=-1300 kJ\cdot mol^{-1}
    \end{matrix}
\]

Debemos calcular, utilizando la Ley de Hess, el incremento de entalía de:

\[
    \begin{matrix}
        D & 2C(s)+H_2(g) \rightarrow C_2H_2(g) & \Delta H=?
      \end{matrix}
\]

Primero sumamos la ecuación A con 2 veces la ecuación B:

\[
    H_2 (g) + 2C(s)+2O_2(g) + \frac{1}{2} O_2 (g) \rightarrow H_2O (l) + 2CO_2(g)
\]

Que simplificado queda:

\[
    2C(s)  +H_2 (g) + \frac{5}{2}O_2(g) \rightarrow 2CO_2(g) + H_2O(l)
\]

A esto le restamos la ecuación C obtenemos:

\[
    2C(s)  +H_2 (g) - C_2H_2(g) \rightarrow 0
\]

Y una vez pasado el término negativo sumando, es equivalente a la ecuación D:

\[
    2C(s)  +H_2 (g) \rightarrow C_2H_2(g)
\]

Ahora aplicando la Ley de Hess, podremos obtener la entalpía estandar. Solamente hay que realizar las mismas operaciones
a las entalpías que a las ecuaciones:

\[
    \Delta H^0_f= -286+2(-393)-(-1300)=228 kJ\cdot mol^{-1}
\]

\newpage



\section*{Problema 3}

En junio del año 1812 Napoleón marchó de Francia con sus tropas. En diciembre, cuando
se retiraron de Moscú, había perdido más de medio millón de soldados. Muchas razones existen para
esta derrota, pero quizás la más interesante tiene que ver con sus botones de estaño. Dice la leyenda
que, en el invierno de Rusia, sus botones se descompusieron, provocando la exposición de sus soldados
al intenso frío. Es lo que se conoce como “peste del estaño”, que consiste en la transformación del
estaño blanco (forma metálica del estaño) se transforma en estaño gris (forma no metálica, con aspecto
de polvo). Determina si está historia pudo ser cierta.

(AYUDA: La forma para determinarlo es probar si existe algún rango de temperaturas en que se
produzca la reacción espontáneamente.)

Datos.


    \begin{center}
        \begin{tabular}{ |c|c|c| }
            \hline
            \textbf{Compuesto}       &  \textbf{$\Delta$ H (kJ/mol)}  &   \textbf{S$\si[]{\degree}$ (J/mol.K)}\\
            \hline
            Sn (blanco)     & 0                         &   51,55\\
            \hline
            Sn (gris)       &  - 2,09                   &   44,14\\
            \hline
        \end{tabular}
    \end{center}


En este caso podemos utilizar la teoría sobre la energía libre de Gibbs ya que nos indicará si
la reacción es espontánea o no:


\[
    \Delta G= \Delta H-T\cdot \Delta S
\]

\[
    \Delta G= \sum G(productos)-\sum G(reactivos)
\]

Entonces necesitamos calcular las variaciones de entropía y entalpía de la reacción
y aplicar la fórmula.

\[
    \Delta S=44.14-51.55=-7.41 J \cdot mol^{-1}\cdot K^{-1}
\]

\[
    \Delta H=-2.09kJ \cdot mol^{-1}=-2090 J \cdot mol^{-1}
\]

\[
    \Delta G= -2090 - T(-7.41)
\]

Según la teoría de la entalpía libre de Gibbs, para que la reacción sea espontánea se debe cuplir que
$\Delta G < 0$:

\[
    0 > -2090 - T(-7.41)
\]

\[
    2090 > -7.41T
\]

\[
    T < 9 \si[]{\degree} \ C
\]

Lo que indica que la condición para que la reacción sea espontánea es de que la temperatura en
la que se produce sea inferior a 9 grados Celsius. En este caso la leyenda puede ser cierta
ya que las temperaturas durante el invierno en Russia llegan a los -40 grados Celsius.


\newpage



\section*{Problema 4}

El fosgeno $(COCl_2)$, utilizado en las cámaras de gas por los alemanes durante la
segunda guerra mundial, se produce a partir del monóxido de carbono y del cloro gas.\\

a) Determina la ley de velocidad para la reacción a partir de los datos de la tabla.\\

\begin{center}
    \includegraphics[height=3.3cm]{D:/KUKADisk/UDIMA/Quimica/Practicas/AEC3/p4e1.PNG}\\ 
\end{center}

b) Calcúlese la constante de velocidad, k.\\


\subsection*{Apartado a}

Escribimos la ecuación cinética de la reacción:

\[
    v=k\left[CO\right]^m \cdot \left[Cl_2\right]^n
\]

Si tratamos los experimentos como sistemas de ecuaciones, podemos dividir y despejar las órdenes de
reacción $m$ y $n$. Hecho esto obtenemos los valores:

\[
    m\approx 1, n\approx \frac{3}{2}
\]

Así que la ley de la velocidad es:

\[
    v\approx k\left[CO\right]^1 \cdot \left[Cl_2\right]^{\frac{3}{2}}
\]

\subsection*{Apartado b}

Despejando la constante obtenemos su valor con cualquiera de los experimentos. El resultado no es exacto
puesto que se han aproximado las órdenes de reacción.
\[
    k\approx
    \frac{v}{[CO][Cl_2]^{\frac{3}{2}}}\approx
    \frac{0.121}{0.12\cdot 0.2^{\frac{3}{2}}}\approx
    11.27 \ M^{-1}\cdot s^{-1}
\]  

\newpage

\section*{Problema 5}

El ozono ($O_3$) reacciona con el dióxido de nitrógeno para dar pentóxido de dinitrógeno y
oxígeno, (todos en fase gas), y presenta una ley de velocidad experimental $v = k[NO_2 ][O_3]$ . Durante la
reacción se ha podido detectar la presencia de trióxido de nitrógeno como especie intermedia. Escribe
la reacción global y propón un mecanismo factible para la reacción.\\

Se nos presenta con la siguiente reacción:

\[
    O_3+NO_2 \rightarrow N_2O_5 + O_2
\]

Y su ley de velocidad experimental:

\[
    v = k[NO_2 ][O_3]
\]

El hecho de que coincidan los coeficientes estequiométricos de la reacción con los órdenes de reacción parciales de los reactivos, indica que
dicha reacción transcurre a lo largo de varias etapas.

Según los datos del enunciado postulamos el siguiente mecanismo para la reacción:\\


\textbf{Etapa 1. Controlante}
\[
    O_3+NO_2 \rightarrow O_2 + NO_3
\]

Obteniendo así el trióxido de nitrógeno como intermedio de reacción.\\

\textbf{Etapa 2. Rápida.}

\[
    NO_3+NO_2 \rightarrow N_2O_5
\]


\textbf{Reacción global.}\\


\[
    2NO_2+O_3 \rightarrow N_2O_5+O_2
\]

Sumando ambas ecuaciones obtenemos la reacción global y eliminamos el intermedio de reacción $NO_3$.

\end{document}