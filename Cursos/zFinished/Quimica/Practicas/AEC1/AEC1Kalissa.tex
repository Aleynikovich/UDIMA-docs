\documentclass{article}
\usepackage{lipsum}
\usepackage{authoraftertitle}
\usepackage[top=2cm,bottom=1.5cm,left=1.5cm, right=3cm,includeheadfoot]{geometry}
\usepackage{graphicx}
\usepackage{fancyhdr}
\usepackage[spanish]{babel}
\usepackage{mathtools}
\usepackage{csquotes}
\usepackage{amssymb}
\usepackage{fancybox, graphicx}
\usepackage{array}
\usepackage{parskip}
\usepackage{hhline}
\usepackage{hyperref}
\usepackage{tikz}
\usepackage{amsmath}
\usepackage{wrapfig}
\usepackage{float}
\usepackage{siunitx}
\usepackage{amsmath}
\usepackage{caption}
\usepackage{esvect}
\usepackage{stackrel}
\usepackage{siunitx}
\usepackage{commath}
%Header & Footer

\pagestyle{fancy}
%\fancyhead[LE]{\MyTitle}
\fancyhead[LO]{Química}
\fancyhead[RO]{\nouppercase\leftmark}
%\fancyhead[RE]{\leftmark}
\fancyfoot[L]{\raisebox{-1cm}{\includegraphics[height=1.5cm]{D:/KUKADisk/UDIMA/DocumentGraphics/LOGOUDIMA.jpg}}}
\fancyfoot[R]{Corregido:\\ Dr. Lucas Castro Martínez}


%Vars
\author{Alexander Sebastian Kalis}
\title{AEC 1 - Prácticas de laboratorio}


%DOC


\begin{document}

\begin{titlepage}

    \begin{center}

        \line(1,0){300}\\
        [0.2in]
        \huge{\bfseries {\MyTitle}}\\
        [1mm]
        \line(2,0){200}\\
        [0.75cm]
        \textsc{\LARGE Química}\\
        [2cm]
        \includegraphics[height=10cm]{D:/KUKADisk/UDIMA/Quimica/Practicas/portada.jpg}\\
        [3cm]

    \end{center}

    \begin{flushright}

        Autor: {\MyAuthor}\\
        Profesor: Dr. Lucas Castro Martínez\\
        Curso: Ingeniería de Organización Industrial\\
        UDIMA\\
        \today

    \end{flushright}
    
\end{titlepage}

\tableofcontents

\newpage

\section{Forma de la molécula}

Los objetivos de esta práctica son:

\begin{itemize}
    \item Explorar las formas moleculares mediante la construcción de moléculas en 3D empleando el
    modelo de repulsión de pares de electrones de la capa valencia (RPECV).
    \item Ver cómo cambia la forma molecular con diferentes números de enlaces y pares de electrones.
    \item Comparar las simulaciones de moléculas ficticias con moléculas reales.
\end{itemize}


\subsection{Pregunta 1}

\textbf{Explique por qué el ángulo de enlace en una molécula de agua es de 104,5º y no 109,5º como se
muestra en la pantalla de modelado.}

Este es un caso muy conocido en la geometría molecular. En la pantalla de modelado se representa la molécula en
la forma tetraédrica. Sin embargo, esta sólo es aplicable en los casos de enlaces ideales.

En realidad lo que sucede, es que, al generar el enlace con los dos átomos de hidrógeno, el oxígeno queda con dos pares
de electrones solitarios en su capa de valencia. Estos ejercen una fuerza de repulsión mayor sobre los pares enlazados, 
reduciendo así el ángulo de 109.5º a 104.5º.

\hspace{1cm}


\begin{figure}[!htb]
    \begin{minipage}{0.48\textwidth}
      \centering
      \includegraphics[height=7.5cm]{D:/KUKADisk/UDIMA/Quimica/Practicas/AEC1/p1p1e1.PNG}
      \caption{Modelo, $H_2O$.}
    \end{minipage}\hfill
    \begin{minipage}{0.48\textwidth}
      \centering
      \includegraphics[height=7.5cm]{D:/KUKADisk/UDIMA/Quimica/Practicas/AEC1/p1p1e2.PNG}
      \caption{Real, $H_2O$.}
    \end{minipage}
\end{figure}

\newpage

\subsection{Pregunta 2}

\textbf{Explique por qué los ángulos de enlace en algunas moléculas reales no coinciden con el ángulo
proyectado por la teoría RPECV – por ejemplo, $H_2O$, $SO_2$, $ClF_3$, $NH_3$, $SF_4$, $BrF_5$. (Fíjate en la ventana
moléculas reales la diferencia entre real y modelo).}

En este caso observemos el ejemplo de la molécula de $NH_3$:

\begin{figure}[!htb]
    \begin{center}
        \includegraphics[height=6cm]{D:/KUKADisk/UDIMA/Quimica/Practicas/AEC1/p1p2e3.PNG}
        \caption{Estructura química del Amoníaco, Wikipedia}
    \end{center}
\end{figure}


Similarmente como sucede en la pregunta 1, todas las moléculas de este apartado tienen la característica de
tener pares de electrones no enlazados (pares solitarios) que ejercen fuerza de repulsión mayor
sobre los pares enlazados a los no enlazados y dismunyen los ángulos entre estos enlaces, generando
la desviación observada.


\begin{figure}[!htb]
    \begin{minipage}{0.48\textwidth}
      \centering
      \includegraphics[height=7.5cm]{D:/KUKADisk/UDIMA/Quimica/Practicas/AEC1/p1p2e1.PNG}
      \caption{Modelo, $NH_3$.}
    \end{minipage}\hfill
    \begin{minipage}{0.48\textwidth}
      \centering
      \includegraphics[height=7.5cm]{D:/KUKADisk/UDIMA/Quimica/Practicas/AEC1/p1p2e2.PNG}
      \caption{Real, $NH_3$.}
    \end{minipage}
\end{figure}

\newpage

\subsection{Pregunta 3}

\textbf{Construya una molécula que tenga una geometría electrónica octaédrica y una geometría
molecular cuadrado planar.}

\hspace{1cm}

\begin{figure}[!htb]
    \begin{center}
        \includegraphics[height=10cm]{D:/KUKADisk/UDIMA/Quimica/Practicas/AEC1/p1p3e1.PNG}
        \caption{Molécula }
    \end{center}
\end{figure}

En este caso hemos construído la molécula con cuatro enlaces y dos pares de electrones solitarios. Ejemplos de
moléculas con esta geometría pueden ser el Tetrafloruro de Xenon $XeF_4$ y el Tetracloruro de Platino $[PtCl_4]^{2-}$.


\subsection{Pregunta 4}

\textbf{Describa la diferencia entre la geometría molecular y la geometría electrónica.}

La geometría molecular es la disposición tridimensional de los átomos que constituyen una molécula mientras que la
geometría electrónica describe como se distribuyen los electrones en el espacio.


\newpage

\section{Solubilidad}


Los objetivos de esta práctica son:

\begin{itemize}
    \item Relacionar volumen y cantidad de soluto en una concentración.
    \item Calcular la concentración de las soluciones en unidades de molaridad (mol/L).
    \item Utilizar la molaridad para calcular la dilución de las soluciones.
    \item Comparar los límites de solubilidad entre solutos.
\end{itemize}

\subsection{Pregunta 1}

\textbf{Elige la disolución que quieras, escoge una cantidad de soluto (entre 0,1 y 0,9), un volumen de
disolución (entre 0,1 y 0,9) siempre que no esté saturada ¿es correcto el valor que proporciona el simulador? ¿Cómo hace el cálculo para obtener el
valor de concentración?}

\begin{figure}[!htb]
    \begin{center}
        \includegraphics[width=\textwidth]{D:/KUKADisk/UDIMA/Quimica/Practicas/AEC1/p2p1e1.PNG}
        \caption{$KMnO_4$}
    \end{center}
\end{figure}

Es correcto aunque redondea al 2o decimal. Haciendo el cálculo manualmente obtenemos el siguiente resultado utilizando la fórmula de la molaridad:

\[
    M=\frac{n}{v} \rightarrow
    M=\frac{0.183}{0.701}=0.2610556348\dots \ M
\]


\subsection{Pregunta 2}
\textbf{Realiza los cálculos para obtener los gramos de sulfato de cobre que hay disueltos en 0,1 L de
disolución saturada.}

Al ser una soluciona saturada, en este caso esta va a contener $0.1 \ mol$ o más de soluto.

Sabiendo la masa molar:

\[
    \begin{matrix}
        Cu && 63.5 \ g/mol\\
        S && 32 \ g/mol \\
        O && 16 \ g/mol
    \end{matrix}
\]

Calculamos el peso molecular del $CuSO_4$. En total $159.5 \ g/mol$ y por lo tanto en nuestra disolución debe haber
$15.95g$ de $CuSO_4$ como mínimo para que sea una disolución saturada.

\subsection{Pregunta 3}


\textbf{Calcula la solubilidad de las siguientes disoluciones y ponlas en la tabla.}

\hspace{1cm}

\begin{center}
    \includegraphics[width=.7\textwidth]{D:/KUKADisk/UDIMA/Quimica/Practicas/AEC1/p2p3e1.PNG}
\end{center}

\textbf{¿Hay algunas que no puedas calcular? Ponlas con una x en tabla anterior.}


Debido a que exceden molaridad 5, que es la máxima que permite el simulador, hay dos compuestos de los cuales no se ha podido calcular la solubilidad utilizando
el simulador.

\textbf{¿Podrías indicar si es mayor o menor de algún valor?}

La solubilidad molar será mayor a 5 mol/L.

\subsection{Pregunta 4}

\textbf{Indica la relación entre las solubilidades y el producto de solubilidad del AuCl3.}

\[
    AuCl_3(s) \xrightleftharpoons[]{} Au^{3+}(ac) + 3Cl^-(ac)
\]

Entonces su Kps nos queda de la siguiente forma:

\[
    Kps=[AU^{3+}][Cl^-]^3=s \cdot (3s)^3=27s^4
\]

La Kps (constante del producto de solubilidad) nos indica entonces la relación entre la solubilidad y el producto de solubilidad.

\subsection{Pregunta 5}

\textbf{Calcula el producto de solubilidad del AuCl3.}

\[
    Kps=27(2.25^4)=692
\]


\section{Ácido-base}

El objetivo de esta práctica es entender las distintas variables que influyen en las disoluciones ácido base
sobre el valor del pH


\subsection{Pregunta 1}

\textbf{En la ventana Introducción calcula la constante del ácido débil o de la base débil (con
uno de los dos es suficiente), necesitaras los valores que se obtienen con la Vistas de la Gráfica.}

\begin{figure}[!htb]
    \begin{center}
        \includegraphics[width=.5\textwidth]{D:/KUKADisk/UDIMA/Quimica/Practicas/AEC1/p3p1e1.PNG}
        \caption{Base Débil}
    \end{center}
\end{figure}



% \[
%     [B]=9.97 \cdot 10^{-3} \ mol/L
% \]
% \[
%     [H_2O]=55.6 \ mol/L
% \]
% \[
%     [BH^+]=3.16 \cdot 10^{-5} \ mol/L
% \]
% \[
%     [OH^-]=3.16 \cdot 10^{-5} \ mol/L
% \]

Con los datos proporcionados por el simulador podemos calcular la constante de equilibrio:

\[
    K_b=\cfrac[]{[BH^+][OH^-]}{[B]}=\cfrac[]{x^2}{c_b}
\]

Sabiendo que $x=3.16 \cdot 10^{-5}$ y $c_b-x = [B]$:

\[
    x=9,99 \cdot 10^{-10} mol/L
\]

\[
    c_b=9.97 \cdot 10^{-3}+3.16 \cdot 10^{-5}=0.01M
\]

\[
    K_b=9,99 \cdot 10^{-8} \ M
\]

Comparamos el pH del simulador con la formula del pH:

\[
    pH=14-\frac{1}{2}pK_b+\frac{1}{2}log c_b=14-3.5-1=9.5
\]


\subsection{Pregunta 2}

\textbf{En la ventana Introducción calcula la concentración inicial del ácido fuerte o de la base
fuerte (con uno de los dos es suficiente), ten en cuenta que necesitaras los valores que se
obtienen con la Vistas de la Gráfica.}

\begin{figure}[!htb]
    \begin{center}
        \includegraphics[width=.5\textwidth]{D:/KUKADisk/UDIMA/Quimica/Practicas/AEC1/p3p2e1.PNG}
        \caption{Base Fuerte}
    \end{center}
\end{figure}


Como indica el simulador, al tratarse de una disolucion diluida, se desprecia [MOH] y por lo tanto
la concentración base es 

\[
    [OH^-]=[M^+]=c_b \rightarrow c_b=1.00 \cdot 10^{-2} \ M
\]

Comparamos el pH:

\[
    14+log c_b=14-2=12
\]

\newpage

\subsection{Pregunta 3}


\textbf{En la ventana Mi disolución tienes que elegir un ácido o una base, una concentración
inicial que no tenga un valor redondo, es decir que no sea 0,001 o 0,01 …, que sea débil. Una
vez elegida debes calcular la constante del ácido o de la base que hayas elegido. Debes indicar
los datos que has elegido, para ello pon un pantallazo de los valores, en los que se vea también
la gráfica, y el valor del pH.}

\begin{figure}[!htb]
    \begin{center}
        \includegraphics[width=.9\textwidth]{D:/KUKADisk/UDIMA/Quimica/Practicas/AEC1/p3p3e1.PNG}
        \caption{Valores ácido débil}
    \end{center}
\end{figure}


Podremos calcular la constante del ácido ayudándonos de la tabla ICE:

\[
    \begin{matrix}
       I && 0.044 && - && 0 && 0\\
       C && 0.044 - x && - && x && x\\
       E && 0.044 - 6.62\cdot10^{-5} && - && 6.62\cdot10^{-5} && 6.62\cdot10^{-5}
    \end{matrix}
\]

Calculamos la constante de acidez:

\[
    K_a=\frac{[H_3O^+][A^-]}{[HA]}=\frac{x^2}{c_a-x}=\frac{(6.62\cdot10^{-5})^2}{0.044-6.62\cdot10^{-5}}=9.975 \cdot 10^{-8}
\]

Comparamos el pH:

\[
    pH=\frac{1}{2}pK_a-\frac{1}{2}log c_a \rightarrow
    pH=\frac{1}{2}7-\frac{1}{2}lg(0.044)=3.5+0.68=4.18
\]



\subsection{Pregunta 4}

\textbf{¿Qué iones están presentes en una disolución ácida?}

Siempre debe haber el ión Hidrógeno y el correspondiente del ácido A. Por ejemplo, HF(ac) sería ácido fluorhídrico.


\subsection{Pregunta 5}

\textbf{Un compañero afirma en un foro: "Los ácidos fuertes siempre tienen un pH más bajo que los
ácidos débiles". ¿Estarías de acuerdo o en desacuerdo con esta afirmación? Usa evidencia de la
simulación para apoyar tu razonamiento.}

No estaría de acuerdo pues dependerá de la concentración del ácido como podemos observar:


\begin{figure}[!htb]
    \begin{center}
        \includegraphics[height=7cm]{D:/KUKADisk/UDIMA/Quimica/Practicas/AEC1/p3p5e1.PNG}
        \caption{Valores ácido débil altamente concentrado}
    \end{center}
\end{figure}

\begin{figure}[!htb]
    \begin{center}
        \includegraphics[height=7cm]{D:/KUKADisk/UDIMA/Quimica/Practicas/AEC1/p3p5e2.PNG}
        \caption{Valores ácido fuerte poco concentrado}
    \end{center}
\end{figure}













\end{document}