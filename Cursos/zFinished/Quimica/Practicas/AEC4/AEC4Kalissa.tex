\documentclass{article}
\usepackage{lipsum}
\usepackage{authoraftertitle}
\usepackage[top=2cm,bottom=1.5cm,left=1.5cm, right=3cm,includeheadfoot]{geometry}
\usepackage{graphicx}
\usepackage{fancyhdr}
\usepackage[spanish]{babel}
\usepackage{mathtools}
\usepackage{csquotes}
\usepackage{amssymb}
\usepackage{fancybox, graphicx}
\usepackage{array}
\usepackage{parskip}
\usepackage{hhline}
\usepackage{hyperref}
\usepackage{tikz}
\usepackage{amsmath}
\usepackage{wrapfig}
\usepackage{float}
\usepackage{siunitx}
\usepackage{amsmath}
\usepackage{caption}
\usepackage{esvect}
\usepackage{stackrel}
\usepackage{siunitx}
\usepackage{commath}
%Header & Footer

\pagestyle{fancy}
%\fancyhead[LE]{\MyTitle}
\fancyhead[LO]{Química}
\fancyhead[RO]{\nouppercase\leftmark}
%\fancyhead[RE]{\leftmark}
\fancyfoot[L]{\raisebox{-1cm}{\includegraphics[height=1.5cm]{D:/KUKADisk/UDIMA/DocumentGraphics/LOGOUDIMA.jpg}}}
\fancyfoot[R]{Corregido:\\ Dr. Lucas Castro Martínez}


%Vars
\author{Alexander Sebastian Kalis}
\title{AEC 4 - Unidades 7, 8 y 9}


%DOC


\begin{document}

\begin{titlepage}

    \begin{center}

        \line(1,0){300}\\
        [0.2in]
        \huge{\bfseries {\MyTitle}}\\
        [1mm]
        \line(2,0){200}\\
        [0.75cm]
        \textsc{\LARGE Química}\\
        [2cm]
        \includegraphics[height=10cm]{D:/KUKADisk/UDIMA/Quimica/Practicas/portada.jpg}\\
        [3cm]

    \end{center}

    \begin{flushright}

        Autor: {\MyAuthor}\\
        Profesor: Dr. Lucas Castro Martínez\\
        Curso: Ingeniería de Organización Industrial\\
        UDIMA\\
        \today

    \end{flushright}
    
\end{titlepage}

\thispagestyle{plain}
\tableofcontents

\newpage


\section{Problema 1}

\textbf{Disponemos de una disolución que se ha obtenido añadiendo 1,17 gramos de
cloruro de sodio a un litro de agua. Además, se le adiciona a esa disolución 42,48 gramos de
bis(trioxidonitrato) de plomo.}

\textbf{Determine si se producirá la precipitación del dicloruro de plomo cuya constante de solubilidad
es $1.7\cdot 10^{-5}$.}

Según los datos proporcionados tenemos:

    1L de $H_2O$.

    $1.17g$ de $NaCl \rightarrow \cfrac{1.17}{23+35.5}=0.02$ moles de $NaCl$.

    $42.48g$ de $Pb(NO_3)_2 \rightarrow \cfrac{42.48}{331}=0.13$ moles de $Pb(NO_3)_2$.







\newpage

\section{Problema 2}

Si medimos el pH de una disolución tampón formada con hidrogenotrioxidocarbonato de sodio (bicarbonato) y trioxidocarbonato
de sodio (carbonato) obtenemos un valor de 9,4.

Datos $K_a(HCO^-_3)=4.710^{-11}$

a) Calcúlese la relación de iones bicarbonato y carbonato, es decir el siguiente cociente $\cfrac{HCO^-_3}{CO^{2-}_3}$.

b) Calcúlese los moles de hidrogenocarbonato de sodio hay que añadir a una disolución 0,225 M de carbonato de sodio
para obtener el pH de 9,4.

\subsection{Apartado A}


\subsection{Apartado B}

\newpage

\section{Problema 3}

\textbf{Se desea construir una pila. Para ello, se conectan dos electrodos, uno de ellos
consiste en una placa de platino sumergida en una disolución de tetraoxidosulfato de cobre
(sulfato de cobre), y el otro está constituido por una placa de hierro que está sumergida en una
disolución de tricloruro de hierro.}

\textbf{Consultar los potenciales necesarios en la tabla de potenciales estándar de reducción.}

\begin{enumerate}
    \item \textbf{Razonar qué electrodo actúa como cátodo y cuál como ánodo. Dibujar el esquema de la
    pila y escribir las reacciones parciales y el proceso global que tiene lugar.}
    \item \textbf{Represente mediante notación la pila formada.}
    \item \textbf{Calcule el potencial estándar de esta pila.}
    \item \textbf{Calcule la variación de energía de Gibbs estándar.}
    \item \textbf{Hallar el potencial de la pila si $[Cu^{2+}]$ es 0,1 M y $[Fe^{3+}]$ es 0,2 M.}
\end{enumerate}

\textbf{Mirar los potenciales que hagan falta en la tabla de potenciales del manual de la asignatura}

\subsection{Apartado 1}

\subsection{Apartado 2}

\subsection{Apartado 3}

\subsection{Apartado 4}

\subsection{Apartado 5}

\end{document}