\documentclass[twoside]{article}
\usepackage{lipsum}
\usepackage{authoraftertitle}
\usepackage[margin=2cm,left=2.5cm,bottom=1.5cm,top=2cm,includeheadfoot]{geometry}
\usepackage{graphicx}
\usepackage{fancyhdr}
\usepackage[spanish]{babel}
\usepackage{mathtools}


%Header & Footer

\pagestyle{fancy}
\fancyhead[LE]{\MyTitle}
\fancyhead[LO]{Tecnologías de la Información y Comunicación}
\fancyhead[RO]{\leftmark}
\fancyhead[RE]{\leftmark}
\fancyfoot[L]{\raisebox{-1cm}{\includegraphics[height=2cm]{C:/Users/alexk/Dropbox/DocumentGraphics/LOGOUDIMA.jpg}}}
\fancyfoot[RE]{Corregido:\\ Dr. Isaac Seaone Pujol}
\fancyfoot[RO]{07/12/2018}


%Vars
\author{Alexander Sebastian Kalis}
\title{Actividad de Evaluación Continua 3, Octave:Opción A}


%DOC


\begin{document}

\begin{titlepage}

    \begin{center}

        \line(1,0){300}\\
        [0.2in]
        \huge{\bfseries {\MyTitle}}\\
        [1mm]
        \line(2,0){200}\\
        [0.75cm]
        \textsc{\LARGE Tecnologías de la Información y Comunicación}\\
        [3cm]
        \includegraphics[height=10cm]{C:/Users/alexk/Dropbox/DocumentGraphics/octave.png}\\
        [2cm]

    \end{center}

    \begin{flushright}

        Autor: {\MyAuthor}\\
        Profesores: Dr. Lucas Castro Martínez, Dr. Isaac Seoane Pujol\\
        Curso: 1o, Ingeniería de Organización Industrial\\
        UDIMA         

    \end{flushright}
    
\end{titlepage}

\newpage

\section{Introducción}

En este documento se pretende mostrar un pequeño resumen del desarrollo de la actividad en Octave. El alumno está familiarizado con la estructuración
y programación en varios lenguajes funcionales y orientados a objetos, sin embargo nunca ha realizado programación en MatLab/Octave.

\section{Desarrollo}
Se ha escogido la opción A a desarrollar:Crear un algoritmo que permita resolver un sistema de ecuaciones como el que se muestra acontinuación y dibujar la progresión de cada una de las 3 variables de salida X1, X2 y X3 para todos las posibles valores del parámetro V. Los valores de los parámetros di son los 8 dígitos de vuestro DNI y el parámetro d es el valor medio de las 8 cifras de dicho DNI. 

Los parámetros V y R varían de la siguiente manera: V tomará 51 valores equiespaciados entre -10 y 10. R es 1 valor aleatorio entre 0 y 10.\\

\begin{center}
    
    $d_1X+d_2Y+d_3Z=d$

    $d_4X+d_5Y+d_6Z=V$
    
    $d_7X+d_8Y+dZ=R$

\end{center}

Como se ha observado en la sesión de videoconferencia de Octave, se propondrá la resolución del problema utilizando la función de la inversa de una matriz.

En la primera parte se plantean las variables y constantes que utilizará el programa:\\

\includegraphics[width=\textwidth]{C:/Users/alexk/Dropbox/DocumentGraphics/AEC3TIC/variables.png}

Seguidamente, mediante un bucle ``for", se crearán varios vectores que corresponderán a los resultados para cada valor de V, es decir, 51 vectores distintos:\\

\includegraphics[width=\textwidth]{C:/Users/alexk/Dropbox/DocumentGraphics/AEC3TIC/for.png}

Se utiliza la función plot() para obtener la gráfica de los resultados obtenidos. Adicionalmente se pone título y leyenda a los valores representados:\\

\includegraphics{C:/Users/alexk/Dropbox/DocumentGraphics/AEC3TIC/plot.png}

Por último se guardan los resultados en un fichero .csv el cual nos permitirá importar estos valores posteriormente, si fueran necesarios:

\includegraphics[width=\textwidth]{C:/Users/alexk/Dropbox/DocumentGraphics/AEC3TIC/csv.png}

\section{Resultados gráficos}

A continuación se muestra un par ejemplos de resultados obtenidos con nuestro script de Octave:

\includegraphics[height=5cm]{C:/Users/alexk/Dropbox/DocumentGraphics/AEC3TIC/graph1.jpg}
\includegraphics[height=5cm]{C:/Users/alexk/Dropbox/DocumentGraphics/AEC3TIC/graph2.jpg}

\section{Conclusión}

Viniendo de programación funcional y secuancial principalmente, me ha costado un poco ``cambiar el chip'' a la forma de pensar
en matrices para utilizar el lenguaje de MatLab/GNU Octave. Realmente es un lenguaje muy potente y espero con ansias poder utilizarlo
en las asignaturas relacionadas con matemáticas, ya que también es divertido ver las representaciones gráficas de lo que uno computa.

\end{document}