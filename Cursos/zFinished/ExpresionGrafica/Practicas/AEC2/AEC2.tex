\documentclass{article}
\usepackage{lipsum}
\usepackage[backend=biber]{biblatex}
\addbibresource{aec2.bib}
\usepackage{authoraftertitle}
\usepackage[top=2cm,bottom=1.5cm,left=1.5cm, right=3cm,includeheadfoot]{geometry}
\usepackage{graphicx}
\usepackage{fancyhdr}
\usepackage[spanish]{babel}
\usepackage{mathtools}
\usepackage{csquotes}
\usepackage{amssymb}
\usepackage{fancybox, graphicx}
\usepackage{array}
\usepackage{hhline}
\usepackage{hyperref}
\usepackage{tikz}
\usepackage{amsmath}
\usepackage{wrapfig}
\usepackage{float}
\usepackage{amsmath}
\usepackage{caption}
\usepackage{esvect}
\usepackage{siunitx}
\usepackage{commath}
\usepackage{tcolorbox}
\usepackage{listings}
\makeatletter
\renewcommand*\env@matrix[1][*\c@MaxMatrixCols c]{%
   \hskip -\arraycolsep
   \let\@ifnextchar\new@ifnextchar
   \array{#1}}
\makeatother

%Header & Footer
\pagestyle{fancy}
\fancyhead[LE]{\MyTitle}
\fancyhead[LO]{Expresión Gráfica}
\fancyhead[RO]{\leftmark}
\fancyhead[RE]{\leftmark}
\fancyfoot[L]{\raisebox{-1cm}{\includegraphics[height=2cm]{C:/Users/alexk/Dropbox/DocumentGraphics/LOGOUDIMA.jpg}}}
\fancyfoot[R]{Corregido:\\ Dra. Isabel Cristina Gil García}
%\fancyfoot[RO]{07/12/2018}
%Vars
\author{Alexander Sebastian Kalis}
\title{ Actividad 5. AEC. Ejercicios propuestos Unidades 4-5}


%DOC

\begin{document}

\begin{titlepage}

    \begin{center}

        \line(1,0){300}\\
        [0.2in]
        \huge{\bfseries {\MyTitle}}\\
        [1mm]
        \line(2,0){200}\\
        [0.75cm]
        \textsc{\LARGE Expresión Gráfica}\\
        [2cm]
        \includegraphics[height=10cm]{C:/Users/alexk/Dropbox/DocumentGraphics/AEC1ExpresionGrafica/expresiongrafica.jpg}\\
        [2.5cm]

    \end{center}

    \begin{flushright}

        Autor: {\MyAuthor}\\
        Profesora: Dra. Isabel Cristina Gil García\\
        Curso: 1o, Ingeniería de Organización Industrial\\
        UDIMA\\         
        \today
    \end{flushright}
    
\end{titlepage}

\newpage 

\section*{Ejercicio 1. Bloques}

\subsection*{Apartado a}

\subsubsection*{Imagen}

\includegraphics[height=10cm]{C:/Users/alexk/Dropbox/DocumentGraphics/AEC2ExpresionGrafica/Ejercicio1a.png}\\

\subsection*{Descripción}

Para la creación de éstos bloques sencillos se han utilizado las medidas proporcionadas por los enunciados. Las medidas que no estaban
explícitas se han deducido o aproximado para crear una representación parecida a la del enunciado.

La base de la centrifugadora se ha realizado con la función de ayuda perpendicular. El tanque se ha dibujados
mediante la utilización de la función para crear círculos tangentes (se pueden consultar las líneas auxiliares en la capa aux).

Finalmente se ha separado cada elemento en archivos .dwg distintos y se ha comprimido en bloques.rar.

\subsection*{Apartado b}

\subsubsection*{Imagen}

\includegraphics[height=10cm]{C:/Users/alexk/Dropbox/DocumentGraphics/AEC2ExpresionGrafica/Ejercicio1b.png}\\

\subsection*{Descripción}


Se ha abierto el ejercicio 1 de la primera AEC. Seguidamente se ha descargado el cajetín del moodle de la
asignatura. Se ha utilizado el modelo horizontal ya que es el que corresponde con las tuberías. Para insertar el bloque
del cajetín, simplemente se ha arrastrado el archivo del cajetín hacia la pestaña de autoCad del ejercicio. 

Posteriormente se ha arrastrado hacia el ejercicio cada bloque necesario creados en el apartado a. En el caso de las válvulas se ha utilizado
la función de copiar para insertar una válvula en cada punto.

\newpage


\section*{Ejercicio 2. Sólidos}

\subsection*{Imagen}
\includegraphics[height=8cm]{C:/Users/alexk/Dropbox/DocumentGraphics/AEC2ExpresionGrafica/Ejercicio2.png}\\
\subsection*{Descripción}

Para el ejercicio 2, se han seguido al pie de la letra las instrucciones prestadas en vídeo por la profesora de cómo dibujar el elemento en 2D.

Una vez terminado el modelo en 2D, se ha utilizado la herramienta "Extrude" para crear un sólido a partir del dibujo en 2D. Primero se ha levantado 21 unidades la
superfície del contorno dentado. Posteriormente se ha utilizado de nuevo la herramienta "Extrude" para levantar y crear un sólido de los elementos interiores, los cuales
se han levantado 21+30 unidades. 

Para conseguir que quede hueca la parte más interna de la pieza se ha utilizado la herramienta "Subtract". El procedimiento fue seleccionar
el cilindro exterior, enter, seleccionar el elemento interior, enter.

Finalmente se ha representado las varias vistas isométricas utilizando la herramienta "Viewport Configuration" de la pestaña "View". Ésta nos 
permite visualizar varias vistas a la vez, según la configuremos. 

\newpage

\section*{Ejercicio 3}

\subsection*{Imagen}
\includegraphics[height=8cm]{C:/Users/alexk/Dropbox/DocumentGraphics/AEC2ExpresionGrafica/Ejercicio3.png}\\
\subsection*{Descripción}
En este ejercicio la principal dificultad fue deducir el tamaño de los chaflanados ya que no tenemos acotaciones disponibles y el dibujo propocionado por el enunciado tiene una resolución muy baja. De nuevo se ha limitado el espacio de trabajo a formato A4 
vertical utilizando los comandos
\begin{lstlisting}[backgroundcolor = \color{lightgray}]
    LIMITS(0,0)(210,297)
    GRID Limits No
\end{lstlisting}

Al ser dibujos simétricos se ha construído un lado del dibujo y luego se ha creado un espejo con la función \textit{mirror} para ahorrarse algo de tiempo.



Para la representación del sistema diédrico se han seguido los pasos siguientes:

\begin{enumerate}
    \item Colocación de la vista en sistema europeo
    \item Selección de alzado.
    \item El croquis no se ha dibujado ya que se dispone de la pieza en 3D previamente.
    \item No se requieren vistas especiales.
    \item Ajuste de la escala:
    
    Para calcular las escalas:

\[
    \begin{cases}
        120+100Eh+20Eh=210\\
        120+45Ev+22.5Ev=297  
    \end{cases} 
\]

Obtenemos que $Eh=\frac{3}{4}$ y $Ev=2.6$ con lo cual escogemos la óptima $\frac{3}{4}$. Hubiese sido mejor utilizar el
cajetín horizontal.

    \item Insertar el cajetín A4 Vertical.
\end{enumerate}



\newpage
\section*{Ejercicio 4. Vistas diédricas 2}

\subsection*{Imagen}
\includegraphics[height=7cm]{C:/Users/alexk/Dropbox/DocumentGraphics/AEC2ExpresionGrafica/Ejercicio4.png}\\
\subsection*{Descripción}

Para la realización del ejercicio 4 se han seguido los pasos descritos en el manual de la asignatura
\begin{enumerate}
    \item Se escoge el sistema Europeo de represetación.
    \item El alzado fue proporcionado por el enunciado.
    \item Se realiza un croquis de las 3 visas principales:\\
    \includegraphics[height=7cm]{C:/Users/alexk/Dropbox/DocumentGraphics/AEC2ExpresionGrafica/croquis4.png}\\
    \item No se definen vistas auxiliares.
    \item Se calcula la escala óptima teniendo en cuenta las medidas del cajetín.
    \[
    \begin{cases}
        30+40Eh+20Eh=277\\
        30+20Ev+8Ev=135  
    \end{cases} 
    \]

    Con la que obtenemos una escala 2:1.

    \item Se definen las diversas capas (Objeto principal, auxiliares, aristas invisibles, cotas, etc).


\end{enumerate}

Para trazar los dibujos simplemente se ha seguido los datos proporcionados por el enunciado para hacer el contrno de las piezas. Posteriormente se han ido trazando
lineas rectas con medidas aproximadas. Los arcos fueron realizados con la herramienta "Fillet" o "Chaflán" ajustando a radio 0.5.

\end{document}

