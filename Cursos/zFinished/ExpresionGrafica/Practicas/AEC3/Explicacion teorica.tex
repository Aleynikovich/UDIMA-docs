\documentclass{article}
\usepackage{lipsum}
\usepackage[backend=biber]{biblatex}
\addbibresource{aec2.bib}
\usepackage{authoraftertitle}
\usepackage[top=2cm,bottom=1.5cm,left=1.5cm, right=3cm,includeheadfoot]{geometry}
\usepackage{graphicx}
\usepackage{fancyhdr}
\usepackage[spanish]{babel}
\usepackage{mathtools}
\usepackage{csquotes}
\usepackage{amssymb}
\usepackage{fancybox, graphicx}
\usepackage{array}
\usepackage{hhline}
\usepackage{hyperref}
\usepackage{tikz}
\usepackage{subfig}
\usepackage{amsmath}
\usepackage{wrapfig}
\usepackage{float}
\usepackage{amsmath}
\usepackage{caption}
\usepackage{esvect}
\usepackage{siunitx}
\usepackage{commath}
\usepackage{tcolorbox}
\usepackage{bookmark}
\usepackage{listings}

%Header & Footer
\pagestyle{fancy}
%\fancyhead[LE]{\MyTitle}
\fancyhead[LO]{Expresión Gráfica}
\fancyhead[RO]{\leftmark}
%\fancyhead[RE]{\leftmark}
\fancyfoot[L]{\raisebox{-1cm}{\includegraphics[height=2cm]{D:/KUKADisk/UDIMA/DocumentGraphics/LOGOUDIMA.jpg}}}
\fancyfoot[R]{Corregido:\\ Dra. Isabel Cristina Gil García}
%\fancyfoot[RO]{07/12/2018}
%Vars
\author{Alexander Sebastian Kalis}
\title{ Actividad 9. AEC. Ejercicios propuestos Unidades 6-8}


%DOC

\begin{document}

\begin{titlepage}

    \begin{center}

        \line(1,0){300}\\
        [0.2in]
        \huge{\bfseries {\MyTitle}}\\
        [1mm]
        \line(2,0){200}\\
        [0.75cm]
        \textsc{\LARGE Expresión Gráfica}\\
        [2cm]
        \includegraphics[height=10cm]{D:/KUKADisk/UDIMA/DocumentGraphics/AEC1ExpresionGrafica/expresiongrafica.jpg}\\
        [2.5cm]

    \end{center}

    \begin{flushright}

        Autor: {\MyAuthor}\\
        Profesora: Dra. Isabel Cristina Gil García\\
        Curso: 1o, Ingeniería de Organización Industrial\\
        UDIMA\\         
        \today
    \end{flushright}
    
\end{titlepage}

\newpage 

\section*{Ejercicio 1. Acotación}

\subsection*{Imagen}


\begin{center}
    \includegraphics[width=.4\linewidth]{D:/KUKADisk/UDIMA/ExpresionGrafica/Practicas/AEC3/Ejercicio1a.png}
    \includegraphics[width=.4\linewidth]{D:/KUKADisk/UDIMA/ExpresionGrafica/Practicas/AEC3/Ejercicio1b.png}\\
\end{center}



\subsection*{Descripción}

Para la realización del ejercicio 1, debido a la falta de impresora en la fecha de realización se ha acotado los dibujos utilizando la opción de importar PDFs a
Autocad y utilizando las herramientas de acotación del propio programa.

La medición se ha realizado considerando cada cuadrado mide 1 unidad a acotar.










\newpage

\section*{Ejercicio 2. Impresión desde espacio modelo}

\subsection*{Imagen}
\includegraphics[height=10cm]{D:/KUKADisk/UDIMA/ExpresionGrafica/Practicas/AEC3/Ejercicio2.png}\\
\subsection*{Descripción}


En la imagen se puede observar la vista presentación del espacio modelo. 

Se han acotado las vistas el ejercicio 3 de la Actividad 5. Posteriormente, siguiendo las instrucciones de la profesora
en el vídeo `Vídeo 8.1 Impresión desde el espacio Modelo' y `Vídeo 8.2. Ventanas gráficas. Presentaciones'

\begin{itemize}
    \item Se ha configurado el tamaño de papel, origen del trazado y escala a 1:1.
    \item Se crean ventanas gráficas de presentación.
    \item Se establece sobre cada ventana la orientación, escala y visibilidad.
\end{itemize}

Se ha añadido además una ventana de presentación del detalle del radio de la cirumferencia y se ha acotado su radio directamente sobre su vista modelo.
























\newpage

\section*{Ejercicio 3. Sólidos 3D. Presentaciones}

\subsection*{Imagen}
\includegraphics[height=10cm]{D:/KUKADisk/UDIMA/ExpresionGrafica/Practicas/AEC3/Ejercicio3a.png}\\
\subsection*{Descripción}

Para la realización de este ejercicio el alumno también se ha apoyado sobre los vídeos instructivos de la profesora.

Primeramente, utilizando las herramientas 3D, se ha dibujado el cubo.

Posteriormente se ha ido configurando los sistemas de coordenadas personales para que el plano de la rejilla caiga sobre cada cara de cada vista diédrica
del cubo.

Una vez estampadas todas las vistas diédricas sobre el cubo, se han utilizado las herramientas de modelación 3D (principalmente `Extrude' y `Subtract') sobre los rectángulos en las vistas 2D
para hacer así el ahuecado del cubo.

Para los huecos redondos, se ha utilizado la herramienta `Esfera' y posteriormente se ha hecho la diferencia entre el cubo y las esferas.\\


\newpage 

\subsection*{Presentación 1}
Una vez terminado el sólido en 3D, se ha procedido a la presetación 1, donde se ha configurado el tamaño de papel A4 horizontal, y se ha insertado una ventana gráfica escala 2:1 de la perspectiva
isométrica del cubo con un estilo visual realista:\\

\includegraphics[height=8cm]{D:/KUKADisk/UDIMA/ExpresionGrafica/Practicas/AEC3/Ejercicio3b.png}\\

\subsection*{Presentación 2}

La presentación 2 también se ha configurado como A4 horizontal. Se han plasmado las vistas diédricas y la isométrica utilizando la herramienta `base' de Autocad
teniendo todas las aristas visibles. La ventana isométrica se ha modificado para que aparezca el sólido en escala 2:1 y con estilo sombreado con líneas visibles:\\

\includegraphics[height=8cm]{D:/KUKADisk/UDIMA/ExpresionGrafica/Practicas/AEC3/Ejercicio3c.png}\\

\end{document}

