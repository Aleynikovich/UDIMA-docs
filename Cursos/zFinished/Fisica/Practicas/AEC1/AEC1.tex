\documentclass{article}
\usepackage{lipsum}
\usepackage{authoraftertitle}
\usepackage[top=2cm,bottom=1.5cm,left=1.5cm, right=3cm,includeheadfoot]{geometry}
\usepackage{graphicx}
\usepackage[parfill]{parskip}
\usepackage{fancyhdr}
\usepackage[spanish]{babel}
\usepackage[utf8]{inputenc}
\usepackage{mathtools}
\usepackage{csquotes}
\usepackage{amssymb}
\usepackage[shortlabels]{enumitem}
\usepackage{fancybox, graphicx}
\usepackage{array}
\usepackage{hhline}
\usepackage{subfigure}
\usepackage{gensymb}
\usepackage{hyperref}
\usepackage{nicefrac}
\usepackage{tikz}
\usepackage{amsmath}
\DeclareUnicodeCharacter{00A0}{ }
\usepackage{leftidx}
\usepackage{wrapfig}
\usepackage{float}
\usepackage{upgreek}
\usepackage{amsmath} 
\usepackage{caption}
\usepackage{esvect}
\usepackage{siunitx}
\usepackage{commath}
\usepackage{bigints}


%Vars
\author{Alexander Sebastian Kalis}
\title{Actividad de Evaluación Continua 1}
\pagestyle{fancy}
\fancyhead[LO]{1507 Fundamentos Físicos}
\fancyhead[RO]{\author}
%\fancyhead[RE]{\leftmark}
\fancyfoot[L]{\raisebox{-1cm}{\includegraphics[height=1.5cm]{E:/KUKADisk/UDIMA/DocumentGraphics/LOGOUDIMA.jpg}}}
\fancyfoot[R]{Corregido:\\ Dra. Celeste Beatriz Justo María}
%\fancyfoot[RO]{07/12/2018}


\begin{document}

\begin{titlepage}

    \begin{center}

        \line(1,0){300}\\
        [0.2in]
        \huge{\bfseries {\MyTitle}}\\
        [1mm]
        \line(2,0){200}\\
        [0.75cm]
        \textsc{\LARGE Fundamentos Físicos}\\
        [2cm]
        \includegraphics[height=10cm]{E:/KUKADisk/UDIMA/Fisica/Practicas/portada.jpg}\\
        [3cm]

    \end{center}

    \begin{flushright}

        {\MyAuthor}\\
        Profesora: Dra. Celeste Beatriz Justo María\\
        Curso: Ingeniería de Organización Industrial\\
        UDIMA\\
        \today        

    \end{flushright}
    
\end{titlepage}

%\tableofcontents \thispagestyle{empty}
%\newpage

\section*{Problema 1}
Tenemos una esfera conductora rodeada por una corteza esférica también conductora como en la
siguiente figura:\\

\begin{center}
    \includegraphics[height=5cm]{E:/KUKADisk/UDIMA/Fisica/Practicas/AEC1/problema1.png}
\end{center}

En un momento determinado la esfera interior se conecta a una batería y se carga hasta que se llega al
equilibrio electrostático de nuevo y a continuación se desconecta la batería, manteniéndose el equilibrio.
Calcula:\\

a) Densidad de carga superficial en la esfera interna.\\


La esfera que hemos conectado a una batería es conductora. Esto significa que sus electrones de valencia
se encuentran libres y están situados en la superfície de la esfera. También sabemos que la carga en una esfera  
se comporta como una carga puntual en su centro.

Por último, el dato que nos piden es la densidad de carga superficial. Por lo tanto necesitaremos saber la carga y la
superfície de la esfera.\\


Potencial  generado por una carga puntual $Q$ a una distancia $d$:

\[
    V=\frac{Q}{4\pi \epsilon_0 d} \implies Q=V 4\pi \epsilon_0 d
\]

Y sustituyendo los datos proporcionados nos queda que:

\[
    Q=V_0 R_0 4 \pi \epsilon_0 \ \  C
\]

Entonces la densidad de carga superficial de la esfera interna será:

\[
    \sigma = \frac{Q}{S} = \frac{V_0 R_0 4 \pi \epsilon_0}{4 \pi {R_0}^2} = \frac{V_0 \epsilon_0}{R_0} \ \nicefrac{C}{m^2}
\]

\newpage

b) ¿Aparece alguna carga en la corteza esférica? ¿qué valores tiene? ¿cuáles son sus
densidades?\\


Por ley de Gauss, sabemos que el campo interior de un conductor en equilibrio equivale a 0. Por lo tanto, en las superfícies de la 
corteza se concentran las cargas, de signo contrario. Por inducción, la superfície interior de la corteza estará cargada con la carga
$-Q$ que hemos calculado en el apartado anterior y por consecuencia, la superfície exterior estará cargada con la carga $Q$.

La densidad de carga superficial de la superfície interior será:

\[
    \sigma = \frac{Q}{S} \implies \sigma_{s2int}=\frac{-Q}{4 \pi R_i} \ \nicefrac{C}{m^2}
\]

Y la de la superfície exterior será:

\[
    \sigma = \frac{Q}{S} \implies \sigma_{s2ext}=\frac{Q}{4 \pi R_e} \ \nicefrac{C}{m^2}
\]



c) Campo eléctrico en todo punto del espacio, en forma vectorial.\\

Si $r < R_0$ entonces $E=0 \nicefrac \ {N}{C}$ pues estamos dentro de un conductor.

Si $R_0 < r < R_i$ se genera un campo radial. Recordamos que se comporta como una carga puntual en el centro:

\[
    \varPhi_{gauss} = \frac{q_{int}}{\epsilon_0} = \frac{Q}{\epsilon_0} \implies \frac{Q}{\epsilon_0} = E4\pi r^2 \implies E=\frac{Q}{4\pi \epsilon r^2} \ \nicefrac{N}{C}
\]

Si $R_i < r < R_e$ volvemos a estar dentro de un conductor por tanto $E=0 \ \nicefrac{N}{C}$.

Si $R_e < r < \infty$ nos encontramos con el mismo caso que en $R_0 < r < R_i$ y por lo tanto será: 

\[
    E=\frac{Q}{4\pi \epsilon_0 r^2} \ \nicefrac{N}{C}
\]


d) Función potencial eléctrico en todo punto del espacio.\\

La función del potencial eléctrico en cualquier punto, de nuevo, será la misma que la generada por una carga puntual en el centro de la 
esfera cargada. Además sabemos que dentro de los conductores el potencial va a ser constante.

\[    
    V(r)=\frac{q}{4\pi \epsilon_0 r} \ V
\]


\newpage

\section*{Problema 2}

Un condensador cilíndrico está compuesto por por dos cilindros metálicos concéntricos de radios “a” y
“b” $(a < b)$ y una longitud $L$. Este condensador está relleno de aire hasta un radio “c” y desde “c” hasta
el conductor exterior de un dieléctrico de constante $r=4$.

a) Haz un dibujo transversal esquemático del problema.\\

\begin{center}
    \includegraphics[height=7cm]{E:/KUKADisk/UDIMA/Fisica/Practicas/AEC1/problema2.png}\\
\end{center}


b) Cuando el cilindro interno se carga con Q y el exterior con -Q: ¿qué forma van a tener el campo
eléctrico y las superficies equipotenciales en el interior del condensador?\\

El campo generado por las cargas Q y -Q tendrá forma radial y apuntará hacia fuera del condensador.
El campo generado por la carga de polarización tendrá la misma forma radial pero en este caso será hacia dentro del condensador.

c) Calcula el campo y el potencial en el interior del condensador.\\

El campo dentro del cilindro interio será $E=0$ pues es un conductor.

El campo en el primer dieléctrico (aire), $k_1$, será el campo generado por la carga Q y disminuido por la constante dieléctrica del aire.
Escogemos entonces una superfície gaussiana de radio $r$ donde $a<r<c$:

\[
    E_1=\frac{E_0}{k_{a}}=\frac{Q}{2\pi \epsilon_0 k_{1} L r } \ \nicefrac{N}{C}
\]

Similarmente ocurrirá en el dieléctrico exterior. El dieléctrico interior está reduciendo el campo total antes de llegar al
siguiente dielétrico. Con lo cual solamente debemos multiplicar el campo anterior por la constante del material exterior por
su constante dielétrica (4) y escogiendo una superfície gaussiana tal que $c<r<b$:

\[
    E_2=\frac{E_1}{4}=\frac{Q}{8\pi \epsilon_0 k_{1} L r} \ \nicefrac{N}{C}
\]

Aplicando el principio de superposición tendremos que el campo total será:

\[
    E_t=E_0-E_1-E_2 \ \nicefrac{N}{C}
\]

Sabemos que la diferencia de potencial también es inversamente proporcional a la constante dieléctrica, de forma que:

\[
    \Delta V = \frac{\Delta V_0}{4 k_1} \ V
\]


d) Calcula la capacidad del condensador.\\

Al tener dos materiales dieléctricos, podemos tomar el condensador como dos condensadores conectados en serie:

\[
    \frac{1}{C_t}=\frac{1}{C_1}+\frac{1}{C_2} \ F
\]


Calculamos $C_1$ y $C_2$:

\[
    C_1 = \frac{2\pi L \epsilon_0 k_{a}}{\log (\nicefrac{a}{c})} \ F
\]

\[
    C_2 = \frac{8\pi L \epsilon_0}{\log (\nicefrac{c}{b})} \ F
\]

Entonces nos queda que:


\[
    C_t = \cfrac{1}{\cfrac{2\pi L \epsilon_0 k_{a}}{\log (\nicefrac{a}{c})} + \cfrac{8\pi L \epsilon_0}{\log (\nicefrac{c}{b})}} \ F
\]

e) En el dieléctrico calcula: densidades de carga ligada volumétrica y superficial.

Calculamos la carga en el primer dieléctrico (aire) y en el segundo dieléctrico:

\[
    q_a=Q(1-\frac{1}{k_1}) \ C
\]
   
\[
    q_r=Q(1-\frac{1}{4}) \ C
\]

Entonces densidad superficial será:

\[
    \sigma_a=\frac{q_a}{2\pi c L} \ \nicefrac{C}{m^2}
\]

\[
    \sigma_r=\frac{q_r}{2\pi b L} \ \nicefrac{C}{m^2}
\]

La densidad volumétrica es nula.\\

f) Demuestra que el dieléctrico sigue siendo neutro.

Para demostrar que los dieléctricos son neutros, la densidad de carga en las superfícies de los conductores deberá ser igual pero opuesta
a las superfícies de los dieléctricos con lo cual se debe cumplir que $\sigma_Q=-\sigma_a$ y $\sigma_{-Q} = \sigma_r$.


\section*{Problema 3}

Tenemos dos esferas metálicas de radios “a” y “b” respectivamente. Podemos suponer que ambas
esferas están alejadas una de la otra una gran distancia. En un momento determinado conectamos
cada esfera a una fuente de tensión y las ponemos a un potencial $V_a$ y $V_b$ respectivamente.
A continuación, desconectamos las fuentes de tensión.

a) ¿Qué cantidad de carga almacenan las esferas?

Como vimos en el problema 1, la carga de una esfera se comporta como una carga puntual en su centro, entonces:

\[
    Q_a=V_a a k \ C
\]

\[
    Q_b=V_b b k \ C
\]

b) ¿Qué energía almacena cada una de las esferas?

La energía que almacena una esfera se puede expresar como:

\[
    U=\frac{1}{2} \frac{Q^2}{4\pi \epsilon_0 R} \ J
\]

Donde R será el radio de la esfera, en este caso $a$ y $b$ y Q será la carga que almacena la esfera, $Q_a$ y $Q_b$.\\


Pasado un tiempo, mediante un hilo conductor, conectamos eléctricamente ambas esferas y esperamos
una gran cantidad de tiempo a que se alcance un equilibrio:\\

c) ¿Qué cantidad de carga va a tener cada una de las esferas?\\

Como hemos llegado a un punto de equilibrio, se debe cumplir que $V_1 = V_2$, entonces:

\[
    k \frac{Q_a}{a}=k \frac{Q_b}{b} \ [V]
\]

Si despejamos nos queda que:

\[
    Q_a=Q_b \frac{a}{b} \ C
\]

\[
    Q_b=Q_a \frac{b}{a} \ C
\]

d) ¿A qué potencial estará cada una de las esferas?

Ambas esferas estarán al mismo potencial, pues al unirlas creamos un sistema equipotencial.

\[V = k \frac{Q_a}{a} =  k \frac{Q_a}{a}=k \frac{Q_b}{b} \ [V] \]\\

e) Qué energía almacena cada una de las esferas?

\[
    U_a=\frac{1}{2} \frac{(Q_b \frac{a}{b})^2}{4\pi \epsilon_0 a} \ J
\]

\[
    U_b=\frac{1}{2} \frac{(Q_a \frac{b}{a})^2}{4\pi \epsilon_0 b} \ J
\]


\section*{Problema 4}

Tenemos dos conos metálicos alineados con el eje Z. Uno está en el semiespacio $z>0$ y el otro está en
el semiespacio $z<0$ y su vértice está en el origen de coordenadas como en la figura. Los dos vértices están
eléctricamente aislados.\\

También como se ve en la figura, el cono superior está sometido a un potencial $V_0$ y el inferior a  y el inferior a $-V_0$.
En estas condiciones, calcula:\\

a) El campo eléctrico entre ambos conos.


\begin{center}
    \includegraphics[height=4cm]{E:/KUKADisk/UDIMA/Fisica/Practicas/AEC1/problema4.png}\\
\end{center}

Sabemos que son superfícies equipotenciales, diferencia de potencial entre ambos conos siempre es de 2 V.\\

El campo eléctrico entre los conos no es uniforme y varía en función de la distancia recorrida por el eje X. A medida que aumenta X,
el campo disminuye.


\begin{center}
    \includegraphics[height=5cm]{E:/KUKADisk/UDIMA/Fisica/Practicas/AEC1/problema4b.PNG}\\
\end{center}

\[
    \frac{d_x}{2}=x\frac{\cos \alpha}{\sin \alpha} \implies
    d_x=2x \cot \alpha
\]
Partiendo de la definición:\\
\[
    \int \vec{E} \cdot \vec{dl}=V
\]
Podemos reescribirla así:\\
\[
    E \cdot  x \cot \alpha = 1
\]
Por lo tanto el campo será, en función de x:\\
\[
    E(x)=\frac{ \tan \alpha}{x} \ \nicefrac{V}{m}
\]


b) La densidad superficial de carga en las superficies de los dos conos. \\


Ya que el campo depende de x, también lo hará la densidad de carga pues la geometría es cónica. Sabemos que
el material es conductor entonces toda su carga se almacena en la superfície:

\[
    E=\frac{\sigma}{\epsilon_0} \implies
    \sigma =  \frac{ \epsilon_0 \tan \alpha}{x} \ \nicefrac{C}{m^2}
\]


c)La densidad de carga a cada lado de la plancha metálica puesta a tierra que pudiésemos colocar
en z=0.

\begin{center}
    \includegraphics[height=5cm]{E:/KUKADisk/UDIMA/Fisica/Practicas/AEC1/problema4c.PNG}\\
\end{center}

Podemos deducir que la densidad superficial de la plancha metálica tampoco se distribuye de forma constante.
Los conos proyectan superfícies circulares equipotenciales de radio $x$. Como un lado es positivo y el otro negativo, esto hace 
que la carga total neta sea 0, pues se anulan entre ellos. 

Dicho en "matemáticas", en la parte superior de la plancha:

\[
    \sigma = - \frac{ \epsilon_0 \tan \alpha}{x} \ \nicefrac{C}{m^2}
\]

En la parte inferior de la plancha:

\[
    \sigma = + \frac{ \epsilon_0 \tan \alpha}{x} \ \nicefrac{C}{m^2}
\]

Carga neta total:

\[
    \sum Q_t = 0 \ C
\]


\end{document}