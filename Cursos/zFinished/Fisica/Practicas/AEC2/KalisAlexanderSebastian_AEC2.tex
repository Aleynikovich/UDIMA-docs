\documentclass{article}
\usepackage{lipsum}
\usepackage{authoraftertitle}
\usepackage[top=2cm,bottom=1.5cm,left=1.5cm, right=3cm,includeheadfoot]{geometry}
\usepackage{graphicx}
\usepackage[parfill]{parskip}
\usepackage{fancyhdr}
\usepackage[spanish]{babel}
\usepackage[utf8]{inputenc}
\usepackage{mathtools}
\usepackage{csquotes}
\usepackage{amssymb}
\usepackage[shortlabels]{enumitem}
\usepackage{fancybox, graphicx}
\usepackage{array}
\usepackage{hhline}
\usepackage{subfigure}
\usepackage{gensymb}
\usepackage{hyperref}
\usepackage{nicefrac}
\usepackage{tikz}
\usepackage{amsmath}
\DeclareUnicodeCharacter{00A0}{ }
\usepackage{leftidx}
\usepackage{wrapfig}
\usepackage{float}
\usepackage{upgreek}
\usepackage{amsmath} 
\usepackage{caption}
\usepackage{esvect}
\usepackage{siunitx}
\usepackage{commath}
\usepackage{bigints}


%Vars
\author{Alexander Sebastian Kalis}
\title{Actividad de Evaluación Continua 2}
\pagestyle{fancy}
\fancyhead[LO]{1507 Fundamentos Físicos}
\fancyhead[RO]{\author}
%\fancyhead[RE]{\leftmark}
\fancyfoot[L]{\raisebox{-1cm}{\includegraphics[height=1.5cm]{E:/KUKADisk/UDIMA/DocumentGraphics/LOGOUDIMA.jpg}}}
\fancyfoot[R]{Corregido:\\ Dra. Celeste Beatriz Justo María}
%\fancyfoot[RO]{07/12/2018}


\begin{document}

\begin{titlepage}

    \begin{center}

        \line(1,0){300}\\
        [0.2in]
        \huge{\bfseries {\MyTitle}}\\
        [1mm]
        \line(2,0){200}\\
        [0.75cm]
        \textsc{\LARGE Fundamentos Físicos}\\
        [2cm]
        \includegraphics[height=10cm]{E:/KUKADisk/UDIMA/Fisica/Practicas/AEC2/img/portada.jpg}\\
        [3cm]

    \end{center}

    \begin{flushright}

        {\MyAuthor}\\
        Profesora: Dra. Celeste Beatriz Justo María\\
        Curso: Ingeniería de Organización Industrial\\
        UDIMA\\
        \today        

    \end{flushright}
    
\end{titlepage}

%\tableofcontents \thispagestyle{empty}
%\newpage

\section*{Problema 1}

En el circuito de la figura, todos los elementos
son conocidos salvo la resistencia R.

\begin{center}
    \includegraphics[height=5cm]{E:/KUKADisk/UDIMA/Fisica/Practicas/AEC2/img/p1.PNG}\\
\end{center}


Se pide:

a) Valor de R que hace que la potencia
consumida por la resistencia sea la máxima
posible.

La forma más sencilla de resolver este problema será mediante el uso del teorema de Thevenin. Para ello cortamos la rama $R$ del sistema y
evaluamos el resto del circuito. La intención será conseguir un circuito equivalente con una sola fuente de tensión y una resistencia.

\begin{center}
    \includegraphics[height=5cm]{E:/KUKADisk/UDIMA/Fisica/Practicas/AEC2/img/p1v1.PNG}\\
\end{center}

Para resolver el circuito se utilizará análisis de nodos. Escribimos todas las ecuaciones del sistema:

\[
    CTL1: \ V_x=440-V_3
\]

\[
   KCL2: \  \frac{440-V_2}{7}+\frac{V_3-V_2}{1}=\frac{V_x}{2}
\]

\[
    KCL3: \ \frac{440-V_3}{2}+\frac{220-V_3}{3}=\frac{V_3-V_2}{1}
\]

Tenemos tres incógnitas y tres ecuaciones. Resolviendo el sistema obtenemos que $V_2=255.2V$. Sabiendo esto podemos calcular:


\[
    V_{th}=V_{oc}=440-255.2=184.8V
\]

Ahora para encontrar $R_{th}$ necesitaremos conocer la corriente $I_{sc}$. Aplicando KCL en el circuito cortocircuitado obtenemos:

\[
    CTL1: \ V_x=440-V_1
\]

\[
    \frac{440-V_1}{2}+\frac{220-V_1}{3}=\frac{V_1-440}{1}
\]

Obtenemos entonces que $V_1$ (el nodo central en este caso) son $400V$. 

\[
    I_{sc}=\frac{440-400}{1}+\frac{V_x}{2}=40+20=60 \ A
\]



De esta forma podemos computar $R_{th}$:

\[
    R_{th}=\frac{V_{oc}}{I_{sc}}=\frac{184.8}{60}=3.08 \ \ohm
\]

Ahora tenemos montado nuestro circuito equivalente con $V_{th}=184.8 \ V$ y resistencia $R_{tv}=3.08 \ \ohm$ con lo cual podemos
calcular el valor de $R$ tal que absorba la máxima potencia.

El teorema de la máxima potencia nos dice que para maximizar la potencia absorbida simplemente colocamos una resistencia
con la misma resistividad que $R_{tv}$. Con lo cual el valor de $R=3.08 \ \ohm$.


b) ¿Cuál es esa potencia?

De nuevo utilizaremos el teorema de máxima potencia:

\[
    P_l=I^2 \cdot R = 30^2 \cdot 3.08 = 2772 \ W
\]

\newpage

\section*{Problema 2}

En el circuito de la figura, la tensión $Vc$ del condensador vale $-4V$ en $t=0$.

\begin{center}
    \includegraphics[height=4.5cm]{E:/KUKADisk/UDIMA/Fisica/Practicas/AEC2/img/p2e.PNG}\\
\end{center}

Se pide:

a) Si $R = 2 \ \ohm$, calcular el tiempo que tardará la tensión $Vc$ en el condensador en alcanzar $+4V$.

Utilizando sustitución de fuentes de alimentación podemos simplificar bastante el circuito para que quede con este aspecto:

\begin{center}
    \includegraphics[height=4.5cm]{E:/KUKADisk/UDIMA/Fisica/Practicas/AEC2/img/p2v1.PNG}\\
\end{center}

Para resolver el problema lo más sencillo será volver a aplicar Thevenin. En este caso sólo existen fuentes de alimentación 
independientes con lo cual encontrar $R_{th}$ resultará muy sencillo. Cortocircuitamos las fuentes y sumamos las resistencias 
según convenga:

\[
    R_{th}=\frac{2 \cdot 8}{2+8}+2=3.6 \ \ohm
\]

Para encontrar $V_{th}$ simplemente aplicamos KCL en los dos nodos:

\[
    KCL2: \ \frac{80-V_2}{8}=\frac{V_2+5}{2}+\frac{V_2-V_1}{2}
\]

y

\[
    KCL1: \ \frac{V_2-V_1}{2}=0
\]

Calculando el valor de $V_2 = 12 \ V$ con lo cual $V_{th}=V_{oc}=12 \ V$.

Ahora sabiendo el comportamiento de un circuito de tipo RC simple como el que tenemos, podemos calcular fácilmente el tiempo en
el que el condensador tardará en alcanzar $4V$.

Sabemos que en el instante $t=0$, el estado inicial del condensador es $-4V$. Solamente debemos aplicar la fórmula y resolver para $t$:

\[
    V(t)=V_0\left(1-\frac{1}{e^{\nicefrac{t}{RC}}}\right)
\]

Sustituyendo con nuestros datos:

\[
    -4=4\left(1-\frac{1}{e^{\frac{t}{3.6\cdot \:0.005}}}\right)
\]

Resolviendo para el tiempo $t$ obtenemos que $t=0.01247 \ s$.

b) ¿Qué valor debería haber tenido R para que ese tiempo hubiera sido la mitad?

Sabmos que el tiempo que tarda en cargarse un condensador depende proporcionalmente de la constante de tiempo $\tau=RC$.

Debemos recordar que en el apartado anterior, $R=1.6+2$. Entonces la resistencia que debemos modificar es la de valor $2 \ohm$. Tenemos
que encontrar un valor que haga que $\tau$ sea la mitad para conseguir que se cargue en la mitad de tiempo.

\[
    \tau=RC=(1.6+2) \cdot 0.005 = 0.018 \ s
\]

Entonces:

\[
    \tau=RC=(1.6+R) \cdot 0.005 = 0.09 \implies
    (1.6+0.2) \cdot 0.005 = 0.09 \ s
\]

Con lo cual el valor de $R$ que hace que el tiempo sea la mitad es $R=0.2 \ \ohm$

\newpage


\section*{Problema 3}

El siguiente circuito representa un conjunto de cargas conectadas a una red de 220V eficaces a 50Hz:
Los datos que se conocen para cada una de las cargas son los siguientes:

\begin{itemize}
    \item $Z1 = 30+40j$
    \item $Z2$ : consume 2KW con f.p. = 0.8 inductivo
    \item $R$: consume 1 KW
    \item $L$: consume 0,5 KVAR
\end{itemize}


\begin{center}
    \includegraphics[height=4.5cm]{E:/KUKADisk/UDIMA/Fisica/Practicas/AEC2/img/p3e.PNG}\\
\end{center}

Se pide:

a) Potencias activa, reactiva y aparente consumidas por cada una de las cargas.

Para $Z1$:

Su impedancia:

\[
    Z1=\sqrt{X^2+Y^2}=\sqrt{30^2+40^2}=50 \ \ohm
\]

La corriente que pasa por Z1:
\[
    I_{Z1}=\frac{V}{Z}=\frac{220}{50}=4.4 \ A
\]

Las potencias:

\[
    P_{Z1}=I^2\cdot R= 4.4^2 \cdot 30 = 580.8 \ W
\]

\[
    Q_{Z1}=I^2 \cdot X = 4.4^2 \cdot 40 = 774.4 \ VAR
\]

\[
    S_{Z1}=I^2 \cdot Z = 4.4^2 \cdot 50 = 968 \ VA
\]


Para $Z2$:

Sabemos que consume 2000 W por lo tanto esa es su potencia activa. $P_{z2}=2000 \ W$.

\[
    S_{Z2}=\frac{2000}{0.8}=2500 \ VA
\]

\[
    Q_{Z2}=\sqrt{S^2-P^2}=1500 \ VAR
\]

Para R:

Sabemos que las resistencias disipan la potencia absorbida en forma de calor y no almacenan energía en forma
de campo magnético, por lo tanto:

\[
    P_R=1000 \ W \implies 
    S_R=1000 \ VA \implies
    Q_R = 0 \ VAR
\]

Para L:

En este caso sabemos que la bobina (perfecta) almacena la energía formando un campo magnético, pero no disipa ninguna 
en forma de calor. Esto significa que:

\[
    Q_L= 500 \ VAR \implies
    S_L= 500 \ VA \implies
    P_L= 0 \ W  
\]


b) Factor de potencia del conjunto de cargas.

Para ello debemos saber las potencias totales del sistema. Sumando los valores obtenidos en el apartado anterior obtenemos:

\[
    P_{sys}=3581 \ W
\]

\[
    Q_{sys}=2774 \ VAR    
\]

\[
    S_{sys}=\sqrt{P_{sys}^2+Q_{sys}^2}=4530 \ VA
\]

Ahora podemos computar el factor de potencia:

\[
    f.p. = \frac{P}{S}=0.79 \ (inductivo)
\]


c) Intensidad i solicitada a la red (valor eficaz).

\[
    I=\frac{S}{V}=\frac{4530}{220}=20.6 \ A
\]

d) Valor del condensador a colocar entre los terminales A y B para reducir esa intensidad un 10 \%.

Calculamos el valor de las potencias que cumplan:

\[
    \frac{S}{220}=\frac{20.6}{1.1} \implies S=4077 \ VA
\]

\[
    Q=\sqrt{S^2-P^2}=1950 \ VAR
\]

Aporte del condensador al sistema:

\[
    1950-2774 = -824 \ VAR  \implies
    \frac{V^2}{\omega \cdot C} \implies
\]

\[
    C=\frac{-824}{-220^2 \cdot 100 \pi} =
    54.1 \ mF  
\]

e) Nuevo factor de potencia para el conjunto de las cargas (incluyendo el condensador).

Repetimos el proceso del apartado b pero con la nueva potencia aparente:

\[
    f.p.=\frac{3581}{4077}=0.88 \ (inductivo)
\]


\end{document}