\documentclass{article}
\usepackage{lipsum}
\usepackage[backend=biber]{biblatex}
\addbibresource{aec2.bib}
\usepackage{authoraftertitle}
\usepackage[top=2cm,bottom=1.5cm,left=1.5cm, right=3cm,includeheadfoot]{geometry}
\usepackage{graphicx}
\usepackage{fancyhdr}
%\usepackage[spanish]{babel}
\usepackage{mathtools}
\usepackage{nicefrac}
\usepackage{csquotes}
\usepackage{amssymb}
\usepackage{fancybox, graphicx}
\usepackage{array}
\usepackage{hhline}
\usepackage{hyperref}
\usepackage{tikz}
\usepackage{amsmath}
\usepackage{wrapfig}
\usepackage{float}
\usepackage{amsmath}
\usepackage{esint}
\usepackage{caption}
\usepackage{esvect}
\usepackage{siunitx}
\usepackage{commath}
\newcommand{\ihat}{\textbf{\^\i}}
\newcommand{\jhat}{\textbf{\^\j}}
%Header & Footer

\pagestyle{fancy}
%\fancyhead[LE]{\MyTitle}
\fancyhead[LO]{Fundamentos de Electricidad y Electrónica}
\fancyhead[RO]{Actividad de Evaluación Contínua 1}
%\fancyhead[RE]{\leftmark}
\fancyfoot[L]{\raisebox{-1cm}{\includegraphics[height=1.5cm]{D:/KUKADisk/OneDrive - KUKA AG/UDIMA/DocumentGraphics/LOGOUDIMA.jpg}}}
\fancyfoot[R]{Corregido:\\ Dra. Teresa Magraner Benedicto}
%\fancyfoot[RO]{07/12/2018}


%Vars
\author{Alexander Sebastian Kalis}
\title{Actividad de Evaluación Contínua 1}


%DOC


\begin{document}

\begin{titlepage}

    \begin{center}

        \line(1,0){300}\\
        [0.2in]
        \huge{\bfseries {\MyTitle}}\\
        [1mm]
        \line(2,0){200}\\
        [0.75cm]
        \textsc{\LARGE Fundamentos de Electricidad y Electrónica}\\
        [2cm]
        \includegraphics[height=10cm]{D:/KUKADisk/OneDrive - KUKA AG/UDIMA/EYE/AEC/AEC1/imgs/portada.jpg}\\
        [3cm]

    \end{center}

    \begin{flushright}

        Autor: {\MyAuthor}\\
        Profesora: Dra. Teresa Magraner Benedicto\\
        Ingeniería de Organización Industrial\\
        UDIMA         

    \end{flushright}
    
\end{titlepage}

\section*{Caso 1}

En el circuito de la figura, calcular el valor de la fuerza magnetomotriz necesaria para 
que circule un flujo de $10^{-4} Wb$ suponiendo que no hay dispersión.

\begin{center}
    \includegraphics[height=8cm]{D:/KUKADisk/OneDrive - KUKA AG/UDIMA/EYE/AEC/AEC1/imgs/p1e1.PNG}\\
\end{center}


Datos\\
$\mu_r = 300$
$\mu_0 = 4\pi 10^{-7} Tm/A$\\

Empezamos calculando la longitud de las lineas medias:\\

$L_n = 2(55-5) + 60 - 5(2) = 1.5m$\\

$L_{e} = 2(1) = 0.02m$\\

$L_a = 60 - 5(2) = 0.5m$\\

Conociendo el valor del flujo y las medidas geométricas podemos determinar la densidad del flujo. Al no haber dispersión el flujo será uniforme
a lo largo de la figura.

\[
    B_{nea} = \cfrac[]{\Phi_{nea}}{A_{nea}}=\frac{10^{-4}}{0.01} = 10^{-2} T
\]

Procedemos a calcular el valor del campo magnético:

\[
    H_e = \cfrac[]{10^{-2}}{4\pi \cdot 10^{-7}}=
    \cfrac[]{10^5}{4\pi} Av/m
\]

\[
    H_{na} = \cfrac[]{10^{-2}}{4\pi \cdot 10^{-7} \cdot 300} = \cfrac[]{10^3}{12\pi} Av/m
\]

Procedemos a hacer la sumatoria por cada trozo de la figura para obtener la f.m.m necesaria:

\[
    f.m.m = NI = \sum H_iL_i 
\]

\[
    H_{na}L_{na} + H_eL_e =
    2 \cfrac[]{10^3}{12\pi}  +  0.02 \cfrac[]{10^5}{4\pi} = 
    212.2 Av
\]
\newpage

\section*{Caso 2}


El anillo magnético de la figura tiene una longitud media de 0,5 metros y una sección 
uniforme de $10 cm^2$, está formado por un material magnético que tiene la siguiente curva de 
magnetización:\\

$B = 15H/(100+H)$ con $B$ en $T$, y $H$ en $Av/m$\\

Si por la bobina arrollada circula una intensidad de $0,1 A$, ¿cuál será su coeficiente de 
autoinducción $L$ sabiendo que tiene 100 espiras?

\begin{center}
    \includegraphics[height=4cm]{D:/KUKADisk/OneDrive - KUKA AG/UDIMA/EYE/AEC/AEC1/imgs/p2e1.PNG}\\
\end{center}

Conociendo la intensidad y el número de espiras calculamos la fuerza magnetomotriz:\\

\[
    f.m.m = NI = 100 \cdot 0.1 = 10 Av  
\]\\

Y por ende la fuerza del campo magnético:\\

\[
    H = \cfrac[]{fmm}{l_c} = \cfrac[]{10}{0.5} = 20 Av/m
\]  \\

Conocemos la curva de magnetización:\\

\[
    B = \cfrac[]{15 \cdot 20}{100+20} = 2.5 T
\]\\

Con estos datos podemos calcular el flujo magnético:\\

\[
    \Phi = BA = 2.5 \cdot 10^{-4} Wb  
\]\\

Finalmente hallamos su coeficiente de autoinducción:\\

\[
    L = \cfrac[]{N\Phi}{I} = \cfrac[]{100 \cdot 2.5 \cdot 10^{-4}}{0.1} = 0.25 H
\]


\newpage

\section*{Caso 3}

El circuito magnético de la figura está formado por un arrollamiento de 500 espiras en 
la rama central del núcleo que induce un campo magnético en el entrehierro de $1.1 Teslas$. Calcular 
el valor de la intensidad que recorre la bobina suponiendo que no hay dispersión de flujo, sabiendo 
que los valores de la curva de magnetización del material de núcleo son los que se recogen en la 
tabla 1 y que el espesor del entrehierro es de $1 mm$. Todas las cotas del dibujo están en $mm$.

\begin{center}
    \includegraphics[height=8cm]{D:/KUKADisk/OneDrive - KUKA AG/UDIMA/EYE/AEC/AEC1/imgs/p3e1.PNG}\\
\end{center}
\begin{center}
    \includegraphics[width=0.9\textwidth]{D:/KUKADisk/OneDrive - KUKA AG/UDIMA/EYE/AEC/AEC1/imgs/p3e2.PNG}\\
\end{center}

Hacemos el esquema del circuito magnético:

\begin{center}
    \includegraphics[height=5cm]{D:/KUKADisk/OneDrive - KUKA AG/UDIMA/EYE/AEC/AEC1/imgs/p3e3.PNG}\\
\end{center}

Para resolver el circuito empezamos interpolando linealmente para obtener $H$ para $B=1.1 T$:

\[
    400 + (1.1-1) \cfrac[]{1000-400}{1.35-1} = 571.43 Av/m
\]

\[
    U_{R4} = (H_4 \cdot l_4) + (H_2 \cdot l_2) = (571.43 \cdot 0.499) + \left(\cfrac[]{1.1}{4\pi \cdot 10^{-7}} \right) = 1169.5 Av
\]

Entonces:

\[
    H_4 = \cfrac[]{1160.5}{0.5} = 2321 Av/m    
\]\\

Podemos calcular $B_4$ para $H_4 = 2321$:\\

\[
    B_4 = 1.45 + \cfrac[]{2321-2000}{3000-2000} \cdot (1.5-1.45) = 1.466 T
\]\\

Con estos datos pasamos a calcular los flujos:\\

\[
    \Phi_1 = B_1A_1 = 1.1 \cdot 10^{-3} Wb
\]
\[
    \Phi_2= B_2A_2 = 1.466 \cdot 10^{-3} Wb
\]
\[
    \Phi_3 = B_3A_3 = 1.1 \cdot 10^{-3}+1.466 \cdot 10 ^{-3} = 2.566 \cdot 10^{-3} Wb
\]\\

Calculo $B_3$ con el dato del flujo:\\

\[
    B_3=\frac{\Phi_1}{A_1}=\cfrac[]{2.566 \cdot 10^{-3}}{1.6 \cdot 10^{-3}}=1.604 T
\]\\


Volvemos a interpolar para obtener $H$:\\

\[
    H_3 = 4000+(1.604 - 1.54) \cfrac[]{6000-4000}{1.61-1.54}=5829 Av/m
\]\\

Finalmente podemos calcular la intensidad tal que:\\

\[
    I = \cfrac[]{H_4l_4+H_3l_3}{N}=\cfrac[]{2321 \cdot 0.5 + 5829 \cdot 0.2}{500} = 4.65 A
\]

\end{document}