\documentclass{article}
\usepackage{lipsum}
\usepackage{authoraftertitle}
\usepackage[top=2cm,bottom=1.5cm,left=1.5cm, right=3cm,includeheadfoot]{geometry}
\usepackage{graphicx}
\usepackage{fancyhdr}
%\usepackage[spanish]{babel}
\usepackage{mathtools}
\usepackage{nicefrac}
\usepackage{csquotes}
\usepackage{amssymb}
\usepackage{fancybox, graphicx}
\usepackage{array}
\usepackage{hhline}
\usepackage{hyperref}
\usepackage{tikz}
\usepackage{amsmath}
\usepackage{wrapfig}
\usepackage{float}
\usepackage{amsmath}
\usepackage{esint}
\usepackage{caption}
\usepackage{esvect}
\usepackage{siunitx}
\usepackage{commath}
\newcommand{\ihat}{\textbf{\^\i}}
\newcommand{\jhat}{\textbf{\^\j}}
%Header & Footer

\pagestyle{fancy}
%\fancyhead[LE]{\MyTitle}
\fancyhead[LO]{Fundamentos de Electricidad y Electrónica}
\fancyhead[RO]{Actividad de Evaluación Contínua 2}
%\fancyhead[RE]{\leftmark}
\fancyfoot[L]{\raisebox{-1cm}{\includegraphics[height=1.5cm]{D:/KUKADisk/OneDrive - KUKA AG/UDIMA/DocumentGraphics/LOGOUDIMA.jpg}}}
\fancyfoot[R]{Corregido:\\ Dra. Teresa Magraner Benedicto}
%\fancyfoot[RO]{07/12/2018}


%Vars
\author{Alexander Sebastian Kalis}
\title{Actividad de Evaluación Contínua 2}


%DOC


\begin{document}

\begin{titlepage}

    \begin{center}

        \line(1,0){300}\\
        [0.2in]
        \huge{\bfseries {\MyTitle}}\\
        [1mm]
        \line(2,0){200}\\
        [0.75cm]
        \textsc{\LARGE Fundamentos de Electricidad y Electrónica}\\
        [2cm]
        \includegraphics[height=10cm]{D:/KUKADisk/OneDrive - KUKA AG/UDIMA/EYE/AEC/AEC1/imgs/portada.jpg}\\
        [3cm]

    \end{center}

    \begin{flushright}

        Autor: {\MyAuthor}\\
        Profesora: Dra. Teresa Magraner Benedicto\\
        Ingeniería de Organización Industrial\\
        UDIMA         

    \end{flushright}
    
\end{titlepage}

\section*{Caso 1}

Para el circuito de la figura se pide:\\

\begin{center}
    \includegraphics*[height=7cm]{D:/KUKADisk/OneDrive - KUKA AG/UDIMA/EYE/AEC/AEC2/imgs/c1e1.png}
\end{center}


a) Determinar la caída de tensión entre los puntos A y B.\\

En este caso podemos aplicar el teorema de Millmann para obtener la caída de tensión de forma simple:

\[
    V_{AB} = \cfrac[]{\cfrac[]{V_1}{R_1}+\cfrac[]{V_3-V_2}{R_2}+\cfrac[]{0}{R_3}}{\cfrac[]{1}{R_1}+\cfrac[]{1}{R_2}+\cfrac[]{1}{R_3}}=
    \cfrac[]{\cfrac[]{10}{2}+\cfrac[]{3}{1}+\cfrac[]{0}{3}}{\cfrac[]{1}{2}+\cfrac[]{1}{1}+\cfrac[]{1}{3}}=
    4.36V
\]\\


b) Comprobar el valor obtenido mediante la simulación del circuito en QUCS.\\

\begin{center}
    \includegraphics*[height=7cm]{D:/KUKADisk/OneDrive - KUKA AG/UDIMA/EYE/AEC/AEC2/imgs/c1r1.png}
\end{center}

\newpage

\section*{Caso 2}

El interruptor de los circuitos de la figura se cierra en $t = 0$, estando el condensador 
descargado $(V_o = 0)$. Se pide para cada circuito:\\

\begin{center}
    \includegraphics*[height=8cm]{D:/KUKADisk/OneDrive - KUKA AG/UDIMA/EYE/AEC/AEC2/imgs/c2e1.png}
\end{center}

a) Determinar la expresión de la tensión en el condensador en función del tiempo $u_c(t)$.\\

Circuito 1\\

Vamos a aplicar Thevenin para simplificar el circuito:

\begin{center}
    \includegraphics*[height=5cm]{D:/KUKADisk/OneDrive - KUKA AG/UDIMA/EYE/AEC/AEC2/imgs/c2c1r2.png}
\end{center}

\[
    R_{eq}=R1+R3=2500 \Omega
\]

\[
    I_{th}=\frac{V}{R_{eq}}=\frac{10}{2500}=4 mA
\]

\[
    V_{th}=R_3 \cdot I_{th}=2000 \cdot 4 \cdot 10^{-3} = 8V
\]

Para calcular la resistencia se cortocircuita la fuente de alimentación y sustituimos el condensador por una. En ese caso nos queda:

\[
    R_{th} = \cfrac[]{500 \cdot 2000}{500+2000} + 400 = 1000 \Omega
\]


\begin{center}
    \includegraphics*[height=5cm]{D:/KUKADisk/OneDrive - KUKA AG/UDIMA/EYE/AEC/AEC2/imgs/c2c1r3.png}
\end{center}

Calculamos la constante de tiempo del circuito:

\[
    \tau = RC = 1000 \cdot 1 \cdot 10 ^ {-7} = 100 \mu s \rightarrow 5\tau = 500 \mu s
\]

Sabiendo que la tensión del condensador en función del tiempo viene dada por:

\[
    U_c(t) = [U_c(0) - U_{cp}(0)]e^{\cfrac[]{-t}{\tau}}+U_{cp}(t)
\]

Simplificamos con los datos calculados:

\[
    U_c(t)=8\left(1-e^{\cfrac[]{-t}{\tau}}\right)
\]





b) Obtener, mediante la simulación en QUCS, la evolución de la tensión en el condensador desde 
el instante $t = 0$ hasta que el circuito alcance el régimen permanente.\\


\begin{center}
    \includegraphics*[height=7cm]{D:/KUKADisk/OneDrive - KUKA AG/UDIMA/EYE/AEC/AEC2/imgs/c2c1r4.png}
\end{center}



\end{document}