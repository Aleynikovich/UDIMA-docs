\documentclass{article}
\usepackage{lipsum}
\usepackage{authoraftertitle}
\usepackage[top=2cm,bottom=1.5cm,left=1.5cm, right=3cm,includeheadfoot]{geometry}
\usepackage{graphicx}
\usepackage{fancyhdr}
%\usepackage[spanish]{babel}
\usepackage{mathtools}
\usepackage{nicefrac}
\usepackage{csquotes}
\usepackage{amssymb}
\usepackage{fancybox, graphicx}
\usepackage{array}
\usepackage{hhline}
\usepackage{hyperref}
\usepackage{tikz}
\usepackage{amsmath}
\usepackage{wrapfig}
\usepackage{float}
\usepackage{amsmath}
\usepackage{esint}
\usepackage{caption}

\usepackage{yhmath}

\usepackage{esvect}
\usepackage{siunitx}
\usepackage{commath}
\usepackage{steinmetz}
\newcommand{\ihat}{\textbf{\^\i}}
\newcommand{\jhat}{\textbf{\^\j}}
%Header & Footer

\pagestyle{fancy}
%\fancyhead[LE]{\MyTitle}
\fancyhead[LO]{Fundamentos de Electricidad y Electrónica}
\fancyhead[RO]{Actividad de Evaluación Contínua 3}
%\fancyhead[RE]{\leftmark}
\fancyfoot[L]{\raisebox{-1cm}{\includegraphics[height=1.5cm]{D:/KUKADisk/OneDrive - KUKA AG/UDIMA/DocumentGraphics/LOGOUDIMA.jpg}}}
\fancyfoot[R]{Corregido:\\ Dra. Teresa Magraner Benedicto}
%\fancyfoot[RO]{07/12/2018}


%Vars
\author{Alexander Sebastian Kalis}
\title{Actividad de Evaluación Contínua 3}


%DOC


\begin{document}

\begin{titlepage}

    \begin{center}

        %\line(1,0){300}\\
        %[0.2in]
        \huge{\bfseries {\MyTitle}}\\
        [1mm]
        %\line(2,0){200}\\
        %[0.75cm]
        \textsc{\LARGE Fundamentos de Electricidad y Electrónica}\\
        [2cm]
        \includegraphics[height=10cm]{D:/KUKADisk/OneDrive - KUKA AG/UDIMA/EYE/AEC/AEC1/imgs/portada.jpg}\\
        [3cm]

    \end{center}

    \begin{flushright}

        Autor: {\MyAuthor}\\
        Profesora: Dra. Teresa Magraner Benedicto\\
        Ingeniería de Organización Industrial\\
        UDIMA         

    \end{flushright}
    
\end{titlepage}

\section*{Caso 1}
En el circuito de la figura el voltímetro marca una tensión de 150 V (valor eficaz), se pide:\\


\begin{center}
    \includegraphics[height=3cm]{D:/KUKADisk/OneDrive - KUKA AG/UDIMA/EYE/AEC/AEC3/imgs/c1e1.PNG}\\
\end{center}

a) Calcular la potencia activa, reactiva y aparente del circuito (1,0 puntos)\\


Calculamos la impedancia total del circuito equivalente:

\[
    Z_T=\cfrac[]{1}{\cfrac[]{1}{20\phase{0^{\circ}}}+\cfrac[]{1}{15\phase{90^{\circ}}}}=12\phase{53^{\circ}} \ \Omega
\]

Con esto podemos averiguar $I_T$:

\[
    I_T= \cfrac[]{V}{Z_t} = \cfrac[]{150}{12\phase{53^{\circ}}} = 12.5 \phase{-53^{\circ}} \  A
\]

A partir de aquí procedemos a calcular las potencias:

\[
    P=VI\cos\phi = 150 \cdot 12.5 \cdot \cos(53) = 1128.4 \  W
\]

\[
    Q=VI\sin\phi = 150 \cdot 12.5 \cdot \sin(53) = 1497.4 \ VAr
\]

\[
    S=VI = 150 \cdot 12.5 = 1875 \ VA
\]



b) Determinar el factor de potencia (0,5 puntos)\\

\[
    f.d.p. = \cfrac[]{P}{S}=\cfrac[]{1128.4}{1875}=0.6018 \rightarrow 60.2\%
\]

c) Calcular la intensidad total circulante IT (0,5 puntos)\\

\[
    I_T= \cfrac[]{V}{Z_t} = \cfrac[]{150}{12\phase{53^{\circ}}} = 12.5 \phase{-53^{\circ}} \  A
\]

d) Si se quiere que la potencia activa del circuito P sea igual a la reactiva Q, determinar el
valor de la resistencia R1 que debe conectarse en paralelo con R (1,0 puntos)\\

En este caso calculamos la intensidad que pasa por la bobina:

\[
    I_L=\cfrac[]{V}{Z_L}=10 \phase{-90}^{\circ} \ A
\]

Para que se cumpla $P=Q$, $\phi = 45^{\circ}$, entonces:


\[
    I_R=\cfrac[]{150}{R_T}=10\phase{0}^{\circ} \ A \rightarrow
    R_T = 0.0\wideparen{6} \ \Omega
\]

Calculando el paralelo de las resistencias, obtenemos que:

\[
    R_1 \approx 0.066889... \ \Omega
\]





\section*{Caso 2}
En el circuito de la figura la lectura del voltímetro V es de 200 V (valor eficaz). Determinar 
la lectura del resto de elementos de medida representados y dibujar el diagrama fasorial de 
intensidades.\\

\begin{center}
    \includegraphics[height=4cm]{D:/KUKADisk/OneDrive - KUKA AG/UDIMA/EYE/AEC/AEC3/imgs/c2e1.PNG}\\
\end{center}

Calculamos la impedancia total:\\

\[
    Z_T= \cfrac[]{1}{\cfrac[]{1}{50}+\cfrac[]{1}{40 \phase{90^{\circ}}}+\cfrac[]{1}{100 \phase{-90^{\circ}}}} = 40 \phase{36.9^{\circ}} \ \Omega
\]\\

Con esto podemos calcular el valor del amperímetro $A_T$:\\

\[
    A_T=\cfrac[]{V}{Z_T}=\cfrac[]{200}{ 40 \phase{36.9^{\circ}}}=  5 \phase{-36.9^{\circ}} \ A = 4 - 3j \ A
\]\\

Podemos también averiguar la potencia activa leída por el watimetro  $W$:\\

\[
    P = W =VI\cos \phi = 200 \cdot 5 \cdot \cos (36.9) = 799.7 \ W
\]\\

Por último calculamos las intensidades en cada rama:\\

\[
    A_R = \cfrac[]{200}{50}=4 + 0j \ A, \  A_L = \cfrac[]{200}{40 \phase{90^{\circ}}}= 0 - 5j \ A, \  A_C= \cfrac[]{200}{100 \phase{-90^{\circ}}}= 0 + 2j \ A
\]\\

\begin{center}
    \includegraphics[height=20cm]{D:/KUKADisk/OneDrive - KUKA AG/UDIMA/EYE/AEC/AEC3/imgs/c2r1.PNG}\\
\end{center}

\newpage


\section*{Caso 3}
El circuito de la figura representa el ensayo de un motor M en el que se han obtenido los 
siguientes datos en los elementos de medida:
\begin{center}
    \includegraphics[height=5cm]{D:/KUKADisk/OneDrive - KUKA AG/UDIMA/EYE/AEC/AEC3/imgs/c3e1.PNG}\\
\end{center}

W = 2208 W

A = 12 A (valor eficaz)

V = 230 V (valor eficaz)

Calcular:

a) El factor de potencia del motor y la potencia reactiva que consume (1,0 puntos)

\[
    f.d.p. = \cfrac[]{2208}{230 \cdot 12} = 0.8
\]

\[
    2760=\sqrt{2208^2+Q^2} \rightarrow Q=1656 \ VAr
\]




b) El valor de la resistencia R y la reactancia XL que representa el circuito equivalente del 
motor si se conectan en serie (1,5 puntos)

\[	
    \phi = \arccos (0.8) = 36.87^\circ
\]

Obtenemos entonces los valores de $R_s$ (eje real) y $X_s$ (eje imaginario):
\[
    Z=\cfrac[]{230}{12}=19.1\widehat{6} \phase{36.87^\circ} \rightarrow 15.3 + 11.5 j \ \Omega
\]


c) El valor de la resistencia R y la reactancia XL que representa el circuito equivalente del 
motor si se conectan en paralelo (1,5 puntos)\\

Transformamos el circuito anterior a paralelo:

\[
    X_p=\cfrac[]{X_s^2+R_s^2}{X_s}=31.85 \phase{90^\circ} \Omega
\]

\[
    R_p=\cfrac[]{X_s^2+R_s^2}{R_s}=23.94 \ \Omega
\]

\end{document}