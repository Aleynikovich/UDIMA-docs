\documentclass[a4paper,12pt]{article}
\usepackage[spanish]{babel}
\usepackage[utf8]{inputenc}
\usepackage{amsmath, amssymb}
\usepackage{graphicx}
\usepackage{geometry}
\usepackage{fancyhdr}
\usepackage{hyperref}
\usepackage{siunitx}

% Configuración de márgenes
\geometry{left=2.5cm, right=2.5cm, top=3cm, bottom=3cm}

% Encabezado y pie de página
\pagestyle{fancy}
\fancyhf{}
\fancyhead[L]{UDIMA}
\fancyhead[R]{Termodinámica y mecánica de fluidos}
\fancyfoot[C]{\thepage}

% Título del documento
\title{\textbf{Actividad de Evaluación Contínua 1}\\[0.5cm]
\Large{Termodinámica y mecánica de fluidos}}
\author{Alumno: Alexander Sebastian Kalis \\ Profesor: Dr.~César Pérez de Villar Palomo}
\date{\today}

\begin{document}

\maketitle
\newpage
\tableofcontents
\newpage

\section{Ejercicio 1}

Disponemos de un mol de un gas ideal que es \textbf{diatómico} y que se encuentra en el estado inicial 
con presión \(p_1 = 5\,\text{bar}\) y temperatura \(T_1 = 1075\,\text{K}\). 
A partir de este estado:

\begin{enumerate}
    \item El gas sufre \textbf{una compresión isoterma} hasta que su presión se multiplica por \(2{,}3\) veces. 
          (Es decir, \(p_2 = 2{,}3\,p_1\).)
    \item A continuación, el gas \textbf{se expande de manera adiabática} hasta alcanzar de nuevo los \(5\,\text{bar}\) de presión. 
          (Segundo estado intermedio: \(p_3 = 5\,\text{bar}\).)
    \item Finalmente, el gas \textbf{vuelve a su estado inicial} mediante un proceso isobaro a \(5\,\text{bar}\).
\end{enumerate}

Se sabe que todas las transformaciones son reversibles y que se trata de un gas diatómico con:
\[
R = 8,31 \,\text{J/(mol\,K)}, 
\quad
C_{v} = \tfrac{5}{2}\,R,
\quad
(\text{y por tanto } C_{p} = C_{v} + R = \tfrac{7}{2}\,R).
\]

Se pide:

\begin{enumerate}
    \item[(a)] Dibujar el ciclo en el \textbf{diagrama \(p\text{-}V\)} y en el \textbf{diagrama \(p\text{-}T\)}.
    \item[(b)] Calcular el \textbf{volumen}, así como la \textbf{temperatura} y la \textbf{presión} después de cada etapa.
    \item[(c)] Calcular para cada transformación la \textbf{variación de energía interna} (\(\Delta U\)), el \textbf{trabajo} (\(W\)) y el \textbf{calor} (\(Q\)).
\end{enumerate}

\textit{(a) Diagramas \(p\text{-}V\) y \(p\text{-}T\)}: 

\begin{figure}[htbp]
    \centering
    \begin{minipage}[b]{0.45\textwidth}
        \centering
        \includegraphics[width=\textwidth]{diagramaPv.png}
        \caption{Diagrama \(p\text{-}V\)}
        \label{fig:pv}
    \end{minipage}
    \hfill
    \begin{minipage}[b]{0.45\textwidth}
        \centering
        \includegraphics[width=\textwidth]{diagramaPt.png}
        \caption{Diagrama \(p\text{-}T\)}
        \label{fig:pt}
    \end{minipage}
\end{figure}

\textit{(b) Cálculo del volumen, temperatura y presión tras cada etapa:}

Para cada estado (2 y 3), se aplican las leyes de los gases ideales y las relaciones específicas (isotermo y adiabático):

\paragraph{Estado 2 (tras la compresión isoterma 1\(\to\)2):}
\[
\begin{aligned}
p_2 &= 2{,}3\,p_1 
     = 2{,}3 \times 5\,\text{bar} 
     = 11{,}5\,\text{bar},\\[6pt]
T_2 &= T_1 
     = 1075\,\text{K}
     \quad\text{(isoterma)},\\[6pt]
V_2 &= \frac{p_1}{p_2}\,V_1 
     = \frac{5}{11{,}5}\,V_1 
     \approx 0{,}43478\,V_1 
     \;\;\Longrightarrow\;\; 
     V_2 \approx 0{,}00778\,\text{m}^3.
\end{aligned}
\]

\paragraph{Estado 3 (tras la expansión adiabática 2\(\to\)3):}
\[
\begin{aligned}
p_3 &= 5\,\text{bar},\\[4pt]
\text{(relación adiabática)}&:\;\; p_2\,V_2^\gamma 
           = p_3\,V_3^\gamma 
\;\Longrightarrow\; 
V_3 
= V_2 \Bigl(\tfrac{p_2}{p_3}\Bigr)^{\!\!1/\gamma}
\approx 0{,}0141\,\text{m}^3,\\[6pt]
T_3 &= T_2\,\Bigl(\tfrac{V_2}{V_3}\Bigr)^{(\gamma-1)}
     \approx 848\,\text{K}.
\end{aligned}
\]

\paragraph{Regreso al estado 1 (3\(\to\)1, isobara a \(5\,\text{bar}\)):}
\[
\begin{aligned}
p_1 &= p_3 = 5\,\text{bar},\\
T_1 &= 1075\,\text{K},\\
V_1 &\approx 0{,}0179\,\text{m}^3 \quad
\text{(el mismo estado inicial, ya conocido)}.
\end{aligned}
\]

\noindent
Por tanto, \textbf{después de cada etapa} se tienen (en unidades SI y bar):

\[
\begin{aligned}
&\text{Estado 2: }
  p_2 = 11{,}5\,\text{bar},\ 
  T_2 = 1075\,\text{K},\ 
  V_2 \approx 0{,}00778\,\text{m}^3,\\
&\text{Estado 3: }
  p_3 = 5\,\text{bar},\ 
  T_3 \approx 848\,\text{K},\ 
  V_3 \approx 0{,}0141\,\text{m}^3.
\end{aligned}
\]

\noindent
El \textbf{estado final} (1) coincide con el inicial, cumpliendo el ciclo completo.


\medskip
\noindent
\textit{(c) Cálculo de \(\Delta U\), \(W\) y \(Q\)}:  
\[
\Delta U_{1\to 2} 
  = n\,C_{v}\,(T_2 - T_1) 
  = 0 \quad (\text{pues }T_2=T_1).
\]
El trabajo (convención: \(W>0\) lo realiza el sistema al expandirse) en un proceso isotermo reversible es
\[
W_{1\to 2} 
  = n\,R\,T_1 \,\ln\!\Bigl(\frac{V_2}{V_1}\Bigr).
\]
Dado que \(V_2<V_1\), este trabajo será negativo (compresión). Numéricamente,
\[
W_{1\to 2} 
 = (1)(8,31\,\text{J/(mol\,K)})(1075\,\text{K})\,
   \ln\!\bigl(0,00778/0,0179\bigr)
 \;\approx\; 8933\,\ln(0,43478)
\]
\[
    \;\approx\; 8933 \times (-0,834)
    \;\approx\; -7444\,\text{J}.
\]
Por la Primera Ley, \(\,Q_{1\to 2} = \Delta U_{1\to 2} + W_{1\to 2}\). Como \(\Delta U_{1\to 2}=0\),
\[
Q_{1\to 2} \;=\; W_{1\to 2} \;\approx\; -7444\,\text{J}.
\]
El sistema, por tanto, \textit{cede calor} al entorno durante la compresión isoterma (calor negativo).

\bigskip

\textbf{Transformación \(2\to 3\): Adiabática reversible}

\[
p_3 = 5\,\text{bar}, 
\quad Q_{2\to 3} = 0 \quad (\text{adiabático}).
\]
Las relaciones adiabáticas para un gas ideal dan:
\[
p_2\,V_2^\gamma 
 = 
 p_3\,V_3^\gamma,
\quad\text{y}\quad
 T_2\,V_2^{\gamma-1}
 = 
 T_3\,V_3^{\gamma-1}.
\]
De la primera, se obtiene
\[
V_{3} 
 = 
 V_{2}\,\Bigl(\frac{p_{2}}{p_{3}}\Bigr)^{\!\!\frac{1}{\gamma}}
 = 
 V_2\,(2{,}3)^{\tfrac{1}{1{,}4}}
 \;\approx\; 1,81\,V_2
 \;\approx\; 0,0141\,\text{m}^3.
\]
La temperatura \(T_3\) se puede hallar de la segunda relación (o empleando \(T\,p^{\tfrac{1-\gamma}{\gamma}}=\text{cte}\)):
\[
T_3
 =
 T_2\,\Bigl(\tfrac{V_2}{V_3}\Bigr)^{\gamma-1}
 =
 T_2\,(0,43478)^{\tfrac{\gamma-1}{\gamma}}
 \;\approx\;
 848\,\text{K}.
\]
(Obsérvese que \(T_3 < T_2\), al expandir adiabáticamente.)

\medskip
\noindent
\textit{(c) Cálculo de \(\Delta U\), \(W\) y \(Q\)}:  
Dado que \(Q_{2\to 3}=0\) (adiabático), por Primera Ley
\[
\Delta U_{2\to 3}
 = 
 n\,C_{v}\,(T_{3}-T_{2})
 =
 1 \times \tfrac{5}{2}R \,(848 - 1075)
 =
 \tfrac{5}{2}\,(8,31)\,(-227)
 \;\approx\; -4715\,\text{J}.
\]
Luego
\[
W_{2\to 3} 
 = 
 -\,\Delta U_{2\to 3}
 = 
 4715\,\text{J}.
\]
Trabajo positivo significa que el gas realiza trabajo (se expande), a costa de reducir su energía interna (\(\Delta U<0\)).

\bigskip

\textbf{Transformación \(3\to 1\): Isobara a \(5\,\text{bar}\)}

Finalmente, el gas pasa de \((p_3=5\,\text{bar},V_3,T_3=848\,\text{K})\) a \((p_1=5\,\text{bar},V_1,T_1=1075\,\text{K})\).  
A \textbf{presión constante}, el trabajo es
\[
W_{3\to 1}
 =
 \int_{V_3}^{V_1} p\,\mathrm{d}V
 =
 p_3\,(V_1 - V_3)
 =
 n\,R\,(T_1 - T_3)
 \;\approx\;
 8,31\,\bigl(1075 - 848\bigr)
 \;\approx\;
 1886\,\text{J}.
\]
El cambio de energía interna:
\[
\Delta U_{3\to 1}
 =
 n\,C_v\,(T_1 - T_3)
 =
 \tfrac{5}{2}\,R\,(1075 - 848)
 \;\approx\;
 4715\,\text{J}.
\]
Por tanto,
\[
Q_{3\to 1}
 =
 \Delta U_{3\to 1} + W_{3\to 1}
 \;\approx\;
 4715 + 1886
 \;=\;
 6601\,\text{J}.
\]
Se trata de un calor positivo (\(Q>0\)), pues el gas recibe energía para aumentar su temperatura a presión constante.

\bigskip
\noindent
\textbf{Resumen de resultados numéricos:}
\[
\begin{aligned}
&\text{Estado 1: } p_1=5\,\text{bar},\; T_1=1075\,\text{K},\; V_1\approx 0,0179\,\text{m}^3,\\
&\text{Estado 2: } p_2=11{,}5\,\text{bar},\; T_2=1075\,\text{K},\; V_2\approx 0,00778\,\text{m}^3,\\
&\text{Estado 3: } p_3=5\,\text{bar},\quad T_3\approx 848\,\text{K},\quad V_3\approx 0,0141\,\text{m}^3.
\end{aligned}
\]

\[
\begin{array}{c|ccc}
\hline
\textbf{Proceso} & \Delta U \,(\text{J}) & W\,(\text{J}) & Q\,(\text{J}) \\ \hline
1\to 2 \;(\text{isoterma}) & 0 & -7444 & -7444 \\
2\to 3 \;(\text{adiab})    & -4715 & +4715 & 0 \\
3\to 1 \;(\text{isobara})  & +4715 & +1886 & +6601 \\ \hline
\end{array}
\]

\noindent
Para el \textbf{ciclo completo} se verifica que \(\sum \Delta U = 0\) y \(\sum Q = \sum W\).


\newpage

\section{Ejercicio 2}

Tenemos un ciclo de Otto ideal con aire, partiendo de una presión \(p_1=100\,\text{kPa}\) y una 
temperatura \(T_1=36^\circ\text{C}\) y una relación de compresión \(V_1/V_2=7{,}6\). 
Estudiaremos el ciclo desde el momento en el que el cilindro ya está lleno con la mezcla de aire 
y gasolina, y que el proceso de compresión se aproxima a un proceso adiabático, la explosión 
de la mezcla a una compresión a volumen constante en el que se suministran 1200 kJ de calor, 
la expansión o descompresión se asemeja a otro proceso adiabático, y la salida de gases de 
combustión a un proceso isócoro donde se cede calor al foco frío. Realizar los cálculos para 
1 kg del gas.

Datos: Coeficiente de dilatación adiabática del aire = 1,4; 
\quad Masa molar del aire = 28,97 g/mol.

\[
R = 8{,}31 \,\text{J/(K·mol)}, 
\quad
C_v\text{(aire)} = 717{,}62 \,\text{J/(kg·K)}
\]

\noindent
\textbf{a)} Dibuje el ciclo en un diagrama p-V.\\
\textbf{b)} Muéstrese la presión, volumen y temperatura de cada uno de los 4 estados en una tabla.\\
\textbf{c)} Calcúlense el calor cedido y el trabajo útil.\\
\textbf{d)} Hálle el rendimiento del ciclo y compárelo con el de una máquina de Carnot en las 
mismas condiciones de temperatura.

\bigskip


\noindent\textit{(a) Diagrama p--V}

\begin{figure}[htbp]
    \centering
    \includegraphics[width=0.5\textwidth]{diagramaOtto.png}
    \caption{Diagrama \(p\text{-}V\)}
    \label{fig:pv}
\end{figure}

\medskip
\noindent

\newpage

\textit{(b) Tabla de presiones, volúmenes y temperaturas en cada estado}

    \begin{table}[h!] % [h!] sugiere a LaTeX poner la tabla "aquí" si es posible
        \centering % Centra la tabla en la página
        \caption{Estados termodinámicos del ciclo Otto ideal y origen de los valores.} % Título de la tabla
        \label{tab:estados_otto_explicado} % Etiqueta para referencias cruzadas
        \begin{tabular}{ccccl} % Define 5 columnas: 4 centradas (c) y la última alineada a la izquierda (l)
        \hline % Línea horizontal superior
        % Encabezados de columna en negrita
        \textbf{E} & \textbf{P} & \textbf{V} & \textbf{T} & \textbf{Datos} \\
        % Unidades debajo de los encabezados correspondientes
         & \textbf{[kPa]} & \textbf{[m$^3$/kg]} & \textbf{[K]} & \\
        \hline\hline % Doble línea horizontal para separar encabezado y datos
        % Fila para el Estado 1
        1 & 100.0  & 0.8873 & 309.1  & ($p_1, T_1$). $v_1 = R_{esp} T_1 / p_1$. \\ % '&' separa columnas, '\\' termina la fila
        % Fila para el Estado 2
        2 & 1788.3 & 0.1168 & 692.5  &  $v_2 = v_1 / r$; $p_2 = p_1 r^\gamma$; $T_2 = T_1 r^{\gamma-1}$. \\
        % Fila para el Estado 3
        3 & 6106.1 & 0.1168 & 2364.6 &  $v_3 = v_2$; $T_3 = T_2 + Q_{in} / C_v$; $p_3 = p_2 (T_3/T_2)$. \\
        % Fila para el Estado 4
        4 & 341.4  & 0.8873 & 1055.6 &  $v_4 = v_1$; $p_4 = p_3 (v_3/v_4)^\gamma$; $T_4 = T_3 (v_3/v_4)^{\gamma-1}$. \\
        \hline % Línea horizontal inferior
        % Nota al pie explicando las constantes usadas, usando multicolumn para abarcar todas las columnas
        \multicolumn{5}{l}{\footnotesize Nota: Se usó $r=7.6$, $\gamma=1.4$, $R_{esp} \approx 0.287$ kJ/kg·K, $C_v \approx 0.718$ kJ/kg·K, $Q_{in}=1200$ kJ/kg.} \\
        \end{tabular}
\end{table}


\textit{(c) Cálculo del Calor Cedido y Trabajo Útil}

A continuación, se calcula el calor cedido al foco frío ($Q_{cedido}$) y el trabajo neto o útil ($W_{neto}$) realizado por el ciclo Otto ideal por unidad de masa (1 kg) de aire. Se utilizan los valores de temperatura de los estados previamente calculados (ver Tabla~\ref{tab:estados_otto_explicado}): $T_1 \approx 309.1$~K, $T_4 \approx 1055.6$~K, y la capacidad calorífica a volumen constante $C_v \approx 0.718$~kJ/kg·K. El calor añadido durante el proceso isócoro 2-3 fue $Q_{in} = 1200$~kJ/kg.

\subsubsection*{Calor Cedido ($Q_{cedido}$)}

El calor se cede al foco frío durante el proceso de enfriamiento a volumen constante (isócoro) que lleva del estado 4 al estado 1. La fórmula para el calor transferido a volumen constante es $Q = m C_v \Delta T$. Como estamos interesados en el calor cedido (una cantidad positiva por convenio en este contexto), calculamos:
\[ Q_{cedido} = |Q_{41}| = |- m C_v (T_1 - T_4)| = m C_v (T_4 - T_1) \]
Considerando $m=1$~kg:
\[ Q_{cedido} = (1 \text{ kg}) \times (0.71762 \text{ kJ/kg·K}) \times (1055.6 \text{ K} - 309.1 \text{ K}) \]
\[ Q_{cedido} = 0.71762 \times (746.5) \text{ kJ} \]
El resultado es:
\[ {Q_{cedido} \approx 535.7 \text{ kJ/kg}} \]
Este es el calor liberado por el sistema al entorno (foco frío) en cada ciclo por kilogramo de aire.

\subsubsection*{Trabajo Neto Útil ($W_{neto}$)}

El Primer Principio de la Termodinámica establece que para un ciclo completo, la variación de la energía interna es cero ($\Delta U_{ciclo} = 0$), ya que la energía interna es una función de estado. Por lo tanto, el trabajo neto realizado por el sistema durante el ciclo ($W_{neto}$) es igual al calor neto absorbido por el sistema ($Q_{neto}$):
\[ \Delta U_{ciclo} = Q_{neto} - W_{neto} = 0 \implies W_{neto} = Q_{neto} \]
El calor neto es la diferencia entre el calor que entra al sistema (absorbido, $Q_{in}$) y el calor que sale del sistema (cedido, $Q_{cedido}$):
\[ Q_{neto} = Q_{in} - Q_{cedido} \]
Así, el trabajo neto útil del ciclo es:
\[ W_{neto} = Q_{in} - Q_{cedido} \]
Sustituyendo los valores conocidos:
\[ W_{neto} = 1200 \text{ kJ/kg} - 535.7 \text{ kJ/kg} \]
El resultado es:
\[ {W_{neto} \approx 664.3 \text{ kJ/kg}} \]
Este valor representa el trabajo mecánico que se puede extraer del ciclo por cada kilogramo de aire que lo recorre.

\textit{(d) Rendimiento del Ciclo Otto y Comparación con Carnot}

Para calcular el rendimiento del ciclo Otto ideal y compararlo con el de una máquina de Carnot operando entre las mismas temperaturas extremas, seguimos los siguientes pasos:

El rendimiento teórico ($\eta_{Otto}$) de un ciclo Otto ideal depende únicamente de la relación de compresión ($r = V_1/V_2$) y del coeficiente adiabático ($\gamma$) del gas de trabajo. La fórmula, es:
\[
\eta_{Otto} = 1 - \frac{1}{r^{\gamma-1}}
\]
Datos proporcionados:
\begin{itemize}
    \item Relación de compresión, $r = V_1/V_2 = 7.6$
    \item Coeficiente adiabático del aire, $\gamma = 1.4$
\end{itemize}
Sustituyendo los valores:

\[
\eta_{Otto} = 1 - \frac{1}{2.248} \approx 1 - 0.4448 = 0.5552
\]
El rendimiento del ciclo Otto ideal es aproximadamente \textbf{55.52\%}.


El rendimiento de una máquina de Carnot ($\eta_{Carnot}$) que opera entre dos focos térmicos a temperaturas $T_{frio}$ (temperatura mínima del ciclo) y $T_{caliente}$ (temperatura máxima del ciclo) viene dado por :
\[
\eta_{Carnot} = 1 - \frac{T_{frio}}{T_{caliente}}
\]
Necesitamos determinar las temperaturas mínima y máxima del ciclo Otto.
Datos iniciales:
\begin{itemize}
    \item $T_1 = 36^\circ \text{C} = 36 + 273.15 = 309.15 \text{ K}$ (Temperatura mínima, $T_{frio}$)
    \item Calor suministrado por kg, $q_{in} = 1200 \text{ kJ/kg} = 1,200,000 \text{ J/kg}$
    \item Capacidad calorífica a volumen constante, $C_v = 717.62 \text{ J/kg·K}$ 
\end{itemize}
Calculamos las temperaturas en los otros estados del ciclo:
\begin{itemize}
    \item \textbf{Estado 2 (fin de compresión adiabática):} Usamos la relación adiabática $T V^{\gamma-1} = \text{constante}$
    \[
    T_1 V_1^{\gamma-1} = T_2 V_2^{\gamma-1} \implies T_2 = T_1 \left(\frac{V_1}{V_2}\right)^{\gamma-1} = T_1 r^{\gamma-1}
    \]
    \[
    T_2 = 309.15 \times (7.6)^{0.4} \approx 309.15 \times 2.248 \approx 694.97 \text{ K}
    \]
    \item \textbf{Estado 3 (fin de adición de calor isócora):} El calor se añade a volumen constante ($V_2 = V_3$). Para $m=1$ kg, $Q_{in} = m C_v (T_3 - T_2)$ 
    \[
    q_{in} = C_v (T_3 - T_2) \implies T_3 = T_2 + \frac{q_{in}}{C_v}
    \]
    \[
    T_3 = 694.97 + \frac{1,200,000}{717.62} \approx 694.97 + 1672.20 \approx 2367.17 \text{ K}
    \]
    Esta es la temperatura máxima del ciclo ($T_{caliente}$).
    \item \textbf{Estado 4 (fin de expansión adiabática):} (Cálculo no estrictamente necesario para $\eta_{Carnot}$, pero útil para completar el ciclo)
    \[
    T_3 V_3^{\gamma-1} = T_4 V_4^{\gamma-1} \implies T_4 = T_3 \left(\frac{V_3}{V_4}\right)^{\gamma-1} = T_3 \left(\frac{V_2}{V_1}\right)^{\gamma-1} = T_3 \left(\frac{1}{r}\right)^{\gamma-1}
    \]
     \[
     T_4 = \frac{T_3}{r^{\gamma-1}} \approx \frac{2367.17}{2.248} \approx 1052.9 \text{ K}
     \]
\end{itemize}
Las temperaturas extremas del ciclo son $T_{frio} = T_1 = 309.15 \text{ K}$ y $T_{caliente} = T_3 = 2367.17 \text{ K}$.
Ahora calculamos el rendimiento de Carnot:
\[
\eta_{Carnot} = 1 - \frac{T_1}{T_3} = 1 - \frac{309.15}{2367.17} \approx 1 - 0.1306 = 0.8694
\]
El rendimiento de la máquina de Carnot operando entre las mismas temperaturas extremas es aproximadamente \textbf{86.94\%}.

El rendimiento del ciclo Otto ideal es $\eta_{Otto} \approx 55.52\%$.
El rendimiento de una máquina de Carnot operando entre la temperatura mínima ($T_1$) y máxima ($T_3$) del ciclo Otto es $\eta_{Carnot} \approx 86.94\%$.

Como se esperaba según el Teorema de Carnot, el rendimiento del ciclo Otto es inferior al rendimiento de una máquina de Carnot que opera entre las mismas temperaturas extremas:
$$ \eta_{Otto} < \eta_{Carnot} $$
$$ 55.52\% < 86.94\% $$
Esto se debe a que los procesos de adición y cesión de calor en el ciclo Otto no ocurren a temperatura constante, a diferencia del ciclo de Carnot.


\section{Ejercicio 3}

\textbf{Enunciado:}
Un ciclo refrigerador reversible de Carnot se emplea para mantener a \SI{-18}{\celsius} el congelador de un frigorífico instalado en un local donde la temperatura es \SI{20}{\celsius}. En este ciclo termodinámico, se realiza cada \SI{2}{\second}, y en él se emplea \SI{1}{\mole} de un gas ideal diatómico ($C_v = 5/2 R$). Por las especificaciones técnicas se sabe que consume una potencia de \SI{50}{\kilo\watt}.

Datos: $R = \SI{8,31}{\joule\per\kelvin\per\mole}$

a) Dibújese el ciclo en un diagrama p-V y especifíquese cada una de las transformaciones que lo componen. Calcúlese su eficiencia.

b) Calcúlese el calor que se intercambia en cada una de las etapas.

c) Calcúlese la variación de entropía del gas en cada transformación y para el ciclo completo. Calcúlese también la variación de entropía del Universo.

d) Si tras la compresión isotérmica el volumen del gas es igual a \SI{5}{\litre}, calcúlese el volumen y la presión tras la expansión adiabática.

\vspace{1em} % Espacio vertical
\bigskip
%-----------------------------------------------------
\textit{(a) Diagrama P-V, Transformaciones y Eficiencia}
\bigskip
%-----------------------------------------------------
Las transformaciones del ciclo refrigerador de Carnot son (en orden):
\begin{enumerate}
    \item $1 \to 2$: Expansión adiabática reversible (el gas se enfría de $T_H$ a $T_C$).
    \item $2 \to 3$: Expansión isotérmica reversible a $T_C$ (el gas absorbe calor $Q_C$ del foco frío).
    \item $3 \to 4$: Compresión adiabática reversible (el gas se calienta de $T_C$ a $T_H$).
    \item $4 \to 1$: Compresión isotérmica reversible a $T_H$ (el gas cede calor $Q_H$ al foco caliente).
\end{enumerate}
\textbf{Diagrama P-V:} No he encontrado la forma de generar una representación gráfica estándar y visualmente útil con los parámetros dados. Hay combinación de un trabajo de entrada muy alto (\SI{100}{\kilo\joule}) para una cantidad pequeña de gas (\SI{1}{\mole}) partiendo de un volumen $V_1=\SI{5}{\litre}$ (asumiendo por el apartado d). Esto distorsiona completamente la escala del eje de volúmenes.

\textbf{Eficiencia:}( $\beta$). Para un ciclo de Carnot reversible, depende solo de las temperaturas:
\[
    \beta_{\text{Carnot}} = \frac{T_C}{T_H - T_C} = \frac{\SI{255,15}{\kelvin}}{\SI{293,15}{\kelvin} - \SI{255,15}{\kelvin}} = \frac{255,15}{38} \approx 6,714
\]

\bigskip
%-----------------------------------------------------
\textit{(b) Calor intercambiado en cada etapa}
\bigskip
%-----------------------------------------------------
Calculamos los calores intercambiados por el gas (el sistema):
\begin{itemize}
    \item \textbf{Etapa $1 \to 2$} (Adiabática): $Q_{12} = \SI{0}{\joule}$
    \item \textbf{Etapa $2 \to 3$} (Isotérmica a $T_C$): Se absorbe calor $Q_C$. $Q_C = \beta \times W_{in}$.
        \[
            Q_{23} = Q_C \approx 6,714 \times \SI{100000}{\joule} \approx \SI{+671400}{\joule}
        \]
    \item \textbf{Etapa $3 \to 4$} (Adiabática): $Q_{34} = \SI{0}{\joule}$
    \item \textbf{Etapa $4 \to 1$} (Isotérmica a $T_H$): Se cede calor $Q_H$. Por el primer principio para el ciclo, $Q_{neto} = Q_C + Q_{41} = -W_{in}$.
        \[
            Q_{41} = -W_{in} - Q_C = \SI{-100000}{\joule} - \SI{671400}{\joule} = \SI{-771400}{\joule}
        \]
        El calor cedido al foco caliente es $|Q_H| = \SI{771400}{\joule}$.
\end{itemize}

%-----------------------------------------------------
\bigskip
\textit{(c) Variación de entropía}\\
\bigskip
%-----------------------------------------------------
\textbf{Variación de entropía del gas ($\Delta S_{gas}$):}
\begin{itemize}
    \item \textbf{Etapa $1 \to 2$} (Adiabática Reversible): $\Delta S_{12} = \SI{0}{\joule\per\kelvin}$
    \item \textbf{Etapa $2 \to 3$} (Isotérmica Reversible a $T_C$):
        \[
            \Delta S_{23} = \frac{Q_{23}}{T_C} = \frac{\SI{671400}{\joule}}{\SI{255,15}{\kelvin}} \approx \SI{+2631,4}{\joule\per\kelvin}
        \]
    \item \textbf{Etapa $3 \to 4$} (Adiabática Reversible): $\Delta S_{34} = \SI{0}{\joule\per\kelvin}$
    \item \textbf{Etapa $4 \to 1$} (Isotérmica Reversible a $T_H$):
        \[
            \Delta S_{41} = \frac{Q_{41}}{T_H} = \frac{\SI{-771400}{\joule}}{\SI{293,15}{\kelvin}} \approx \SI{-2631,4}{\joule\per\kelvin}
        \]
\end{itemize}
\textbf{Variación de entropía del gas en el ciclo completo:}
\[
    \Delta S_{\text{ciclo}} = \Delta S_{12} + \Delta S_{23} + \Delta S_{34} + \Delta S_{41} \approx 0 + 2631,4 + 0 - 2631,4 = \SI{0}{\joule\per\kelvin}
\]
\textbf{Variación de entropía del Universo ($\Delta S_{Universo}$):}
Para un proceso reversible, la variación de entropía del universo es siempre cero. Comprobación:
$\Delta S_{\text{Universo}} = \Delta S_{\text{ciclo}} + \Delta S_{\text{foco frío}} + \Delta S_{\text{foco caliente}}$
$\Delta S_{\text{foco frío}} = -Q_C / T_C \approx \SI{-2631,4}{\joule\per\kelvin}$
$\Delta S_{\text{foco caliente}} = |Q_H| / T_H = -Q_{41} / T_H \approx \SI{+2631,4}{\joule\per\kelvin}$
\[
    \Delta S_{\text{Universo}} \approx 0 - 2631,4 + 2631,4 = \SI{0}{\joule\per\kelvin}
\]
\bigskip
%-----------------------------------------------------
\textit{(d) Volumen y Presión tras la expansión adiabática}
%-----------------------------------------------------
Se pide el estado 2 ($V_2, P_2$) que se alcanza tras la expansión adiabática $1 \to 2$.
Estado 1: $T_1 = T_H = \SI{293,15}{\kelvin}$, $V_1 = \SI{0,005}{\cubic\metre}$.
Estado 2: $T_2 = T_C = \SI{255,15}{\kelvin}$.
Relación adiabática $T V^{\gamma-1} = \text{constante}$: $T_1 V_1^{\gamma-1} = T_2 V_2^{\gamma-1}$.
\[
    V_2 = V_1 \left( \frac{T_1}{T_2} \right)^{\frac{1}{\gamma-1}} = \SI{0,005}{\cubic\metre} \times \left( \frac{293,15}{255,15} \right)^{\frac{1}{1,4-1}} = \SI{0,005}{\cubic\metre} \times \left( 1,1489... \right)^{2,5}
\]
\[
    V_2 \approx \SI{0,005}{\cubic\metre} \times 1,3976 \approx \SI{0,006988}{\cubic\metre} \quad (\approx \SI{6,99}{\litre})
\]
Ley del gas ideal para $P_2$: $P_2 V_2 = n R T_2$.
\[
    P_2 = \frac{n R T_2}{V_2} = \frac{\SI{1}{\mole} \times \SI{8,31}{\joule\per\kelvin\per\mole} \times \SI{255,15}{\kelvin}}{\SI{0,006988}{\cubic\metre}} \approx \frac{2119,30}{0,006988} \approx \SI{303277}{\pascal}
\]
Tras la expansión adiabática, el volumen es $\mathbf{V_2 \approx \SI{0,00699}{\cubic\metre}}$ y la presión es $\mathbf{P_2 \approx \SI{303,3}{\kilo\pascal}}$.


\section{Ejercicio 4}

\textbf{Enunciado:}
Disponemos de una máquina térmica que trabaja entre dos focos térmicos, cuyas temperaturas son la del foco caliente \SI{500}{\kelvin} y la del frío \SI{100}{\kelvin}, y además sabemos que en cada ciclo se absorben del foco caliente \SI{1000}{\joule}. Si el rendimiento es del 20\%,

a) ¿Cómo funciona la máquina térmica, reversible o irreversiblemente? Explique por qué.

b) Determínese la variación de entropía del gas de trabajo de la máquina térmica, así como de sus alrededores y la del universo en cada ciclo.

c) Repítase los cálculos realizados en el apartado anterior para una máquina térmica de Carnot que funcionase entre los dos mismos focos.

d) Calcúlese el trabajo que se obtiene de la máquina en un ciclo, y la potencia que puede desarrollar si se completa un ciclo cada \SI{500}{\milli\second}.

\vspace{1em} % Espacio vertical

\bigskip
%-----------------------------------------------------
\textit{(a) ¿Cómo funciona la máquina térmica, reversible o irreversiblemente? Explique por qué.}
%-----------------------------------------------------
\bigskip

Para determinar si la máquina funciona reversible o irreversiblemente, comparamos su rendimiento real con el rendimiento máximo posible entre esas temperaturas, que es el rendimiento de una máquina de Carnot.
El rendimiento de Carnot ($\eta_{\text{Carnot}}$) es:
\[
\eta_{\text{Carnot}} = 1 - \frac{T_C}{T_H} = 1 - \frac{\SI{100}{\kelvin}}{\SI{500}{\kelvin}} = 1 - 0,2 = 0,80 \quad (\text{o } 80\%)
\]
El rendimiento dado de la máquina es $\eta = 0,20$ (o 20\%).
Como el rendimiento real de la máquina ($\eta = 0,20$) es menor que el rendimiento máximo posible o de Carnot ($\eta_{\text{Carnot}} = 0,80$), la máquina funciona \textbf{irreversiblemente}. El Teorema de Carnot establece que ninguna máquina térmica real puede tener un rendimiento mayor que el de una máquina de Carnot operando entre las mismas temperaturas, y solo una máquina reversible alcanza dicho rendimiento.

\bigskip
%-----------------------------------------------------
\textit{(b) Determínese la variación de entropía del gas de trabajo de la máquina térmica, así como de sus alrededores y la del universo en cada ciclo.}
%-----------------------------------------------------
\bigskip


\textit{Variación de entropía del gas de trabajo ($\Delta S_{\text{gas}}$):}
La entropía es una función de estado. Como la máquina opera en un ciclo, el estado final del gas de trabajo es idéntico a su estado inicial. Por lo tanto, la variación de entropía del gas en un ciclo completo es cero.
\[
\Delta S_{\text{gas}} = \SI{0}{\joule\per\kelvin}
\]

\textit{Variación de entropía de los alrededores ($\Delta S_{\text{alr}}$):}
Los alrededores son los focos térmicos.
Primero calculamos el trabajo neto ($W_{\text{net}}$) y el calor cedido al foco frío ($|Q_C|$).
\[
W_{\text{net}} = \eta \times Q_H = 0,20 \times \SI{1000}{\joule} = \SI{200}{\joule}
\]
Por el primer principio para el ciclo ($\Delta U = 0$), $W_{\text{net}} = Q_H - |Q_C|$.
\[
|Q_C| = Q_H - W_{\text{net}} = \SI{1000}{\joule} - \SI{200}{\joule} = \SI{800}{\joule}
\]
El foco caliente cede $Q_H = \SI{1000}{\joule}$, su cambio de entropía es:
\[
\Delta S_{H} = \frac{-Q_H}{T_H} = \frac{-\SI{1000}{\joule}}{\SI{500}{\kelvin}} = \SI{-2,0}{\joule\per\kelvin}
\]
El foco frío recibe $|Q_C| = \SI{800}{\joule}$, su cambio de entropía es:
\[
\Delta S_{C} = \frac{+|Q_C|}{T_C} = \frac{+\SI{800}{\joule}}{\SI{100}{\kelvin}} = \SI{+8,0}{\joule\per\kelvin}
\]
La variación de entropía total de los alrededores es:
\[
\Delta S_{\text{alr}} = \Delta S_{H} + \Delta S_{C} = \SI{-2,0}{\joule\per\kelvin} + \SI{8,0}{\joule\per\kelvin} = \SI{+6,0}{\joule\per\kelvin}
\]

\textit{Variación de entropía del Universo ($\Delta S_{\text{uni}}$):}
\[
\Delta S_{\text{uni}} = \Delta S_{\text{gas}} + \Delta S_{\text{alr}} = \SI{0}{\joule\per\kelvin} + \SI{6,0}{\joule\per\kelvin} = \SI{+6,0}{\joule\per\kelvin}
\]
Como $\Delta S_{\text{uni}} > 0$, se confirma que el proceso es irreversible.

\bigskip
%-----------------------------------------------------
\textit{(c) Repítase los cálculos realizados en el apartado anterior para una máquina térmica de Carnot que funcionase entre los dos mismos focos.}
%-----------------------------------------------------
\bigskip

Consideramos una máquina de Carnot reversible operando entre $T_H = \SI{500}{\kelvin}$ y $T_C = \SI{100}{\kelvin}$, absorbiendo $Q_H = \SI{1000}{\joule}$.

\textit{Variación de entropía del gas de trabajo ($\Delta S_{\text{gas, Carnot}}$):}
Sigue siendo un ciclo, por lo que:
\[
\Delta S_{\text{gas, Carnot}} = \SI{0}{\joule\per\kelvin}
\]

\textit{Variación de entropía de los alrededores ($\Delta S_{\text{alr, Carnot}}$):}
Para una máquina de Carnot, la relación de calores es $\frac{|Q_C|}{Q_H} = \frac{T_C}{T_H}$.
\[
|Q_{C, \text{Carnot}}| = Q_H \frac{T_C}{T_H} = \SI{1000}{\joule} \times \frac{\SI{100}{\kelvin}}{\SI{500}{\kelvin}} = \SI{1000}{\joule} \times 0,2 = \SI{200}{\joule}
\]
El cambio de entropía del foco caliente es:
\[
\Delta S_{H, \text{Carnot}} = \frac{-Q_H}{T_H} = \frac{-\SI{1000}{\joule}}{\SI{500}{\kelvin}} = \SI{-2,0}{\joule\per\kelvin}
\]
El cambio de entropía del foco frío es:
\[
\Delta S_{C, \text{Carnot}} = \frac{+|Q_{C, \text{Carnot}}|}{T_C} = \frac{+\SI{200}{\joule}}{\SI{100}{\kelvin}} = \SI{+2,0}{\joule\per\kelvin}
\]
La variación de entropía total de los alrededores para la máquina de Carnot es:
\[
\Delta S_{\text{alr, Carnot}} = \Delta S_{H, \text{Carnot}} + \Delta S_{C, \text{Carnot}} = \SI{-2,0}{\joule\per\kelvin} + \SI{2,0}{\joule\per\kelvin} = \SI{0}{\joule\per\kelvin}
\]

\textit{Variación de entropía del Universo ($\Delta S_{\text{uni, Carnot}}$):}
\[
\Delta S_{\text{uni, Carnot}} = \Delta S_{\text{gas, Carnot}} + \Delta S_{\text{alr, Carnot}} = \SI{0}{\joule\per\kelvin} + \SI{0}{\joule\per\kelvin} = \SI{0}{\joule\per\kelvin}
\]
Como se esperaba para un proceso reversible.

\bigskip
%-----------------------------------------------------
\textit{(d) Calcúlese el trabajo que se obtiene de la máquina en un ciclo, y la potencia que puede desarrollar si se completa un ciclo cada \SI{500}{\milli\second}.}
%-----------------------------------------------------
\bigskip

\textit{Trabajo por ciclo ($W_{\text{net}}$) (para la máquina real con $\eta=0,20$):}
El trabajo neto obtenido se calcula a partir del rendimiento y el calor absorbido:
\[
W_{\text{net}} = \eta \times Q_H = 0,20 \times \SI{1000}{\joule} = \SI{200}{\joule}
\]
El trabajo obtenido en cada ciclo es $\mathbf{\SI{200}{\joule}}$.

\textit{Potencia desarrollada ($P$):}
La potencia es el trabajo realizado por unidad de tiempo. El tiempo por ciclo es $\Delta t = \SI{0,5}{\second}$.
\[
P = \frac{W_{\text{net}}}{\Delta t} = \frac{\SI{200}{\joule}}{\SI{0,5}{\second}} = \SI{400}{\joule\per\second} = \SI{400}{\watt}
\]
La potencia que puede desarrollar la máquina es $\mathbf{\SI{400}{\watt}}$.

\end{document}
