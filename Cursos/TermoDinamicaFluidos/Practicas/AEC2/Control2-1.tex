\documentclass[a4paper,12pt]{article}
\usepackage[spanish]{babel}
\usepackage[utf8]{inputenc}
\usepackage{amsmath, amssymb}
\usepackage{graphicx}
\usepackage{geometry}
\usepackage{fancyhdr}
\usepackage{hyperref}
\usepackage{siunitx}
\usepackage{enumitem}

% Configuración de márgenes
\geometry{left=2.5cm, right=2.5cm, top=3cm, bottom=3cm}

% Encabezado y pie de página
\pagestyle{fancy}
\fancyhf{}
\fancyhead[L]{UDIMA}
\fancyhead[R]{Fundamentos de Termodinámica y Mecánica de Fluidos}
\fancyfoot[C]{\thepage}

% Título del documento
\title{\textbf{Resolución del Control de Evaluación}\\[0.5cm]
\Large{Unidades 5 y 6 - Transferencia de Calor}}
\author{Alumno: Alexander Sebastian Kalis \\ Profesor: Dr.~César Pérez de Villar Palomo}
\date{\today}

\begin{document}

\maketitle
\newpage

\section*{Preguntas y Resoluciones (Primera tanda)}

\subsection*{Pregunta 1}
\textit{La transmisión de calor por convección:}
\begin{enumerate}[label=\alph*)]
    \item Se produce únicamente cuando la superficie está a la misma temperatura que el fluido.
    \item Se produce únicamente cuando la superficie está a menor temperatura.
    \item Se produce únicamente cuando la superficie está a mayor temperatura.
    \item \textbf{Se produce cuando la superficie está a distinta temperatura que el fluido.}
\end{enumerate}

\paragraph{Respuesta Correcta: d)}

\hrulefill

\subsection*{Pregunta 2}
\textit{Para el caso de paredes planas en paralelo el calor global debido a conducción será:}
\begin{enumerate}[label=\alph*)]
    \item Igual a la que pasa por cada pared.
    \item Fórmula en serie.
    \item \textbf{Suma de flujos por cada pared (fórmula en paralelo).}
    \item Ninguna es correcta.
\end{enumerate}

\paragraph{Respuesta Correcta: c)}

\hrulefill

\subsection*{Pregunta 3}
\textit{Conducción de 3 kW a través de aislante ($A=8.4$ m$^2$, $L=9.1$ mm, $k=0.2$). $T_{int}=415^\circ$C. Calcular $T_{ext}$.}

\paragraph{Respuesta: 399}
\textbf{Justificación:} $\Delta T=\frac{Q L}{kA}=\frac{3000\cdot0.0091}{0.2\cdot8.4}=16.25$.  
$T_{ext}=415-16.25=399^\circ$C.  
(Corrección: en tu versión tenías 413.4 ºC, pero el resultado correcto es $\approx 399^\circ$C).

\hrulefill

\subsection*{Pregunta 4}
\textit{Madera de $L=45.2$ mm, $q=40$ W/m$^2$, $\Delta T=26.4$ ºC. Calcular $k$.}

\paragraph{Respuesta: 0.068}
\textbf{Justificación:} $k=\frac{q L}{\Delta T}=\frac{40\cdot0.0452}{26.4}=0.068$ W/m·K.  
(Corrección: en tu versión habías puesto 0.061, pero el cálculo da $\approx 0.068$).

\hrulefill

\subsection*{Pregunta 5}
\textit{La conductancia convectiva tiene unidades de $\frac{J}{m^2 s K}$.}
\begin{enumerate}[label=\alph*)]
    \item \textbf{Verdadero}
    \item Falso
\end{enumerate}

\paragraph{Respuesta Correcta: a)}

\hrulefill

\subsection*{Pregunta 6}
\textit{¿Qué mecanismo no necesita medio físico?}
\begin{enumerate}[label=\alph*)]
    \item Conducción
    \item Todas necesitan medio
    \item \textbf{Radiación}
    \item Convección
\end{enumerate}

\paragraph{Respuesta Correcta: c)}

\hrulefill

\subsection*{Pregunta 7}
\textit{En paredes planas en paralelo, el calor total es la suma de los calores parciales.}
\begin{enumerate}[label=\alph*)]
    \item \textbf{Verdadero}
    \item Falso
\end{enumerate}

\paragraph{Respuesta Correcta: a)}

\hrulefill

\subsection*{Pregunta 8}
\textit{La conductividad térmica $k$ tiene unidades de $\frac{J}{m K}$.}
\begin{enumerate}[label=\alph*)]
    \item Verdadero
    \item \textbf{Falso}
\end{enumerate}

\paragraph{Respuesta Correcta: b)}  
\textbf{Justificación:} Correctamente es $\frac{W}{m K} = \frac{J}{s m K}$.

\hrulefill

\newpage
\section*{Preguntas y Resoluciones (Segunda tanda)}

\subsection*{Pregunta 9}
\textit{Para que se produzca transferencia de calor por convección siempre es necesario que exista diferencia de temperaturas entre la superficie y el fluido.}
\begin{enumerate}[label=\alph*)]
    \item \textbf{Verdadero}
    \item Falso
\end{enumerate}

\paragraph{Respuesta Correcta: a)}

\hrulefill

\subsection*{Pregunta 10}
\textit{La conductividad térmica tiene unidades de $W/(m \cdot K)$.}
\begin{enumerate}[label=\alph*)]
    \item \textbf{Verdadero}
    \item Falso
\end{enumerate}

\paragraph{Respuesta Correcta: a)}

\hrulefill

\subsection*{Pregunta 11}
\textit{En paredes planas en serie, el calor total es la suma de los calores que atraviesan cada pared.}
\begin{enumerate}[label=\alph*)]
    \item Verdadero
    \item \textbf{Falso}
\end{enumerate}

\paragraph{Respuesta Correcta: b)}  
\textbf{Justificación:} En serie, el calor es el mismo en todas las capas, no se suma. Lo que se suma son las resistencias térmicas.

\hrulefill

\subsection*{Pregunta 12}
\textit{La convección forzada:}
\begin{enumerate}[label=\alph*)]
    \item \textbf{Necesita fuerza motriz externa}
    \item Se produce por diferencia de densidad
    \item Ninguna es correcta
    \item No necesita fluido
\end{enumerate}

\paragraph{Respuesta Correcta: a)}

\hrulefill

\subsection*{Pregunta 13}
\textit{La transmisión de calor por radiación:}
\begin{enumerate}[label=\alph*)]
    \item Es independiente de la temperatura
    \item \textbf{Se produce por ondas electromagnéticas}
    \item No se puede producir en fluido
    \item Necesita vacío
\end{enumerate}

\paragraph{Respuesta Correcta: b)}

\hrulefill

\subsection*{Pregunta 14}
\textit{Conducción de 3 kW, $A=16$ m$^2$, $L=2.2$ mm, $k=0.2$, $T_{int}=415^\circ$C. Calcular $T_{ext}$.}

\paragraph{Respuesta: 415}
\textbf{Justificación:} $\Delta T=\frac{Q L}{kA}=\frac{3000\cdot0.0022}{0.2\cdot16}=2.06$.  
$T_{ext}=415-2.06\approx 413^\circ$C.

\hrulefill

\subsection*{Pregunta 15}
\textit{Madera de $L=43.5$ mm, $\Delta T=19.7$ ºC, $q=40$ W/m$^2$. Calcular $k$.}

\paragraph{Respuesta: 0.088}
\textbf{Justificación:} $k=\frac{q L}{\Delta T}=\frac{40\cdot0.0435}{19.7}=0.088$ W/m·K.

\hrulefill

\subsection*{Pregunta 16}
\textit{La conducción estacionaria unidimensional se rige por la ley de Fourier. ¿Cuál es la forma correcta?}
\begin{enumerate}[label=\alph*)]
    \item Incorrecta
    \item Incorrecta
    \item Incorrecta
    \item \textbf{Proporcional a $A$, $k$, $\Delta T$ e inversamente a $L$}
\end{enumerate}

\paragraph{Respuesta Correcta: d)}

\end{document}
