\documentclass[a4paper,12pt]{article}
\usepackage[spanish]{babel}
\usepackage[utf8]{inputenc}
\usepackage{amsmath, amssymb}
\usepackage{graphicx}
\usepackage{geometry}
\usepackage{fancyhdr}
\usepackage{hyperref}
\usepackage{siunitx}
\usepackage{pdflscape}
% Configuración de márgenes
\geometry{left=2.5cm, right=2.5cm, top=3cm, bottom=3cm}

% Encabezado y pie de página
\pagestyle{fancy}
\fancyhf{}
\fancyhead[L]{UDIMA}
\fancyhead[R]{Fundamentos de Termodinámica y Mecánica de Fluidos}
\fancyfoot[C]{\thepage}

% Título del documento
\title{\textbf{Actividad de Evaluación Contínua 3}\\[0.5cm]
\Large{Unidades 7 a 10 - Mecánica de Fluidos}}
\author{Alumno: Alexander Sebastian Kalis \\ Profesor: Dr.~César Pérez de Villar Palomo}
\date{\today}

\begin{document}

\maketitle
\newpage
\tableofcontents
\newpage



\newpage
\begin{landscape}
\small
\begin{tabular}{|p{0.20\textwidth}|p{0.20\textwidth}|p{0.20\textwidth}|p{0.20\textwidth}|p{0.20\textwidth}|}
\hline
\textbf{Aspecto Comparado} & \textbf{Principio básico/funcionamiento} & \textbf{Tipo de fuente/recurso} & \textbf{Rol en el sistema/gestión de red} & \textbf{Característica/ventaja/desafío clave} \\
\hline
\textbf{Centrales de ciclo combinado} & Producen electricidad utilizando una turbina de gas y aprovechando el calor residual para generar vapor en un ciclo secundario, que mueve una turbina de vapor [2.1]. & No renovable (gas natural), un combustible fósil cuya disponibilidad depende de los mercados energéticos globales. & Actúan como \textbf{generación de respaldo} para garantizar la estabilidad de la red, cubriendo picos de demanda o supliendo la falta de producción de renovables [12.1]. & Son altamente flexibles, pudiendo arrancar y parar rápidamente. Tienen menores emisiones de CO$_2$ que otras centrales térmicas, pero siguen contribuyendo al cambio climático. \\
\hline
\textbf{Centrales nucleares} & Generan electricidad a partir de la \textbf{fisión nuclear} de uranio. El calor producido calienta agua, creando vapor que mueve una turbina conectada a un generador [3.1]. & No renovable (uranio), aunque una pequeña cantidad puede producir una gran cantidad de energía, prolongando su disponibilidad. & Son la base del \textbf{suministro eléctrico (base load)}. Ofrecen una generación de electricidad constante, robusta y predecible, con muy alta disponibilidad [13.1]. & Producen grandes cantidades de energía con muy bajas emisiones de gases de efecto invernadero. El principal desafío es la gestión segura de los residuos radiactivos de larga vida y el riesgo de accidentes [18.1]. \\
\hline
\textbf{Instalaciones de energía solar fotovoltaica} & Convierten la radiación solar directamente en electricidad mediante el \textbf{efecto fotovoltaico}. Los paneles solares, formados por celdas, liberan electrones al ser expuestos a la luz solar [4.1, 9.1]. & Renovable (solar), con disponibilidad variable y dependiente de las condiciones meteorológicas y el ciclo día-noche. & Su rol es el de \textbf{generación variable}. Su producción se adapta a la demanda diurna, pero requiere fuentes de respaldo o almacenamiento para las horas sin sol [14.1]. & No generan emisiones ni residuos durante su operación, pero su intermitencia y dependencia del clima son desafíos importantes [19.1]. \\
\hline
\textbf{Instalaciones de energía eólica} & Aprovechan la energía cinética del viento para mover las palas de un aerogenerador, que a su vez acciona un generador para producir electricidad [5.1, 10.1]. & Renovable (eólica), con disponibilidad variable según las condiciones climáticas. | Contribuyen como \textbf{generación variable}, aportando una parte significativa al mix energético. Los aerogeneradores modernos pueden apoyar la estabilidad de la red al inyectar energía reactiva [15.1]. & Son una fuente de energía limpia y con bajos costes operativos. El desafío principal es su intermitencia, que puede requerir de sistemas de respaldo o almacenamiento [20.1]. \\
\hline
\textbf{Sistemas de almacenamiento de electricidad} & Almacenan la energía eléctrica en grandes cantidades, como en baterías (químicas) o centrales hidroeléctricas de bombeo (energía potencial). El agua se bombea a un embalse superior cuando la energía es barata para luego liberarla y generar electricidad en picos de demanda [6.1, 11.1]. & Varía (puede ser de fuentes renovables o no renovables), pero su rol es el de \textbf{gestión de la energía}. & Su rol es fundamental en la \textbf{regulación de precisión y el respaldo}. Almacenan el excedente de energía de las fuentes intermitentes y la liberan cuando la demanda es alta o la generación es baja [16.1]. & Son clave para la integración de las renovables. Tienen una alta flexibilidad y capacidad de respuesta. Su principal desafío es el coste elevado, especialmente el de las baterías [21.1]. \\
\hline
\end{tabular}
\end{landscape}

\end{document}
