\documentclass[a4paper,12pt]{article}
\usepackage[utf8]{inputenc}
\usepackage{amsmath, amssymb}
\usepackage{graphicx}
\usepackage{geometry}
\usepackage{float}
\usepackage{caption}
\usepackage[spanish]{babel}
\geometry{left=3cm, right=3cm, top=2.5cm, bottom=2.5cm}
\usepackage{fancyhdr}
\pagestyle{fancy}
\fancyhf{}
\fancyhead[L]{Tecnología Eléctrica}
\fancyhead[R]{Universidad a Distancia de Madrid (UDIMA)}
\fancyfoot[C]{\thepage}

\title{AEC1. Prácticas de máquinas eléctricas. Simulación y circuitos equivalentes}
\author{Alexander Sebastian Kalis}
\date{9 de noviembre de 2024}

\begin{document}

\maketitle

\newpage

\tableofcontents

\newpage 

\section{Problema 1}
Un transformador monofásico de 1 MVA, relación de tensiones 10000/1000 V y 50 Hz, ha dado los siguientes resultados en unos ensayos:

\begin{itemize}
    \item \textbf{Ensayo de vacío} (lado de baja tensión): $U_0 = 1000$ V, $I_0 = 30$ A, $P_0 = 10$ kW.
    \item \textbf{Ensayo de cortocircuito} (lado de alta tensión): $U_{\text{cc}} = 540$ V, $I_{\text{cc}} = 90$ A, $P_{\text{cc}} = 12$ kW.
\end{itemize}



\subsection{a) Circuito equivalente}
Dibuja el circuito equivalente del transformador e indica los valores numéricos de las resistencias y reactancias calculadas:


A partir de los datos obtenidos en los ensayos de vacío y cortocircuito, determinamos los parámetros del circuito equivalente del transformador.

\subsubsection{Ensayo de vacío}
Datos del ensayo de vacío en el lado de baja tensión:
\[
U_0 = 1000 \, \text{V}, \quad I_0 = 30 \, \text{A}, \quad P_0 = 10,000 \, \text{W}
\]

Para calcular el factor de potencia en vacío:
\[
\cos \phi_0 = \frac{P_0}{U_0 \cdot I_0} = \frac{10,000}{1000 \times 30} = 0.333
\]
y el ángulo de fase:
\[
\phi_0 = \cos^{-1}(0.333) \approx 70.53^\circ
\]

Descomponemos la corriente de vacío \( I_0 \) en sus componentes activa \( I_{\text{Fe}} \) y reactiva \( I_m \):
\[
I_{\text{Fe}} = I_0 \cos \phi_0 = 10 \, \text{A}, \quad I_m = I_0 \sin \phi_0 \approx 28.29 \, \text{A}
\]

Calculamos los valores de la rama en paralelo:
\[
R_{\text{Fe}} = \frac{U_0}{I_{\text{Fe}}} = \frac{1000}{10} = 100 \, \Omega
\]
\[
X_m = \frac{U_0}{I_m} = \frac{1000}{28.29} \approx 35.35 \, \Omega
\]

\subsubsection{Ensayo de cortocircuito}
Datos del ensayo de cortocircuito en el lado de alta tensión:
\[
U_{\text{cc}} = 540 \, \text{V}, \quad I_{\text{cc}} = 90 \, \text{A}, \quad P_{\text{cc}} = 12,000 \, \text{W}
\]

Calculamos la resistencia y reactancia en serie total:
\[
R_{\text{cc}} = \frac{P_{\text{cc}}}{I_{\text{cc}}^2} = \frac{12,000}{90^2} \approx 1.48 \, \Omega
\]
\[
Z_{\text{cc}} = \frac{U_{\text{cc}}}{I_{\text{cc}}} = 6 \, \Omega
\]
\[
X_{\text{cc}} = \sqrt{Z_{\text{cc}}^2 - R_{\text{cc}}^2} = \sqrt{6^2 - 1.48^2} \approx 5.81 \, \Omega
\]

Distribuimos estos valores entre \( R_1 \), \( X_1 \), \( R_2 \), y \( X_2 \) suponiendo que:
\[
R_1 = R_2 = \frac{R_{\text{cc}}}{2} = 0.74 \, \Omega, \quad X_1 = X_2 = \frac{X_{\text{cc}}}{2} = 2.905 \, \Omega
\]

\subsubsection{Resumen del circuito equivalente}
Los parámetros obtenidos para el circuito equivalente del transformador son:
\begin{itemize}
    \item Resistencia en paralelo (pérdidas en el núcleo): \( R_{\text{Fe}} = 100 \, \Omega \)
    \item Reactancia de magnetización: \( X_m = 35.35 \, \Omega \)
    \item Resistencia en el lado primario: \( R_1 = 0.74 \, \Omega \)
    \item Reactancia en el lado primario: \( X_1 = 2.905 \, \Omega \)
    \item Resistencia en el lado secundario: \( R_2 = 0.74 \, \Omega \)
    \item Reactancia en el lado secundario: \( X_2 = 2.905 \, \Omega \)
\end{itemize}

Estos valores conforman el circuito equivalente completo del transformador.




\begin{figure}[H]
    \centering
    \includegraphics[width=0.8\textwidth]{circuito_equivalente.png} % Sustituye con la ruta de tu imagen
    \caption{Circuito equivalente del transformador}
\end{figure}

\subsection{b) Tensión de alimentación}
Calcula la tensión con la que se debe alimentar el transformador en el primario para proporcionar la tensión nominal en el secundario al suministrar a una carga de 800 kVA con un factor de potencia 0.8 inductivo.


Para determinar la tensión en el primario \( U_1 \) que permita obtener la tensión nominal en el secundario bajo una carga de 800 kVA con un factor de potencia de 0.8 inductivo, seguimos el procedimiento descrito.

\subsubsection*{1. Cálculo de la corriente en el secundario}
Dado:
\begin{itemize}
    \item Potencia aparente de la carga: \( S = 800 \, \text{kVA} = 800,000 \, \text{VA} \)
    \item Tensión nominal en el secundario: \( U_{\text{sec}} = 1000 \, \text{V} \)
    \item Factor de potencia: \( \cos \phi = 0.8 \) (inductivo)
\end{itemize}

La corriente en el secundario se calcula como:
\[
I_{\text{sec}} = \frac{S}{U_{\text{sec}}} = \frac{800,000}{1000} = 800 \, \text{A}
\]

\subsubsection*{2. Reflexión de la corriente al primario}
Usamos la relación de transformación \( \text{rt} = \frac{10000}{1000} = 10 \) para reflejar la corriente al lado primario:
\[
I'_2 = \frac{I_{\text{sec}}}{\text{rt}} = \frac{800}{10} = 80 \, \text{A}
\]

\subsubsection*{3. Cálculo de la tensión en el primario}
Aplicamos la segunda ley de Kirchhoff en el circuito equivalente reflejado al lado primario, despreciando la rama en paralelo. La ecuación para la tensión en el primario es:
\[
U_1 = I'_2 \cdot (R_{\text{cc}} + jX_{\text{cc}}) + U'_2
\]

Donde:
\begin{itemize}
    \item \( R_{\text{cc}} = 1.48 \, \Omega \)
    \item \( X_{\text{cc}} = 5.81 \, \Omega \)
    \item \( U'_2 = U_{\text{sec}} \times \text{rt} = 1000 \times 10 = 10,000 \, \text{V} \)
    \item \( I'_2 = 80 \, \text{A} \)
\end{itemize}

Calculamos la caída de tensión en la impedancia serie:
\[
I'_2 \cdot (R_{\text{cc}} + jX_{\text{cc}}) = 80 \times (1.48 + j5.81) = 80 \times 1.48 + j(80 \times 5.81)
\]
\[
= 118.4 + j464.8 \, \text{V}
\]

Convertimos esta expresión a forma polar:
\[
|I'_2 \cdot (R_{\text{cc}} + jX_{\text{cc}})| = \sqrt{118.4^2 + 464.8^2} \approx 478.65 \, \text{V}
\]
\[
\theta = \tan^{-1}\left(\frac{464.8}{118.4}\right) \approx 75.7^\circ
\]

Finalmente, sumamos esta caída de tensión a \( U'_2 \) en forma polar para obtener \( U_1 \):
\[
U_1 \approx 10,000 + 478.65 \angle 75.7^\circ
\]

El resultado de \( U_1 \) en forma polar proporciona la tensión requerida en el primario para mantener la tensión nominal en el secundario bajo carga.


\section{Problema 2}

\subsection{a) Ensayo de vacío}
Simula el circuito equivalente del ensayo de vacío en QUCS y determina la intensidad de vacío $I_0$ y sus componentes de la rama en paralelo.
\begin{figure}[H]
    \centering
    \includegraphics[width=0.75\textwidth]{simulacion_vacio.png} % Ruta de la captura de pantalla de QUCS
    \caption{Simulación del circuito de vacío en QUCS}
\end{figure}

\subsection{b) Ensayo de cortocircuito}
Simula el circuito equivalente del ensayo de cortocircuito en QUCS y verifica el valor de la intensidad de cortocircuito $I_{\text{cc}}$.
\begin{figure}[H]
    \centering
    \includegraphics[width=0.75\textwidth]{simulacion_cortocircuito.png} % Ruta de la captura de pantalla de QUCS
    \caption{Simulación del circuito de cortocircuito en QUCS}
\end{figure}

\section{Problema 3}
Un generador trifásico conectado en estrella con una potencia de 4500 kVA y conectado a una red de 6000 V tiene una reactancia síncrona de $j0.7 \, \Omega/$fase. Calcular el valor de la f.e.m. resultante y la f.m.m. necesaria para una carga con factor de potencia 0.85 inductivo, interpolando en la tabla dada.

\begin{figure}[H]
    \centering
    \includegraphics[width=0.75\textwidth]{p3e1.png} % Ruta de la captura de pantalla de QUC
\end{figure}
    

1. \textbf{Cálculo de la corriente de línea \(I_{\text{L}}\)}

   Dado:
   \begin{itemize}
       \item Potencia aparente del generador: \( S = 4500 \, \text{kVA} = 4500000 \, \text{VA} \)
       \item Tensión de línea: \( U_{\text{L}} = 6000 \, \text{V} \)
       \item Factor de potencia: \( \cos \varphi = 0.85 \)
   \end{itemize}

   La corriente de línea en un sistema trifásico es:
   \[
   I_{\text{L}} = \frac{S}{\sqrt{3} \cdot U_{\text{L}}} = \frac{4500000}{\sqrt{3} \cdot 6000} \approx 433 \, \text{A}
   \]

2. \textbf{Conversión a tensión de fase}

   La tensión de fase en un sistema conectado en estrella se calcula como:
   \[
   U_{\text{fase}} = \frac{U_{\text{L}}}{\sqrt{3}} = \frac{6000}{\sqrt{3}} \approx 3464.1 \, \text{V}
   \]

3. \textbf{Cálculo de la fuerza electromotriz (f.e.m.)}

   La reactancia síncrona del generador es \( X_s = 0.7 \, \Omega \). Con la carga inductiva (\( \cos \varphi = 0.85 \)), descomponemos la corriente en componentes activa y reactiva:
   \[
   I_{\text{act}} = I_{\text{L}} \cos \varphi = 433 \times 0.85 \approx 368.05 \, \text{A}
   \]
   \[
   I_{\text{react}} = I_{\text{L}} \sin \varphi = 433 \sin(\cos^{-1}(0.85)) \approx 233.92 \, \text{A}
   \]

   Calculamos la caída de tensión debido a la reactancia:
   \[
   \Delta V = I_{\text{react}} \cdot X_s = 233.92 \times 0.7 \approx 163.74 \, \text{V}
   \]

   Finalmente, la fuerza electromotriz \( E \) se calcula como:
   \[
   E = \sqrt{U_{\text{fase}}^2 + (\Delta V)^2} = \sqrt{3464.1^2 + 163.74^2} \approx 3471.97 \, \text{V}
   \]

4. \textbf{Interpolación para la fuerza magnetomotriz (f.m.m.)}

   Con \( E \approx 3471.97 \, \text{V} \), interpolamos en la tabla. Los valores más cercanos son:
   \[
   \text{E (V)} = 3005 \rightarrow F = 10000 \quad \text{y} \quad \text{E (V)} = 3580 \rightarrow F = 15000
   \]

   Aplicando la interpolación lineal:
   \[
   F = 10000 + \left( \frac{3471.97 - 3005}{3580 - 3005} \right) \cdot (15000 - 10000)
   \]
   \[
   F \approx 10000 + \frac{466.97}{575} \cdot 5000 \approx 14062 \, \text{Av/polo}
   \]

\begin{itemize}
    \item Fuerza electromotriz \( E \approx 3472 \, \text{V} \)
    \item Fuerza magnetomotriz \( F \approx 14062 \, \text{Av/polo} \)
\end{itemize}

\section{Problema 4}
Calcular el rendimiento de un motor asíncrono trifásico de 4 polos que absorbe una potencia de 5.5 kW y está conectado a una red de 50 Hz, girando a 1470 r.p.m., con pérdidas en el estátor y otros componentes.

\begin{enumerate}
    \item \textbf{Datos:}
    \begin{itemize}
        \item Potencia eléctrica absorbida: $ P_{\text{el}} = 5.5 \, \text{kW} $
        \item Frecuencia de la red: $ f = 50 \, \text{Hz} $
        \item Velocidad del rotor: $ n_{\text{r}} = 1470 \, \text{r.p.m.} $
        \item Número de polos: $ p = 4 $
        \item Pérdidas en el estátor: $ 3\% \text{ de } P_{\text{el}} $
        \item Pérdidas por rozamiento y ventilación: $ P_{\text{rv}} = 180 \, \text{W} $
    \end{itemize}

    \item \textbf{Cálculo de la velocidad síncrona} $ n_s $:
    \[
    n_s = \frac{120 \times f}{p} = \frac{120 \times 50}{4} = 1500 \, \text{r.p.m.}
    \]

    \item \textbf{Cálculo del deslizamiento} $ s $:
    \[
    s = \frac{n_s - n_r}{n_s} = \frac{1500 - 1470}{1500} = 0.02
    \]

    \item \textbf{Pérdidas en el estátor}:
    \[
    P_{\text{estator}} = 0.03 \times P_{\text{el}} = 0.03 \times 5500 \, \text{W} = 165 \, \text{W}
    \]

    \item \textbf{Cálculo de la potencia que llega al rotor} $ P_{\text{rotor}} $:
    \[
    P_{\text{rotor}} = P_{\text{el}} - P_{\text{estator}} = 5500 - 165 = 5335 \, \text{W}
    \]

    \item \textbf{Pérdidas en el cobre del rotor} $ P_{\text{cobre, rotor}} $:
    \[
    P_{\text{cobre, rotor}} = P_{\text{rotor}} \cdot s = 5335 \times 0.02 = 106.7 \, \text{W}
    \]

    \item \textbf{Cálculo de la potencia útil} $ P_{\text{out}} $:
    \[
    P_{\text{out}} = P_{\text{rotor}} - P_{\text{cobre, rotor}} - P_{\text{rv}} 
    \]
    \[
    P_{\text{out}} = 5335 - 106.7 - 180 = 5048.3 \, \text{W}
    \]

    \item \textbf{Cálculo del rendimiento} $ \eta $:
    \[
    \eta = \frac{P_{\text{out}}}{P_{\text{el}}} \times 100 = \frac{5048.3}{5500} \times 100 \approx 91.78\%
    \]
\end{enumerate}



\end{document}
