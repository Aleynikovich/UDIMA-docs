\documentclass[a4paper,12pt]{article}
\usepackage[utf8]{inputenc}
\usepackage[spanish]{babel}
\usepackage{amsmath, amssymb, amsthm}
\usepackage{geometry}
\usepackage{graphicx}
\geometry{margin=1in}
\title{Actividad de Evaluación Continua 3\\Diseño de una instalación en baja tensión}
\author{Alexander Sebastián Kalis}
\date{\today}

\begin{document}

\maketitle
\newpage
\tableofcontents
\newpage

\section{Ejercicio 1: Cálculo de las intensidades circulantes}

Se debe diseñar una instalación eléctrica trifásica en baja tensión de una nave industrial que 
alimenta a tres cargas principales cuyo diagrama unifilar se muestra en la siguiente figura.  

\begin{figure}[h!]
    \centering
    \includegraphics[width=0.8\textwidth]{image.png}
    \caption{Esquema eléctrico de la instalación.}
    \label{fig:esquema-electrico}
\end{figure} 

\subsection{Revisión del enunciado}
Debemos calcular las intensidades circulantes por las líneas asociadas a las cargas indicadas. Los datos proporcionados son:
\begin{itemize}
    \item \textbf{Motor C1:} Potencia 10 CV, $cos\phi = 0.8$, rendimiento $\eta = 0.87$.
    \item \textbf{Motor C2:} Potencia 20 CV, $cos\phi = 0.86$, rendimiento $\eta = 0.89$.
    \item \textbf{Alumbrado C3:} Potencia 18.25 kW, $cos\phi = 0.95$.
\end{itemize}

Además, el enunciado indica que debemos prever un incremento del 25\% en el consumo del motor alimentado por cada línea para las líneas que alimentan a los motores.

\subsection{Fórmula utilizada}
La intensidad de una carga trifásica se calcula con la fórmula:
\[
I = \frac{P}{\sqrt{3} \cdot V \cdot \cos\phi \cdot \eta}
\]
Donde:
\begin{itemize}
    \item $P$: Potencia activa de la carga (kW).
    \item $V$: Tensión de línea (V).
    \item $\cos\phi$: Factor de potencia.
    \item $\eta$: Rendimiento del motor (para motores).
\end{itemize}

Para el caso del alumbrado, al no tener rendimiento (\(\eta\)), se utiliza directamente:
\[
I = \frac{P}{\sqrt{3} \cdot V \cdot \cos\phi}
\]

\subsection{Conversión de potencias}
Para convertir las potencias de los motores de caballos de vapor (CV) a kilovatios (kW), se utiliza la relación:
\[
P_{\text{kW}} = P_{\text{CV}} \cdot 0.7355
\]
\begin{itemize}
    \item Motor C1: $10 \cdot 0.7355 = 7.355 \, \text{kW}$.
    \item Motor C2: $20 \cdot 0.7355 = 14.710 \, \text{kW}$.
    \item Alumbrado C3: Ya está dado en kilovatios, $18.25 \, \text{kW}$.
\end{itemize}


\subsection{Resultados finales}
Las intensidades circulantes para cada carga son:
\begin{itemize}
    \item \textbf{Motor C1:} $I_{1,\text{nominal}} = 13.80 \, \text{A}$, $I_{1,\text{incrementado}} = 17.25 \, \text{A}$.
    \item \textbf{Motor C2:} $I_{2,\text{nominal}} = 27.20 \, \text{A}$, $I_{2,\text{incrementado}} = 34.00 \, \text{A}$.
    \item \textbf{Alumbrado C3:} $I_{3} = 28.00 \, \text{A}$.
\end{itemize}


\section{Ejercicio 2: Dimensionamiento de las secciones de las líneas}

\subsection{Revisión del enunciado}
El objetivo es dimensionar las secciones de las líneas considerando:
\begin{itemize}
    \item La caída de tensión permitida (\( \Delta U_{\text{máx}} = 5\% \, \text{de} \, 380 \, \text{V} = 19 \, \text{V}\)).
    \item La condición adicional del proyectista:
    \[
    0.9 \cdot I_z > I_B
    \]
    \item El cálculo del conductor de protección (\( S_{\text{PE}} \)) en cada línea, siguiendo el criterio:
    \[
    S_{\text{PE}} =
    \begin{cases}
    S_{\text{Fase}}, & \text{si } S_{\text{Fase}} \leq 16 \, \text{mm}^2, \\
    16 \, \text{mm}^2, & \text{si } S_{\text{Fase}} > 16 \, \text{mm}^2 \, \text{y} \, S_{\text{Fase}} \leq 35 \, \text{mm}^2, \\
    0.5 \cdot S_{\text{Fase}}, & \text{si } S_{\text{Fase}} > 35 \, \text{mm}^2.
    \end{cases}
    \]
\end{itemize}

\subsection{Cálculo de las secciones de las líneas}

\subsubsection{Línea Acometida (LA)}
\textbf{Conductor activo:}
\[
I_{\text{LA}} = 79.25 \, \text{A}, \quad S_{\text{LA}} = 16 \, \text{mm}^2 \, \text{(comercial)}.
\]

\textbf{Conductor de protección:}
\[
S_{\text{PE,LA}} = 16 \, \text{mm}^2 \, \text{(por ser igual a \( S_{\text{Fase}} \))}.
\]

\subsubsection{Línea L1}
\textbf{Conductor activo:}
\[
I_{\text{L1}} = 17.25 \, \text{A}, \quad S_{\text{L1}} = 6 \, \text{mm}^2 \, \text{(comercial)}.
\]

\textbf{Conductor de protección:}
\[
S_{\text{PE,L1}} = 6 \, \text{mm}^2 \, \text{(por ser igual a \( S_{\text{Fase}} \))}.
\]

\subsubsection{Línea L2}
\textbf{Conductor activo:}
\[
I_{\text{L2}} = 34.00 \, \text{A}, \quad S_{\text{L2}} = 10 \, \text{mm}^2 \, \text{(comercial)}.
\]

\textbf{Conductor de protección:}
\[
S_{\text{PE,L2}} = 10 \, \text{mm}^2 \, \text{(por ser igual a \( S_{\text{Fase}} \))}.
\]

\subsubsection{Línea LC1}
\textbf{Conductor activo:}
\[
I_{\text{LC1}} = 17.25 \, \text{A}, \quad S_{\text{LC1}} = 2.5 \, \text{mm}^2 \, \text{(comercial)}.
\]

\textbf{Conductor de protección:}
\[
S_{\text{PE,LC1}} = 2.5 \, \text{mm}^2 \, \text{(por ser igual a \( S_{\text{Fase}} \))}.
\]

\subsubsection{Línea LC2}
\textbf{Conductor activo:}
\[
I_{\text{LC2}} = 34.00 \, \text{A}, \quad S_{\text{LC2}} = 6 \, \text{mm}^2 \, \text{(comercial)}.
\]

\textbf{Conductor de protección:}
\[
S_{\text{PE,LC2}} = 6 \, \text{mm}^2 \, \text{(por ser igual a \( S_{\text{Fase}} \))}.
\]

\subsubsection{Línea LC3}
\textbf{Conductor activo:}
\[
I_{\text{LC3}} = 28.00 \, \text{A}, \quad S_{\text{LC3}} = 4 \, \text{mm}^2 \, \text{(comercial)}.
\]

\textbf{Conductor de protección:}
\[
S_{\text{PE,LC3}} = 4 \, \text{mm}^2 \, \text{(por ser igual a \( S_{\text{Fase}} \))}.
\]

\subsection{Resultados finales}
Las secciones finales, incluyendo el conductor de protección, son:
\begin{itemize}
    \item \textbf{Acometida (LA):} $S_{\text{Fase}} = 16 \, \text{mm}^2, \, S_{\text{PE}} = 16 \, \text{mm}^2$.
    \item \textbf{Línea L1:} $S_{\text{Fase}} = 6 \, \text{mm}^2, \, S_{\text{PE}} = 6 \, \text{mm}^2$.
    \item \textbf{Línea L2:} $S_{\text{Fase}} = 10 \, \text{mm}^2, \, S_{\text{PE}} = 10 \, \text{mm}^2$.
    \item \textbf{Línea LC1:} $S_{\text{Fase}} = 2.5 \, \text{mm}^2, \, S_{\text{PE}} = 2.5 \, \text{mm}^2$.
    \item \textbf{Línea LC2:} $S_{\text{Fase}} = 6 \, \text{mm}^2, \, S_{\text{PE}} = 6 \, \text{mm}^2$.
    \item \textbf{Línea LC3:} $S_{\text{Fase}} = 4 \, \text{mm}^2, \, S_{\text{PE}} = 4 \, \text{mm}^2$.
\end{itemize}
\section{Ejercicio 3: Cálculo de intensidades de cortocircuito}

\subsection{Enunciado}
El objetivo es calcular las intensidades de cortocircuito en cada línea de la instalación eléctrica. Se debe determinar:
\begin{itemize}
    \item La intensidad máxima de cortocircuito (\( I_{\text{cc,max}} \)).
    \item La intensidad mínima de cortocircuito (\( I_{\text{cc,min}} \)).
\end{itemize}
Para ello, se considera la impedancia total de cada línea:
\[
Z = R + jX
\]
Donde:
\begin{itemize}
    \item \( R \): Resistencia de la línea, calculada como:
    \[
    R = \frac{\rho \cdot L}{S}
    \]
    \item \( X \): Reactancia de la línea, calculada como:
    \[
    X = \frac{x' \cdot L}{1000}
    \]
\end{itemize}
La intensidad de cortocircuito se calcula como:
\[
I_{\text{cc}} = \frac{V}{Z}
\]

\subsection{Datos iniciales}
\begin{itemize}
    \item Tensión nominal: \( V = 380 \, \text{V} \).
    \item Resistividad del cobre: \( \rho = 0.02198 \, \Omega \cdot \text{mm}^2/\text{m} \).
    \item Reactancia de los cables: \( x' = 0.08 \, \Omega/\text{km} \).
    \item Secciones comerciales obtenidas en el ejercicio anterior:
    \begin{itemize}
        \item Acometida (LA): \( 16 \, \text{mm}^2 \).
        \item Línea L1: \( 6 \, \text{mm}^2 \).
        \item Línea L2: \( 10 \, \text{mm}^2 \).
        \item Línea LC1: \( 2.5 \, \text{mm}^2 \).
        \item Línea LC2: \( 6 \, \text{mm}^2 \).
        \item Línea LC3: \( 4 \, \text{mm}^2 \).
    \end{itemize}
    \item Longitudes de las líneas:
    \begin{itemize}
        \item Acometida (LA): \( 5 \, \text{m} \).
        \item Línea L1: \( 83 \, \text{m} \).
        \item Línea L2: \( 25 \, \text{m} \).
        \item Línea LC1: \( 7 \, \text{m} \).
        \item Línea LC2: \( 7 \, \text{m} \).
        \item Línea LC3: \( 23 \, \text{m} \).
    \end{itemize}
\end{itemize}

\subsection{Cálculos}

\subsubsection{Línea Acometida (LA)}
\[
R_{\text{LA}} = \frac{\rho \cdot L}{S} = \frac{0.02198 \cdot 5}{16} = 0.00687 \, \Omega
\]
\[
X_{\text{LA}} = \frac{x' \cdot L}{1000} = \frac{0.08 \cdot 5}{1000} = 0.0004 \, \Omega
\]
\[
Z_{\text{máx}} = R_{\text{LA}} + X_{\text{LA}} = 0.00687 + 0.0004 = 0.00727 \, \Omega
\]
\[
Z_{\text{mín}} = R_{\text{LA}} = 0.00687 \, \Omega
\]
\[
I_{\text{cc,max}} = \frac{V}{Z_{\text{mín}}} = \frac{380}{0.00687} = 55,323.02 \, \text{A}
\]
\[
I_{\text{cc,min}} = \frac{V}{Z_{\text{máx}}} = \frac{380}{0.00727} = 52,278.59 \, \text{A}
\]

\subsubsection{Línea L1}
\[
R_{\text{L1}} = \frac{\rho \cdot L}{S} = \frac{0.02198 \cdot 83}{6} = 0.3039 \, \Omega
\]
\[
X_{\text{L1}} = \frac{x' \cdot L}{1000} = \frac{0.08 \cdot 83}{1000} = 0.00664 \, \Omega
\]
\[
Z_{\text{máx}} = R_{\text{L1}} + X_{\text{L1}} = 0.3039 + 0.00664 = 0.31054 \, \Omega
\]
\[
Z_{\text{mín}} = R_{\text{L1}} = 0.3039 \, \Omega
\]
\[
I_{\text{cc,max}} = \frac{V}{Z_{\text{mín}}} = \frac{380}{0.3039} = 1,249.77 \, \text{A}
\]
\[
I_{\text{cc,min}} = \frac{V}{Z_{\text{máx}}} = \frac{380}{0.31054} = 1,223.06 \, \text{A}
\]

\subsubsection{Línea L2}
\[
R_{\text{L2}} = \frac{\rho \cdot L}{S} = \frac{0.02198 \cdot 25}{10} = 0.05495 \, \Omega
\]
\[
X_{\text{L2}} = \frac{x' \cdot L}{1000} = \frac{0.08 \cdot 25}{1000} = 0.002 \, \Omega
\]
\[
Z_{\text{máx}} = R_{\text{L2}} + X_{\text{L2}} = 0.05495 + 0.002 = 0.05695 \, \Omega
\]
\[
Z_{\text{mín}} = R_{\text{L2}} = 0.05495 \, \Omega
\]
\[
I_{\text{cc,max}} = \frac{V}{Z_{\text{mín}}} = \frac{380}{0.05495} = 6,915.38 \, \text{A}
\]
\[
I_{\text{cc,min}} = \frac{V}{Z_{\text{máx}}} = \frac{380}{0.05695} = 6,672.52 \, \text{A}
\]

\subsubsection{Línea LC1}
\[
R_{\text{LC1}} = \frac{\rho \cdot L}{S} = \frac{0.02198 \cdot 7}{2.5} = 0.06154 \, \Omega
\]
\[
X_{\text{LC1}} = \frac{x' \cdot L}{1000} = \frac{0.08 \cdot 7}{1000} = 0.00056 \, \Omega
\]
\[
Z_{\text{máx}} = R_{\text{LC1}} + X_{\text{LC1}} = 0.06154 + 0.00056 = 0.0621 \, \Omega
\]
\[
Z_{\text{mín}} = R_{\text{LC1}} = 0.06154 \, \Omega
\]
\[
I_{\text{cc,max}} = \frac{V}{Z_{\text{mín}}} = \frac{380}{0.06154} = 6,174.44 \, \text{A}
\]
\[
I_{\text{cc,min}} = \frac{V}{Z_{\text{máx}}} = \frac{380}{0.0621} = 6,118.77 \, \text{A}
\]

\subsubsection{Línea LC2}
\[
R_{\text{LC2}} = \frac{\rho \cdot L}{S} = \frac{0.02198 \cdot 7}{6} = 0.02564 \, \Omega
\]
\[
X_{\text{LC2}} = \frac{x' \cdot L}{1000} = \frac{0.08 \cdot 7}{1000} = 0.00056 \, \Omega
\]
\[
Z_{\text{máx}} = R_{\text{LC2}} + X_{\text{LC2}} = 0.02564 + 0.00056 = 0.0262 \, \Omega
\]
\[
Z_{\text{mín}} = R_{\text{LC2}} = 0.02564 \, \Omega
\]
\[
I_{\text{cc,max}} = \frac{V}{Z_{\text{mín}}} = \frac{380}{0.02564} = 14,818.67 \, \text{A}
\]
\[
I_{\text{cc,min}} = \frac{V}{Z_{\text{máx}}} = \frac{380}{0.0262} = 14,501.97 \, \text{A}
\]

\subsubsection{Línea LC3}
\[
R_{\text{LC3}} = \frac{\rho \cdot L}{S} = \frac{0.02198 \cdot 23}{4} = 0.12639 \, \Omega
\]
\[
X_{\text{LC3}} = \frac{x' \cdot L}{1000} = \frac{0.08 \cdot 23}{1000} = 0.00184 \, \Omega
\]
\[
Z_{\text{máx}} = R_{\text{LC3}} + X_{\text{LC3}} = 0.12639 + 0.00184 = 0.12823 \, \Omega
\]
\[
Z_{\text{mín}} = R_{\text{LC3}} = 0.12639 \, \Omega
\]
\[
I_{\text{cc,max}} = \frac{V}{Z_{\text{mín}}} = \frac{380}{0.12639} = 3,006.69 \, \text{A}
\]
\[
I_{\text{cc,min}} = \frac{V}{Z_{\text{máx}}} = \frac{380}{0.12823} = 2,963.54 \, \text{A}
\]

\subsection{Resultados finales}
\begin{itemize}
    \item Línea LA: $I_{\text{cc,max}} = 55,323.02 \, \text{A}$, $I_{\text{cc,min}} = 52,278.59 \, \text{A}$.
    \item Línea L1: $I_{\text{cc,max}} = 1,249.77 \, \text{A}$, $I_{\text{cc,min}} = 1,223.06 \, \text{A}$.
    \item Línea L2: $I_{\text{cc,max}} = 6,915.38 \, \text{A}$, $I_{\text{cc,min}} = 6,672.52 \, \text{A}$.
    \item Línea LC1: $I_{\text{cc,max}} = 6,174.44 \, \text{A}$, $I_{\text{cc,min}} = 6,118.77 \, \text{A}$.
    \item Línea LC2: $I_{\text{cc,max}} = 14,818.67 \, \text{A}$, $I_{\text{cc,min}} = 14,501.97 \, \text{A}$.
    \item Línea LC3: $I_{\text{cc,max}} = 3,006.69 \, \text{A}$, $I_{\text{cc,min}} = 2,963.54 \, \text{A}$.
\end{itemize}

\section{Ejercicio 4: Diseño de la protección contra cortocircuitos y sobrecargas}

\subsection{Enunciado}
El diseño de la protección contra cortocircuitos y sobrecargas se realizará mediante:
\begin{itemize}
    \item Interruptores automáticos en las líneas de acometida (LA), L1 y L2.
    \item Interruptores magnetotérmicos en las líneas LC1, LC2 y LC3.
\end{itemize}
Se tendrán en cuenta las siguientes consideraciones:
\begin{itemize}
    \item Tanto en los interruptores automáticos como en los magnetotérmicos, se cumple la condición:
    \[
    I_2 = 1.45 \cdot I_n
    \]
    \item En la protección contra cortocircuitos, no es necesario comprobar la condición \( I_{\text{cc,max}} < I_B \), ya que se cumple siempre para la aparamenta indicada.
\end{itemize}

\subsection{Datos iniciales}
Los modelos disponibles para los dispositivos de protección son los siguientes:
\begin{itemize}
    \item \textbf{Interruptores automáticos (IA):}
    \begin{itemize}
        \item Modelo 3VF5: \( I_a = 1575 - 3150 \, \text{A}, \, I_n = 315 \, \text{A}, \, \text{Poder de corte} = 20 \, \text{kA} \).
        \item Modelo 3VF4: \( I_a = 500 - 1000 \, \text{A}, \, I_n = 90 \, \text{A}, \, \text{Poder de corte} = 18 \, \text{kA} \).
        \item Modelo 3VF3: \( I_a = 500 \, \text{A}, \, I_n = 63 \, \text{A}, \, \text{Poder de corte} = 18 \, \text{kA} \).
    \end{itemize}
    \item \textbf{Interruptores magnetotérmicos:}
    \begin{itemize}
        \item Modelo 5SN3: \( I_a = 5 \cdot I_n, \, I_n = 20 \, \text{A}, \, \text{Poder de corte} = 6 \, \text{kA} \).
        \item Modelo 5SN4: \( I_a = 5 \cdot I_n, \, I_n = 40 \, \text{A}, \, \text{Poder de corte} = 4.5 \, \text{kA} \).
    \end{itemize}
\end{itemize}

\subsection{Selección de dispositivos}

\subsubsection{Línea Acometida (LA)}
\begin{itemize}
    \item \textbf{Intensidad nominal} (\( I_B \)): \( 79.25 \, \text{A} \).
    \item Se selecciona el interruptor automático \textbf{3VF4}, con:
    \[
    I_n = 90 \, \text{A}, \, I_a = 500 - 1000 \, \text{A}, \, \text{Poder de corte} = 18 \, \text{kA}.
    \]
\end{itemize}

\subsubsection{Línea L1}
\begin{itemize}
    \item \textbf{Intensidad nominal} (\( I_B \)): \( 17.25 \, \text{A} \).
    \item Se selecciona el interruptor automático \textbf{3VF3}, con:
    \[
    I_n = 63 \, \text{A}, \, I_a = 500 \, \text{A}, \, \text{Poder de corte} = 18 \, \text{kA}.
    \]
\end{itemize}

\subsubsection{Línea L2}
\begin{itemize}
    \item \textbf{Intensidad nominal} (\( I_B \)): \( 34.00 \, \text{A} \).
    \item Se selecciona el interruptor automático \textbf{3VF3}, con:
    \[
    I_n = 63 \, \text{A}, \, I_a = 500 \, \text{A}, \, \text{Poder de corte} = 18 \, \text{kA}.
    \]
\end{itemize}

\subsubsection{Línea LC1}
\begin{itemize}
    \item \textbf{Intensidad nominal} (\( I_B \)): \( 17.25 \, \text{A} \).
    \item Se selecciona el interruptor magnetotérmico \textbf{5SN3}, con:
    \[
    I_n = 20 \, \text{A}, \, I_a = 5 \cdot I_n = 100 \, \text{A}, \, \text{Poder de corte} = 6 \, \text{kA}.
    \]
\end{itemize}

\subsubsection{Línea LC2}
\begin{itemize}
    \item \textbf{Intensidad nominal} (\( I_B \)): \( 34.00 \, \text{A} \).
    \item Se selecciona el interruptor magnetotérmico \textbf{5SN4}, con:
    \[
    I_n = 40 \, \text{A}, \, I_a = 5 \cdot I_n = 200 \, \text{A}, \, \text{Poder de corte} = 4.5 \, \text{kA}.
    \]
\end{itemize}

\subsubsection{Línea LC3}
\begin{itemize}
    \item \textbf{Intensidad nominal} (\( I_B \)): \( 28.00 \, \text{A} \).
    \item Se selecciona el interruptor magnetotérmico \textbf{5SN4}, con:
    \[
    I_n = 40 \, \text{A}, \, I_a = 5 \cdot I_n = 200 \, \text{A}, \, \text{Poder de corte} = 4.5 \, \text{kA}.
    \]
\end{itemize}



\end{document}
