\documentclass[a4paper,12pt]{article}
\usepackage[utf8]{inputenc}
\usepackage{amsmath, amssymb}
\usepackage{graphicx}
\usepackage{geometry}
\usepackage{float}
\usepackage{caption}
\usepackage[spanish]{babel}
\geometry{left=3cm, right=3cm, top=2.5cm, bottom=2.5cm}
\usepackage{fancyhdr}
\pagestyle{fancy}
\fancyhf{}
\fancyhead[L]{Tecnología Eléctrica}
\fancyhead[R]{Universidad a Distancia de Madrid (UDIMA)}
\fancyfoot[C]{\thepage}

\title{AEC2. Prácticas de Dimensionado de Circuitos Eléctricos}
\author{Alexander Sebastian Kalis}
\date{1 de diciembre de 2024}

\begin{document}



\maketitle

\newpage

\tableofcontents

\newpage 

\section{Problema 1}

Se desea determinar la sección de los conductores eléctricos para alimentar un motor trifásico de una amasadora industrial con las siguientes condiciones: potencia del motor de $10 \, \text{CV}$, factor de potencia $\cos \varphi = 0.85$, tensión de $380 \, \text{V}$, longitud de la línea de $12 \, \text{m}$ y una caída de tensión máxima permitida del $1\%$. Los conductores son de cobre, instalados en tubos de PVC, con una resistencia específica de $\rho_{\text{Cu}} = 0.018 \, \Omega \, \text{mm}^2/\text{m}$.

La potencia del motor, en kilovatios, se convierte como:
\[
P_{\text{kW}} = 10 \, \text{CV} \cdot 0.736 = 7.36 \, \text{kW}.
\]
La intensidad de corriente se determina mediante la fórmula:
\[
I_B = \frac{P_{\text{kW}}}{\sqrt{3} \cdot V \cdot \cos \varphi},
\]
donde $P_{\text{kW}} = 7.36 \, \text{kW}$, $V = 380 \, \text{V}$ y $\cos \varphi = 0.85$. Sustituyendo valores:
\[
I_B = \frac{7.36}{\sqrt{3} \cdot 380 \cdot 0.85} \approx 13.2 \, \text{A}.
\]

Para la caída de tensión, se usa la relación:
\[
\Delta V = \frac{2 \cdot \rho_{\text{Cu}} \cdot L \cdot I_B}{S},
\]
donde $\Delta V_{\text{máx}} = 0.01 \cdot 380 = 3.8 \, \text{V}$. Despejando $S$:
\[
S = \frac{2 \cdot \rho_{\text{Cu}} \cdot L \cdot I_B}{\Delta V}.
\]
Sustituyendo los valores:
\[
S = \frac{2 \cdot 0.018 \cdot 12 \cdot 13.2}{3.8} \approx 1.5 \, \text{mm}^2.
\]

Finalmente, verificamos el criterio del proyectista: $0.9 \cdot I_z > I_B$. Para un conductor de sección $1.5 \, \text{mm}^2$ tripolar empotrado, la intensidad admisible es $I_z \approx 15 \, \text{A}$. Entonces:
\[
0.9 \cdot I_z = 0.9 \cdot 15 = 13.5 \, \text{A}.
\]
Como $13.5 > 13.2$, se cumple el criterio adicional. Por lo tanto, la sección de los conductores requerida es de $1.5 \, \text{mm}^2$.



\section{Problema 2}

Para la instalación de la figura, se realiza el análisis con los siguientes datos:

\begin{itemize}
    \item Potencia absorbida por la carga: $P_{\text{abs}} = 20800 \, \text{W}$.
    \item Tensión nominal: $V = 380 \, \text{V}$.
    \item Factor de potencia: $\cos \varphi = 0.92$.
    \item Longitud de la línea: $L = 45 \, \text{m}$.
    \item Sección de los conductores: $S = 6 \, \text{mm}^2$.
    \item Resistividad del cobre: $\rho_{\text{Cu}} = 0.018 \, \Omega \, \text{mm}^2/\text{m}$.
    \item Impedancia en el punto A: $Z_A = 25 + 22j \, \text{m}\Omega$.
\end{itemize}

\subsection*{a) Verificación de la caída de tensión}

La intensidad de corriente absorbida por la carga se calcula como:
\[
I_B = \frac{P_{\text{abs}}}{\sqrt{3} \cdot V \cdot \cos \varphi}.
\]
Sustituyendo los valores:
\[
I_B = \frac{20800}{\sqrt{3} \cdot 380 \cdot 0.92} \approx 33.14 \, \text{A}.
\]

La caída de tensión se calcula con la fórmula:
\[
\Delta V = \frac{2 \cdot \rho_{\text{Cu}} \cdot L \cdot I_B}{S},
\]
donde $L$ es la longitud total del cableado (ida y vuelta). Sustituyendo los valores:
\[
\Delta V = \frac{2 \cdot 0.018 \cdot 45 \cdot 33.14}{6} \approx 8.93 \, \text{V}.
\]

El porcentaje de caída de tensión respecto a la tensión nominal es:
\[
\text{Caída (\%)} = \frac{\Delta V}{V} \cdot 100 = \frac{8.93}{380} \cdot 100 \approx 2.35\%.
\]

Como el porcentaje de caída excede el límite del $1.5\%$ establecido por la ITC-BT-19 del REBT, es necesario aumentar la sección del conductor.

\subsection*{b) Cálculo de la corriente de cortocircuito}

La impedancia del punto A está dada como $Z_A = 25 + 22j \, \text{m}\Omega$, lo que equivale a:
\[
Z_A = 0.025 + 0.022j \, \Omega.
\]

El módulo de la impedancia es:
\[
|Z_A| = \sqrt{(0.025)^2 + (0.022)^2} \approx 0.0336 \, \Omega.
\]

La corriente máxima de cortocircuito se calcula suponiendo que $\cos \varphi = 1$ (impedancia puramente resistiva):
\[
I_{\text{cc máx}} = \frac{V}{|Z_A|} = \frac{380}{0.0336} \approx 11310 \, \text{A}.
\]

Para la corriente mínima de cortocircuito, consideramos el ángulo de desfase $\theta$ entre la resistencia y la reactancia, calculado como:
\[
\theta = \arctan\left(\frac{\text{Im}(Z_A)}{\text{Re}(Z_A)}\right) = \arctan\left(\frac{0.022}{0.025}\right) \approx 41.19^\circ.
\]

El factor de potencia asociado es:
\[
\cos \theta = \cos(41.19^\circ) \approx 0.754.
\]

La corriente mínima de cortocircuito es:
\[
I_{\text{cc mín}} = I_{\text{cc máx}} \cdot \cos \theta = 11310 \cdot 0.754 \approx 8530 \, \text{A}.
\]

Por lo tanto, las corrientes de cortocircuito en el punto A son:
\[
I_{\text{cc máx}} \approx 11310 \, \text{A}, \quad I_{\text{cc mín}} \approx 8530 \, \text{A}.
\]

\section{Problema 3}

Se solicita determinar las corrientes de cortocircuito en el punto A, basándonos en los siguientes datos:

\begin{itemize}
    \item Potencia nominal del transformador: $S_n = 800 \, \text{kVA}$.
    \item Tensiones: $20 \, \text{kV}/400 \, \text{V}$.
    \item Impedancia relativa del transformador: $\varepsilon_{cc} = 6\%$.
    \item Impedancia relativa de la red: $\varepsilon_{xcc} = 4.5\%$.
    \item Longitud de la línea: $L = 60 \, \text{m}$.
    \item Sección del conductor: $S = 16 \, \text{mm}^2$.
    \item Resistividad del cobre: $\rho_{\text{Cu}} = 0.018 \, \Omega \, \text{mm}^2/\text{m}$.
\end{itemize}

\subsection*{Cálculo de la impedancia equivalente}

\subsubsection*{Impedancia del transformador}
La impedancia relativa del transformador, expresada en ohmios, se calcula utilizando la fórmula:
\[
Z_{\text{cc}} = \varepsilon_{cc} \cdot \frac{V_{\text{sec}}^2}{S_n},
\]
donde $V_{\text{sec}} = 400 \, \text{V}$ es la tensión en el lado secundario y $S_n = 800 \, \text{kVA}$ es la potencia nominal. Sustituyendo valores:
\[
Z_{\text{cc}} = 0.06 \cdot \frac{400^2}{800 \cdot 10^3} = 0.012 \, \Omega.
\]

\subsubsection*{Impedancia de la red}
De manera análoga, para la red:
\[
Z_{\text{xcc}} = \varepsilon_{xcc} \cdot \frac{V_{\text{sec}}^2}{S_n},
\]
con $\varepsilon_{xcc} = 4.5\%$. Sustituyendo valores:
\[
Z_{\text{xcc}} = 0.045 \cdot \frac{400^2}{800 \cdot 10^3} = 0.009 \, \Omega.
\]

\subsubsection*{Impedancia de la línea}
La impedancia de la línea se calcula considerando su resistencia por unidad de longitud:
\[
Z_{\text{línea}} = \frac{2 \cdot \rho_{\text{Cu}} \cdot L}{S},
\]
donde $L = 60 \, \text{m}$ es la longitud y $S = 16 \, \text{mm}^2$ es la sección del conductor. Sustituyendo:
\[
Z_{\text{línea}} = \frac{2 \cdot 0.018 \cdot 60}{16} = 0.135 \, \Omega.
\]

\subsubsection*{Impedancia total}
La impedancia equivalente en el punto A se obtiene como la suma de las impedancias:
\[
Z_{\text{total}} = Z_{\text{cc}} + Z_{\text{xcc}} + Z_{\text{línea}}.
\]
Sustituyendo los valores:
\[
Z_{\text{total}} = 0.012 + 0.009 + 0.135 = 0.156 \, \Omega.
\]

\subsection*{Cálculo de la corriente de cortocircuito}
La corriente de cortocircuito en el punto A se determina como:
\[
I_{\text{cc}} = \frac{V_{\text{sec}}}{Z_{\text{total}}},
\]
donde $V_{\text{sec}} = 400 \, \text{V}$. Sustituyendo:
\[
I_{\text{cc}} = \frac{400}{0.156} \approx 2564 \, \text{A}.
\]

Por lo tanto, la corriente de cortocircuito en el punto A es de aproximadamente $2564 \, \text{A}$.


\end{document}
