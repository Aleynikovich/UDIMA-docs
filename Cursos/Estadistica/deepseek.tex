\documentclass{article}
\usepackage[utf8]{inputenc}
\usepackage{amsmath}
\usepackage{amssymb}
\usepackage{enumerate}

\title{Guía de Estudio para el Examen de Estadística y Probabilidad}
\author{Tu Nombre}
\date{\today}

\begin{document}

\maketitle

\section*{Introducción}
Este documento está diseñado para ayudarte a prepararte para el examen de Estadística y Probabilidad. Incluye explicaciones sencillas, conceptos clave y ejercicios resueltos paso a paso.

\section*{Conceptos Clave}

\subsection*{1. Probabilidad Básica}
La probabilidad es una medida de qué tan probable es que algo ocurra. Se expresa como un número entre 0 y 1, donde:
\begin{itemize}
    \item 0 significa que el evento \textbf{no ocurrirá}.
    \item 1 significa que el evento \textbf{seguro ocurrirá}.
    \item Un valor entre 0 y 1 indica la \textbf{posibilidad} de que el evento ocurra.
\end{itemize}

La probabilidad de un evento \( A \) se calcula como:
\[
P(A) = \frac{\text{Número de resultados favorables}}{\text{Número total de resultados posibles}}
\]
\begin{itemize}
    \item \( P(A) \): Probabilidad del evento \( A \).
    \item \textbf{Resultados favorables}: Son los resultados que nos interesan.
    \item \textbf{Resultados posibles}: Son todos los resultados que pueden ocurrir.
\end{itemize}

\subsection*{2. Probabilidad Condicional}
La probabilidad condicional es la probabilidad de que ocurra un evento \( A \), dado que otro evento \( B \) ya ha ocurrido. Se denota como \( P(A|B) \) y se calcula como:
\[
P(A|B) = \frac{P(A \cap B)}{P(B)}
\]
\begin{itemize}
    \item \( P(A|B) \): Probabilidad de \( A \) dado \( B \).
    \item \( P(A \cap B) \): Probabilidad de que ocurran tanto \( A \) como \( B \).
    \item \( P(B) \): Probabilidad de que ocurra \( B \).
\end{itemize}

\subsection*{3. Teorema de la Probabilidad Total}
Este teorema se usa cuando un evento puede ocurrir de varias maneras diferentes. Si \( B_1, B_2, \dots, B_n \) son eventos que cubren todas las posibilidades (es decir, no se solapan y uno de ellos siempre ocurre), entonces la probabilidad de un evento \( A \) se calcula como:
\[
P(A) = P(A|B_1)P(B_1) + P(A|B_2)P(B_2) + \dots + P(A|B_n)P(B_n)
\]
\begin{itemize}
    \item \( P(A|B_i) \): Probabilidad de \( A \) dado que \( B_i \) ha ocurrido.
    \item \( P(B_i) \): Probabilidad de que ocurra \( B_i \).
\end{itemize}

\subsection*{4. Independencia de Sucesos}
Dos eventos \( A \) y \( B \) son independientes si el hecho de que ocurra uno no afecta la probabilidad de que ocurra el otro. Matemáticamente, esto se expresa como:
\[
P(A \cap B) = P(A) \cdot P(B)
\]
\begin{itemize}
    \item \( P(A \cap B) \): Probabilidad de que ocurran tanto \( A \) como \( B \).
    \item \( P(A) \) y \( P(B) \): Probabilidades individuales de \( A \) y \( B \).
\end{itemize}

\section*{Ejercicios Resueltos}

\subsection*{Ejercicio 1: Probabilidad de Eventos}
\textbf{Enunciado:} Entre los deportistas profesionales, el 50\% disfrutan de una beca de alto rendimiento y el 30\% está cursando estudios superiores. Se sabe también que el 10\% de los deportistas profesionales disfrutan de una beca de alto rendimiento y además están cursando estudios superiores. Seleccionado un deportista profesional al azar, calcule la probabilidad de que:
\begin{enumerate}
    \item Disfrute de una beca de alto rendimiento o esté cursando estudios superiores.
    \item No disfrute de una beca de alto rendimiento, sabiendo que no está cursando estudios superiores.
\end{enumerate}

\textbf{Solución:}
\begin{enumerate}
    \item Usamos la fórmula de la unión de dos eventos:
    \[
    P(A \cup B) = P(A) + P(B) - P(A \cap B)
    \]
    Donde:
    \begin{itemize}
        \item \( P(A) = 0.5 \): Probabilidad de disfrutar una beca.
        \item \( P(B) = 0.3 \): Probabilidad de cursar estudios superiores.
        \item \( P(A \cap B) = 0.1 \): Probabilidad de disfrutar una beca y cursar estudios superiores.
    \end{itemize}
    Entonces:
    \[
    P(A \cup B) = 0.5 + 0.3 - 0.1 = 0.7
    \]
    \item Usamos la probabilidad condicional:
    \[
    P(\overline{A}| \overline{B}) = \frac{P(\overline{A} \cap \overline{B})}{P(\overline{B})}
    \]
    Donde:
    \begin{itemize}
        \item \( P(\overline{B}) = 1 - P(B) = 0.7 \): Probabilidad de no cursar estudios superiores.
        \item \( P(\overline{A} \cap \overline{B}) = 1 - P(A \cup B) = 1 - 0.7 = 0.3 \): Probabilidad de no disfrutar una beca y no cursar estudios superiores.
    \end{itemize}
    Finalmente:
    \[
    P(\overline{A}| \overline{B}) = \frac{0.3}{0.7} \approx 0.4286
    \]
\end{enumerate}

\subsection*{Ejercicio 2: Probabilidad con Urnas}
\textbf{Enunciado:} Una urna contiene 7 bolas blancas y 12 bolas negras. Se extrae al azar una bola de la urna y se sustituye por dos del otro color. A continuación, se extrae una segunda bola de la urna. Se pide:
\begin{enumerate}
    \item Calcular la probabilidad de que la segunda bola extraída sea blanca.
    \item Calcular la probabilidad de que la segunda bola extraída sea de distinto color que la primera.
    \item Calcular la probabilidad de que la primera bola extraída haya sido negra, sabiendo que la segunda bola fue blanca.
\end{enumerate}

\textbf{Solución:}
\begin{enumerate}
    \item Usamos el teorema de la probabilidad total. Hay dos casos:
    \begin{itemize}
        \item La primera bola es blanca (probabilidad \( \frac{7}{19} \)), y se sustituye por dos negras. Entonces, la urna tendrá 6 blancas y 14 negras.
        \item La primera bola es negra (probabilidad \( \frac{12}{19} \)), y se sustituye por dos blancas. Entonces, la urna tendrá 9 blancas y 11 negras.
    \end{itemize}
    La probabilidad de que la segunda bola sea blanca es:
    \[
    P(\text{Blanca}) = \frac{7}{19} \cdot \frac{6}{20} + \frac{12}{19} \cdot \frac{9}{20} = \frac{42}{380} + \frac{108}{380} = \frac{150}{380} \approx 0.3947
    \]
    \item La probabilidad de que la segunda bola sea de distinto color que la primera es:
    \[
    P(\text{Distinto color}) = P(\text{Blanca primero}) \cdot P(\text{Negra segundo}) + P(\text{Negra primero}) \cdot P(\text{Blanca segundo})
    \]
    \[
    P(\text{Distinto color}) = \frac{7}{19} \cdot \frac{14}{20} + \frac{12}{19} \cdot \frac{9}{20} = \frac{98}{380} + \frac{108}{380} = \frac{206}{380} \approx 0.5421
    \]
    \item Usamos la probabilidad condicional:
    \[
    P(\text{Negra primero} | \text{Blanca segundo}) = \frac{P(\text{Negra primero} \cap \text{Blanca segundo})}{P(\text{Blanca segundo})}
    \]
    Ya calculamos \( P(\text{Blanca segundo}) \approx 0.3947 \).
    La probabilidad de que la primera bola sea negra y la segunda blanca es:
    \[
    P(\text{Negra primero} \cap \text{Blanca segundo}) = \frac{12}{19} \cdot \frac{9}{20} = \frac{108}{380} \approx 0.2842
    \]
    Entonces:
    \[
    P(\text{Negra primero} | \text{Blanca segundo}) = \frac{0.2842}{0.3947} \approx 0.72
    \]
\end{enumerate}

\section*{Conclusión}
Este documento cubre los conceptos básicos y ejercicios resueltos que te ayudarán a prepararte para el examen. Recuerda practicar con más ejercicios similares para afianzar tus conocimientos.

\section*{Unidades 4, 5 y 6: Variables Aleatorias y Distribuciones}

En estas unidades trabajaremos con **variables aleatorias** y sus **distribuciones de probabilidad**. Vamos a explicar los conceptos básicos y cómo resolver los ejercicios paso a paso.

\subsection*{1. Variables Aleatorias}
Una **variable aleatoria** es una función que asigna un valor numérico a cada resultado posible de un experimento aleatorio. Hay dos tipos:
\begin{itemize}
    \item \textbf{Variable aleatoria discreta}: Toma valores enteros (por ejemplo, el número de clientes en una tienda).
    \item \textbf{Variable aleatoria continua}: Toma valores en un intervalo (por ejemplo, el tiempo que tarda un coche en aparcar).
\end{itemize}

\subsection*{2. Función de Densidad y Función de Distribución}
\begin{itemize}
    \item \textbf{Función de densidad} (para variables continuas): Es una función \( f(x) \) que describe la probabilidad de que la variable aleatoria \( X \) tome un valor cercano a \( x \). La probabilidad se calcula como el área bajo la curva de \( f(x) \).
    \item \textbf{Función de distribución}: Es una función \( F(x) \) que da la probabilidad de que \( X \) sea menor o igual a \( x \).
\end{itemize}

\subsection*{3. Media y Desviación Típica}
\begin{itemize}
    \item \textbf{Media} (\( \mu \)): Es el valor promedio de la variable aleatoria.
    \item \textbf{Desviación típica} (\( \sigma \)): Mide cuánto se dispersan los valores de la variable aleatoria respecto a la media.
\end{itemize}

\subsection*{4. Distribuciones Importantes}
\begin{itemize}
    \item \textbf{Distribución Uniforme}: Todos los valores tienen la misma probabilidad.
    \item \textbf{Distribución Exponencial}: Describe el tiempo entre eventos en un proceso de Poisson.
    \item \textbf{Distribución Normal}: Es la distribución más común en estadística, con forma de campana.
    \item \textbf{Distribución Poisson}: Describe el número de eventos que ocurren en un intervalo de tiempo.
\end{itemize}

\section*{Ejercicios Resueltos}

\subsection*{Problema 1: Función de Densidad}
\textbf{Enunciado:} Una variable aleatoria \( X \) tiene la función de densidad:
\[
f(x) = 
\begin{cases}
c(1 - x^2) & \text{si } 0 \leq x \leq 1 \\
0 & \text{en otro caso}
\end{cases}
\]
Se pide:
\begin{enumerate}
    \item Calcular \( c \) para que \( f(x) \) sea una función de densidad.
    \item Calcular la media, la desviación típica y \( P(0.5 \leq X \leq 1) \).
\end{enumerate}

\textbf{Solución:}
\begin{enumerate}
    \item Para que \( f(x) \) sea una función de densidad, el área bajo la curva debe ser 1. Entonces:
    \[
    \int_{0}^{1} c(1 - x^2) \, dx = 1
    \]
    Resolviendo la integral:
    \[
    c \int_{0}^{1} (1 - x^2) \, dx = c \left[ x - \frac{x^3}{3} \right]_0^1 = c \left(1 - \frac{1}{3}\right) = \frac{2c}{3}
    \]
    Igualamos a 1:
    \[
    \frac{2c}{3} = 1 \implies c = \frac{3}{2}
    \]
    \item \textbf{Media (\( \mu \))}:
    \[
    \mu = \int_{0}^{1} x \cdot f(x) \, dx = \int_{0}^{1} x \cdot \frac{3}{2}(1 - x^2) \, dx = \frac{3}{2} \int_{0}^{1} (x - x^3) \, dx
    \]
    Resolviendo:
    \[
    \mu = \frac{3}{2} \left[ \frac{x^2}{2} - \frac{x^4}{4} \right]_0^1 = \frac{3}{2} \left( \frac{1}{2} - \frac{1}{4} \right) = \frac{3}{2} \cdot \frac{1}{4} = \frac{3}{8}
    \]
    \textbf{Desviación típica (\( \sigma \))}:
    Primero calculamos \( E(X^2) \):
    \[
    E(X^2) = \int_{0}^{1} x^2 \cdot f(x) \, dx = \int_{0}^{1} x^2 \cdot \frac{3}{2}(1 - x^2) \, dx = \frac{3}{2} \int_{0}^{1} (x^2 - x^4) \, dx
    \]
    Resolviendo:
    \[
    E(X^2) = \frac{3}{2} \left[ \frac{x^3}{3} - \frac{x^5}{5} \right]_0^1 = \frac{3}{2} \left( \frac{1}{3} - \frac{1}{5} \right) = \frac{3}{2} \cdot \frac{2}{15} = \frac{1}{5}
    \]
    Luego, la varianza es:
    \[
    \sigma^2 = E(X^2) - \mu^2 = \frac{1}{5} - \left( \frac{3}{8} \right)^2 = \frac{1}{5} - \frac{9}{64} = \frac{64 - 45}{320} = \frac{19}{320}
    \]
    Finalmente, la desviación típica es:
    \[
    \sigma = \sqrt{\frac{19}{320}} \approx 0.24
    \]
    \textbf{Probabilidad \( P(0.5 \leq X \leq 1) \)}:
    \[
    P(0.5 \leq X \leq 1) = \int_{0.5}^{1} f(x) \, dx = \int_{0.5}^{1} \frac{3}{2}(1 - x^2) \, dx
    \]
    Resolviendo:
    \[
    P(0.5 \leq X \leq 1) = \frac{3}{2} \left[ x - \frac{x^3}{3} \right]_{0.5}^1 = \frac{3}{2} \left( \left(1 - \frac{1}{3}\right) - \left(0.5 - \frac{0.125}{3}\right) \right) = \frac{3}{2} \left( \frac{2}{3} - 0.4583 \right) \approx 0.3125
    \]
\end{enumerate}

\subsection*{Problema 2: Distribución Exponencial}
\textbf{Enunciado:} El tiempo de estacionamiento en la zona azul de Madrid sigue una distribución exponencial con media 0.5 horas. Se pide:
\begin{enumerate}
    \item Calcular la probabilidad de que un estacionamiento supere las 3 horas.
    \item Calcular la probabilidad de que un estacionamiento sea inferior a 30 minutos.
\end{enumerate}

\textbf{Solución:}
\begin{enumerate}
    \item La distribución exponencial tiene la función de densidad:
    \[
    f(x) = \lambda e^{-\lambda x}
    \]
    Donde \( \lambda = \frac{1}{\text{media}} = \frac{1}{0.5} = 2 \).
    La probabilidad de que \( X > 3 \) es:
    \[
    P(X > 3) = \int_{3}^{\infty} 2 e^{-2x} \, dx = e^{-6} \approx 0.0025
    \]
    \item La probabilidad de que \( X < 0.5 \) horas (30 minutos) es:
    \[
    P(X < 0.5) = \int_{0}^{0.5} 2 e^{-2x} \, dx = 1 - e^{-1} \approx 0.6321
    \]
\end{enumerate}

\section*{Unidades 7, 8, 9 y 10: Inferencia Estadística}

En estas unidades trabajaremos con **inferencia estadística**, que es el proceso de usar datos de una muestra para hacer conclusiones sobre una población. Vamos a explicar los conceptos básicos y cómo resolver los ejercicios paso a paso.

\subsection*{1. Intervalos de Confianza}
Un **intervalo de confianza** es un rango de valores que probablemente contiene el valor verdadero de un parámetro de la población (como la media o la varianza). Se calcula usando datos de una muestra y un nivel de confianza (por ejemplo, 95\%).

\subsection*{2. Pruebas de Hipótesis}
Una **prueba de hipótesis** es un procedimiento para decidir si una afirmación sobre una población es válida, basándose en datos de una muestra. Se comparan dos hipótesis:
\begin{itemize}
    \item \textbf{Hipótesis nula} (\( H_0 \)): Es la afirmación que se quiere probar (por ejemplo, "no hay diferencia").
    \item \textbf{Hipótesis alternativa} (\( H_1 \)): Es la afirmación contraria a \( H_0 \) (por ejemplo, "hay diferencia").
\end{itemize}

\subsection*{3. Distribuciones Importantes}
\begin{itemize}
    \item \textbf{Distribución t de Student}: Se usa cuando la muestra es pequeña y la desviación típica de la población es desconocida.
    \item \textbf{Distribución Chi-cuadrado (\( \chi^2 \))}: Se usa para intervalos de confianza de la varianza.
    \item \textbf{Distribución F}: Se usa para comparar varianzas de dos poblaciones.
\end{itemize}

\section*{Ejercicios Resueltos}

\subsection*{Problema 1: Intervalo de Confianza para la Varianza}
\textbf{Enunciado:} Se realizan 10 mediciones de radiactividad en el agua de una fuente natural, obteniendo una media de 12 y una varianza muestral de 34. Calcular un intervalo de confianza para la varianza al 90\%.

\textbf{Solución:}
\begin{enumerate}
    \item \textbf{Datos:}
    \begin{itemize}
        \item Tamaño de la muestra (\( n \)): 10
        \item Varianza muestral (\( s^2 \)): 34
        \item Nivel de confianza: 90\%
    \end{itemize}
    \item \textbf{Distribución Chi-cuadrado:} Para un intervalo de confianza de la varianza, usamos la distribución \( \chi^2 \).
    \item \textbf{Grados de libertad (\( \nu \)):} \( \nu = n - 1 = 9 \).
    \item \textbf{Valores críticos de \( \chi^2 \):} Para un nivel de confianza del 90\%, los valores críticos son \( \chi^2_{0.05} = 3.325 \) y \( \chi^2_{0.95} = 16.919 \).
    \item \textbf{Intervalo de confianza:}
    \[
    \left( \frac{(n-1)s^2}{\chi^2_{0.95}}, \frac{(n-1)s^2}{\chi^2_{0.05}} \right) = \left( \frac{9 \cdot 34}{16.919}, \frac{9 \cdot 34}{3.325} \right) = (18.12, 92.03)
    \]
    \item \textbf{Conclusión:} Con un 90\% de confianza, la varianza de la radiactividad en el agua está entre 18.12 y 92.03.
\end{enumerate}

\subsection*{Problema 2: Intervalo de Confianza para la Diferencia de Medias}
\textbf{Enunciado:} Se tienen dos muestras con los siguientes datos:
\begin{itemize}
    \item Muestra 1: \( \bar{x}_1 = 3565 \), \( s_1 = 150 \), \( n_1 = 25 \)
    \item Muestra 2: \( \bar{x}_2 = 3280 \), \( s_2 = 170 \), \( n_2 = 12 \)
\end{itemize}
Calcular el intervalo de confianza al 95\% de la diferencia de medias.

\textbf{Solución:}
\begin{enumerate}
    \item \textbf{Datos:}
    \begin{itemize}
        \item Medias muestrales: \( \bar{x}_1 = 3565 \), \( \bar{x}_2 = 3280 \)
        \item Desviaciones típicas: \( s_1 = 150 \), \( s_2 = 170 \)
        \item Tamaños de muestra: \( n_1 = 25 \), \( n_2 = 12 \)
        \item Nivel de confianza: 95\%
    \end{itemize}
    \item \textbf{Distribución t de Student:} Como las muestras son pequeñas y las varianzas son desconocidas, usamos la distribución \( t \).
    \item \textbf{Grados de libertad (\( \nu \)):} Usamos la fórmula de Welch:
    \[
    \nu = \frac{ \left( \frac{s_1^2}{n_1} + \frac{s_2^2}{n_2} \right)^2 }{ \frac{(s_1^2/n_1)^2}{n_1-1} + \frac{(s_2^2/n_2)^2}{n_2-1} } \approx 15.5 \approx 16
    \]
    \item \textbf{Valor crítico de \( t \):} Para un nivel de confianza del 95\% y \( \nu = 16 \), el valor crítico es \( t_{0.025} = 2.120 \).
    \item \textbf{Intervalo de confianza:}
    \[
    (\bar{x}_1 - \bar{x}_2) \pm t_{0.025} \cdot \sqrt{ \frac{s_1^2}{n_1} + \frac{s_2^2}{n_2} }
    \]
    Calculamos:
    \[
    3565 - 3280 = 285
    \]
    \[
    \sqrt{ \frac{150^2}{25} + \frac{170^2}{12} } = \sqrt{900 + 2408.33} = \sqrt{3308.33} \approx 57.52
    \]
    Entonces:
    \[
    285 \pm 2.120 \cdot 57.52 = 285 \pm 122.0
    \]
    \item \textbf{Conclusión:} Con un 95\% de confianza, la diferencia de medias está entre 163 y 407.
\end{enumerate}

\subsection*{Problema 3: Prueba de Hipótesis para la Igualdad de Varianzas}
\textbf{Enunciado:} Para comprobar si las cotizaciones de dos tipos de renta fija (A y B) presentan la misma dispersión, se obtienen dos muestras aleatorias e independientes de 17 días de cotización cada una. Las cotizaciones al cierre de A presentaron una varianza de 125.25 y de B una varianza de 638.5. ¿Podemos decir que ambos tipos de renta fija presentan la misma estabilidad en su cotización al 10\% de significación?

\textbf{Solución:}
\begin{enumerate}
    \item \textbf{Hipótesis:}
    \begin{itemize}
        \item \( H_0 \): \( \sigma_A^2 = \sigma_B^2 \) (las varianzas son iguales).
        \item \( H_1 \): \( \sigma_A^2 \neq \sigma_B^2 \) (las varianzas son diferentes).
    \end{itemize}
    \item \textbf{Distribución F:} Usamos la distribución \( F \) para comparar varianzas.
    \item \textbf{Estadístico de prueba:}
    \[
    F = \frac{s_A^2}{s_B^2} = \frac{125.25}{638.5} \approx 0.196
    \]
    \item \textbf{Valores críticos de \( F \):} Para un nivel de significación del 10\% y grados de libertad \( \nu_1 = 16 \), \( \nu_2 = 16 \), los valores críticos son \( F_{0.05} = 0.46 \) y \( F_{0.95} = 2.17 \).
    \item \textbf{Decisión:} Como \( F = 0.196 \) está fuera del intervalo \( [0.46, 2.17] \), rechazamos \( H_0 \).
    \item \textbf{Conclusión:} Con un 10\% de significación, podemos decir que las varianzas de las cotizaciones de A y B son diferentes.
\end{enumerate}

\end{document}