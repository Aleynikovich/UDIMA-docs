\documentclass[a4paper,12pt]{article}
\usepackage[spanish]{babel}
\usepackage{amsmath}
\usepackage{amssymb}
\usepackage{geometry}
\usepackage{fancyhdr}
\usepackage{graphicx}

% Configuración de márgenes
\geometry{top=2.5cm, bottom=2.5cm, left=3cm, right=3cm}

% Configuración de encabezado y pie de página
\pagestyle{fancy}
\fancyhf{}
\fancyhead[L]{Universidad a Distancia de Madrid (UDIMA)}
\fancyhead[R]{Grado en Ingeniería Industrial}
\fancyfoot[C]{\thepage}

% Título y datos
\title{Actividad de Evaluación Continua de las Unidades 1, 2 y 3\\
\large Asignatura: Estadística y Probabilidad}
\author{Alexander Sebastian Kalis\\Grado en Ingeniería Industrial}
\date{\today}

\begin{document}

% Portada
\maketitle
\begin{center}
    \includegraphics[width=0.4\textwidth]{logo.png} % Puedes insertar el logo de la universidad aquí, si lo tienes.
\end{center}
\vfill
\begin{center}
    \textbf{Profesora:} Profa. Dra. Vanessa Fernández Chamorro
\end{center}
\newpage

% Índice
\tableofcontents

\newpage

% Ejercicio 1
\section{Ejercicio 1 (2 Puntos)}
Entre los deportistas profesionales, el 50\% disfruta de una beca de alto rendimiento y el 30\% está cursando estudios superiores. Se sabe que el 10\% de los deportistas disfruta de una beca y está cursando estudios superiores. Seleccionado un deportista al azar, calcule:
\begin{enumerate}
    \item La probabilidad de que disfrute de una beca de alto rendimiento o esté cursando estudios superiores.
    \item La probabilidad de que no disfrute de una beca, sabiendo que no está cursando estudios superiores.
\end{enumerate}

\textbf{Solución:} \\
% Escribe aquí la solución al ejercicio 1. Puedes usar las fórmulas necesarias en formato LaTeX.
Definimos los sucesos siguientes:
- Sea $A$ el suceso ``el deportista disfruta de una beca de alto rendimiento''.
- Sea $B$ el suceso ``el deportista está cursando estudios superiores''.

Se nos dan las probabilidades:
\begin{itemize}
    \item $P(A) = 0.50$
    \item $P(B) = 0.30$
    \item $P(A \cap B) = 0.10$
\end{itemize}

\subsection*{1. Probabilidad de que disfrute de una beca de alto rendimiento o esté cursando estudios superiores}

Queremos encontrar la probabilidad de que el deportista disfrute de una beca de alto rendimiento o esté cursando estudios superiores, es decir, $P(A \cup B)$.

Utilizamos la fórmula de la probabilidad de la unión de dos sucesos:
\[
P(A \cup B) = P(A) + P(B) - P(A \cap B)
\]

Sustituyendo los valores conocidos:
\[
P(A \cup B) = 0.50 + 0.30 - 0.10 = 0.70
\]

Por lo tanto, la probabilidad de que el deportista disfrute de una beca de alto rendimiento o esté cursando estudios superiores es:
\[
\boxed{P(A \cup B) = 0.70}
\]

\subsection*{2. Probabilidad de que no disfrute de una beca de alto rendimiento, sabiendo que no está cursando estudios superiores}

Queremos encontrar $P(A^c | B^c)$, es decir, la probabilidad de que el deportista no disfrute de una beca de alto rendimiento dado que no está cursando estudios superiores.

Usamos la fórmula de la probabilidad condicionada:
\[
P(A^c | B^c) = \frac{P(A^c \cap B^c)}{P(B^c)}
\]

Primero calculamos $P(B^c)$ y $P(A^c \cap B^c)$.

\begin{itemize}
    \item $P(B^c) = 1 - P(B) = 1 - 0.30 = 0.70$
    \item $P(A^c \cap B^c) = 1 - P(A \cup B) = 1 - 0.70 = 0.30$
\end{itemize}

Sustituyendo en la fórmula:
\[
P(A^c | B^c) = \frac{0.30}{0.70} = 0.4286
\]

Por lo tanto, la probabilidad de que el deportista no disfrute de una beca de alto rendimiento, sabiendo que no está cursando estudios superiores, es:
\[
\boxed{P(A^c | B^c) = 0.4286}
\]

\newpage

% Ejercicio 2
\section{Ejercicio 2 (2 Puntos)}
Una urna contiene 7 bolas blancas y 12 bolas negras. Se extrae una bola al azar y se sustituye por dos del otro color. A continuación, se extrae una segunda bola. Calcule:
\begin{enumerate}
    \item La probabilidad de que la segunda bola extraída sea blanca.
    \item La probabilidad de que la segunda bola extraída sea de distinto color que la primera.
    \item La probabilidad de que la primera bola extraída haya sido negra, sabiendo que la segunda bola fue blanca.
\end{enumerate}

\textbf{Solución:}

Definimos los sucesos de la siguiente manera:
- Sea $B_1$ el suceso "la primera bola extraída es blanca".
- Sea $N_1$ el suceso "la primera bola extraída es negra".
- Sea $B_2$ el suceso "la segunda bola extraída es blanca".
- Sea $N_2$ el suceso "la segunda bola extraída es negra".

Dado que inicialmente hay 7 bolas blancas y 12 negras, tenemos en total 19 bolas en la urna.

\subsection*{1. Probabilidad de que la segunda bola extraída sea blanca}

Para calcular esta probabilidad, analizamos dos casos según el color de la primera bola extraída, ya que cambia la composición de la urna:

\begin{itemize}
    \item \textbf{Caso 1: La primera bola extraída es blanca ($B_1$)} \\
    Si la primera bola es blanca, queda una bola blanca menos, y añadimos 2 negras, lo que da una nueva composición de 6 blancas y 14 negras (20 en total). La probabilidad de extraer una bola blanca en la segunda extracción en este caso es:
    \[
    P(B_2 | B_1) = \frac{6}{20} = 0.30
    \]

    \item \textbf{Caso 2: La primera bola extraída es negra ($N_1$)} \\
    Si la primera bola es negra, queda una bola negra menos, y añadimos 2 blancas, lo que da una nueva composición de 9 blancas y 11 negras (20 en total). La probabilidad de extraer una bola blanca en la segunda extracción en este caso es:
    \[
    P(B_2 | N_1) = \frac{9}{20} = 0.45
    \]
\end{itemize}

La probabilidad total de que la segunda bola extraída sea blanca es entonces:
\[
P(B_2) = P(B_2 | B_1) \cdot P(B_1) + P(B_2 | N_1) \cdot P(N_1)
\]
donde $P(B_1) = \frac{7}{19}$ y $P(N_1) = \frac{12}{19}$. Sustituyendo valores:
\[
P(B_2) = \left(0.30 \cdot \frac{7}{19}\right) + \left(0.45 \cdot \frac{12}{19}\right)
\]
\[
P(B_2) = \frac{2.1}{19} + \frac{5.4}{19} = \frac{7.5}{19} \approx 0.3947
\]

Por lo tanto, la probabilidad de que la segunda bola extraída sea blanca es:
\[
\boxed{P(B_2) \approx 0.3947}
\]

\subsection*{2. Probabilidad de que la segunda bola extraída sea de distinto color que la primera}

Examinamos ambos casos posibles:

\begin{itemize}
    \item \textbf{Caso 1: La primera bola es blanca ($B_1$)} \\
    Si la primera bola es blanca, la probabilidad de que la segunda bola sea negra es $P(N_2 | B_1) = \frac{14}{20} = 0.70$.

    \item \textbf{Caso 2: La primera bola es negra ($N_1$)} \\
    Si la primera bola es negra, la probabilidad de que la segunda bola sea blanca es $P(B_2 | N_1) = 0.45$ (ya calculado anteriormente).
\end{itemize}

Entonces, la probabilidad de que las bolas sean de distinto color es:
\[
P(\text{distinto color}) = P(N_2 | B_1) \cdot P(B_1) + P(B_2 | N_1) \cdot P(N_1)
\]
Sustituyendo los valores:
\[
P(\text{distinto color}) = \left(0.70 \cdot \frac{7}{19}\right) + \left(0.45 \cdot \frac{12}{19}\right)
\]
\[
P(\text{distinto color}) = \frac{4.9}{19} + \frac{5.4}{19} = \frac{10.3}{19} \approx 0.5421
\]

Por lo tanto, la probabilidad de que la segunda bola extraída sea de distinto color que la primera es:
\[
\boxed{P(\text{distinto color}) \approx 0.5421}
\]

\subsection*{3. Probabilidad de que la primera bola extraída haya sido negra, sabiendo que la segunda bola fue blanca}

Queremos calcular $P(N_1 | B_2)$. Usamos la fórmula de la probabilidad condicionada:
\[
P(N_1 | B_2) = \frac{P(B_2 | N_1) \cdot P(N_1)}{P(B_2)}
\]
Sustituyendo los valores conocidos:
\[
P(N_1 | B_2) = \frac{0.45 \cdot \frac{12}{19}}{0.3947}
\]
\[
P(N_1 | B_2) \approx \frac{5.4/19}{0.3947} \approx 0.7301
\]

Por lo tanto, la probabilidad de que la primera bola extraída haya sido negra, sabiendo que la segunda bola fue blanca, es:
\[
\boxed{P(N_1 | B_2) \approx 0.7301}
\]


\newpage

% Ejercicio 3
\section{Ejercicio 3 (2 Puntos)}
En un mercado agropecuario, el 70\% de las verduras son de proximidad y el resto no. El 30\% de las de proximidad son ecológicas, y solo el 10\% de las no de proximidad lo son. Calcule:
\begin{enumerate}
    \item La probabilidad de que la verdura comprada no sea ecológica.
    \item La probabilidad de que la verdura sea de proximidad o ecológica.
\end{enumerate}
\textbf{Solución:}

Definimos los sucesos de la siguiente manera:
- Sea $P$ el suceso ``la verdura es de proximidad''.
- Sea $E$ el suceso ``la verdura es ecológica''.

Se nos proporcionan las siguientes probabilidades:
\begin{itemize}
    \item $P(P) = 0.70$ (probabilidad de que la verdura sea de proximidad).
    \item $P(E | P) = 0.30$ (probabilidad de que la verdura sea ecológica dado que es de proximidad).
    \item $P(E | P^c) = 0.10$ (probabilidad de que la verdura sea ecológica dado que no es de proximidad).
\end{itemize}

\subsection*{1. Probabilidad de que la verdura comprada no sea ecológica}

Queremos calcular $P(E^c)$, la probabilidad de que la verdura no sea ecológica. Usamos el teorema de la probabilidad total:

\[
P(E) = P(E | P) \cdot P(P) + P(E | P^c) \cdot P(P^c)
\]

Calculamos $P(P^c)$, la probabilidad de que la verdura no sea de proximidad:
\[
P(P^c) = 1 - P(P) = 1 - 0.70 = 0.30
\]

Sustituyendo los valores en la fórmula:

\[
P(E) = (0.30 \cdot 0.70) + (0.10 \cdot 0.30)
\]
\[
P(E) = 0.21 + 0.03 = 0.24
\]

Ahora calculamos $P(E^c)$ como la probabilidad complementaria de $P(E)$:
\[
P(E^c) = 1 - P(E) = 1 - 0.24 = 0.76
\]

Por lo tanto, la probabilidad de que la verdura comprada no sea ecológica es:
\[
\boxed{P(E^c) = 0.76}
\]

\subsection*{2. Probabilidad de que la verdura sea de proximidad o ecológica}

Queremos calcular $P(P \cup E)$, la probabilidad de que la verdura sea de proximidad o ecológica. Usamos la fórmula de la probabilidad de la unión de dos sucesos:

\[
P(P \cup E) = P(P) + P(E) - P(P \cap E)
\]

Ya conocemos $P(P) = 0.70$ y $P(E) = 0.24$. Ahora necesitamos calcular $P(P \cap E)$, la probabilidad de que la verdura sea de proximidad y ecológica.

Sabemos que:
\[
P(P \cap E) = P(E | P) \cdot P(P) = 0.30 \cdot 0.70 = 0.21
\]

Sustituyendo los valores en la fórmula de la unión:
\[
P(P \cup E) = 0.70 + 0.24 - 0.21
\]
\[
P(P \cup E) = 0.94 - 0.21 = 0.73
\]

Por lo tanto, la probabilidad de que la verdura sea de proximidad o ecológica es:
\[
\boxed{P(P \cup E) = 0.73}
\]


\newpage

% Ejercicio 4
\section{Ejercicio 4 (2 Puntos)}
Dados dos sucesos $A$ y $B$, con probabilidades: $P(A \cup B) = 0.55$, $P(\overline{A} \cup \overline{B}) = 0.90$, y $P(B|A) = 0.25$. Calcule:
\begin{enumerate}
    \item $P(A \cap B)$, $P(A)$, $P(B)$ y $P(B|\overline{A})$
    \item Determine si los sucesos $A$ y $B$ son independientes.
\end{enumerate}

\textbf{Solución:}

Dados los datos:
\[
P(A \cup B) = 0.55, \quad P(A^c \cup B^c) = 0.90, \quad P(B|A) = 0.25
\]

Calcularemos los valores requeridos.

\subsection*{1. Cálculo de \( P(A \cap B) \), \( P(A) \), \( P(B) \) y \( P(B|A^c) \)}

\subsubsection*{1.1 Cálculo de \( P(A \cap B) \)}

Sabemos que:
\[
P(A^c \cup B^c) = 1 - P(A \cap B)
\]
Por lo tanto,
\[
P(A \cap B) = 1 - P(A^c \cup B^c) = 1 - 0.90 = 0.10
\]

\subsubsection*{1.2 Cálculo de \( P(A) \)}

Utilizando la definición de probabilidad condicional:
\[
P(B|A) = \frac{P(A \cap B)}{P(A)}
\]
Despejando \( P(A) \):
\[
P(A) = \frac{P(A \cap B)}{P(B|A)} = \frac{0.10}{0.25} = 0.40
\]

\subsubsection*{1.3 Cálculo de \( P(B) \)}

Usamos la fórmula de la probabilidad de la unión de dos eventos:
\[
P(A \cup B) = P(A) + P(B) - P(A \cap B)
\]
Despejando \( P(B) \):
\[
P(B) = P(A \cup B) - P(A) + P(A \cap B) = 0.55 - 0.40 + 0.10 = 0.25
\]

\subsubsection*{1.4 Cálculo de \( P(B|A^c) \)}

Usamos la fórmula de probabilidad condicional, considerando el complemento de \( A \):
\[
P(B|A^c) = \frac{P(A^c \cap B)}{P(A^c)}
\]
Sabemos que:
\[
P(A^c) = 1 - P(A) = 1 - 0.40 = 0.60
\]
Y \( P(A^c \cap B) = P(B) - P(A \cap B) \):
\[
P(A^c \cap B) = 0.25 - 0.10 = 0.15
\]
Entonces,
\[
P(B|A^c) = \frac{0.15}{0.60} = 0.25
\]

\subsection*{2. Independencia de los sucesos \( A \) y \( B \)}

Dos eventos \( A \) y \( B \) son independientes si \( P(A \cap B) = P(A) \cdot P(B) \).

Calculamos \( P(A) \cdot P(B) \):
\[
P(A) \cdot P(B) = 0.40 \cdot 0.25 = 0.10
\]
Como \( P(A \cap B) = 0.10 \), entonces:
\[
P(A \cap B) = P(A) \cdot P(B)
\]
Por lo tanto, los eventos \( A \) y \( B \) son \textbf{independientes}.

\newpage

% Ejercicio 5
\section{Ejercicio 5 (2 Puntos)}
Una empresa de reparto de comida reparte platos de dos restaurantes. El 60\% de los platos provienen del primer restaurante, y el 40\% del segundo. El 50\% de los platos del primer restaurante están cocinados con productos ecológicos, y el 80\% del segundo lo están. Calcule:
\begin{enumerate}
    \item La probabilidad de que un plato esté cocinado con productos ecológicos.
    \item La probabilidad de que provenga del segundo restaurante, dado que no está cocinado con productos ecológicos.
\end{enumerate}

\textbf{Solución:} \\
% Escribe aquí la solución al ejercicio 5.

Definimos los eventos:
\begin{itemize}
    \item \( A \): El plato proviene del primer restaurante.
    \item \( B \): El plato proviene del segundo restaurante.
    \item \( E \): El plato está cocinado con productos ecológicos.
\end{itemize}

Los datos se traducen a probabilidades:
\[
P(A) = 0.60, \quad P(B) = 0.40
\]
\[
P(E|A) = 0.50, \quad P(E|B) = 0.80
\]

\subsection*{1. Probabilidad de que un plato esté cocinado con productos ecológicos}

Queremos encontrar \( P(E) \). Usamos la ley de la probabilidad total:
\[
P(E) = P(E|A) \cdot P(A) + P(E|B) \cdot P(B)
\]
Sustituyendo los valores:
\[
P(E) = (0.50)(0.60) + (0.80)(0.40)
\]
\[
P(E) = 0.30 + 0.32 = 0.62
\]

\subsection*{2. Probabilidad de que provenga del segundo restaurante, dado que no está cocinado con productos ecológicos}

Queremos encontrar \( P(B|E^c) \), donde \( E^c \) es el complemento de \( E \) (plato no cocinado con productos ecológicos).

Usamos la fórmula de la probabilidad condicional:
\[
P(B|E^c) = \frac{P(E^c|B) \cdot P(B)}{P(E^c)}
\]
Primero calculamos \( P(E^c) \) utilizando la ley de la probabilidad total:
\[
P(E^c) = 1 - P(E) = 1 - 0.62 = 0.38
\]
Sabemos que \( P(E^c|B) = 1 - P(E|B) \):
\[
P(E^c|B) = 1 - 0.80 = 0.20
\]
Por lo tanto,
\[
P(B|E^c) = \frac{(0.20)(0.40)}{0.38}
\]
\[
P(B|E^c) = \frac{0.08}{0.38} \approx 0.2105
\]

\subsection*{Resultados}

\begin{itemize}
    \item La probabilidad de que un plato esté cocinado con productos ecológicos es \( P(E) = 0.62 \).
    \item La probabilidad de que provenga del segundo restaurante, dado que no está cocinado con productos ecológicos, es \( P(B|E^c) \approx 0.2105 \).
\end{itemize}

\end{document}
