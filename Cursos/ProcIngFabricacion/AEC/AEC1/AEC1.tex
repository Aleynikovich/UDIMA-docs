\documentclass{article}
\usepackage[spanish,es-nodecimaldot]{babel}
\usepackage[utf8]{inputenc}
\usepackage{geometry}
\geometry{a4paper, margin=1in}
\usepackage{fancyhdr}
\usepackage{titling}
\usepackage{graphicx}
\usepackage{parskip}
\usepackage{amsmath}
\usepackage{float}

\pagestyle{fancy}
\fancyhf{}
\rhead{Dr. Lucas Castro Martínez}
\lhead{Procesos e Ingeniería de Fabricación}
\rfoot{Página \thepage}

\title{Actividad de evaluación contínua 1}
\author{Alexander Kalis}
\date{Fecha de Entrega: \today}

\begin{document}

\begin{titlepage}
    \centering
    \vspace*{1cm}
    \includegraphics[width=0.15\textwidth]{logo-universidad.jpg}\par\vspace{1cm}
    {\scshape\LARGE Universidad de Ingeniería Industrial \par}
    \vspace{1cm}
    {\scshape\Large Procesos e Ingeniería de Fabricación\par}
    \vspace{1.5cm}
    {\huge\bfseries Actividad de evaluación contínua 1\par}
    \vspace{2cm}
    {\Large\itshape Alexander Kalis\par}
    \vfill
    Profesor\par
    Dr. Lucas Castro Martínez

    \vfill

    % Bottom of the page
    {\large \today\par}
\end{titlepage}

\maketitle

\section{Introducción}
En este documento se analizan diversas muestras de vidrio para determinar su adecuación para distintas aplicaciones en el campo de la ingeniería de organización industrial. Las aplicaciones son variadas e incluyen pizarras de vidrio, depósitos para residuos químicos, vidrio decorativo y recipientes para calentar líquidos.

\section{Análisis de Muestras para Aplicaciones Específicas}
\subsection{Pizarra de Vidrio}
La \textbf{Muestra 9} es la más adecuada para la fabricación de una pizarra de vidrio, dada su alta transparencia e incoloridad gracias a su elevado contenido de SiO2 (80\%) y la ausencia de Fe2O3.

\subsection{Depósito de Residuos Químicos}
Para el depósito de residuos químicos, se requiere una alta resistencia a los agentes químicos. La \textbf{Muestra 1}, con un 9\% de B2O3, ofrece una resistencia química adecuada para esta aplicación.

\subsection{Vidrio Decorativo para Paneles Luminosos}
Para la tonalidad verdosa, se pueden utilizar las \textbf{Muestras 2 o 11}, aunque es posible que sea necesario añadir colorantes para lograr el tono específico. Para la tonalidad rojiza, ninguna muestra es directamente adecuada, siendo necesario también el uso de colorantes.

\subsection{Vidrio para Recipiente de Calentamiento}
La \textbf{Muestra 1} es preferible para recipientes que serán sometidos a ciclos de calentamiento y enfriamiento, debido a su contenido de SiO2 y B2O3, que confieren resistencia a los choques térmicos.

\section{Conclusión}
La selección de la muestra de vidrio adecuada para cada aplicación es crucial para el éxito en la fabricación y la funcionalidad del producto final. Las consideraciones basadas en la composición química proveen una guía preliminar, pero la confirmación a través de pruebas prácticas y el ajuste en la formulación son pasos esenciales.

\end{document}
