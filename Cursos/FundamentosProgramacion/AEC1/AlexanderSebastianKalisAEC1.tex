\documentclass[a4paper]{article}
\usepackage[spanish]{babel}
\usepackage[utf8]{inputenc}
\usepackage{amsmath, amssymb}
\usepackage{graphicx}
\usepackage{geometry}
\usepackage{fancyhdr}
\usepackage{enumitem}
\usepackage{hyperref}
\usepackage{booktabs}
\usepackage{array}

% --- PAQUETES AÑADIDOS PARA CÓDIGO C++ ---
\usepackage{listings}
\usepackage{xcolor}

% Configuración de colores para el código
\definecolor{codegreen}{rgb}{0,0.6,0}
\definecolor{codegray}{rgb}{0.5,0.5,0.5}
\definecolor{codepurple}{rgb}{0.58,0,0.82}
\definecolor{backcolour}{rgb}{0.95,0.95,0.92}

\lstdefinestyle{mystyle}{
    backgroundcolor=\color{backcolour},   
    commentstyle=\color{codegreen},
    keywordstyle=\color{blue},
    numberstyle=\tiny\color{codegray},
    stringstyle=\color{codepurple},
    basicstyle=\ttfamily\footnotesize,
    breakatwhitespace=false,         
    breaklines=true,                 
    captionpos=b,                    
    keepspaces=true,                 
    numbers=left,                    
    numbersep=5pt,                  
    showspaces=false,                
    showstringspaces=false,
    showtabs=false,                  
    tabsize=4,
    language=C++
}

\lstset{style=mystyle}
% -----------------------------------------

% Configuración de márgenes (Misma que tu plantilla)
\geometry{left=2.5cm, right=2.5cm, top=3cm, bottom=3cm}

% Configuración de Encabezado y Pie de Página
\pagestyle{fancy}
\fancyhf{}
\fancyhead[L]{Universidad a Distancia de Madrid (UDIMA)}
\fancyhead[R]{Práctica 1 - Fundamentos de Programación}
\fancyfoot[C]{\thepage}
\renewcommand{\headrulewidth}{0.4pt}
\renewcommand{\footrulewidth}{0.4pt}

% Título del documento
\title{{Actividad de Evaluación Continua 1}\\[0.5cm]
\Large{Práctica 1 de programación con C++}}
\author{Alumno: Alexander Sebastian Kalis \\ Profesor: Javier Llorente Ayuso} % He actualizado el profesor
\date{\today}

\begin{document}

% PÁGINA 1: PORTADA
\maketitle
\thispagestyle{empty}

\newpage

% PÁGINA 2: ÍNDICE
\tableofcontents
\thispagestyle{empty}
\newpage

% REINICIO DE PAGINACIÓN
\setcounter{page}{1} 

\section{Introducción y Descripción del Algoritmo}

El objetivo de esta práctica es desarrollar un programa en C++ que permita detectar patrones en series de caracteres introducidas por el usuario y generar series aleatorias basadas en patrones predefinidos. El programa ha sido diseñado siguiendo los principios de la programación estructurada y cumpliendo estrictamente con los requisitos del enunciado.

\subsection{Estructura del Programa}
El código se ha organizado en tres bloques funcionales principales, gestionados a través de un menú interactivo:

\begin{itemize}
    \item \textbf{Menú Principal:} Un bucle \texttt{do-while} que presenta las opciones al usuario y valida la entrada hasta que se selecciona la opción de salir.
    \item \textbf{Detección de Patrones (Opción 1):} Permite al usuario introducir 4 caracteres. El algoritmo analiza si estos caracteres pertenecen exclusivamente a uno de los conjuntos definidos (vocales, números, caracteres especiales) o si forman una serie general.
    \item \textbf{Generación de Series (Opción 2):} Utiliza la generación de números pseudoaleatorios para crear una serie de 10 elementos basada en la selección del usuario.
\end{itemize}

\subsection{Decisiones de Diseño y Funciones Empleadas}

Para cumplir con las especificaciones y asegurar la robustez del código, se han tomado las siguientes decisiones de implementación:

\begin{enumerate}
    \item \textbf{Librerías Estándar:} Se han utilizado únicamente librerías básicas:
    \begin{itemize}
        \item \texttt{iostream}: Para la entrada y salida de datos.
        \item \texttt{cstdlib} y \texttt{ctime}: Para la gestión de números aleatorios (\texttt{rand}, \texttt{srand}) y la semilla de tiempo.
        \item \texttt{cctype}: Para la función \texttt{tolower}, fundamental para estandarizar la entrada del usuario.
    \end{itemize}
    
    \item \textbf{Normalización de Datos:} Dado que el enunciado especifica que las vocales y consonantes deben tratarse en minúsculas, se aplica la función \texttt{tolower()} a cada carácter introducido. Además, se ha implementado un filtro para ignorar las comas (\texttt{,}) en la entrada de datos, permitiendo al usuario introducir series como "1, 2, 3, 4" sin errores.

    \item \textbf{Lógica de Detección (Banderas):} Para detectar el tipo de serie en la Opción 1, se utiliza una lógica de "banderas" booleanas (\texttt{sonVocales}, \texttt{sonNumeros}, \texttt{sonEspeciales}). Se asume inicialmente que la serie cumple todas las condiciones y se descartan a medida que se analizan los caracteres. Si al final del análisis ninguna bandera específica se mantiene verdadera, o si hay una mezcla, se determina que es una \textbf{Serie General}.
    
    \item \textbf{Generación Aleatoria:} Se emplea \texttt{rand() \% N} para generar índices aleatorios. Para los caracteres especiales, se ha definido un array constante estricto: \texttt{\{'{\#}', '\$', '\%', '\&'\}}, evitando así generar caracteres no permitidos por el enunciado.
\end{enumerate}

\newpage

\section{Código Fuente en C++}

A continuación se presenta el código fuente completo desarrollado para la actividad.

\begin{lstlisting}[caption={Código fuente main.cpp}]
/*
 * Practica 1: Fundamentos de programacion con C++
 * Asignatura: Fundamentos de programacion (Cod. 1375)
 * Autor: Alexander Sebastian Kalis
 * Fecha: 14 dic 2025
 *
 * Descripcion:
 * Programa para detectar patrones en series de 4 caracteres introducidos
 * por el usuario y generar series aleatorias de 10 elementos basadas
 * en patrones predefinidos (vocales, numericos, especiales o general).
 */

#include <iostream>
#include <cstdlib> // Necesario para rand() y srand()
#include <ctime>   // Necesario para time()
#include <cctype>  // Necesario para tolower()

using namespace std;

// --- Prototipos de funciones ---
void mostrarMenuPrincipal();
void opcionDetectarPatron();
void opcionGenerarSerie();
bool esVocal(char c);
bool esNumero(char c);
bool esEspecialPermitido(char c);

int main() {
    // Inicializamos la semilla aleatoria una unica vez al principio
    srand(time(NULL));

    int opcion = 0;

    do {
        mostrarMenuPrincipal();
        cin >> opcion;

        switch (opcion) {
            case 1:
                opcionDetectarPatron();
                break;
            case 2:
                opcionGenerarSerie();
                break;
            case 3:
                cout << "Se cerrara la aplicacion ..." << endl;
                break;
            default:
                cout << "Opcion incorrecta. Por favor, elija 1, 2 o 3." << endl;
        }
        cout << endl; // Salto de linea estetico

    } while (opcion != 3);

    return 0;
}

// --- Implementacion de funciones ---

void mostrarMenuPrincipal() {
    cout << "------------------------------------------------------" << endl;
    cout << "1) Detectar un patron en una serie." << endl;
    cout << "2) Generar una serie alfanumerica de entre las posibles." << endl;
    cout << "3) Salir del programa" << endl;
    cout << "Opcion elegida: ";
}

/**
 * Opcion 1: El usuario introduce 4 caracteres.
 * El programa determina si son vocales, numeros, especiales (#$%&) o una mezcla (General).
 */
void opcionDetectarPatron() {
    char serie[4];
    char entrada;
    int contador = 0;

    cout << "Introduce cuatro caracteres de la serie (puedes separarlos por espacios o comas):" << endl;

    // Leemos 4 caracteres validos.
    // El bucle ignora las comas ',' para que el caso de prueba 1.2 funcione correctamente.
    while (contador < 4) {
        cin >> entrada;
        
        // Convertimos a minuscula inmediatamente para estandarizar
        entrada = tolower(entrada);

        // Si es una coma, la ignoramos y pasamos al siguiente ciclo
        if (entrada == ',') {
            continue;
        }

        serie[contador] = entrada;
        contador++;
    }

    // Banderas para comprobar los tipos. Asumimos que son verdaderas hasta que se demuestre lo contrario.
    bool sonVocales = true;
    bool sonNumeros = true;
    bool sonEspeciales = true;

    for (int i = 0; i < 4; i++) {
        if (!esVocal(serie[i])) {
            sonVocales = false;
        }
        if (!esNumero(serie[i])) {
            sonNumeros = false;
        }
        if (!esEspecialPermitido(serie[i])) {
            sonEspeciales = false;
        }
    }

    // Resultados segun las banderas
    if (sonVocales) {
        cout << "La serie esta formada por vocales" << endl;
    } else if (sonNumeros) {
        cout << "La serie esta formada por caracteres numericos" << endl;
    } else if (sonEspeciales) {
        cout << "La serie esta formada por caracteres especiales" << endl;
    } else {
        // Si no cumple ninguno de los patrones especificos, es General por descarte.
        cout << "La serie es general" << endl;
    }
}

/**
 * Opcion 2: Genera 10 caracteres aleatorios segun el tipo elegido.
 */
void opcionGenerarSerie() {
    int tipo;
    cout << "Introducir el tipo de serie que queremos generar" << endl;
    cout << "1. Serie formada por vocales exclusivamente" << endl;
    cout << "2. Serie formada por caracteres numericos exclusivamente" << endl;
    cout << "3. Serie formada por caracteres especiales exclusivamente" << endl;
    cout << "4. Serie general" << endl; // Corregida la errata "geneal" del enunciado
    cin >> tipo;

    cout << "Serie: ";

    // Definimos los conjuntos de caracteres permitidos
    char vocales[] = {'a', 'e', 'i', 'o', 'u'};
    char especiales[] = {'#', '$', '%', '&'};
    // Nota: Para numeros usamos aritmetica de caracteres ('0' + n)
    // Para general usamos una mezcla

    for (int i = 0; i < 10; i++) {
        char generado;

        switch (tipo) {
            case 1: // Vocales
                generado = vocales[rand() % 5];
                break;
            case 2: // Numeros ('0' a '9')
                generado = '0' + (rand() % 10);
                break;
            case 3: // Especiales (#, $, %, &)
                generado = especiales[rand() % 4];
                break;
            case 4: // General
                // Estrategia: Elegimos aleatoriamente que tipo de caracter generar
                // 0: Vocal, 1: Numero, 2: Especial, 3: Consonante
                int subtipo = rand() % 4;
                if (subtipo == 0) {
                    generado = vocales[rand() % 5];
                } else if (subtipo == 1) {
                    generado = '0' + (rand() % 10);
                } else if (subtipo == 2) {
                    generado = especiales[rand() % 4];
                } else {
                    // Generar una letra minuscula cualquiera (a-z)
                    // Validamos que no sea vocal para que sea estrictamente consonante (opcional, pero limpio)
                    do {
                        generado = 'a' + (rand() % 26);
                    } while (esVocal(generado));
                }
                break;
        }

        cout << generado;
        // Anadimos coma y espacio si no es el ultimo elemento
        if (i < 9) {
            cout << ", ";
        }
    }
    cout << endl;
}
// --- Funciones auxiliares ---
bool esVocal(char c) {
    // Asumimos que c ya viene en minuscula
    return (c == 'a' || c == 'e' || c == 'i' || c == 'o' || c == 'u');
}

bool esNumero(char c) {
    // Comprobamos rango ASCII de digitos
    return (c >= '0' && c <= '9');
}

bool esEspecialPermitido(char c) {
    // Los caracteres especiales que indica el enunciado
    return (c == '#' || c == '$' || c == '%' || c == '&');
}

\end{lstlisting}
\newpage

\section{Pruebas Realizadas (Casos de Prueba)}

Se ha verificado el funcionamiento del programa sometiéndolo a los casos de prueba especificados en el enunciado de la práctica. A continuación se muestran las capturas de pantalla de la ejecución.

\subsection*{Caso 1.1: Detección de Vocales}
El usuario introduce una secuencia de vocales (a, e, u, a). El programa debe detectar el patrón correctamente.

% INSTRUCCIONES: Descomenta la línea de includegraphics y pon el nombre de tu archivo de imagen
\begin{figure}[h!]
    \centering
    \includegraphics[width=0.8\textwidth]{11.png}

    \caption{Ejecución del Caso 1.1}
\end{figure}

\subsection*{Caso 1.2: Detección de Números}
El usuario introduce números separados por comas (1, 9, 8, 0). El programa filtra las comas y detecta el patrón numérico.

\begin{figure}[h!]
    \centering
    \includegraphics[width=0.8\textwidth]{12.png}

    \caption{Ejecución del Caso 1.2}
\end{figure}

\subsection*{Caso 1.3: Detección de Caracteres Especiales}
El usuario introduce los caracteres especiales permitidos (\%, \&, \$, \#).

\begin{figure}[h!]
    \centering
    \includegraphics[width=0.8\textwidth]{13.png}

    \caption{Ejecución del Caso 1.3}
\end{figure}

\subsection*{Caso 1.4: Detección de Serie General}
El usuario introduce una mezcla de caracteres (1, a, b, 2). El programa determina que es una serie general.

\begin{figure}[h!]
    \centering
    \includegraphics[width=0.8\textwidth]{14.png}

    \caption{Ejecución del Caso 1.4}
\end{figure}

\newpage

\subsection*{Casos de Generación (2.1 a 2.4)}
Pruebas de la opción 2 del menú para generar series aleatorias de los distintos tipos.

\begin{figure}[h!]
    \centering
    \includegraphics[width=0.6\textwidth]{21.png}

    \caption{Ejecución de los casos de generación de series}
\end{figure}


\end{document}